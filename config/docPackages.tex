% Packages für dieses Dokument
% Sprachen
\usepackage{csquotes}
\usepackage[ngerman]{babel} % Sprache festlegen

% Kopf und Fußzeilen
\usepackage[
    automark, % Kapitelangaben in Kopfzeile automatisch erstellen
    headsepline, % Trennlinie unter Kopfzeile
    footsepline, % Trennlinie über Fußzeile
    ilines % Trennlinie linksbündig ausrichten
]{scrlayer-scrpage}

%% Für \makeglossaries
%\usepackage{makeidx}

% Schriften und Zeichenencodierung
\usepackage[T1]{fontenc} % Für < und > im Text
\usepackage{lmodern} % Scalable Font für microtype
\usepackage{textcomp} % Für ° Zeichen über \textdegree
\usepackage{amsmath} % Math-Umgebung im Text
\usepackage{siunitx} % Units in Math und Text-Umgebungen
\usepackage[gen]{eurosym} % official für immer das selbe € Symbol, gen für Anpassungen (Kursiv, etc)
\usepackage{microtype} % Wortumbrechungen verringern

\ifcsname{counterwithout}\endcsname%
%
\else%
%\usepackage{chngcntr} % Für \counterwithout auf Windows
\fi%

% Fußnoten auch bei Verwendung von hyperref
\usepackage{footnotehyper}

\usepackage{setspace}
\usepackage[
	%showframe, % Seitenlayout anzeigen, auskommentieren für finales Dokument
	left=25mm,
	right=25mm,
	top=25mm,
	bottom=25mm,
	includeheadfoot
]{geometry}

\usepackage{pdflscape} % Für gedrehte Seiten in der PDF.

\usepackage{xcolor} % Farbboxen im Text
\definecolor{mygreen}{rgb}{0,0.6,0}
\definecolor{mygray}{rgb}{0.5,0.5,0.5}
\definecolor{mymauve}{rgb}{0.58,0,0.82}

\usepackage{listings} % Code-Ausschnitte
% Formatierung von Listings
\lstset{ %
  float=hbp,
  backgroundcolor=\color{white},   % choose the background color; you must add \usepackage{color} or \usepackage{xcolor}
  %basicstyle=\footnotesize,        % the size of the fonts that are used for the code
    basicstyle=\ttfamily\color{black}\small, %\smaller,
  breakatwhitespace=false,         % sets if automatic breaks should only happen at whitespace
  breaklines=true,                 % sets automatic line breaking
    breakautoindent=true,
  captionpos=b,                    % sets the caption-position to bottom
    columns=flexible,
    tabsize=2,
    frame=false,
  commentstyle=\color{mygreen},    % comment style
  deletekeywords={...},            % if you want to delete keywords from the given language
  escapeinside={(*@}{@*)},          % if you want to add LaTeX within your code
  extendedchars=true,              % lets you use non-ASCII characters; for 8-bits encodings only, does not work with UTF-8
  %frame=single,                    % adds a frame around the code
  keepspaces=true,                 % keeps spaces in text, useful for keeping indentation of code (possibly needs columns=flexible)
  keywordstyle=\color{blue},       % keyword style
  morekeywords={*,...},            % if you want to add more keywords to the set
    emph={decltype,string,constexpr,static_assert},
    emphstyle=\color{blue},
    emph=[2]{NULL,nullptr},
    emphstyle=[2]\color{mymauve},
  numbers=left,                    % where to put the line-numbers; possible values are (none, left, right)
  numbersep=5pt,                   % how far the line-numbers are from the code
  numberstyle=\tiny\color{mygray}, % the style that is used for the line-numbers
  rulecolor=\color{black},         % if not set, the frame-color may be changed on line-breaks within not-black text (e.g. comments (green here))
  showspaces=false,                % show spaces everywhere adding particular underscores; it overrides 'showstringspaces'
  showstringspaces=false,          % underline spaces within strings only
  showtabs=false,                  % show tabs within strings adding particular underscores
  stepnumber=1,                    % the step between two line-numbers. If it's 1, each line will be numbered
  stringstyle=\color{mygreen},     % string literal style
  tabsize=2                        % sets default tabsize to 2 spaces
  %title=\lstname                   % show the filename of files included with \lstinputlisting; also try caption instead of title
}

\usepackage{bera}% optional: just to have a nice mono-spaced font

\colorlet{punct}{red!60!black}
\definecolor{background}{HTML}{EEEEEE}
\definecolor{delim}{RGB}{20,105,176}
\colorlet{numb}{magenta!60!black}

\lstdefinelanguage{json}{
	basicstyle=\normalfont\ttfamily,
	numbers=left,
	numberstyle=\scriptsize,
	stepnumber=1,
	numbersep=8pt,
	showstringspaces=false,
	breaklines=true,
	frame=lines,
	backgroundcolor=\color{background},
	literate=
	*{0}{{{\color{numb}0}}}{1}
	{1}{{{\color{numb}1}}}{1}
	{2}{{{\color{numb}2}}}{1}
	{3}{{{\color{numb}3}}}{1}
	{4}{{{\color{numb}4}}}{1}
	{5}{{{\color{numb}5}}}{1}
	{6}{{{\color{numb}6}}}{1}
	{7}{{{\color{numb}7}}}{1}
	{8}{{{\color{numb}8}}}{1}
	{9}{{{\color{numb}9}}}{1}
	{:}{{{\color{punct}{:}}}}{1}
	{,}{{{\color{punct}{,}}}}{1}
	{\{}{{{\color{delim}{\{}}}}{1}
	{\}}{{{\color{delim}{\}}}}}{1}
	{[}{{{\color{delim}{[}}}}{1}
	{]}{{{\color{delim}{]}}}}{1},
}

\usepackage{appendix} % Anhang

\usepackage[german]{fancyref}
\fancyrefchangeprefix{\fancyrefchaplabelprefix}{cha}
\fancyrefchangeprefix{\fancyreftablabelprefix}{tbl}

\usepackage[final]{graphicx} % Einbinden von jpeg-Dateien

% \usepackage{awesomebox} % Für Infoboxen (benötigt XeLaTeX)
\usepackage{mdframed} % Für Infoboxen
\usepackage{afterpage} % Pages einfügen (um Floats zu flushen).
\usepackage{placeins} % Floats flushen (erzwungen).

\usepackage{placeins}
\usepackage{tikz-uml}
\usepackage{here} %großes H um Bilder an genau der Stelle zu erzwingen
\usepackage{tabularx} %für einheitliche Tabellenbreite
\usepackage{longtable} % Für tabellen über mehrere Seiten
\usepackage{tocbasic}

% Url mit Optionen vor hyperref und biblatex!
\usepackage[hyphens]{url}

% Zitieren aus Literaturverzeichnis
\usepackage[style=numeric, backend=biber]{biblatex}

% Hyperref so spät wie möglich laden, damit keine Probleme mit Verlinkungen im Dokument entstehen
\usepackage[
    bookmarks=true,
    bookmarksopen=true,
    %hyperfootnotes=false,
    hypertexnames=false,
    linktocpage=true,
    pdfpagelabels=true,
    plainpages=false,
    % Farben für finalen Druck auf black setzen
    anchorcolor=black,
    citecolor=blue,
    colorlinks=true,
    filecolor=magenta,
    linkcolor=red,
    menucolor=red,
    urlcolor=cyan
]{hyperref}

\hypersetup{
	breaklinks=true,
	final=true,
    pdftitle={\ctitle},
    pdfauthor={\cauthor},
    pdfcreator={\cauthor},
    pdfsubject={\ctitle},
    pdfkeywords={\ctitle}
}
\usepackage{array}

