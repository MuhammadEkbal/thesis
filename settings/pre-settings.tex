% Variable je nach dem wie sie gesetzt ist, wird die Druckversion beziehungsweise die digitale Version erstellt
% Bedeutet Links farbig oder schwarz
\newboolean{print_media}
\setboolean{print_media}{false}

% Ob alle Teile angezeigt werden sollen
\newboolean{show_all}
\setboolean{show_all}{false}
%\includecomment{myComment}

%\KOMAoption{BCOR}{7mm}
%\KOMAoption{parskip}{yes}
\KOMAoption{toc}{bib,index,listof}

% Abstand itemize "andern
%\setitemize{itemsep=-0.5cm}

% Seitenränder
\geometry{
	includehead,
	includefoot,
	inner=3cm, % min 3 max 4
	outer=3cm, % min 3 max 4
	top=2cm, % min 2 max 3
	bottom=2cm % min 2 max 3
}
%\geometry{showframe}

% Farben
\definecolor{gruen}{rgb}{.10, .4, .50}
\definecolor{blau}{rgb}{.2, .2, 0.6}
\definecolor{grau}{gray}{.75}
%\definecolor{white}{gray}{.75}
\definecolor{lstColor}{rgb}{0.901,0.901,0.901} %% Back (Almost White)
\definecolor{darkred}{rgb}{0.601,0.001,0.001} %% dunkles Rot
\definecolor{darkgreen}{rgb}{.10, .4, .20}% dunkelgruen
\definecolor{cdc_Blue}{rgb}{0.0,0.355,0.652} %% Heading Blue (Oldenburg CD Ultramarine)
\definecolor{cdc_BlueM}{rgb}{0.25,0.473,0.722} %% Heading Blue Medium (Oldenburg CD Ultramarine 60%)
\definecolor{cdc_BlueL}{rgb}{0.5,0.589,0.793} %% Heading Blue Light (Oldenburg CD Ultramarine 20%)
\definecolor{cdc_Green}{rgb}{0.390,0.695,0.285} %% Heading Green (Oldenburg CD Chartreuse)
\definecolor{cdc_GreenM}{rgb}{0.559,0.758,0.438} %% Heading Green Medium (Oldenburg CD Chartreuse 60%)
\definecolor{cdc_GreenL}{rgb}{0.688,0.828,0.605} %% Heading Green Light (Oldenburg CD Chartreuse 20%)
\definecolor{color_back}{rgb}{0.941,0.941,0.941} %% Back (Almost White)

% Zeilenabstand
\onehalfspacing
% Index
\newindex{default}{idx}{ind}{Index}
%\newindex{name}{adx}{and}{Namensverzeichnis}
%\newindex{defi}{ddx}{dnd}{Definitionsverzeichnis}
%\newindex{satz}{sdx}{snd}{Satzverzeichnis}
%\newindex{lem}{ldx}{lnd}{Lemmataverzeichnis}
%\newindex{kor}{kdx}{knd}{Korollarverzeichnis}
%\newindex{bemerkung}{bdx}{bnd}{Bemerkungsverzeichnis}
%\newindex{beispiel}{bsdx}{bsnd}{Beispielverzeichnis}

%Einstellungen Code-Fragmente
%\lstset{
%  xleftmargin=13pt,
%  xrightmargin = 5pt,
%  basicstyle=\small\ttfamily,
%  columns=fullflexible,
%  showstringspaces=false,
%  commentstyle=\color{gray}\upshape,
%  literate={"a}{\"a}1 {"o}{\"o}1 {"u}{\"u}1 {"s}{\ss}1
%  {"A}{\"A}1 {"O}{\"O}1 {"U}{\"U}1
%}

%\lstloadlanguages{Java }

%% Listingstyles
\lstdefinelanguage{pseudocode}
{
	alsolanguage=Java,
	morekeywords=[1]{INPUT, OUTPUT, then, od, in, fi, foreach, begin, end, endfor, endforeach, endwhile},
	morecomment=[s][\color{darkred}]{"}{"},
	escapeinside={\%*}{*)},
	commentstyle=\color{cdc_Green},
	keywordstyle=\color{blue},
	frame=single,
	backgroundcolor=\color{white},
	xleftmargin=5pt,
	xrightmargin = 5pt,
	breaklines=true,
}
%\lstdefinelanguage{apt}
%{
%	alsolanguage=Java,
%	morekeywords=[2]{Place, Transition, Arc, Flow, PetriNet, TransitionSystem, Marking, Node, Token, ArcKey, State, Edge, EdgeKey, IEdge, IGraph, INode, PetriNetOrTransitionSystem},
%	keywordstyle=[2]\color{cdc_BlueM},
%	morecomment=[s][\color{darkred}]{"}{"},
%	commentstyle=\color{cdc_Green},
%	keywordstyle=\color{blue},
%	escapeinside={\%*}{*)},
%	frame=single,
%	%backgroundcolor=\color{lstColor},
%	breaklines=true,
%	xleftmargin=5pt,
%	xrightmargin = 5pt,
%	literate= {Ö}{{\"O}}1 {Ä}{{\"A}}1 {Ü}{{\"U}}1 {ß}{{\ss}}2 {ü}{{\"u}}1 {ä}{{\"a}}1 {ö}{{\"o}}1,
%	extendedchars=true
%}

%\lstdefinelanguage{apt-parser}
%{
%	alsolanguage=Java,
%	morekeywords=[2]{IPNParserOutput, AptPNFormatParser, AptPNFormatLexer,APTPNParserOutput,ParserContext, APTParserContext, PetriNet, ANTLRParser, TransitionSystem, SynetLTSParser, SynetPNParser, APTLTSParser, APTPNParser, APTParser,APTRenderer},
%	keywordstyle=[2]\color{cdc_BlueM},
%	morekeywords=[3]{CommonTokenStream, RecognitionException},
%	keywordstyle=[3]\color{cdc_BlueL},
%	morecomment=[s][\color{darkred}]{"}{"},
%	commentstyle=\color{cdc_Green},
%	keywordstyle=\color{blue},
%	escapeinside={\%*}{*)},
%	frame=single,
%	%backgroundcolor=\color{lstColor},
%	breaklines=true,
%	xleftmargin=5pt,
%	xrightmargin = 5pt,
%	literate= {Ö}{{\"O}}1 {Ä}{{\"A}}1 {Ü}{{\"U}}1 {ß}{{\ss}}2 {ü}{{\"u}}1 {ä}{{\"a}}1 {ö}{{\"o}}1,
%	extendedchars=true
%}

%\lstdefinelanguage{apt-format}
%{
%	morekeywords=[1]{.name, .type, .description, .transitions, .places, .flows, .initial_marking, .final_markings,
%			.states, .labels, .arcs},
%	alsoletter={.},
%	keywordstyle=[1]\color{cdc_BlueM},
%	morecomment=[s][\color{darkred}]{"}{"},
%	morecomment=[s][\color{cdc_Green}]{/*}{*/},
%	morecomment=[l][\color{cdc_Green}]{//},
%	escapeinside={\%*}{*)},
%	frame=single,
%	%backgroundcolor=\color{lstColor},
%	breaklines=true,
%	xleftmargin=5pt,
%	xrightmargin = 5pt,
%	numbers=none,
%}

%\lstdefinelanguage{ebnf}
%{
%	morecomment=[s][\color{darkred}]{'}{'},
%	escapeinside={\%*}{*)},
%	morecomment=[s][\color{cdc_Green}]{(*}{*)},
%	morecomment=[s][\color{cdc_BlueM}]{?}{?},
%	frame=single,
%	%backgroundcolor=\color{lstColor},
%	breaklines=true,
%	xleftmargin=5pt,
%	xrightmargin = 5pt,
%	showspaces=false,
%	numbers=none  
%}

%\lstdefinelanguage{my_xml}
%{
%	morestring=[b]",
%	morestring=[s]{>}{<},
%	morecomment=[s][\color{darkred}]{"}{"},
%	morecomment=[s]{<?}{?>},
%	stringstyle=\color{black},
%	identifierstyle=\color{blue},
%	keywordstyle=\color{cdc_BlueM},
%        morekeywords={classpathref,classname,fork, failonerror, value, path}
%	xleftmargin=5pt,
%	xrightmargin = 5pt,
%	numbers=none,
%}

%\lstdefinelanguage{synet}
%{
%	morecomment=[s][\color{darkred}]{'}{'},
%	escapeinside={\%*}{*)},
%	frame=single,
%	%backgroundcolor=\color{lstColor},
%	breaklines=true,
%	xleftmargin=5pt,
%	xrightmargin = 5pt,
%	numbers=none
%}

% Mathematik Umgebungen
\makeatletter
\newtheoremstyle{plainWithSeparator}
	{\item[\hskip\labelsep \theorem@headerfont ##1\ ##2\theorem@separator]}%
	{\item[\hskip\labelsep \theorem@headerfont ##1\ ##2\ (##3)\theorem@separator]}
\makeatother
\makeatletter
\newtheoremstyle{nonumberPlainWithSeparator}
	{\item[\hskip\labelsep \theorem@headerfont ##1\theorem@separator]}%
	{\item[\hskip\labelsep \theorem@headerfont ##1\ (##3)\theorem@separator]}
\makeatother
\theoremstyle{plainWithSeparator}
\theoremheaderfont{\normalfont\bfseries}
\theorembodyfont{\itshape}
\theoremseparator{}
%\theoremindent{0.5cm}%0.8cm

% theorems
\theoremsymbol{\ensuremath{\square}}
\newtheorem{theorem}{Theorem}[section]

% Lemmata
\theoremsymbol{\ensuremath{\diamondsuit}}
\newtheorem{lemma}{Lemma}[section]

% conventions
\theoremsymbol{\ensuremath{\circ}}
\newtheorem{conv}{Convention}[section]

% conjectures
\theoremsymbol{\ensuremath{\circ}}
\newtheorem{conject}{Conjecture}[section]

%% corollaries
\theoremsymbol{\ensuremath{\square}}
\newtheorem{cor}{Corollary}[section]

%%Propositions
%\theoremsymbol{\ensuremath{\square}}
%\newtheorem{proposition}{Proposition}[section]

% examples
\theorembodyfont{\upshape}
\theoremsymbol{\ensuremath{\ast}}
\newtheorem{example}{Example}[section]

% Bemerkung
%\theoremsymbol{}
%\newtheorem{bemerkung}{Bemerkung}[section]

% Definitionen
\theoremsymbol{\ensuremath{\triangle}}
\newtheorem{definition}{Definition}[section]

% Beweise
%\theoremstyle{plain}
\theoremstyle{nonumberPlainWithSeparator}
\theoremseparator{:}
\theoremindent0cm
\theoremsymbol{\rule{1.5ex}{1.5ex}}
\newtheorem{prf}{Proof}

% Hyperlinks
\ifthenelse{\boolean{print_media}}{
	\hypersetup{
		colorlinks=false
		,unicode=true
		,pdfborder={0 0 0}
		,breaklinks=true
		%,linktocpage % falls dort Umbrueche noetig sind
	}
}{
	\hypersetup{colorlinks
		,linkcolor=blau
		,unicode=true
		,urlcolor=blau
		,citecolor=gruen
		,filecolor=gruen
		,breaklinks=true
		%,linktocpage % falls dort Umbrueche noetig sind
	}
}

% Bibliographie
% urldate-format sprachabh"angig machen
%\setbibliographyfont{urldate}{\printdate}
% Definition des Aussehens der Literaturlisten
\bibliographystyle{alpha}

% Erlaubt gr"o"sere Abst"ande, damit nicht "uber den Rand geschrieben wird
%\emergencystretch=1ex

% Erlaubt den Umbruch von align-Umgebungen (Seitenumbruch)
\allowdisplaybreaks

% TIKZ
\usetikzlibrary{arrows, automata, positioning, decorations.pathmorphing}
\tikzstyle{every state}=[draw=none, rectangle, text=black, yshift=10mm]
\tikzstyle{sdots}=[yshift=5mm]
\tikzstyle{ldiag}=[left=2mm,pos=0.4]
\tikzstyle{rdiag}=[right=2mm,pos=0.4]
\tikzstyle{dopac}=[dashed, opacity=0.35]
\tikzstyle{every picture}=[
				->,
				>=stealth',
				%shorten >=1pt,
				%auto, % position of labels
				node distance=3cm,
                    		semithick, % thickness of arrows
				initial text={},
				initial where=above]

%\typearea[current]{calc} % berechnet den Divisor für das Raster zur Berechnung des Satzspiegels
