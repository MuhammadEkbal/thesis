% fix oz and amsmath conflitct part1
\makeatletter
\let\latex@dot\dot
\usepackage{oz}
\let\oz@implies\implies
\let\oz@mod\mod
\let\oz@dot\dot
\let\implies\relax
\let\mod\relax
\let\dot\latex@dot
% fix oz and amsmath conflitct end part1

% for better fitting on a4 paper
%\usepackage{a4} % kein guter Stil
% Umlaute erm"oglichen
\usepackage[utf8]{inputenc}
% Erm"oglicht beim markieren/kopieren des Textes auch die Umlaute
% au"serdem funktioniert damit auch das Trennen von W"ortern mit Umlauten
% Dadurch aber sehr h"assliche Schrift und Schriftbild, aber mit \usepackage{lmodern}

% l"asst es sich wieder heile machen
\usepackage[T1]{fontenc}
% bindet neue Schriften ein und l"asst das Dokument wieder vom Schriftbild normal aussehen
% was durch \usepackage[T1]{fontenc} zerst"ort wurde
\usepackage{lmodern}
% Neue Rechtschreibung %english f"ur Bibliographie
% Index
\usepackage{index}
% Berechnungen mit Befehlen
%\usepackage{calc}
% deutsches Literaturverzeichnis
%\usepackage{babelbib}
% sprachabh"angiges Datum
\usepackage[english,ngerman]{isodate}
% Einzelne Seiten landscape anzeigen
%\usepackage{lscape}
% Tabellen rotieren k"onnen
%\usepackage{rotating}
% F"ur etwas besser einstellbare Listen
%\usepackage{enumerate} kann weniger als enumitem
%\usepackage{enumitem}

% letztes Argument legt die Sprache des Dokuments fest
%\usepackage[english,ngerman]{babel}
\usepackage[english]{babel}
% Eigentlich veraltet, aber babel kann anscheinend keine richtige Silbentrennung
\usepackage{ngerman}
% F"ur Seitenlayout und Darstellung (vorallem eigener Seitenstil) und header und footer
\usepackage[automark,headsepline]{scrlayer-scrpage}
% f"ur den Zeilenabstand
\usepackage{setspace}
% f"ur farbigen Text
\usepackage{color}
\usepackage{xcolor}
% Times Roman (Microsoft's Times New Roman)
%\usepackage{txfonts}
% Helvetica (Microsoft's Arial)
%\usepackage{helvet}
% Matheumgebungen
%\usepackage{amsmath}
\usepackage{amssymb}
%\usepackage{amsfonts}
%\usepackage[mathcal]{euscript}
%\usepackage{mathrsfs}
%\usepackage{MnSymbol} % attention overwrites many math symbols for example the frown is very short an object with ^ appear very low.
%% for xLeftarrow and so on and mathclap for spaces
\usepackage{mathtools}
% for a fat ;
\usepackage{stmaryrd}
\SetSymbolFont{stmry}{bold}{U}{stmry}{m}{n}
% f"ur nicefrac
\usepackage{units}
% der Shuffle Operator
\usepackage{shuffle}
%\usepackage{amsthm}
% f"ur die theoreme

% fix oz and amsmath conflitct part2
\let\proof\relax
\let\endproof\relax
\let\example\relax
\let\endexample\relax
% fix oz and amsmath conflitct end part2

\usepackage[amsmath,amsthm,thmmarks]{ntheorem}

% fix oz and amsmath conflitct part3
\let\implies\oz@implies
\let\mod\oz@mod
\let\dot\oz@dot
\makeatother
% fix oz and amsmath conflitct end part3

% Bilder
%% zwei Bilder in einer Umgebung
%\usepackage{subfig} putt
%\usepackage{subfigure} putt
%\usepackage{subcaption}
% Graphiken einbinden
%\usepackage{graphicx}

% Tabellen
% Um zeilenweise Tabellen zu formatieren
%\usepackage{hhline}
% Tabellen mit einstellbarer Breite
%\usepackage{tabularx}
% Tabellen "uber mehrere Seiten
%\usepackage{longtable}
% passt longtable so an, dass auch die Eigenschaften von tabularx verwandt werden k"onnen
%\usepackage{ltxtable}
% Entzerrt Tabellenzeilen
%\usepackage{booktabs}
% Tabellen farblich markieren
%\usepackage{colortbl}
% Figuren und tables mit "H" exakt an einen Platz binden
\usepackage{float}


% Glossar
%\usepackage[ngerman]{translator}
\usepackage[english]{translator}
\usepackage[
	nowarn,       %stop glossar warninings
	nonumberlist, %keine Seitenzahlen anzeigen
	acronym,      %ein Abkürzungsverzeichnis erstellen
	toc]          %Einträge im Inhaltsverzeichnis
{glossaries}

%Umgebung für Code-Fragmente
\usepackage{listings} \lstset{	
	morecomment=[s][\color{darkred}]{"}{"},
	escapeinside={\%*}{*)},
	commentstyle=\color{cdc_Green},
	keywordstyle=\color{blue},
	frame=single,
	backgroundcolor=\color{lstColor},
	breaklines=true,
	tabsize=4
}


% f"ur Hyperlinks
\usepackage[
	plainpages=false,
	pdfpagelabels,
	breaklinks=true,
	bookmarks,
	bookmarksopen=true,
	bookmarksnumbered=true,
	pdfauthor={Muhammad Ekbal Ahmad},
	pdftitle={Transformational semantics of the combination pi-OZ for mobile processes with data},
]{hyperref}

% f"ur for-Schleifen
%\usepackage{forloop}
% f"ur if-then-else
\usepackage{ifthen}
% for margins foot header titlepage
\usepackage[a4paper]{geometry}

% TODO notes
%\usepackage[disable]{todonotes} % notes not showed
\usepackage[draft]{todonotes}   % notes showed

% Lorem ipsum blindtext \lipsum or lipsum[n]
\usepackage{lipsum}

% Block comments
%\usepackage{comment}
\usepackage{verbatim}

% Seitenlinien
%\usepackage{framed}
%\usepackage{mdframed}

%% Tikz
\usepackage{pgf}
\usepackage{tikz}

%% ebproof package to write inference tree
\usepackage{ebproof}

%% subcaption package for subfigures
\usepackage{subcaption}
