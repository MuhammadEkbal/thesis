% fix \bf issue
\DeclareOldFontCommand{\bf}{\normalfont\bfseries}{\mathbf}
\DeclareOldFontCommand{\sf}{\normalfont\bfseries}{\mathbf}

% Punkte auch bei section
%\renewcommand\l@section{\@dottedtocline{1}{1.5em}{2.3em}}

% buffer commands
%\newcommand\todo[1]{{\marginpar{\color{red}TODO: #1}}}
%\newcommand\todoI[1]{{\color{red}TODO: #1}}
\newcommand\bla{\todo[inline]{Erkl"arung/Beschreibung/F"ulltext}}
%\newcommand\lorem{Lorem ipsum dolor sit amet, consetetur sadipscing elitr, sed diam nonumy eirmod tempor invidunt ut labore et dolore magna aliquyam erat, sed diam voluptua. At vero eos et accusam et justo duo dolores et ea rebum. Stet clita kasd gubergren, no sea takimata sanctus est Lorem ipsum dolor sit amet.}

% partition commands
\newcommand{\newpart}[1]{{\color{darkred}\noindent\rule{0.5\textwidth-1cm}{1pt} #1 new \rule{0.5\textwidth-1cm}{1pt}}\newline}
\newenvironment{old}[1]
{
	\ifthenelse{\boolean{show_all}}{
		\begin{color}{darkgreen}
		\par\noindent\rule{\textwidth}{1pt}
		\begin{center}\underline{#1}\end{center}
	}{\comment{}}
}
{
	\ifthenelse{\boolean{show_all}}{
		\par\noindent\rule{\textwidth}{1pt}\newline
		\end{color}
	}{\endcomment{}}
}
\newenvironment{new}{
%   \begin{mdframed}[outerlinewidth=2,leftmargin=10,%
%     rightmargin=-10pt,backgroundcolor=white,hidealllines=true,leftline=true,%
%     innertopmargin=0pt,splittopskip=\topskip,skipbelow=\baselineskip,innerbottommargin=0pt%
%     skipabove=\baselineskip]%
	%\begin{leftbar}
%\def\FrameCommand{\vrule width 1pt \hspace{1cm}}%
%\MakeFramed {\advance\hsize\width \FrameRestore}
	\marginpar{\color{darkred}\rule{0.5cm}{1pt} $\downarrow$new \rule{0.5cm}{1pt}}
} {
	\marginpar{\color{darkred}\rule{0.5cm}{1pt} $\uparrow$new \rule{0.5cm}{1pt}}
%\endMakeFramed
	%\end{leftbar}
	%\end{mdframed}
}
% Notizen die bei druckversion nicht angezeigt werden
\newcommand\note[1]{\ifthenelse{\boolean{print_media}}{}{{\color{darkgreen}<Note: #1>}}}

% Referenzieren
\newcommand{\refDef}[1]{Definition~\ref{#1}}
\newcommand{\refFig}[1]{Figure~\ref{#1}}
\newcommand{\refLis}[1]{Listing~\ref{#1}}
\newcommand{\refTab}[1]{Table~\ref{#1}}
\newcommand{\refSec}[1]{Section~\ref{#1}}
\newcommand{\refSubSec}[1]{~\ref{#1}}
\newcommand{\refChap}[1]{Chapter~\ref{#1}}
\newcommand{\refLem}[1]{Lemma~\ref{#1}}
\newcommand{\refTheo}[1]{Theorem~\ref{#1}}
\newcommand{\refConv}[1]{Convention~\ref{#1}}
\newcommand{\refConj}[1]{Conjecture~\ref{#1}}
\newcommand{\refEq}[1]{Equation~\ref{#1}}
\newcommand{\refEx}[1]{Example~\ref{#1}}
\newcommand{\refCor}[1]{Corollary~\ref{#1}}
\newcommand{\refPro}[1]{Property~\ref{#1}}
\newcommand{\refCon}[1]{Conclusion~\ref{#1}}

% Referenzieren auf Textstelle
%\newcommand\fullRef[1]{Abschnitt \ref{#1} \glqq\nameref{#1}\grqq{} auf Seite~\pageref{#1}}
%% Referenz auf Bild
%\newcommand\imgRef[1]{Abbildung \ref{#1} auf Seite~\pageref{#1}}
%% Referenz auf Tabelle
%\newcommand\tabRef[1]{Tabelle \ref{#1} auf Seite~\pageref{#1}}
%% Referenz auf Definition
%\newcommand\defRef[1]{Definition \ref{#1} auf Seite~\pageref{#1}}

%Funktionale Anforderungen
%\newcounter{functCounter}
%\newcounter{subFunctCounter}[functCounter]
%\renewcommand\thesubFunctCounter{\Alph{functCounter}\arabic{subFunctCounter}}

%\newcommand\functRequirementPackage{
%\refstepcounter{functCounter}
%\setcounter{subFunctCounter}{0}
%}
%\newcommand\functRequirement[4][\empty]{%%% \empty: Standardwert des optionalen Parameters
%\refstepcounter{subFunctCounter}
%\begin{description}
%\itemsep-0.7cm
%\item[\thesubFunctCounter:] #2 \label{fa_\thesubFunctCounter}
%\item[Eingabe:] #3
%\item[Ausgabe:] #4
%\ifthenelse{\NOT\equal{#1}{\empty}}%%
%		{\item[Optional:] #1}%%
%{}
%\end{description}
%}

% Absatz
%\newcommand\abs{\par\medskip}

% Fetter Index
\newcommand{\findex}[2][\empty]{%%% \empty: Standardwert des optionalen Parameters
	\ifthenelse{\equal{#1}{\empty}}%%
		{\index{#2}\emph{#2}}%%
		{\index{#1}\emph{#2}}%%
	}

% In enumerate-Umgebung (a) solche item
\newcommand{\alphaEnums}{\renewcommand{\labelenumi}{(\alph{enumi})}}

%% Mathematik
\newcommand{\nin}{\not\in}
% Elementzeichen mit anschlie"sendem Mengenzeichen
\newcommand\isin{\in \mathbb}
% Potenzmenge
\newcommand{\pom}[1]{\mathbb P\left(#1\right)}
% Komplexe Zahlen
\newcommand\K{\mathbb C}
% Reelle Zahlen
\newcommand\R{\mathbb R}
% Rationale Zahlen
\newcommand\Q{\mathbb Q}
% Ganze Zahlen
\newcommand\Z{\mathbb Z}
% Nat"urliche Zahlen
\newcommand\N{\mathbb N}
% Kaligraphische Zeichen
\newcommand\calN{\mathcal N}
\newcommand\calL{\mathcal L}
%% backard-Matrix
%\newcommand\B{\mathbb B}
%% forward-Matrix
%\newcommand\F{\mathbb F}
% Menge
\newcommand{\set}[2][\empty]{%%% \empty: Standardwert des optionalen Parameters
	\ifthenelse{\equal{#1}{\empty}}
		{\left\{#2\right\}}
		{\left\{#2\; \mid \; #1 \right\}}
}
\newcommand{\pair}[2]{(#1,#2)}
\newcommand{\tripple}[3]{(#1,#2,#3)}

\newcommand{\setmulti}[2][\empty]{%%% \empty: Standardwert des optionalen Parameters
	\ifthenelse{\equal{#1}{\empty}}
		{\bigl\{#2\bigr\}}
		{\bigl\{#2\; \mid \; #1 \bigr\}}
}
\newcommand{\card}[1]{\left|#1\right|}
% Betrag
%\newcommand{\betrag}[1]{\left\lvert #1 \right\rvert}
% Das zu zeigen - Zeichen ZZ
%\newcommand{\zz}{Z\kern-.3em\raise-0.5ex\hbox{Z}:}
% Kongruenz
%\newcommand{\kong}[3]{#1\equiv#2\; mod \;#3}
%\newcommand{\inkong}[3]{#1\nequiv#2\; mod \;#3}
% Vektoren
%\newcommand{\vect}[1]{
%	\left(\begin{array}{c}
%			#1
%		\end{array}
%	\right)
%}
% Matritzen
%\newcommand{\mat}[2]{
%	\left(\begin{array}{#1}
%			#2
%		\end{array}
%	\right)
%}

%% Petri-Netz
% Pre-/Postset
\newcommand{\preset}[1]{{^\bullet{}#1}}
\newcommand{\postset}[1]{{#1^\bullet{}}}

% scalable arrows
\makeatletter
\def\slashedarrowfill@#1#2#3#4#5{%
  $\m@th\thickmuskip0mu\medmuskip\thickmuskip\thinmuskip\thickmuskip
   \relax#5#1\mkern-7mu%
   \cleaders\hbox{$#5\mkern-2mu#2\mkern-2mu$}\hfill
   \mathclap{#3}\mathclap{#2}%
   \cleaders\hbox{$#5\mkern-2mu#2\mkern-2mu$}\hfill
   \mkern-7mu#4$%
}
% already defined in mathtools
%\makeatother
%\makeatletter
%\renewcommand{\xLeftrightarrow}[2][]{\ext@arrow 0359\Leftrightarrowfill@{#1}{#2}}
%\makeatother
%\makeatletter
%\renewcommand{\xleftrightarrow}[2][]{\ext@arrow 0359\leftrightarrowfill@{#1}{#2}}
%\makeatother
%\makeatletter
%\renewcommand{\xLeftarrow}[2][]{\ext@arrow 0359\Leftarrowfill@{#1}{#2}}
%\makeatother
%\makeatletter
%\renewcommand{\xRightarrow}[2][]{\ext@arrow 0359\Rightarrowfill@{#1}{#2}}
%\makeatother
%\makeatletter
\newcommand*{\simfill@}{\arrowfill@\cdot\sim\succ}
\newcommand{\xsimrightarrow}[2][]{\ext@arrow 0359\simfill@{#1}{#2}}
\makeatother
%% negations
\makeatletter
\newcommand*{\nrightarrowfill@}{\slashedarrowfill@\relbar\relbar{\raisebox{.12em}{\tiny/}}\rightarrow}
\newcommand{\nxrightarrow}[2][]{\ext@arrow 0359\nrightarrowfill@{#1}{#2}}
\makeatother

\newcommand{\functText}[1]{\mathtt{#1}}

\newcommand{\ebnf}{\;\bigm|\;}
\newcommand{\gdw}{\mathrm{iff}}
\newcommand{\falls}{\mathrm{if}}
\newcommand{\eq}[1]{\stackrel{#1}{=}}
\newcommand{\pref}[1]{\functText{pref}(#1)}
\newcommand{\conj}[1]{\overline{#1}}
%\newcommand{\true}{\functText{true}}
%\newcommand{\false}{\functText{false}}

%% pi-calculus
\newcommand{\names}{\calN}
\newcommand{\relNames}{\mathfrak{N}}
\newcommand{\conames}{\overline{\calN}}
\newcommand{\labels}{\calL}
\newcommand{\actions}{\functText{Act}}
\newcommand{\outA}{\functText{Out}}
\newcommand{\inA}{\functText{In}}
\newcommand{\boutA}{\functText{Bout}}
\newcommand{\picalc}{$\pi$-calculus}
\newcommand{\oz}{OZ}
\newcommand{\syntdef}{::=}
\newcommand{\define}{=_{\mathrm{def}}}
\newcommand{\fnF}{\functText{fn}}
\newcommand{\bnF}{\functText{bn}}
\newcommand{\fn}[1]{\functText{fn}(#1)}
\newcommand{\bn}[1]{\functText{bn}(#1)}
\newcommand{\nF}{\functText{n}}
\newcommand{\n}[1]{\functText{n}(#1)}
\newcommand{\subF}{\functText{sub}}
\newcommand{\objF}{\functText{obj}}
\newcommand{\bindF}{\functText{bind}}
\newcommand{\sub}[1]{\functText{sub}(#1)}
\newcommand{\obj}[1]{\functText{obj}(#1)}
%\newcommand{\objbn}[1]{\functText{obj}_\functText{bn}(#1)}
\newcommand{\bind}[2]{\functText{bind}(#1,#2)}
\newcommand{\bnsubstF}{\functText{bn}_{\functText{subst}}}
\newcommand{\bnsubst}[3]{\functText{bn}_{\functText{subst}}(\subs{#1}{#2},#3)}
\newcommand{\struc}{\equiv{}}
\newcommand{\struct}[2]{#1\struc{}#2}
% sequences
\newcommand{\seqset}[1]{\functText{seq}(#1)}
\newcommand{\eseq}{\langle\rangle}
%\newcommand{\seq}[1]{\langle#1\rangle}
\newcommand{\seqconc}[2]{{#1}^{\,\frown{}\,}{#2}}
\newcommand{\seqcom}[2]{#1\leftrightharpoons{}#2}
\newcommand{\subsetsim}{\mathrel{\substack{\textstyle\subset\\[-0.5ex]\textstyle\sim}}}
\newcommand{\simcirc}{\mathrel{\substack{\textstyle\sim\\[-0.4ex]\textstyle\circ}}}
\newcommand{\ec}[1]{\left[#1\right]}%_{\alpha}}
\newcommand{\bigstep}[1]{\xRightarrow{#1}}
\newcommand{\len}[1]{\#(#1)}
%\newcommand{\parl}[1]{\overrightarrow{#1}}
\newcommand{\parl}[1]{\vec{#1}}
\newcommand{\substF}{\sigma}
\newcommand{\subst}[1]{\left(#1\right)\sigma}
%\newcommand{\subs}[2]{\left\{\nicefrac{#1}{#2}\right\}}
\newcommand{\substitue}[2]{\left\{\nicefrac{#1}{#2}\right\}}

% prefix
\newcommand{\singleout}[1]{\bar{#1}}
\newcommand{\out}[2]{\overline{#1}\langle#2\rangle}
\newcommand{\outa}[2]{\overline{#1}\langle#2\rangle}
\newcommand{\bout}[2]{\overline{#1}(#2)}
\newcommand{\bouta}[2]{\overline{#1}(#2)}
\newcommand{\inp}[2]{#1(#2)}
\newcommand{\inpa}[2]{#1\,#2}
% processes
\newcommand{\procs}{\mathcal P^{\pi}}
\newcommand{\procsApha}{\mathcal P^{\pi}_\alpha}


\newcommand{\stats}{\mathcal S^{\oz}}
\newcommand{\ozclasses}{\mathcal C^{\oz}}

\newcommand{\sums}{\procs_{M}}
\newcommand{\procsesf}{\procs_{\functText{esf}}}
\newcommand{\procsresf}{\procs_{\functText{resf}}}
\newcommand{\procsrecf}{\procs_{\functText{recf}}}
\newcommand{\procchoice}[2]{#1 + #2}
\newcommand{\procsum}{\sum_{i\in I}{\pi_i.P_i}}
\newcommand{\procpar}[2]{#1 \mid #2}
\newcommand{\procparM}[2]{#1 \mid\mid #2}

%\newcommand{\procres}[3][\empty]{%%% \empty: Standardwert des optionalen Parameters
%	\ifthenelse{\equal{#1}{\empty}}{
%		\ifthenelse{\equal{#2}{\empty}}{
%			\textnormal{\underline{\texttt{new}}}
%		}{
%			\textnormal{\underline{\texttt{new}}}\,#2\;#3
%		}
%	}{
%		\textnormal{\underline{\texttt{new}}}\,#2\left(#3\right)
%	}
%}
\newcommand{\procres}[3][\empty]{%%% \empty: Standardwert des optionalen Parameters
	\ifthenelse{\equal{#1}{\empty}}{
		\ifthenelse{\equal{#2}{\empty}}{
			\underline{\functText{new}}
		}{
			\underline{\functText{new}}\,#2\;#3
		}
	}{
		\underline{\functText{new}}\,#2\left(#3\right)
	}
}
\newcommand{\proccall}[2]{#1\langle#2\rangle}
\newcommand{\procdef}[2]{#1(#2)}
\newcommand{\proczero}{\mathbf{0}}
% substitution
\newcommand{\transp}[2]{\left\{#1\leftrightarrow{}#2\right\}}% substitution
\newcommand{\transpT}[2]{\theta_{#1}(#2)}
\newcommand{\supp}[1]{\functText{supp}(#1)}
\newcommand{\cosupp}[1]{\functText{cosupp}(#1)}
% alpha conversion
\newcommand{\alphaeq}{=_\alpha}
% transitions
%% sangiorgi
\newcommand{\transs}[1]{\xrightarrow{#1}}
\newcommand{\intrans}[2]{\transs{\inpa{#1}{#2}}}
\newcommand{\outtrans}[2]{\transs{\out{#1}{#2}}}
\newcommand{\bouttrans}[2]{\transs{\bout{#1}{#2}}}
\newcommand{\tautrans}{\transs{\tau}}
%% mine
\newcommand{\trans}[1]{\xsimrightarrow{#1}}
%\newcommand{\trans}[1]{\rightsquigarrow}
\newcommand{\transin}[2]{\trans{\inp{#1}{#2}}}
\newcommand{\transout}[2]{\trans{\out{#1}{#2}}}
\newcommand{\transbout}[2]{\trans{\bout{#1}{#2}}}
\newcommand{\transtau}{\trans{\tau}}
% rules name, pr"amisse1, pr"amisse2, konklusion, optional Anwendungsbedingung
\newcommand{\kalRule}[5][\empty]{%%% \empty: Standardwert des optionalen Parameters
	\underline{\scriptstyle#2}:\;
	\ifthenelse{\equal{#3}{\empty}}{
		\ifthenelse{\equal{#4}{\empty}}{
			\dfrac{}{#5}
		}{
			\dfrac{#4}{#5}
		}
	}{
		\dfrac{#3 \quad #4}{#5}
	}
	\ifthenelse{\NOT\equal{#1}{\empty}}{\;\;#1}{}
}

\newcommand{\kalRuleM}[7][\empty]{%%% \empty: Standardwert des optionalen Parameters
	\underline{\scriptstyle#2}:\;
	\ifthenelse{\equal{#3}{\empty}}
		{
			\dfrac{#4 \quad #5 \quad #6}{#7}
		}
		{
			\dfrac{#3 \quad #4 \quad #5 \quad #6}{#7}
		}
	\ifthenelse{\NOT\equal{#1}{\empty}}{\;\;#1}{}
}
% names of rules
\newcommand{\rulename}[1]{$#1$}
\newcommand{\etau}{\rulename{E-TAU}}
\newcommand{\ecall}{\rulename{E-CALL}}
\newcommand{\eout}{\rulename{E-OUT}}
\newcommand{\ein}{\rulename{E-IN}}
\newcommand{\esuml}{\rulename{E-SUM_L}}
\newcommand{\esumr}{\rulename{E-SUM_R}}
\newcommand{\eres}{\rulename{E-RES}}
\newcommand{\eparl}{\rulename{E-PAR_L}}
\newcommand{\eparr}{\rulename{E-PAR_R}}
\newcommand{\eopen}{\rulename{E-OPEN}}
\newcommand{\ecoml}{\rulename{E-COM_L}}
\newcommand{\ecomr}{\rulename{E-COM_R}}
\newcommand{\eclosel}{\rulename{E-CLOSE_L}}
\newcommand{\ecloser}{\rulename{E-CLOSE_R}}
% strong / weak Bisimulation
\newcommand{\sbisim}{\sim}
\newcommand{\wbisim}{\approx}
\newcommand{\simu}{\mathcal{S}}
\newcommand{\weaksimuset}[2]{\simu^{#1,#2}}
\newcommand{\simuset}{\weaksimuset{P}{Q}}
% denotational semantics
\newcommand{\tr}{\functText{Traces}}
\newcommand{\trI}[1]{\mathcal{T_I}(\ec{#1})}
\newcommand{\tracesI}[3]{\mathcal T_{#1}(\ec{\procpar{#2}{#3}})}
\newcommand{\traces}[1][\empty]{
		\ifthenelse{\equal{#1}{\empty}}
		{\mathcal T}
		{\mathcal T(\ec{#1})}
}
\newcommand{\tracesR}[1]{\mathcal{T}_\relNames(\ec{#1})}

\newcommand{\res}[3]{\functText{res}(#1,#2,\traces[#3])}
\newcommand{\failures}[1][\empty]{
		\ifthenelse{\equal{#1}{\empty}}
		{\mathcal F}
		{\mathcal F(\ec{#1})}
}
\newcommand{\accepatances}[1][\empty]{
		\ifthenelse{\equal{#1}{\empty}}
		{\mathcal AC}
		{\mathcal AC(\ec{#1})}
}
\newcommand{\fd}[1][\empty]{
		\ifthenelse{\equal{#1}{\empty}}
		{\mathcal{FD}}
		{\mathcal{FD}(#1)}
}
% refinement
\newcommand{\refi}[1][\empty]{
		\ifthenelse{\equal{#1}{\empty}}
		{\,\sqsubseteq_{\mathcal T}\,}
		{\,\sqsubseteq_{\mathcal{#1}}\,}
}
\newcommand{\rrefi}[1][\empty]{
		\ifthenelse{\equal{#1}{\empty}}
		{\,\sqsupseteq_{\mathcal T}\,}
		{\,\sqsupseteq_{\mathcal{#1}}\,}
}
\newcommand{\nrefi}[1][\empty]{
		\ifthenelse{\equal{#1}{\empty}}
		{\,\not\sqsubseteq_{\mathcal T}\,}
		{\,\not\sqsubseteq_{\mathcal{#1}}\,}
}
\newcommand{\refiF}[1][\empty]{
		\ifthenelse{\equal{#1}{\empty}}
		{\,\sqsubseteq_{\mathcal F}\,}
		{\,\sqsubseteq_{\mathcal{#1}}\,}
}
\newcommand{\refiAC}[1][\empty]{
		\ifthenelse{\equal{#1}{\empty}}
		{\,\sqsubseteq_{\mathcal AC}\,}
		{\,\sqsubseteq_{\mathcal{#1}}\,}
}
\newcommand{\refiFSimp}[1][\empty]{
		\ifthenelse{\equal{#1}{\empty}}
		{\,\sqsubseteq^{'}_{\mathcal F}\,}
		{\,\sqsubseteq_{\mathcal{#1}}\,}
}

\newcommand{\nrefiF}[1][\empty]{
		\ifthenelse{\equal{#1}{\empty}}
		{\,\not\sqsubseteq_{\mathcal F}\,}
		{\,\not\sqsubseteq_{\mathcal{#1}}\,}
}

\newcommand{\subseteqSimplified}[1]{\subseteq_{#1}}
