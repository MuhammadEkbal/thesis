% mainfile: Refinement.tex
%%% =============================================== Preamble BEGINN =====================================================================
%% Lade:
% Dokumentenklasse
\documentclass[
	%draft,	% Zeigt nur Rahmen f"ur Bilder an
	english,%german,
	12pt,		% Schriftgr"o"se 11 Arial oder 12 Times New Roman
	%fleqn,	% fleqn für Matheumgebungen linksb"undig
		% Layout:
	open=right,    % vor dem chapter eine leere linke Seite
	twoside,	% zweiseitiges
	a4paper,	% Papierformat
	%normalheadings,
	%DIV=calc, % berechnet mit \typearea[current]{calc} den richtigen Seitendivisor für die Rasterung der Satzspiegels
	%DIV11,
	%BCOR5mm % Bindekorrektur, was weiß bleibt beim druck und nicht für die Berechnung des Satzspiegels benützt wird
	%left=3cm, % max 4
	%right=3cm, % max 4
	%top=2cm, % max 3
	%bottom=2cm % max 3,
	enabledeprecatedfontcommands
] {scrreprt}		% Art des Dokumentes

% Pakete
% for better fitting on a4 paper
%\usepackage{a4} % kein guter Stil
% Umlaute erm"oglichen
\usepackage[utf8]{inputenc}
% Erm"oglicht beim markieren/kopieren des Textes auch die Umlaute
% au"serdem funktioniert damit auch das Trennen von W"ortern mit Umlauten
% Dadurch aber sehr h"assliche Schrift und Schriftbild, aber mit \usepackage{lmodern}
% l"asst es sich wieder heile machen
\usepackage[T1]{fontenc}
% bindet neue Schriften ein und l"asst das Dokument wieder vom Schriftbild normal aussehen
% was durch \usepackage[T1]{fontenc} zerst"ort wurde
\usepackage{lmodern}
% Neue Rechtschreibung %english f"ur Bibliographie

% Index
\usepackage{index}
% Berechnungen mit Befehlen
%\usepackage{calc}
% deutsches Literaturverzeichnis
%\usepackage{babelbib}
% sprachabh"angiges Datum
\usepackage[english,ngerman]{isodate}
% Einzelne Seiten landscape anzeigen
%\usepackage{lscape}
% Tabellen rotieren k"onnen
%\usepackage{rotating}
% F"ur etwas besser einstellbare Listen
%\usepackage{enumerate} kann weniger als enumitem
%\usepackage{enumitem}

% letztes Argument legt die Sprache des Dokuments fest
%\usepackage[english,ngerman]{babel}
\usepackage[english]{babel}
% Eigentlich veraltet, aber babel kann anscheinend keine richtige Silbentrennung
\usepackage{ngerman}
% F"ur Seitenlayout und Darstellung (vorallem eigener Seitenstil) und header und footer
\usepackage[automark,headsepline]{scrpage2}

% f"ur den Zeilenabstand
\usepackage{setspace}
% f"ur farbigen Text
\usepackage{color}
\usepackage{xcolor}
% Times Roman (Microsoft's Times New Roman)
%\usepackage{txfonts}
% Helvetica (Microsoft's Arial)
%\usepackage{helvet}
% Matheumgebungen
%\usepackage{amsmath}
\usepackage{amssymb}
%\usepackage{amsfonts}
%\usepackage[mathcal]{euscript}
%\usepackage{mathrsfs}
%\usepackage{MnSymbol} % attention overwrites many math symbols for example the frown is very short an object with ^ appear very low.
%% for xLeftarrow and so on and mathclap for spaces
\usepackage{mathtools}
% for a fat ;
\usepackage{stmaryrd}
% f"ur nicefrac
\usepackage{units}
% der Shuffle Operator
\usepackage{shuffle}
%\usepackage{amsthm}
% f"ur die theoreme
\usepackage[amsmath,amsthm,thmmarks]{ntheorem}

% Bilder
%% zwei Bilder in einer Umgebung
%\usepackage{subfig} putt
%\usepackage{subfigure} putt
%\usepackage{subcaption}
% Graphiken einbinden
%\usepackage{graphicx}

% Tabellen
% Um zeilenweise Tabellen zu formatieren
%\usepackage{hhline}
% Tabellen mit einstellbarer Breite
%\usepackage{tabularx}
% Tabellen "uber mehrere Seiten
%\usepackage{longtable}
% passt longtable so an, dass auch die Eigenschaften von tabularx verwandt werden k"onnen
%\usepackage{ltxtable}
% Entzerrt Tabellenzeilen
%\usepackage{booktabs}
% Tabellen farblich markieren
%\usepackage{colortbl}
% Figuren und tables mit "H" exakt an einen Platz binden
\usepackage{float}


% Glossar
%\usepackage[ngerman]{translator}
\usepackage[english]{translator}
\usepackage[
	nowarn,       %stop glossar warninings
	nonumberlist, %keine Seitenzahlen anzeigen
	acronym,      %ein Abkürzungsverzeichnis erstellen
	toc]          %Einträge im Inhaltsverzeichnis
{glossaries}

%Umgebung für Code-Fragmente
\usepackage{listings} \lstset{	
	morecomment=[s][\color{darkred}]{"}{"},
	escapeinside={\%*}{*)},
	commentstyle=\color{cdc_Green},
	keywordstyle=\color{blue},
	frame=single,
	backgroundcolor=\color{lstColor},
	breaklines=true,
	tabsize=4
}


% f"ur Hyperlinks
\usepackage[
	plainpages=false,
	pdfpagelabels,
	breaklinks=true,
	a4paper,
	bookmarks,
	bookmarksopen=true,
	bookmarksnumbered=true,
	pdfauthor={Manuel Gieseking},
	pdftitle={Refinement of pi-calculus processes},
]{hyperref}

% f"ur for-Schleifen
%\usepackage{forloop}
% f"ur if-then-else
\usepackage{ifthen}
% for margins foot header titlepage
\usepackage[a4paper]{geometry}

% TODO notes
%\usepackage[disable]{todonotes} % notes not showed
\usepackage[draft]{todonotes}   % notes showed

% Lorem ipsum blindtext \lipsum or lipsum[n]
\usepackage{lipsum}

% Block comments
%\usepackage{comment}
\usepackage{verbatim}

% Seitenlinien
%\usepackage{framed}
%\usepackage{mdframed}

%% Tikz
\usepackage{pgf}
\usepackage{tikz}

% Neue Befehle
% fix \bf issue
\DeclareOldFontCommand{\bf}{\normalfont\bfseries}{\mathbf}

% Punkte auch bei section
%\renewcommand\l@section{\@dottedtocline{1}{1.5em}{2.3em}}

% buffer commands
%\newcommand\todo[1]{{\marginpar{\color{red}TODO: #1}}}
%\newcommand\todoI[1]{{\color{red}TODO: #1}}
\newcommand\bla{\todo[inline]{Erkl"arung/Beschreibung/F"ulltext}}
%\newcommand\lorem{Lorem ipsum dolor sit amet, consetetur sadipscing elitr, sed diam nonumy eirmod tempor invidunt ut labore et dolore magna aliquyam erat, sed diam voluptua. At vero eos et accusam et justo duo dolores et ea rebum. Stet clita kasd gubergren, no sea takimata sanctus est Lorem ipsum dolor sit amet.}

% partition commands
\newcommand{\newpart}[1]{{\color{darkred}\noindent\rule{0.5\textwidth-1cm}{1pt} #1 new \rule{0.5\textwidth-1cm}{1pt}}\newline}
\newenvironment{old}[1]
{
	\ifthenelse{\boolean{show_all}}{
		\begin{color}{darkgreen}
		\par\noindent\rule{\textwidth}{1pt}
		\begin{center}\underline{#1}\end{center}
	}{\comment{}}
}
{
	\ifthenelse{\boolean{show_all}}{
		\par\noindent\rule{\textwidth}{1pt}\newline
		\end{color}
	}{\endcomment{}}
}
\newenvironment{new}{
%   \begin{mdframed}[outerlinewidth=2,leftmargin=10,%
%     rightmargin=-10pt,backgroundcolor=white,hidealllines=true,leftline=true,%
%     innertopmargin=0pt,splittopskip=\topskip,skipbelow=\baselineskip,innerbottommargin=0pt%
%     skipabove=\baselineskip]%
	%\begin{leftbar}
%\def\FrameCommand{\vrule width 1pt \hspace{1cm}}%
%\MakeFramed {\advance\hsize\width \FrameRestore}
	\marginpar{\color{darkred}\rule{0.5cm}{1pt} $\downarrow$new \rule{0.5cm}{1pt}}
} {
	\marginpar{\color{darkred}\rule{0.5cm}{1pt} $\uparrow$new \rule{0.5cm}{1pt}}
%\endMakeFramed
	%\end{leftbar}
	%\end{mdframed}
}
% Notizen die bei druckversion nicht angezeigt werden
\newcommand\note[1]{\ifthenelse{\boolean{print_media}}{}{{\color{darkgreen}<Note: #1>}}}

% Referenzieren
\newcommand{\refDef}[1]{Definition~\ref{#1}}
\newcommand{\refFig}[1]{Figure~\ref{#1}}
\newcommand{\refSec}[1]{Section~\ref{#1}}
\newcommand{\refChap}[1]{Chapter~\ref{#1}}
\newcommand{\refLem}[1]{Lemma~\ref{#1}}
\newcommand{\refTheo}[1]{Theorem~\ref{#1}}
\newcommand{\refConv}[1]{Convention~\ref{#1}}
\newcommand{\refConj}[1]{Conjecture~\ref{#1}}
\newcommand{\refEq}[1]{Equation~\ref{#1}}
\newcommand{\refEx}[1]{Example~\ref{#1}}
\newcommand{\refCor}[1]{Corollary~\ref{#1}}
% Referenzieren auf Textstelle
%\newcommand\fullRef[1]{Abschnitt \ref{#1} \glqq\nameref{#1}\grqq{} auf Seite~\pageref{#1}}
%% Referenz auf Bild
%\newcommand\imgRef[1]{Abbildung \ref{#1} auf Seite~\pageref{#1}}
%% Referenz auf Tabelle
%\newcommand\tabRef[1]{Tabelle \ref{#1} auf Seite~\pageref{#1}}
%% Referenz auf Definition
%\newcommand\defRef[1]{Definition \ref{#1} auf Seite~\pageref{#1}}

%Funktionale Anforderungen
%\newcounter{functCounter}
%\newcounter{subFunctCounter}[functCounter]
%\renewcommand\thesubFunctCounter{\Alph{functCounter}\arabic{subFunctCounter}}

%\newcommand\functRequirementPackage{
%\refstepcounter{functCounter}
%\setcounter{subFunctCounter}{0}
%}
%\newcommand\functRequirement[4][\empty]{%%% \empty: Standardwert des optionalen Parameters
%\refstepcounter{subFunctCounter}
%\begin{description}
%\itemsep-0.7cm
%\item[\thesubFunctCounter:] #2 \label{fa_\thesubFunctCounter}
%\item[Eingabe:] #3
%\item[Ausgabe:] #4
%\ifthenelse{\NOT\equal{#1}{\empty}}%%
%		{\item[Optional:] #1}%%
%{}
%\end{description}
%}

% Absatz
%\newcommand\abs{\par\medskip}

% Fetter Index
\newcommand{\findex}[2][\empty]{%%% \empty: Standardwert des optionalen Parameters
	\ifthenelse{\equal{#1}{\empty}}%%
		{\index{#2}\emph{#2}}%%
		{\index{#1}\emph{#2}}%%
	}

% In enumerate-Umgebung (a) solche item
\newcommand{\alphaEnums}{\renewcommand{\labelenumi}{(\alph{enumi})}}

%% Mathematik
\newcommand{\nin}{\not\in}
% Elementzeichen mit anschlie"sendem Mengenzeichen
\newcommand\isin{\in \mathbb}
% Potenzmenge
\newcommand{\pom}[1]{\mathbb P\left(#1\right)}
% Komplexe Zahlen
\newcommand\K{\mathbb C}
% Reelle Zahlen
\newcommand\R{\mathbb R}
% Rationale Zahlen
\newcommand\Q{\mathbb Q}
% Ganze Zahlen
\newcommand\Z{\mathbb Z}
% Nat"urliche Zahlen
\newcommand\N{\mathbb N}
% Kaligraphische Zeichen
\newcommand\calN{\mathcal N}
\newcommand\calL{\mathcal L}
%% backard-Matrix
%\newcommand\B{\mathbb B}
%% forward-Matrix
%\newcommand\F{\mathbb F}
% Menge
\newcommand{\set}[2][\empty]{%%% \empty: Standardwert des optionalen Parameters
	\ifthenelse{\equal{#1}{\empty}}
		{\left\{#2\right\}}
		{\left\{#2\; \mid \; #1 \right\}}
}
\newcommand{\setmulti}[2][\empty]{%%% \empty: Standardwert des optionalen Parameters
	\ifthenelse{\equal{#1}{\empty}}
		{\bigl\{#2\bigr\}}
		{\bigl\{#2\; \mid \; #1 \bigr\}}
}
\newcommand{\card}[1]{\left|#1\right|}
% Betrag
%\newcommand{\betrag}[1]{\left\lvert #1 \right\rvert}
% Das zu zeigen - Zeichen ZZ
%\newcommand{\zz}{Z\kern-.3em\raise-0.5ex\hbox{Z}:}
% Kongruenz
%\newcommand{\kong}[3]{#1\equiv#2\; mod \;#3}
%\newcommand{\inkong}[3]{#1\nequiv#2\; mod \;#3}
% Vektoren
%\newcommand{\vect}[1]{
%	\left(\begin{array}{c}
%			#1
%		\end{array}
%	\right)
%}
% Matritzen
%\newcommand{\mat}[2]{
%	\left(\begin{array}{#1}
%			#2
%		\end{array}
%	\right)
%}

%% Petri-Netz
% Pre-/Postset
\newcommand{\preset}[1]{{^\bullet{}#1}}
\newcommand{\postset}[1]{{#1^\bullet{}}}

% scalable arrows
\makeatletter
\def\slashedarrowfill@#1#2#3#4#5{%
  $\m@th\thickmuskip0mu\medmuskip\thickmuskip\thinmuskip\thickmuskip
   \relax#5#1\mkern-7mu%
   \cleaders\hbox{$#5\mkern-2mu#2\mkern-2mu$}\hfill
   \mathclap{#3}\mathclap{#2}%
   \cleaders\hbox{$#5\mkern-2mu#2\mkern-2mu$}\hfill
   \mkern-7mu#4$%
}
% already defined in mathtools
%\makeatother
%\makeatletter
%\renewcommand{\xLeftrightarrow}[2][]{\ext@arrow 0359\Leftrightarrowfill@{#1}{#2}}
%\makeatother
%\makeatletter
%\renewcommand{\xleftrightarrow}[2][]{\ext@arrow 0359\leftrightarrowfill@{#1}{#2}}
%\makeatother
%\makeatletter
%\renewcommand{\xLeftarrow}[2][]{\ext@arrow 0359\Leftarrowfill@{#1}{#2}}
%\makeatother
%\makeatletter
%\renewcommand{\xRightarrow}[2][]{\ext@arrow 0359\Rightarrowfill@{#1}{#2}}
%\makeatother
%\makeatletter
\newcommand*{\simfill@}{\arrowfill@\cdot\sim\succ}
\newcommand{\xsimrightarrow}[2][]{\ext@arrow 0359\simfill@{#1}{#2}}
\makeatother
%% negations
\makeatletter
\newcommand*{\nrightarrowfill@}{\slashedarrowfill@\relbar\relbar{\raisebox{.12em}{\tiny/}}\rightarrow}
\newcommand{\nxrightarrow}[2][]{\ext@arrow 0359\nrightarrowfill@{#1}{#2}}
\makeatother

\newcommand{\functText}[1]{\mathtt{#1}}

\newcommand{\ebnf}{\;\bigm|\;}
\newcommand{\gdw}{\mathrm{iff}}
\newcommand{\falls}{\mathrm{if}}
\newcommand{\eq}[1]{\stackrel{#1}{=}}
\newcommand{\pref}[1]{\functText{pref}(#1)}
\newcommand{\conj}[1]{\overline{#1}}
%\newcommand{\true}{\functText{true}}
%\newcommand{\false}{\functText{false}}

%% pi-calculus
\newcommand{\names}{\calN}
\newcommand{\relNames}{\mathfrak{N}}
\newcommand{\conames}{\overline{\calN}}
\newcommand{\labels}{\calL}
\newcommand{\actions}{\functText{Act}}
\newcommand{\outA}{\functText{Out}}
\newcommand{\inA}{\functText{In}}
\newcommand{\boutA}{\functText{Bout}}
\newcommand{\picalc}{$\pi$-calculus}
\newcommand{\syntdef}{::=}
\newcommand{\define}{=_{\mathrm{def}}}
\newcommand{\fnF}{\functText{fn}}
\newcommand{\bnF}{\functText{bn}}
\newcommand{\fn}[1]{\functText{fn}(#1)}
\newcommand{\bn}[1]{\functText{bn}(#1)}
\newcommand{\nF}{\functText{n}}
\newcommand{\n}[1]{\functText{n}(#1)}
\newcommand{\subF}{\functText{sub}}
\newcommand{\objF}{\functText{obj}}
\newcommand{\bindF}{\functText{bind}}
\newcommand{\sub}[1]{\functText{sub}(#1)}
\newcommand{\obj}[1]{\functText{obj}(#1)}
%\newcommand{\objbn}[1]{\functText{obj}_\functText{bn}(#1)}
\newcommand{\bind}[2]{\functText{bind}(#1,#2)}
\newcommand{\bnsubstF}{\functText{bn}_{\functText{subst}}}
\newcommand{\bnsubst}[3]{\functText{bn}_{\functText{subst}}(\subs{#1}{#2},#3)}
\newcommand{\struc}{\equiv{}}
\newcommand{\struct}[2]{#1\struc{}#2}
% sequences
\newcommand{\seqset}[1]{\functText{seq}(#1)}
\newcommand{\eseq}{\langle\rangle}
%\newcommand{\seq}[1]{\langle#1\rangle}
\newcommand{\seqconc}[2]{{#1}^{\,\frown{}\,}{#2}}
\newcommand{\seqcom}[2]{#1\leftrightharpoons{}#2}
\newcommand{\subsetsim}{\mathrel{\substack{\textstyle\subset\\[-0.5ex]\textstyle\sim}}}
\newcommand{\simcirc}{\mathrel{\substack{\textstyle\sim\\[-0.4ex]\textstyle\circ}}}
\newcommand{\ec}[1]{\left[#1\right]}%_{\alpha}}
\newcommand{\bigstep}[1]{\xRightarrow{#1}}
\newcommand{\len}[1]{\#(#1)}
%\newcommand{\parl}[1]{\overrightarrow{#1}}
\newcommand{\parl}[1]{\vec{#1}}
\newcommand{\substF}{\sigma}
\newcommand{\subst}[1]{\left(#1\right)\sigma}
%\newcommand{\subs}[2]{\left\{\nicefrac{#1}{#2}\right\}}
\newcommand{\substitue}[2]{\left\{\nicefrac{#1}{#2}\right\}}

% prefix
\newcommand{\singleout}[1]{\bar{#1}}
\newcommand{\out}[2]{\overline{#1}\langle#2\rangle}
\newcommand{\outa}[2]{\overline{#1}\langle#2\rangle}
\newcommand{\bout}[2]{\overline{#1}(#2)}
\newcommand{\bouta}[2]{\overline{#1}(#2)}
\newcommand{\inp}[2]{#1(#2)}
\newcommand{\inpa}[2]{#1\,#2}
% processes
\newcommand{\procs}{\mathcal P^{\pi}}
\newcommand{\sums}{\procs_{M}}
\newcommand{\procsesf}{\procs_{\functText{esf}}}
\newcommand{\procsresf}{\procs_{\functText{resf}}}
\newcommand{\procsrecf}{\procs_{\functText{recf}}}
\newcommand{\procchoice}[2]{#1 + #2}
\newcommand{\procsum}{\sum_{i\in I}{\pi_i.P_i}}
\newcommand{\procpar}[2]{#1 \mid #2}
%\newcommand{\procres}[3][\empty]{%%% \empty: Standardwert des optionalen Parameters
%	\ifthenelse{\equal{#1}{\empty}}{
%		\ifthenelse{\equal{#2}{\empty}}{
%			\textnormal{\underline{\texttt{new}}}
%		}{
%			\textnormal{\underline{\texttt{new}}}\,#2\;#3
%		}
%	}{
%		\textnormal{\underline{\texttt{new}}}\,#2\left(#3\right)
%	}
%}
\newcommand{\procres}[3][\empty]{%%% \empty: Standardwert des optionalen Parameters
	\ifthenelse{\equal{#1}{\empty}}{
		\ifthenelse{\equal{#2}{\empty}}{
			\underline{\functText{new}}
		}{
			\underline{\functText{new}}\,#2\;#3
		}
	}{
		\underline{\functText{new}}\,#2\left(#3\right)
	}
}
\newcommand{\proccall}[2]{#1\langle#2\rangle}
\newcommand{\procdef}[2]{#1(#2)}
\newcommand{\proczero}{\mathbf{0}}
% substitution
\newcommand{\transp}[2]{\left\{#1\leftrightarrow{}#2\right\}}% substitution
\newcommand{\transpT}[2]{\theta_{#1}(#2)}
\newcommand{\supp}[1]{\functText{supp}(#1)}
\newcommand{\cosupp}[1]{\functText{cosupp}(#1)}
% alpha conversion
\newcommand{\alphaeq}{=_\alpha}
% transitions
%% sangiorgi
\newcommand{\transs}[1]{\xrightarrow{#1}}
\newcommand{\intrans}[2]{\transs{\inpa{#1}{#2}}}
\newcommand{\outtrans}[2]{\transs{\out{#1}{#2}}}
\newcommand{\bouttrans}[2]{\transs{\bout{#1}{#2}}}
\newcommand{\tautrans}{\transs{\tau}}
%% mine
\newcommand{\trans}[1]{\xsimrightarrow{#1}}
%\newcommand{\trans}[1]{\rightsquigarrow}
\newcommand{\transin}[2]{\trans{\inp{#1}{#2}}}
\newcommand{\transout}[2]{\trans{\out{#1}{#2}}}
\newcommand{\transbout}[2]{\trans{\bout{#1}{#2}}}
\newcommand{\transtau}{\trans{\tau}}
% rules name, pr"amisse1, pr"amisse2, konklusion, optional Anwendungsbedingung
\newcommand{\kalRule}[5][\empty]{%%% \empty: Standardwert des optionalen Parameters
	\underline{\scriptstyle#2}:\;
	\ifthenelse{\equal{#3}{\empty}}{
		\ifthenelse{\equal{#4}{\empty}}{
			\dfrac{}{#5}
		}{
			\dfrac{#4}{#5}
		}
	}{
		\dfrac{#3 \quad #4}{#5}
	}
	\ifthenelse{\NOT\equal{#1}{\empty}}{\;\;#1}{}
}
% names of rules
\newcommand{\rulename}[1]{$#1$}
\newcommand{\etau}{\rulename{E-TAU}}
\newcommand{\ecall}{\rulename{E-CALL}}
\newcommand{\eout}{\rulename{E-OUT}}
\newcommand{\ein}{\rulename{E-IN}}
\newcommand{\esuml}{\rulename{E-SUM_L}}
\newcommand{\esumr}{\rulename{E-SUM_R}}
\newcommand{\eres}{\rulename{E-RES}}
\newcommand{\eparl}{\rulename{E-PAR_L}}
\newcommand{\eparr}{\rulename{E-PAR_R}}
\newcommand{\eopen}{\rulename{E-OPEN}}
\newcommand{\ecoml}{\rulename{E-COM_L}}
\newcommand{\ecomr}{\rulename{E-COM_R}}
\newcommand{\eclosel}{\rulename{E-CLOSE_L}}
\newcommand{\ecloser}{\rulename{E-CLOSE_R}}
% strong / weak Bisimulation
\newcommand{\sbisim}{\sim}
\newcommand{\wbisim}{\approx}
\newcommand{\simu}{\mathcal{S}}
\newcommand{\weaksimuset}[2]{\simu^{#1,#2}}
\newcommand{\simuset}{\weaksimuset{P}{Q}}
% denotational semantics
\newcommand{\tr}{\functText{Traces}}
\newcommand{\trI}[1]{\mathcal{T_I}(\ec{#1})}
\newcommand{\tracesI}[3]{\mathcal T_{#1}(\ec{\procpar{#2}{#3}})}
\newcommand{\traces}[1][\empty]{
		\ifthenelse{\equal{#1}{\empty}}
		{\mathcal T}
		{\mathcal T(\ec{#1})}
}

\newcommand{\tracesR}[1]{\mathcal{T}_\relNames(\ec{#1})}

\newcommand{\res}[3]{\functText{res}(#1,#2,\traces[#3])}
\newcommand{\failures}[1][\empty]{
		\ifthenelse{\equal{#1}{\empty}}
		{\mathcal F}
		{\mathcal F(#1)}
}
\newcommand{\fd}[1][\empty]{
		\ifthenelse{\equal{#1}{\empty}}
		{\mathcal{FD}}
		{\mathcal{FD}(#1)}
}
% refinement
\newcommand{\refi}[1][\empty]{
		\ifthenelse{\equal{#1}{\empty}}
		{\,\sqsubseteq_{\mathcal T}\,}
		{\,\sqsubseteq_{\mathcal{#1}}\,}
}
\newcommand{\rrefi}[1][\empty]{
		\ifthenelse{\equal{#1}{\empty}}
		{\,\sqsupseteq_{\mathcal T}\,}
		{\,\sqsupseteq_{\mathcal{#1}}\,}
}

% Header/Footer
% style
%\pagestyle{scrheadings}
\pagestyle{empty} % first empty, then until introduction scrheadings
% Konfiguration der Kopf- und Fu"szeilen: siehe scrguide.pdf Kapitel 5 (ab Seite 214)
% pagenumber first roman until introduction arabic
\pagenumbering{Roman} 

% Settings
% Variable je nach dem wie sie gesetzt ist, wird die Druckversion beziehungsweise die digitale Version erstellt
% Bedeutet Links farbig oder schwarz
\newboolean{print_media}
\setboolean{print_media}{false}

% Ob alle Teile angezeigt werden sollen
\newboolean{show_all}
\setboolean{show_all}{false}
%\includecomment{myComment}

%\KOMAoption{BCOR}{7mm}
%\KOMAoption{parskip}{yes}
\KOMAoption{toc}{bib,index,listof}

% Abstand itemize "andern
%\setitemize{itemsep=-0.5cm}

% Seitenränder
\geometry{
	includehead,
	includefoot,
	inner=3cm, % min 3 max 4
	outer=3cm, % min 3 max 4
	top=2cm, % min 2 max 3
	bottom=2cm % min 2 max 3
}
%\geometry{showframe}

% Farben
\definecolor{gruen}{rgb}{.10, .4, .50}
\definecolor{blau}{rgb}{.2, .2, 0.6}
\definecolor{grau}{gray}{.75}
%\definecolor{white}{gray}{.75}
\definecolor{lstColor}{rgb}{0.901,0.901,0.901} %% Back (Almost White)
\definecolor{darkred}{rgb}{0.601,0.001,0.001} %% dunkles Rot
\definecolor{darkgreen}{rgb}{.10, .4, .20}% dunkelgruen
\definecolor{cdc_Blue}{rgb}{0.0,0.355,0.652} %% Heading Blue (Oldenburg CD Ultramarine)
\definecolor{cdc_BlueM}{rgb}{0.25,0.473,0.722} %% Heading Blue Medium (Oldenburg CD Ultramarine 60%)
\definecolor{cdc_BlueL}{rgb}{0.5,0.589,0.793} %% Heading Blue Light (Oldenburg CD Ultramarine 20%)
\definecolor{cdc_Green}{rgb}{0.390,0.695,0.285} %% Heading Green (Oldenburg CD Chartreuse)
\definecolor{cdc_GreenM}{rgb}{0.559,0.758,0.438} %% Heading Green Medium (Oldenburg CD Chartreuse 60%)
\definecolor{cdc_GreenL}{rgb}{0.688,0.828,0.605} %% Heading Green Light (Oldenburg CD Chartreuse 20%)
\definecolor{color_back}{rgb}{0.941,0.941,0.941} %% Back (Almost White)

% Zeilenabstand
\onehalfspacing
% Index
\newindex{default}{idx}{ind}{Index}
%\newindex{name}{adx}{and}{Namensverzeichnis}
%\newindex{defi}{ddx}{dnd}{Definitionsverzeichnis}
%\newindex{satz}{sdx}{snd}{Satzverzeichnis}
%\newindex{lem}{ldx}{lnd}{Lemmataverzeichnis}
%\newindex{kor}{kdx}{knd}{Korollarverzeichnis}
%\newindex{bemerkung}{bdx}{bnd}{Bemerkungsverzeichnis}
%\newindex{beispiel}{bsdx}{bsnd}{Beispielverzeichnis}

%Einstellungen Code-Fragmente
%\lstset{
%  xleftmargin=13pt,
%  xrightmargin = 5pt,
%  basicstyle=\small\ttfamily,
%  columns=fullflexible,
%  showstringspaces=false,
%  commentstyle=\color{gray}\upshape,
%  literate={"a}{\"a}1 {"o}{\"o}1 {"u}{\"u}1 {"s}{\ss}1
%  {"A}{\"A}1 {"O}{\"O}1 {"U}{\"U}1
%}

%\lstloadlanguages{Java }

%% Listingstyles
\lstdefinelanguage{pseudocode}
{
	alsolanguage=Java,
	morekeywords=[1]{INPUT, OUTPUT, then, od, in, fi, foreach, begin, end, endfor, endforeach, endwhile},
	morecomment=[s][\color{darkred}]{"}{"},
	escapeinside={\%*}{*)},
	commentstyle=\color{cdc_Green},
	keywordstyle=\color{blue},
	frame=single,
	backgroundcolor=\color{white},
	xleftmargin=5pt,
	xrightmargin = 5pt,
	breaklines=true,
}
%\lstdefinelanguage{apt}
%{
%	alsolanguage=Java,
%	morekeywords=[2]{Place, Transition, Arc, Flow, PetriNet, TransitionSystem, Marking, Node, Token, ArcKey, State, Edge, EdgeKey, IEdge, IGraph, INode, PetriNetOrTransitionSystem},
%	keywordstyle=[2]\color{cdc_BlueM},
%	morecomment=[s][\color{darkred}]{"}{"},
%	commentstyle=\color{cdc_Green},
%	keywordstyle=\color{blue},
%	escapeinside={\%*}{*)},
%	frame=single,
%	%backgroundcolor=\color{lstColor},
%	breaklines=true,
%	xleftmargin=5pt,
%	xrightmargin = 5pt,
%	literate= {Ö}{{\"O}}1 {Ä}{{\"A}}1 {Ü}{{\"U}}1 {ß}{{\ss}}2 {ü}{{\"u}}1 {ä}{{\"a}}1 {ö}{{\"o}}1,
%	extendedchars=true
%}

%\lstdefinelanguage{apt-parser}
%{
%	alsolanguage=Java,
%	morekeywords=[2]{IPNParserOutput, AptPNFormatParser, AptPNFormatLexer,APTPNParserOutput,ParserContext, APTParserContext, PetriNet, ANTLRParser, TransitionSystem, SynetLTSParser, SynetPNParser, APTLTSParser, APTPNParser, APTParser,APTRenderer},
%	keywordstyle=[2]\color{cdc_BlueM},
%	morekeywords=[3]{CommonTokenStream, RecognitionException},
%	keywordstyle=[3]\color{cdc_BlueL},
%	morecomment=[s][\color{darkred}]{"}{"},
%	commentstyle=\color{cdc_Green},
%	keywordstyle=\color{blue},
%	escapeinside={\%*}{*)},
%	frame=single,
%	%backgroundcolor=\color{lstColor},
%	breaklines=true,
%	xleftmargin=5pt,
%	xrightmargin = 5pt,
%	literate= {Ö}{{\"O}}1 {Ä}{{\"A}}1 {Ü}{{\"U}}1 {ß}{{\ss}}2 {ü}{{\"u}}1 {ä}{{\"a}}1 {ö}{{\"o}}1,
%	extendedchars=true
%}

%\lstdefinelanguage{apt-format}
%{
%	morekeywords=[1]{.name, .type, .description, .transitions, .places, .flows, .initial_marking, .final_markings,
%			.states, .labels, .arcs},
%	alsoletter={.},
%	keywordstyle=[1]\color{cdc_BlueM},
%	morecomment=[s][\color{darkred}]{"}{"},
%	morecomment=[s][\color{cdc_Green}]{/*}{*/},
%	morecomment=[l][\color{cdc_Green}]{//},
%	escapeinside={\%*}{*)},
%	frame=single,
%	%backgroundcolor=\color{lstColor},
%	breaklines=true,
%	xleftmargin=5pt,
%	xrightmargin = 5pt,
%	numbers=none,
%}

%\lstdefinelanguage{ebnf}
%{
%	morecomment=[s][\color{darkred}]{'}{'},
%	escapeinside={\%*}{*)},
%	morecomment=[s][\color{cdc_Green}]{(*}{*)},
%	morecomment=[s][\color{cdc_BlueM}]{?}{?},
%	frame=single,
%	%backgroundcolor=\color{lstColor},
%	breaklines=true,
%	xleftmargin=5pt,
%	xrightmargin = 5pt,
%	showspaces=false,
%	numbers=none  
%}

%\lstdefinelanguage{my_xml}
%{
%	morestring=[b]",
%	morestring=[s]{>}{<},
%	morecomment=[s][\color{darkred}]{"}{"},
%	morecomment=[s]{<?}{?>},
%	stringstyle=\color{black},
%	identifierstyle=\color{blue},
%	keywordstyle=\color{cdc_BlueM},
%        morekeywords={classpathref,classname,fork, failonerror, value, path}
%	xleftmargin=5pt,
%	xrightmargin = 5pt,
%	numbers=none,
%}

%\lstdefinelanguage{synet}
%{
%	morecomment=[s][\color{darkred}]{'}{'},
%	escapeinside={\%*}{*)},
%	frame=single,
%	%backgroundcolor=\color{lstColor},
%	breaklines=true,
%	xleftmargin=5pt,
%	xrightmargin = 5pt,
%	numbers=none
%}

% Mathematik Umgebungen
\makeatletter
\newtheoremstyle{plainWithSeparator}
	{\item[\hskip\labelsep \theorem@headerfont ##1\ ##2\theorem@separator]}%
	{\item[\hskip\labelsep \theorem@headerfont ##1\ ##2\ (##3)\theorem@separator]}
\makeatother
\makeatletter
\newtheoremstyle{nonumberPlainWithSeparator}
	{\item[\hskip\labelsep \theorem@headerfont ##1\theorem@separator]}%
	{\item[\hskip\labelsep \theorem@headerfont ##1\ (##3)\theorem@separator]}
\makeatother
\theoremstyle{plainWithSeparator}
\theoremheaderfont{\normalfont\bfseries}
\theorembodyfont{\itshape}
\theoremseparator{}
%\theoremindent{0.5cm}%0.8cm

% theorems
\theoremsymbol{\ensuremath{\square}}
\newtheorem{theorem}{Theorem}[section]

% Lemmata
\theoremsymbol{\ensuremath{\diamondsuit}}
\newtheorem{lemma}{Lemma}[section]

% conventions
\theoremsymbol{\ensuremath{\circ}}
\newtheorem{conv}{Convention}[section]

% conjectures
\theoremsymbol{\ensuremath{\circ}}
\newtheorem{conject}{Conjecture}[section]

%% corollaries
\theoremsymbol{\ensuremath{\square}}
\newtheorem{cor}{Corollary}[section]

%%Propositions
%\theoremsymbol{\ensuremath{\square}}
%\newtheorem{proposition}{Proposition}[section]

% examples
\theorembodyfont{\upshape}
\theoremsymbol{\ensuremath{\ast}}
\newtheorem{example}{Example}[section]

% Bemerkung
%\theoremsymbol{}
%\newtheorem{bemerkung}{Bemerkung}[section]

% Definitionen
\theoremsymbol{\ensuremath{\ast}}
\newtheorem{definition}{Definition}[section]

% Beweise
%\theoremstyle{plain}
\theoremstyle{nonumberPlainWithSeparator}
\theoremseparator{:}
\theoremindent0cm
\theoremsymbol{\rule{1.5ex}{1.5ex}}
\newtheorem{prf}{Proof}

% Hyperlinks
\ifthenelse{\boolean{print_media}}{
	\hypersetup{
		colorlinks=false
		,unicode=true
		,pdfborder={0 0 0}
		,breaklinks=true
		%,linktocpage % falls dort Umbrueche noetig sind
	}
}{
	\hypersetup{colorlinks
		,linkcolor=blau
		,unicode=true
		,urlcolor=blau
		,citecolor=gruen
		,filecolor=gruen
		,breaklinks=true
		%,linktocpage % falls dort Umbrueche noetig sind
	}
}

% Bibliographie
% urldate-format sprachabh"angig machen
%\setbibliographyfont{urldate}{\printdate}
% Definition des Aussehens der Literaturlisten
\bibliographystyle{alpha}

% Erlaubt gr"o"sere Abst"ande, damit nicht "uber den Rand geschrieben wird
%\emergencystretch=1ex

% Erlaubt den Umbruch von align-Umgebungen (Seitenumbruch)
\allowdisplaybreaks

% TIKZ
\usetikzlibrary{arrows, automata, positioning, decorations.pathmorphing}
\tikzstyle{every state}=[draw=none, rectangle, text=black, yshift=10mm]
\tikzstyle{sdots}=[yshift=5mm]
\tikzstyle{ldiag}=[left=2mm,pos=0.4]
\tikzstyle{rdiag}=[right=2mm,pos=0.4]
\tikzstyle{dopac}=[dashed, opacity=0.35]
\tikzstyle{every picture}=[
				->,
				>=stealth',
				%shorten >=1pt,
				%auto, % position of labels
				node distance=3cm,
                    		semithick, % thickness of arrows
				initial text={},
				initial where=above]

%\typearea[current]{calc} % berechnet den Divisor für das Raster zur Berechnung des Satzspiegels

%%%  =============================================== Preamble END ======================================================================

%%Ein eigenes Symbolverzeichnis erstellen
%\newglossary[slg]{symbolslist}{syi}{syg}{Symbolverzeichnis}
%\newglossary[tlg]{translationlist}{tyi}{tyg}{Übersetzungsliste Deutsch-Englisch}

%%Den Punkt am Ende jeder Beschreibung deaktivieren
%\renewcommand*{\glspostdescription}{}

%Befehle für Symbole
% Zahlmengen
%\newglossaryentry{symb:N}{
%name=$\N$,
%description={nat\"urliche Zahlen, einschlie\ss{}lich $0$},
%sort=00n, type=symbolslist
%}
%\newglossaryentry{symb:Z}{
%name=$\Z$,
%description={ganze Zahlen},
%sort=01z, type=symbolslist
%}
%\newglossaryentry{symb:Q}{
%name=$\Q$,
%description={rationale Zahlen},
%sort=02q, type=symbolslist
%}
%\newglossaryentry{symb:R}{
%name=$\R$,
%description={reelle Zahlen},
%sort=03r, type=symbolslist
%}
%\newglossaryentry{symb:C}{
%name=$\K$,
%description={komplexe Zahlen},
%sort=04k, type=symbolslist
%}

%\newglossaryentry{symb:pot}{
%name=$\mathscr{P}$ ,
%description={Potenzmenge},
%sort=Potenzmenge, type=symbolslist
%}

%\newglossaryentry{symb:dis_ver}{
%name=$\bigcupdot$ ,
%description={die Vereinigung paarweise disjunkter Mengen},
%sort=disjunkte Vereinigung, type=symbolslist
%}

%\newglossaryentry{symb:emptyset}{
%name=$\emptyset$ ,
%description={leere Menge},
%sort=leere Menge, type=symbolslist
%}

%Befehle für Abkürzungen
\newacronym{CCS}{CCS}{Calculus of Communicating Systems}
\newacronym{CSP}{CSP}{Communicating Sequential Processes}
%Eine Abkürzung mit Glossareintrag
%\newacronym{AD}{AD}{Active Directory\protect\glsadd{glos:AD}}

%Übersetzungen Deutsch-Englisch
%\loadglsentries [translationlist] {translationlist.tex} 

\makeglossaries

% Einf"ugen der Zeichen ohne anzeige
%\glsadd{symb:N}
%\glsadd{symb:Z}
%\glsadd{symb:Q}
%\glsadd{symb:R}
%\glsadd{symb:C}
%\glsadd{symb:pot}
%\glsadd{symb:dis_ver}
%\glsadd{symb:emptyset}


%%%  =============================================== Document BEGINN ===================================================================
\begin{document}

	%% ------------------------------------------------- Main ----------------------------------------------------------------------
	%% Titel
	\selectlanguage{ngerman}
	% mainfile: Refinement.tex
\def\myTitle{Refinement of $\pi$-calculus processes}
%\def\mySubtitle{asdfasdf}
\def\myDocumentType{Masterarbeit}
\def\myName{Manuel Gieseking}
\def\myAddress{Gaststra\"se 4}
\def\myPlz{26122}
\def\myCity{Oldenburg}
\def\myEmail{manuel.gieseking@informatik.uni-oldenburg.de}
\def\myBirth{07.07.1985}
\def\myUni{Universit"at Oldenburg}
\def\myMatrNr{9102550}
\def\myFakultaet{II}
\def\myFachbereich{Informatik, Wirtschafts- und Rechtswissenschaften}
\def\myInstitut{Department f"ur Informatik}
\def\myStudiengang{Fach-Master Informatik}
\def\myFirstPruefer{Prof. Dr. Ernst-R"udiger Olderog}
\def\mySecondPruefer{Dipl.-Inform. Sven Linker}
\def\myBetreuer{}
\def\myAbteilung{Entwicklung korrekter Systeme}

% Zur Seiteneinstellung
\def\myTopMargin{15mm}
%\def\myBottomMargin{50mm}
\def\myEnlargement{30mm}

% Kopf- und Fu"szeilen l"oschen
\thispagestyle{empty}
% Einstellungen der Seitenr"ander
\addtolength{\topmargin}{-\myTopMargin}
% Textheigth + x
\enlargethispage{\myEnlargement}
\vspace*{-2cm}
\begin{figure}[htbp]
	\centering
	%\subfloat{\includegraphics[width=0.5\textwidth]{./chapters/0_title/logo_first.eps}}
	%\quad \subfloat{\includegraphics[width=0.25\textwidth]{./chapters/0_title/logo_second.jpg}}
	\includegraphics[width=0.5\textwidth]{./images/logoUniOL.pdf}
\end{figure}
\begin{center}
	\bf{Fakult"at \myFakultaet: \myFachbereich \newline \myInstitut}

	%\vspace*{-0.6cm}
	\bf{Abteilung: \myAbteilung}    
	\vspace*{0.5cm}

	{\color{grau}\noindent\rule{\textwidth-5cm}{1pt}}

	\vspace*{4.5cm}

	{\bf \huge \myTitle \\}
	\vspace*{1.5cm}
	%   \vspace*{0.5cm}
	%  {\bf \large \mySubtitle}
	% \vspace*{1cm} \\
	{\Large \myDocumentType\\}
	\vspace*{0.5cm}
	{\large \it\texttt{-- post version --}}
\end{center}
\vfill
\normalsize{
	\begin{tabular}{ll}
	    	Name: & {\myName} \\
		%Stra"se: & \myAddress \\
		%Wohnort: & \myPlz \ \myCity \\ 
		%Geburtstag: & \myBirth \\ \\
	    	%Matrikelnr.: & {\myMatrNr} \\
		E-Mail: & \myEmail \\ \\
	    	Studiengang: & \myStudiengang\\
		%Betreuer: & \myBetreuer \\
		Erstgutachter: & \myFirstPruefer \\
		Zweitgutachter: & \mySecondPruefer \\
		Datum: & \today \\
	\end{tabular}
}
% Seitenlayout zur"ucksetzen
\addtolength{\topmargin}{\myTopMargin}

	\selectlanguage{english}
	%\cleardoublepage
	%% mainfile: Refinement.tex
\begin{abstract}
\todo{asdf}
\end{abstract}


	%\cleardoublepage
	%\newpage
	%% Inhaltsverzeichnis
	\singlespacing
	\tableofcontents
	\onehalfspacing
	% Abbildungsverzeichnis
	\listoffigures
	% Tabellenverzeichnis
	%\listoftables	
	%Listingsverzeichnis
	%\renewcommand{\lstlistlistingname}{Listingsverzeichnis}
	%\lstlistoflistings
	%% Main
	% mainfile: ../Refinement.tex
\chapter{Introduction}
\pagestyle{scrheadings}	
\setcounter{page}{0}
\pagenumbering{arabic} 
\label{sec_introduction}
In many cases of modern computing it is of interest to describe and model concurrency. Computers no longer just solve a problem by subsequently working off the single tasks of their own, but they decompose and concurrently calculate the problem even together in a network. The increase in the number of CPU cores and more heavily of GPU cores within one single computer convincingly demonstrates how fundamental concurrency is for modern computing. Moreover, the rapidly increasing spread of the Internet is one of the most common examples which shows the importance of networks.

This thesis is divided into five chapters. In \refChap{sec_preliminaries} we briefly introduce sequences and properly investigate the \picalc{} and its operational semantics (the \emph{early transition system} \cite{sangiorgi}). Thereby, a new structural form of processes -- the \emph{extended standard form} -- is introduced and it is shown that every process can be transformed into a structural congruent one in this standard form. We proceed in \refChap{sec_big-step_semantics} by defining a big-step semantics which bases on the early transition system by observing an arbitrary number of external steps within the early semantics. Subsequently, in \refChap{sec_de_sem_trace}, we introduce the trace semantics which is based on the big-step semantics of the previous chapter. Thereby, we investigate its properties and define the refinement based on the trace semantics. Finally, the conclusion in \refChap{sec_conclusion} gives a brief summary of our results and presents ideas for future work.


	% mainfile: ../Refinement.tex
\chapter{Preliminaries}
\label{sec_preliminaries}
% mainfile: ../../Refinement.tex
At the heart of the refinement of \findex{\picalc{}} processes is the theory of \findex[sequence]{sequences}. Thus, in this chapter, we recall the model of sequences to gain a formal construct to handle ordered elements.% in an intuitive way.

Furthermore, we introduce the \picalc{} and investigate its behavior properly. In particular, we carefully explain the operational semantics of \picalc{} processes, since its peculiarities induce the characteristics of the refinement and its properties. Moreover, we discuss why we choose this particular operational semantics for the following work in this thesis and compare it to other semantics.

The majority of those definitions and notions can, for example, be found in \cite{milner,sangiorgi}.

As mathematical notations, we consider the natural numbers starting with zero ($\N=\set{0,1,2,\ldots}$) and use $\fatsemi$ as the composition of relations. Furthermore, we denote $R^*$ as the reflexive and transitive closure of a relation $R$.


\section{The \texorpdfstring{$\pi$}{pi}-calculus}
\label{sec_pi_calculus}
% mainfile: ../../Refinement.tex
The \findex[\picalc{}|(]{\picalc{}} belongs to the family of ...

\newpage %%%%%%%%%%%%%%% NEWPAGE!


\section{The OZ}
\label{sec_oz}
% mainfile: ../../Refinement.tex
The OZ

\newpage %%%%%%%%%%%%%%% NEWPAGE!





\chapter{Big-step semantics}
\label{sec_big-step_semantics}
\input{chapters/big-step_semantics/big-step_semantics}

\chapter{Trace semantics}
\label{sec_de_sem_trace}
\input{chapters/trace_semantics/traceSemantics}

%\chapter{Denotational semantics}
%\label{sec_de_semantics}
%% mainfile: ../Refinement.tex
In this chapter we introduce three denotational semantics for \picalc{} processes. The definitions are very similar to the denotational semantics of \gls{CSP} processes defined by Bill Roscoe in \cite{roscoe}. With attention to an easy definition of the refinement of \picalc{} processes, the denotational semantics consists of sets retrieved from the operational semantics presented in \refSec{sec_pi_op_sem}.

Therefore, in \refSec{sec_de_sem_trace} we define a trace semantics $\traces$, which describes a process' behavior regarding its external, neglecting its internal ability. The internal behavior is taken to account in the failures semantics $\failures$ presented in \refSec{sec_de_sem_failures} and a somewhat different view on internal divergence of a process is constituted with the failures-divergence semantics $\fd$ in \refSec{sec_de_sem_fd}.

\section{Trace semantics}
\label{sec_de_sem_trace}
\input{chapters/trace_semantics/traceSemantics}

%\section{Failures semantics}
%\label{sec_de_sem_failures}
%% mainfile: ../Refinement.tex
\todo{instead of just traces is not nice}The failures semantics collects instead of just traces, tuple of traces and a set of so-called failures. These failure sets describe for the process reachable with the respected trace and if no internal transition is possible, which transitions cannot be used for the next step. This takes the internal behavior of a process into account.

Like the trace semantics, the failures semantics do not really differ from the definition of Roscoe and this section has a similar structure as the previous one.
\subsection{Definition}
\label{sec_de_sem_failures_def}

\subsection{Examples}
\label{sec_de_sem_failures_bsp}
\todo[inline]{example why not just adding taus to the traces}

\subsection{Properties}
\label{sec_de_sem_failures_prop}

\subsection{Refinement}
\label{sec_de_sem_failures_ref}



%\section{Failures-divergence semantics}
%\label{sec_de_sem_fd}
%% mainfile: ../Refinement.tex
\subsection{Definition}
\label{sec_de_sem_fd_def}

\subsection{Properties}
\label{sec_de_sem_fd_prop}

\subsection{Refinement}
\label{sec_de_sem_fd_ref}


%\section{Related work}
%\label{sec_de_sem_rel_work}
%\todo{Relation for inheritance of Canal et al. adaption to refinement?}


%\chapter{Reduction of semantics}
%\label{sec_reduc_semantics}
%% mainfile: ../Refinement.tex
By reason of the dimension of the transition systems resulting already from small processes, \refSec{sec_op_sem_coverage} introduces an adaption of Sangiorgi's semantics and shows the equivalence of both. This notion with the associated adaption of the denotational semantics also helps to simplify the proofs, that one process is a refinement of an other.
\section{Operational semantics}
\label{sec_op_sem_coverage}
For this section we follow this notion but for the sense of simplification for the denotational semantics we get rid of them in \refSec{sec_op_sem_coverage} for the rest of the thesis.\todo{alpha conversion equiv classes of processes}
\subsection{Definition}
\label{sec_op_sem_cov_def}
% mainfile: ../../Refinement.tex
\begin{figure}
\begin{gather*}
\kalRule{TAU}{}{}{\tau.P \trans{\tau} P} \\\\
\kalRule{OUT}{}{}{\out{x}{y}.P \trans{\out{x}{y}} P} \quad\quad \kalRule{IN}{}{}{\inp{x}{z}.P \transin{x}{z} P} \\\\
\kalRule{SUM_L}{}{P \trans{\alpha} P'}{P+Q \trans{\alpha} P'} \quad\quad\kalRule{SUM_R}{}{P \trans{\alpha} P'}{Q+P \trans{\alpha} P'} \\\\
\kalRule{PAR_L}{P \transin{a}{x} P'}{z\in\fn{P}\cup\fn{Q}\cup\set{x}}{\procpar{P}{Q} \transin{a}{z} \procpar{P'\subs{x}{z}}{Q}} \quad\quad \kalRule{PAR_R}{}{P \trans{\alpha} P'}{\procpar{Q}{P} \trans{\alpha} \procpar{Q}{P'}} \\\\
\kalRule[x\nin n(\alpha)]{RES}{}{P \trans{\alpha} P'}{\procres{x}{P} \trans{\alpha} \procres{x}{P'}} \quad\quad \kalRule[z\neq x]{OPEN}{}{P \transout{x}{z} P'}{\procres{z}{P} \transbout{x}{z}P'} \\\\
\kalRule{COM_L}{P \transin{x}{y} P'}{Q \transout{x}{z} Q'}{\procpar{P}{Q} \transtau \procpar{P'\subs{z}{y}}{Q'}} \quad\quad\kalRule{COM_R}{P \transin{x}{y} P'}{Q \transout{x}{z} Q'}{\procpar{Q}{P} \transtau \procpar{Q'}{P'\subs{z}{y}}} \\\\
\kalRule{CLOSE_L}{P \transbout{x}{z} P'}{Q \transin{x}{z} Q'}{\procpar{P}{Q} \transtau \procres[a]{z}{\procpar{P'}{Q'}}} \quad\quad\kalRule{CLOSE_R}{P \transbout{x}{z} P'}{Q \transin{x}{z} Q'}{\procpar{Q}{P} \transtau \procres[a]{z}{\procpar{Q'}{P'}}}\\\\
\kalRule[\procdef{A}{\parl{w}}:=P]{CALL}{}{}{\proccall{A}{\parl{v}} \transtau P\subs{\parl{v}}{\parl{w}}} 
\end{gather*}
\label{fig_ts_mine}
\caption{Transition system for $\pi$-calculus}
\end{figure}


\subsection{Properties and proof}
\label{sec_op_sem_cov_prop}
\begin{theorem}[Coverability]
	Sangiorgi's transition system mentioned in \refFig{fig_ts_early} can be covered by the transition system specified in \refFig{fig_ts_mine}; meaning:
	for all $P,Q\in\procs$ and exists $a,x\in\names$ hold
	\begin{enumerate}
		\item $\ec{P}\tautrans\ec{Q} \gdw P\transtau Q$
		\item $\ec{P}\outtrans{a}{x}\ec{Q} \gdw P\transout{a}{x} Q$
		\item $\ec{P}\intrans{a}{x}\ec{Q} \gdw \exists y\in\fn{P}\cup\set{x}: P\transin{a}{y} Q$
		\item $\ec{P}\bouttrans{a}{x}\ec{Q} \gdw \exists y\in\fn{P}\cup\set{x}: P\transbout{a}{y} Q$
	\end{enumerate}
\end{theorem}

\section{Denotational semantics}
\label{sec_op_sem_coverage}

\begin{align*}
\underbrace{\seq{s_1,\dots,s_n}}_{=_{def}seq_1} \simeq \underbrace{\seq{s_1',\dots,s_m'}}_{=_{def} seq_2} \gdw & \exists l,o,k\in\N\; with\; l+o=k: \\
& \exists x_1,\dots,x_k\in\names\setminus\left(\fn{seq_1}\cup\fn{seq_2}\right) \\
& \exists y_1,\dots,y_l\in\bn{seq_1} \exists z_1,\dots,z_o\in\bn{seq_2}: \\
& seq_1\subs{x_1,\dots,x_l}{y_1,\dots,y_l} = seq_2\subs{x_{l+1},\dots,x_k}{z_1,\dots,z_o}
\end{align*}

\[\traces[P] \subsetsim \traces[Q] \gdw \forall s\in\traces[P]\exists s'\in\traces[Q]: s\simeq s' \]




	% mainfile: ../Refinement.tex
\chapter{Conclusion and future work}
\label{sec_conclusion}
In this thesis ...


	\singlespacing
	%% ------------------------------------------------- Anhang -------------------------------------------------------------------
	%% Anhang
	%\appendix
	%
\section{Tabellen}
\begin{landscape}
	\begin{table}[htb]
	\caption{Credentials der PG-RiO}
	\begin{tabular}{|p{0.15\linewidth}|p{0.20\linewidth}|p{0.60\linewidth}|}
		\hline
		Zweck & Benutzername & Passwort \\
		\hline
		Dockerhub & pgrio & ~\newline \\
		\hline
		VMs & pgrio & ~\newline \\
		\hline
		SK"=Verwaltung & svadmin & ~\newline \\
		\hline
		Jenkins & pgrio & ~\newline \\
		\hline
		DB"=Writer & writer & ~\newline \\
		\hline
		DB"=Admin & admin & ~\newline \\
		\hline
		MQTT"=Datacollector & pgrio"=datacollector & ~\newline \\
		\hline
		API"=Routing & routing & ~\newline \\
		\hline
		App"=Signierung Google Playstore & \textit{Keyfile} & ~\newline \\
		\hline
	\end{tabular}
	\label{tbl:appendix:infrastruktur:cred}
	\end{table}
\end{landscape}


	%% ------------------------------------------------- Glossar ------------------------------------------------------------------
	% Glossar
	%\printglossary[style=altlist,title=Glossar]
	% Abk"urzungen
	%\deftranslation[to=German]{Acronyms}{Abk\"urzungsverzeichnis}
	%\printglossary[type=\acronymtype,style=long]
	% Symbole
	%\printglossary[type=symbolslist,style=long]
	% Englisch-Deutsch
	%\printglossary[type= translationlist, style=long]

	%% ------------------------------------------------- Literaturverzeichnis -----------------------------------------------------
	% Literaturverzeichnis
	\bibliography{bib/bib}

	%% ------------------------------------------------- Index --------------------------------------------------------------------
	% Index - Schlagworte
	\printindex
	% Index - Namensverzeichnis
	%\printindex[name]
	% Eigenst"andigkeitserkl"arung
	\newpage{}
	\pagestyle{empty}
	%\pagenumbering{} 
	\selectlanguage{ngerman}
	\section*{Erkl"arung}
	Hiermit versichere ich, dass ich diese Arbeit selbstst"andig verfasst und keine anderen als die angegebenen Quellen und Hilfsmittel benutzt habe. Au"serdem versichere ich, dass ich die allgemeinen Prinzipien wissenschaftlicher Arbeit und Ver"offentlichung, wie sie in den Leitlinien guter wissenschaftlicher Praxis der Carl von Ossietzky Universit"at Oldenburg festgelegt sind, befolgt habe.
	
	\vspace{1cm}
	Oldenburg, den \today
	
	\vspace{0.5cm}
	\rule{6cm}{0.5pt}
	
	(Manuel Gieseking)
\end{document}
%%%  =============================================== Document END =====================================================================
