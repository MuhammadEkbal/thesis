% mainfile: ../Refinement.tex
\chapter{Introduction}
\pagestyle{scrheadings}	
\setcounter{page}{0}
\pagenumbering{arabic} 
\label{chp_introduction}
Modeling distributed computing systems exhibit various aspects such as modeling the components of the system and modeling the behavior and communication between the concurrent mobile components. Formal specification techniques for such systems have to be able to describe all these aspects. Unfortunately, a single specification technique that is well suited for all these aspects is not yet available. Instead one finds techniques that are very good at describing individual aspects of system. This observation has led to research into the combination and semantic integration of specification techniques. In this thesis we research combining the two specification techniques:
\picalc{} and Object-Z.

\picalc{} is a process algebra originally introduced by Robin Milner \cite{milner}. The central concepts of \picalc{} are communication via channels between different processes, creating new channels parallel composition and the mobility of channels, which we will use to model the mobility of components. Tool support comes through the Stargazer and ABC. ABC The name stands for bi simulation checking.

Object-Z is an object-oriented extension to the Z notation, developed at the University of Queensland, Australia, by the addition of language constructs resembling the object-oriented paradigm, most notably, classes. Object-Z represent a set-theoretic and predicate language for the specification of data, state spaces and state transformations.

The main contribution of this thesis is to develop the combination $\pi$-OZ and to transform it into a \picalc{} process, similarly to the approach for CSP-OZ in \cite{olderog}. $\pi$-OZ is a new combination of formal techniques
for the specification the components and their behavior. The basic idea is to use a \picalc{} process to specify the behavior of an Object-Z class. This enables the mobility of Object-Z classes through the use of the state pattern. Syntactically, the $\pi$-OZ specification is divided into an OZ part specifying the data and the possible state transitions, and a $\pi$ part specifying the behavior.  This combination is illustrated by the example of a mobile vending machine and two shops which alternate their behavior when they connect to the mobile vending machine.

 Transforming the combination $\pi$-OZ into a \picalc{} process will enable checking the simulation of mobile processes with data. The transformation will give the combination a \picalc{} semantics. The idea is that this semantics come from the parallel composition of the $\pi$ part and a $\pi$ process, representing the OZ part transformed syntactically into a \picalc{} process.

 Furthermore, we investigate a failure-model and a success-model for the \picalc{} processes show its relation to the simulation and point out its limitations.

This thesis is divided into five chapters. In Chapter 2 we overview the \picalc{} and Object-Z. Thereby, Dynamic OZ, a new extension of Object-Z is introduced to model mobile components with alternating behavior. We proceed in Chapter 3 by transforming OZ into \picalc{}. Subsequently, in Chapter 4, we introduce the combination $\pi$-OZ and transform it into \picalc{}. Chapter 5, introduces the failure and the success refinement of \picalc{} their relation to the simulation. Finally, the conclusion in Chapter 6 gives a brief summary of our results and presents ideas for future work.

