% mainfile: ../Refinement.tex
\chapter{Introduction}
\pagestyle{scrheadings}	
\setcounter{page}{0}
\pagenumbering{arabic} 
\label{chp_introduction}
Modeling Complex computing systems exhibit various aspects such as modeling the components of the system and modeling the communication between the components. Formal specification techniques for such systems have to be able to describe all these aspects. Unfortunately, a single specification technique that is well suited for all these aspects is not yet available. Instead one finds techniques that are very good at describing individual aspects of system. This observation has led to research into the combination and semantic integration of specification techniques. In this thesis we research combining the two specification techniques:
\picalc{} and Object-Z.

Object-Z is an object-oriented extension to the Z notation, developed at the University of Queensland, Australia, by the addition of language constructs resembling the object-oriented paradigm, most notably, classes. Object-Z represent a set-theoretic and predicate language for the specification of data, state spaces and state transformations.

\picalc{} were originally introduced by Milner. The central concepts of \picalc{} are communication via channels between different processes, creating new channels parallel composition and the mobility of channels, which we will use to model the mobility of components. Tool support comes through the Stargazer and ABC. ABC The name stands for bi simulation checking.

$\pi$-OZ is a new combination of formal techniques
for the specification the components and their behavior. Transforming the combination $\pi$-OZ into \picalc{] process will enable checking the simulation of mobile processes with data. Additional study the refinement of \picalc{} processes. This combination is illustrated by specifying
illustrated by the example of a vending machine.

This thesis is divided into five chapters. In Chapter 2 we overview the \picalc{} and Object-Z. Thereby, Dynamic OZ, a new extension of Object-Z is introduced to model mobile components with alternating behavior. We proceed in Chapter 3 by transforming OZ into \picalc{}. Subsequently, in Chapter 4, we introduce the combination $\pi$-OZ and transform it into \picalc{}. Chapter 5, introduces the failure and the success refinement of \picalc{} their relation to the simulation. Finally, the conclusion in Chapter 6 gives a brief summary of our results and presents ideas for future work.

