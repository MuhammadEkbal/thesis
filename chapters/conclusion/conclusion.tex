% mainfile: ../Refinement.tex
\chapter{Conclusion and future work}
\label{sec_conclusion}
In this thesis we investigated the ransformational semantics of the combination $\pi$-OZ for mobile processes with data.

The aim of the work is to combine the \oz{} with \picalc{}, and to tranform the combination into a \picalc{} processes, in a similar way to CSP-OZ \cite{olderog} approach. Unfortunately, we found out that the transformation is Cumbersome, since \picalc{} has only elementary constructs and not suitable to express complex constructs like \oz{} class. On the one hand, we showed how to integrate a \picalc{} process, describing the desired sequence of behavior, into an \oz{} class to form the $\pi$-OZ combination. On the other hand, we explained how to transform the combination $\pi$-OZ into \picalc{} process through transforming \oz{} class constructs value, state variable, state schema, initial state schema and operation schema into \picalc{} processes and names accompanied by the processes of the desired sequence of behavior using the parallel operator. In spite of that, we introduced the Failure-Refinement model for \picalc{} and showed that the strong simulation does not imply the failure-refinement. Finally, we introduced the Acceptance-Refinement model and showed that the strong simulation implies acceptance-refinement. 

The elementary nature of the \picalc{} is a a good start for future work through introducing a state-full \picalc{}, which is an extension of \picalc{} that supports data, data types, variable and conditions. This will ease the mapping between \oz{} and \picalc{} constructs. Additionaly, a tool can be developed to visualize the new sate-full \picalc{} in similyr way to Stargazer\cite{stargazer}. Furthermore,it would be good to extend the ABC simulation-checker \cite{abc} to support simulation checking and acceptance-refinement verification of the new calculus. Finally, it would be interesting to extend \cite{gieseking} trace semantics to support recursive processes through developing a fixed-point algorithm for the recursive processes. This will facilitate the study of the composition properties of the Acceptance-Refinement model, which will permit to study the possibility of introducing an Acceptance-Divergence model and to explore its composition properties.
