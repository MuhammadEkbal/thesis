% mainfile: ../../Refinement.tex
For comparing processes based on their behavior there is already a notion called \index{bisimulation!strong}\findex[strong!bisimulation]{strong bisimulation} defined in \cite{milnerParrowWalker}. Since they use a slightly different semantics for the transitions, we stick to the notion of bisimulations in \cite{sangiorgi}.

\begin{definition}[Strong simulation and bisimulation]
\label{def_strong_sim_bisim}
A relation $\mathcal{S}\subseteq\procs_\alpha\times\procs_\alpha$ is called a \index{simulation!strong}\findex[strong!simulation]{strong simulation}, if $(\ec{P},\ec{Q})\in\mathcal{S}$ implies that
\[\text{if } \ec{P}\transs{\alpha}\ec{P'} \text{ then }Q'\in\procs \text{ exists such that } \ec{Q}\transs{\alpha}\ec{Q'} \text{ and } (\ec{P'},\ec{Q'})\in\mathcal{S}.\]
$\mathcal{S}$ is called a \index{bisimulation!strong}\findex[strong!bisimulation]{strong bisimulation} if $\mathcal{S}$ and $\mathcal{S}^{-1}$ are strong simulations.

Furthermore, for a strong bisimulation $\mathcal{S}$, we call $\ec{P}$ \findex[strong!bisimilar]{strong bisimilar} to $\ec{Q}$, written as $\ec{P}\sbisim\ec{Q}$, if $(\ec{P},\ec{Q})\in\mathcal{S}$.
\end{definition}

Thereby, the internal as well as the external behavior are taken into account. To abstract from the invisible actions, the \index{bisimulation!weak}\findex[weak!bisimulation]{weak bisimulation} relates processes only according to their external behavior.

\begin{definition}[Weak simulation and bisimulation]
\label{def_weak_sim_bisim}
A relation $\mathcal{S}\subseteq\procs_\alpha\times\procs_\alpha$ is called a \index{simulation!weak}\findex[weak!simulation]{weak simulation}, if $(\ec{P},\ec{Q})\in\mathcal{S}$ implies 
\begin{itemize}%\itemsep-5mm
\item[(1)] if $\ec{P}\transs{\tau}\ec{P'}$ then $Q'\in\procs$ exists such that $\ec{Q}(\tautrans)^*\ec{Q'}$ and $(\ec{P'},\ec{Q'})\in\mathcal{S}$,
\item[(2)] if $\ec{P}\transs{\alpha}\ec{P'}$ then $Q'\in\procs$ exists such that $\ec{Q}(\tautrans)^*\fatsemi\transs{\alpha}\fatsemi(\tautrans)^*\ec{Q'}$ and $(\ec{P'},\ec{Q'})\in\mathcal{S}$, for $\alpha\in\actions\setminus\set{\tau}$.
\end{itemize}
A relation $\mathcal{S}\subseteq\procs_\alpha\times\procs_\alpha$ is called a \index{bisimulation!weak}\findex[weak!bisimulation]{weak bisimulation}, if $\mathcal{S}$ and $\mathcal{S}^{-1}$ are weak simulations.

Furthermore, for a weak bisimulation $\mathcal{S}$, we call $\ec{P}$ \findex[weak!bisimilar]{weak bisimilar} to $\ec{Q}$, written as $\ec{P}\wbisim\ec{Q}$, if $(\ec{P},\ec{Q})\in\mathcal{S}$.
\end{definition}
Note that with the definition of the big-step semantics in \refChap{sec_big-step_semantics} there is a somehow simpler way to describe the weak bisimulation, since there is an abbreviation for an arbitrary number of $\tau$ transitions.

\newpage %%%%%%%%%%%%%%% NEWPAGE!
