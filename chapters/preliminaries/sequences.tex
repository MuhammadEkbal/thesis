% mainfile: ../../Refinement.tex
In this section we recall the definition of sequences and some of its properties. This well-known information can be found in standard textbooks.

\begin{definition}[Sequence]
\label{def_seq}
A \findex{sequence} is a partial function which maps natural numbers to values of a given set. We define the set of all finite sequences for a given set $X$ by
\[\seqset{X}\define\set[\exists{}n\in\N:\texttt{dom}(f) = \set{1,\ldots,n} \vee \texttt{dom}(f)=\emptyset]{f:\N\rightarrow{}X},\]
with $\texttt{dom}(f)$ denotes the domain of a function $f$.

 %$\set[i\in\set{1,\ldots,n}]{\left(i,a_i\right)}\in\seqset{X}$ with $\set{a_1,\ldots,a_n}\subseteq{}X$.

The collection of all sequences we denote by $\texttt{seqs}$. As an abbreviation we write $\seq{a_1,\ldots,a_n}$ for the function $f:\N\rightarrow{}\set{a_1,\ldots,a_n}$ with $i\mapsto{}a_i$ for all $i\in\set{1,\ldots,n}$. Furthermore, we also use $\alpha\in{}s$ instead of $(\cdot,\alpha)\in{}s$, for a sequence $s\in\seqset{X}$ and $\alpha\in{}X$.

The special case that $\texttt{dom}(f)=\emptyset$ holds, so no element is mapped for a function $f$, is called an \index{sequence!empty}\findex{empty sequence}, and we denote it by $\eseq{}$. The collection of all sequences apart from the empty sequence is denoted by $\texttt{seqs}_1$. Similarly, $\texttt{seq}_1(X)$ describes all sequences over a given set $X$ except of the empty sequence. Furthermore, we often use the index notation for a sequence $s=\seq{a_1,\ldots,a_n}$ to address the single elements: $s_i=s(i)$.
\end{definition}

To ease the usage of sequences, we list some operations handling them and their properties.

\begin{definition}[Operations on sequences]
\label{def_seqs_ops}
We define the \index{sequence!length}\findex{length of a sequence} by the cardinality of its domain:
\begin{align}
\#: \left\{ \begin{array}{cl}
		 \texttt{seqs}&\rightarrow\N \\
		s&\mapsto\card{\texttt{dom}(s)}.
		\end{array}\right.
\tag{length}
\end{align}
For the \index{sequence!concatenation}\findex{concatenation} of two sequences we use the $\seqconc{}{}$ operator, defined by
\begin{align}
\seqconc{}{}:\left\{ \begin{array}{cl}
		 \texttt{seqs}\times\texttt{seqs}&\rightarrow\texttt{seqs} \\
		  (s,t)&\mapsto{}s\cup\set[(i,x)\in{}t]{(i+\len{s},x)}.		
		\end{array}\right.
\tag{concatenation}
\end{align}
%\begin{old}[not needed definition] %%%%%%%%%%%%%%%%%%%%%%%%%%%%%%%%%%%% OLD: NOT NEEDED DEFINITION
%To address the first element and the rest of the list, we define the $\texttt{head}$\index{sequence!head}\index{head} and the $\texttt{tail}$\index{sequence!tail}\index{tail} function for any set $X$:
%\begin{equation*}
%\begin{aligned}
%\texttt{head}:\left\{ \begin{array}{cl}
%		 \texttt{seq}_1(X)&\rightarrow{}X \\
%		  s&\mapsto{}s_1		
%		\end{array}\right.
%%\tag{head}
%\end{aligned}\quad
%\begin{aligned}
%\texttt{tail}:\left\{ \begin{array}{cl}
%		 \texttt{seqs}_1&\rightarrow{}\texttt{seqs}\\
%		  s&\mapsto{}\set[(i,x)\in{}s\wedge i\neq{}1]{(i-1,x)}	
%		\end{array}\right.
%%\tag{tail}
%\end{aligned}
%\end{equation*} All in an comment. Otherwise there is an indent at the beginning of the next sentence
%\end{old}%%%%%%%%%%%%%%%%%%%%%%%%%%%%%%%%%%%%%%%%%%%%%%%%%%%%%%%%%%%%%%%%%%%%%%%%%%%%%%%%% OLD: NOT NEEDED DEFINITION
The \index{sequence!iteration}\findex{iteration} of a sequence is inductively defined by
\begin{align}
	\begin{array}{lcl}
		s^0&\define&\eseq \\
		s^{n+1}&\define&\seqconc{s^n}{s}.
	\end{array}
\tag{iteration}
\end{align}
We lift both, the concatenation for two sets $S,T\subseteq\texttt{seqs}$ by
\begin{align*}\seqconc{S}{T}\define\set[s\in{}S,t\in{}T]{\seqconc{s}{t}}\tag{concatenation}\end{align*}
and the iteration by
\begin{align}
	\begin{array}{lcl}
		S^0&\define&\set{\eseq} \\
		S^{n+1}&\define&\seqconc{S^n}{S}
	\end{array}
\tag{iteration}
\end{align}
to sets of sequences.

Furthermore, we define the \findex{Kleene star} as $S^*\define\bigcup_{n\in\N}S^n$, as well as the \findex[prefix!operator]{prefix operator}:
\begin{align}
\pref{S}\define\set[\exists{}s\in{}S,u\in\texttt{seqs}:s=\seqconc{t}{u}]{t\in{}\texttt{seqs}}\tag{prefix}
\end{align}
for a set $S\subseteq\texttt{seqs}$.
%\begin{description}
%\item[the \index{sequence!length}\findex{length of a sequence}:] $\#:\texttt{seqs}\rightarrow\N$ with $\#(s)\define\texttt{max}(\texttt{dom}(s))$.
%\item[\index{sequence!concatenation}\findex{concatenation of sequences}:] $\seqconc{}{}:\texttt{seqs}\times\texttt{seqs}\rightarrow\texttt{seqs}$ with $\seqconc{s}{t}\define{}s\cup\set[(i,x)\in{}t]{(i+\#s,x)}$
%\item[\index{sequence!head}\findex{head of a sequence}:] For any set $X$ $\texttt{head}:\texttt{seqs}(X)\setminus\eseq\rightarrow{}X$ with $\texttt{head}(s)\define{}s_1$
%\item[\index{sequence!tail}\findex{tail of a sequence}:]  $\texttt{tail}:\texttt{seqs}\setminus\eseq\rightarrow{}\texttt{seqs}$ with $\texttt{tail}(s\right)\define{}\set[(i,x)\in{}s\wedge i\neq{}1]{(i-1,x)}$.
%\item[\index{sequence!iteration}\findex{iteration of a sequence}:] $\circ:\texttt{seqs}\times\N\rightarrow{}\texttt{seqs}$ written $\circ(s,i)$ as $s^i$.  $s^0\define\eseq$, $s^{n+1}\define\seqconc{s^n}{s}$.
%\item[\index{sequence!prefix}\findex{prefix of a sequence}:]  $\texttt{pref}:\texttt{seqs}\times\texttt{seqs}\rightarrow{}\B$ $s^0\define\eseq$, $s^{n+1}\define\seqconc{s^n}{s}$.
%\end{description}
\end{definition}

\begin{old}[not needed properties] %%%%%%%%%%%%%%%%%%% OLD: NOT NEEDED PROPERTIES
There are many well-known facts about sets of sequences and the given operators. We collect a few in the following lemma.

\begin{lemma}[Properties]
\label{lem_seq_props}
	Given $S,T,R\in\texttt{seqs}$ it holds:\todo{Till now just distributivity used. Delete the others, if not needed.}
	\begin{align*}
		\begin{array}{rcl}
			\seqconc{\left(S\cup{}T\right)}{R} &=& \left(\seqconc{S}{R}\right)\cup\left(\seqconc{T}{R}\right) \\
			\seqconc{S}{\left(T\cup{}R\right)} &=& \left(\seqconc{S}{T}\right)\cup\left(\seqconc{S}{R}\right) \\
			S&\subseteq{}&S^*\\
			\left(S^*\right)^*&=&S^*\\
			S &\subseteq{}&\pref{S} \\
			\pref{\pref{S}} &=& \pref{S}.
		\end{array}
	\end{align*}
Furthermore, \findex{Arden's rule} holds, which means that the equation $S=\left(\seqconc{T}{S}\right)\cup{}R$ has the solution $S=\seqconc{T^*}{R}$; moreover, the solution is unique if $\seq{}\nin{}T$.
\end{lemma}
\end{old} %%%%%%%%%%%%%%%%%%% OLD: NOT NEEDED PROPERTIES

With the definition of the concatenation of sets of sequences, we get that the union and the concatenation of sets of sequences are distributive.
\begin{lemma}[Distributivity]
\label{lem_seq_props}
	Given $S,T,R\in\texttt{seqs}$, then
	\begin{align*}
		\begin{array}{rcl}
			\seqconc{\left(S\cup{}T\right)}{R} &=& \left(\seqconc{S}{R}\right)\cup\left(\seqconc{T}{R}\right) \\
			\seqconc{S}{\left(T\cup{}R\right)} &=& \left(\seqconc{S}{T}\right)\cup\left(\seqconc{S}{R}\right) 
		\end{array}
	\end{align*}
	holds.
\end{lemma}
The proof follow directly from the definition of the concatenation and union of sets of sequences.
