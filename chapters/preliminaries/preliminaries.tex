% mainfile: ../../Refinement.tex
At the heart of the refinement of \findex{\picalc{}} processes is the theory of \findex[sequence]{sequences}. Thus, in this chapter, we recall the model of sequences to gain a formal construct to handle ordered elements.% in an intuitive way.

Furthermore, we introduce the \picalc{} and investigate its behavior properly. In particular, we carefully explain the operational semantics of \picalc{} processes, since its peculiarities induce the characteristics of the refinement and its properties. Moreover, we discuss why we choose this particular operational semantics for the following work in this thesis and compare it to other semantics.

The majority of those definitions and notions can, for example, be found in \cite{milner,sangiorgi}.

As mathematical notations, we consider the natural numbers starting with zero ($\N=\set{0,1,2,\ldots}$) and use $\fatsemi$ as the composition of relations. Furthermore, we denote $R^*$ as the reflexive and transitive closure of a relation $R$.


\section{The \texorpdfstring{$\pi$}{pi}-calculus}
\label{sec_pi_calculus}
% mainfile: ../../Refinement.tex
The \findex[\picalc{}|(]{\picalc{}} belongs to the family of ...

\newpage %%%%%%%%%%%%%%% NEWPAGE!


\section{The OZ}
\label{sec_oz}
% mainfile: ../../Refinement.tex
The OZ

\newpage %%%%%%%%%%%%%%% NEWPAGE!



