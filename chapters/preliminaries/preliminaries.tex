% mainfile: ../../Refinement.tex
At the heart of the refinement of \findex{\picalc{}} processes is the theory of \findex[sequence]{sequences}. Thus, in this chapter, we recall the model of sequences to gain a formal construct to handle ordered elements.% in an intuitive way.

Furthermore, we introduce the \picalc{} and investigate its behavior properly. In particular, we carefully explain the operational semantics of \picalc{} processes, since its peculiarities induce the characteristics of the refinement and its properties. Moreover, we discuss why we choose this particular operational semantics for the following work in this thesis and compare it to other semantics.

The majority of those definitions and notions can, for example, be found in \cite{milner,sangiorgi,meyer,caires}.

As mathematical notations, we consider the natural numbers starting with zero ($\N=\set{0,1,2,\ldots}$) and use $\fatsemi$ as the composition of relations. Furthermore, we denote $R^*$ as the reflexive and transitive closure of a relation $R$.
%%%%%%%%%%%%%%%%%%%%%%%%%%%%%%%%%%%%%%%%%%%%%%%%%%%%%%%%%%% START OLD DESCRIPTION TEXT %%%%%%%%%%%%%%%%%%%%%%%%%%%%%%%%%%%%%%%%%%%%%%%%%%%%%%%%%%%%%%%%
\begin{old}{some old description text}
In this chapter we provide a collection of basic knowledge, which is further used in this thesis. Those definitions and notions can for example be found in \cite{milner,sangiorgi,meyer,caires}.

In \refSec{sec_seqs} we present \findex[sequence]{sequences}, a formal construct to handle ordered elements in an intuitive way.

In the second part (\refSec{sec_pi_calculus}) we introduce a process algebra -- the \findex{\picalc{}} -- published by Robin Milner, Joachim Parrow, and David Walker in \cite{milnerParrowWalker}, which is used to model concurrency and communication of mobile processes. In addition to that, we present an adaption of an operational semantics for \picalc{} processes -- the \findex{early transition system}\index{transition system!early} --, which is defined in \cite{sangiorgi} by Davide Sangiorgi and David Walker. Furthermore we discuss why we choose this operational semantics for the following work in this thesis and compare it to other semantics.
\end{old}
%%%%%%%%%%%%%%%%%%%%%%%%%%%%%%%%%%%%%%%%%%%%%%%%%%%%%%%%%%% END OLD DESCRIPTION TEXT %%%%%%%%%%%%%%%%%%%%%%%%%%%%%%%%%%%%%%%%%%%%%%%%%%%%%%%%%%%%%%%%

\section{Sequences}
\label{sec_seqs}
% mainfile: ../../Refinement.tex
In this section we recall the definition of sequences and some of its properties. This well-known information can be found in standard textbooks.

\begin{definition}[Sequence]
\label{def_seq}
A \findex{sequence} is a partial function which maps natural numbers to values of a given set. We define the set of all finite sequences for a given set $X$ by
\[\seqset{X}\define\set[\exists{}n\in\N:\texttt{dom}(f) = \set{1,\ldots,n} \vee \texttt{dom}(f)=\emptyset]{f:\N\rightarrow{}X},\]
with $\texttt{dom}(f)$ denotes the domain of a function $f$.

 %$\set[i\in\set{1,\ldots,n}]{\left(i,a_i\right)}\in\seqset{X}$ with $\set{a_1,\ldots,a_n}\subseteq{}X$.

The collection of all sequences we denote by $\texttt{seqs}$. As an abbreviation we write $\seq{a_1,\ldots,a_n}$ for the function $f:\N\rightarrow{}\set{a_1,\ldots,a_n}$ with $i\mapsto{}a_i$ for all $i\in\set{1,\ldots,n}$. Furthermore, we also use $\alpha\in{}s$ instead of $(\cdot,\alpha)\in{}s$, for a sequence $s\in\seqset{X}$ and $\alpha\in{}X$.

The special case that $\texttt{dom}(f)=\emptyset$ holds, so no element is mapped for a function $f$, is called an \index{sequence!empty}\findex{empty sequence}, and we denote it by $\eseq{}$. The collection of all sequences apart from the empty sequence is denoted by $\texttt{seqs}_1$. Similarly, $\texttt{seq}_1(X)$ describes all sequences over a given set $X$ except of the empty sequence. Furthermore, we often use the index notation for a sequence $s=\seq{a_1,\ldots,a_n}$ to address the single elements: $s_i=s(i)$.
\end{definition}

To ease the usage of sequences, we list some operations handling them and their properties.

\begin{definition}[Operations on sequences]
\label{def_seqs_ops}
We define the \index{sequence!length}\findex{length of a sequence} by the cardinality of its domain:
\begin{align}
\#: \left\{ \begin{array}{cl}
		 \texttt{seqs}&\rightarrow\N \\
		s&\mapsto\card{\texttt{dom}(s)}.
		\end{array}\right.
\tag{length}
\end{align}
For the \index{sequence!concatenation}\findex{concatenation} of two sequences we use the $\seqconc{}{}$ operator, defined by
\begin{align}
\seqconc{}{}:\left\{ \begin{array}{cl}
		 \texttt{seqs}\times\texttt{seqs}&\rightarrow\texttt{seqs} \\
		  (s,t)&\mapsto{}s\cup\set[(i,x)\in{}t]{(i+\len{s},x)}.		
		\end{array}\right.
\tag{concatenation}
\end{align}
%\begin{old}[not needed definition] %%%%%%%%%%%%%%%%%%%%%%%%%%%%%%%%%%%% OLD: NOT NEEDED DEFINITION
%To address the first element and the rest of the list, we define the $\texttt{head}$\index{sequence!head}\index{head} and the $\texttt{tail}$\index{sequence!tail}\index{tail} function for any set $X$:
%\begin{equation*}
%\begin{aligned}
%\texttt{head}:\left\{ \begin{array}{cl}
%		 \texttt{seq}_1(X)&\rightarrow{}X \\
%		  s&\mapsto{}s_1		
%		\end{array}\right.
%%\tag{head}
%\end{aligned}\quad
%\begin{aligned}
%\texttt{tail}:\left\{ \begin{array}{cl}
%		 \texttt{seqs}_1&\rightarrow{}\texttt{seqs}\\
%		  s&\mapsto{}\set[(i,x)\in{}s\wedge i\neq{}1]{(i-1,x)}	
%		\end{array}\right.
%%\tag{tail}
%\end{aligned}
%\end{equation*} All in an comment. Otherwise there is an indent at the beginning of the next sentence
%\end{old}%%%%%%%%%%%%%%%%%%%%%%%%%%%%%%%%%%%%%%%%%%%%%%%%%%%%%%%%%%%%%%%%%%%%%%%%%%%%%%%%% OLD: NOT NEEDED DEFINITION
The \index{sequence!iteration}\findex{iteration} of a sequence is inductively defined by
\begin{align}
	\begin{array}{lcl}
		s^0&\define&\eseq \\
		s^{n+1}&\define&\seqconc{s^n}{s}.
	\end{array}
\tag{iteration}
\end{align}
We lift both, the concatenation for two sets $S,T\subseteq\texttt{seqs}$ by
\begin{align*}\seqconc{S}{T}\define\set[s\in{}S,t\in{}T]{\seqconc{s}{t}}\tag{concatenation}\end{align*}
and the iteration by
\begin{align}
	\begin{array}{lcl}
		S^0&\define&\set{\eseq} \\
		S^{n+1}&\define&\seqconc{S^n}{S}
	\end{array}
\tag{iteration}
\end{align}
to sets of sequences.

Furthermore, we define the \findex{Kleene star} as $S^*\define\bigcup_{n\in\N}S^n$, as well as the \findex[prefix!operator]{prefix operator}:
\begin{align}
\pref{S}\define\set[\exists{}s\in{}S,u\in\texttt{seqs}:s=\seqconc{t}{u}]{t\in{}\texttt{seqs}}\tag{prefix}
\end{align}
for a set $S\subseteq\texttt{seqs}$.
%\begin{description}
%\item[the \index{sequence!length}\findex{length of a sequence}:] $\#:\texttt{seqs}\rightarrow\N$ with $\#(s)\define\texttt{max}(\texttt{dom}(s))$.
%\item[\index{sequence!concatenation}\findex{concatenation of sequences}:] $\seqconc{}{}:\texttt{seqs}\times\texttt{seqs}\rightarrow\texttt{seqs}$ with $\seqconc{s}{t}\define{}s\cup\set[(i,x)\in{}t]{(i+\#s,x)}$
%\item[\index{sequence!head}\findex{head of a sequence}:] For any set $X$ $\texttt{head}:\texttt{seqs}(X)\setminus\eseq\rightarrow{}X$ with $\texttt{head}(s)\define{}s_1$
%\item[\index{sequence!tail}\findex{tail of a sequence}:]  $\texttt{tail}:\texttt{seqs}\setminus\eseq\rightarrow{}\texttt{seqs}$ with $\texttt{tail}(s\right)\define{}\set[(i,x)\in{}s\wedge i\neq{}1]{(i-1,x)}$.
%\item[\index{sequence!iteration}\findex{iteration of a sequence}:] $\circ:\texttt{seqs}\times\N\rightarrow{}\texttt{seqs}$ written $\circ(s,i)$ as $s^i$.  $s^0\define\eseq$, $s^{n+1}\define\seqconc{s^n}{s}$.
%\item[\index{sequence!prefix}\findex{prefix of a sequence}:]  $\texttt{pref}:\texttt{seqs}\times\texttt{seqs}\rightarrow{}\B$ $s^0\define\eseq$, $s^{n+1}\define\seqconc{s^n}{s}$.
%\end{description}
\end{definition}

\begin{old}[not needed properties] %%%%%%%%%%%%%%%%%%% OLD: NOT NEEDED PROPERTIES
There are many well-known facts about sets of sequences and the given operators. We collect a few in the following lemma.

\begin{lemma}[Properties]
\label{lem_seq_props}
	Given $S,T,R\in\texttt{seqs}$ it holds:\todo{Till now just distributivity used. Delete the others, if not needed.}
	\begin{align*}
		\begin{array}{rcl}
			\seqconc{\left(S\cup{}T\right)}{R} &=& \left(\seqconc{S}{R}\right)\cup\left(\seqconc{T}{R}\right) \\
			\seqconc{S}{\left(T\cup{}R\right)} &=& \left(\seqconc{S}{T}\right)\cup\left(\seqconc{S}{R}\right) \\
			S&\subseteq{}&S^*\\
			\left(S^*\right)^*&=&S^*\\
			S &\subseteq{}&\pref{S} \\
			\pref{\pref{S}} &=& \pref{S}.
		\end{array}
	\end{align*}
Furthermore, \findex{Arden's rule} holds, which means that the equation $S=\left(\seqconc{T}{S}\right)\cup{}R$ has the solution $S=\seqconc{T^*}{R}$; moreover, the solution is unique if $\seq{}\nin{}T$.
\end{lemma}
\end{old} %%%%%%%%%%%%%%%%%%% OLD: NOT NEEDED PROPERTIES

With the definition of the concatenation of sets of sequences, we get that the union and the concatenation of sets of sequences are distributive.
\begin{lemma}[Distributivity]
\label{lem_seq_props}
	Given $S,T,R\in\texttt{seqs}$, then
	\begin{align*}
		\begin{array}{rcl}
			\seqconc{\left(S\cup{}T\right)}{R} &=& \left(\seqconc{S}{R}\right)\cup\left(\seqconc{T}{R}\right) \\
			\seqconc{S}{\left(T\cup{}R\right)} &=& \left(\seqconc{S}{T}\right)\cup\left(\seqconc{S}{R}\right) 
		\end{array}
	\end{align*}
	holds.
\end{lemma}
The proof follow directly from the definition of the concatenation and union of sets of sequences.


\section{The \texorpdfstring{$\pi$}{pi}-calculus}
\label{sec_pi_calculus}
% mainfile: ../../Refinement.tex
The \findex[\picalc{}|(]{\picalc{}} belongs to the family of \findex[process!algebra]{process algebras} and is used to model communicating and concurrent systems of mobile processes with changing structure. The roots of process algebras can be found in Tony Hoare's \gls{CSP} from the late $70$'s and Milner's \gls{CCS} invented around $1980$ \cite{milnerCCS}. Moreover, the \picalc{} can be seen as a continuation and extension of \gls{CCS} and is inspired by the work of Engberg and Nielsen in \cite{engbertNielsen}, where they extended \gls{CCS} with mobility \cite{milnerParrowWalker}. 

The first published paper treating the \picalc{} is called ``A calculus of mobile processes'' \cite{milnerParrowWalker} written by Robin Milner, Joachim Parrow, and David Walker in the late $80$'s \cite{milner}. While in \gls{CCS} only messages can be sent over a channel during a communication, in this paper they developed the capability to send a channel over a channel. In particular, messages and channels have the same type.%Therein the \picalc{} is defined and it is presented how the calculus can be used to model communicating systems with processes with changing structure; also notions like \findex{strong bisimilarity} and \findex{strong equivalence} are already defined.

Hence, the main advantage of the \picalc{} is its capability to send names over a channel during a communication which can later be used as channels themselves. This provides the ability to change the connection structure of a system.

In this thesis we choose a \picalc{} variant which uses parameterized recursion instead of the replication operator. Thereby, we have no loss in expressiveness, since there is a transformation from recursion to replication and vice versa \cite{milner,sangiorgi}. But since the definition of the examples in the topic of computer science with recursion appeared to be more intuitive to us and recursion is present in \gls{CCS}, we decided to use a variant with recursion rather than replication.%We use this variant at the expense of loosing the possibility of using structural induction \cite{sangiorgi}\todo{instead: fixed point induction?}, because the definition of the examples in the topic of computer science seemed more intuitive to us with recursion.

Furthermore, we define a variant without the match prefix, since ``[m]atch, and especially mismatch, prefixes are seldom useful for describing systems''\footnote{\cite{sangiorgi}, page $13$.}, but inflating the proofs and definitions.

Apart from that, the definition of the syntax as well as the definition of the semantics and the treatment of the \picalc{} refers to \cite{sangiorgi}. Especially the \findex[guarded choice]{guarded choice} -- which means every process in a choice has to perform a prefix action first -- differs from Milner's, Parrow's and Walker's syntax in \cite{milnerParrowWalker}. This restriction is well-accepted in literature \cite{meyer}, since non-guarded choice ``is considered of minor practical importance''\footnote{\cite{meyer}, page $15$.} and ``complicates the theory''\footnote{\cite{sangiorgi}, page $13$.}. However, using guarded choice ``does not delimit the computational expressiveness of the calculus''\footnote{\cite{meyer}, page $15$.}. We slightly differ from the standard definition of the guarded choice. That is, we are not only allowed to have prefix processes in a choice, but we additionally allow that such processes may be wrapped in restrictions. This syntactically extension is needed to transform an arbitrary process into a process in our extended standard form. However, this adaption is of minor influence for the remaining theory.

Finally, we use the \index{\picalc{}!monadic}\findex[monadic \picalc{}]{monadic} \picalc{} -- meaning just a single name can be sent (respectively received) by an output (respectively input) action --, since Milner also gave a transformation from the \index{\picalc{}!polyadic}\findex[polyadic \picalc{}]{polyadic} to the monadic version, which proves their equal expressiveness \cite{milner} and the monadic variant simplifies the handling of processes in this thesis.

\subsection{Syntax}
\label{sec_pi_syntax}
% mainfile: ../../Refinement.tex
Let $\names$ be the countably infinite set of \findex[name]{names}, with typical representatives as lower case letters like $a,b,c,x,y,z,\ldots\in\names$ and let $\conames\define\set[a\in\names]{\overline{a}}$ be the set of \findex{co-names}. In communications the names of $\names$ are used as \findex[channel]{channels} as well as as \findex[message]{messages} / \findex[object]{objects}.

With the help of these names we can define \findex[prefix]{prefixes} of \picalc{} processes. The possible prefixes $\pi$ of a \picalc{} process are
\[\pi \syntdef \inp{x}{y} \ebnf \out{x}{y} \ebnf \tau,\]
with $x,y\in\names$. The prefix $\out{x}{y}$ stands for the \index{output!prefix}\findex[prefix!output]{output prefix}, which means the process can send the name $y$ over the channel $x$. For receiving a value $y$ along the channel $x$, the \index{input!prefix}\findex[prefix!input]{input prefix} $\inp{x}{y}$ is used. After receiving, every occurrence of $y$ in the remaining process is replaced by the received name. The third prefix is called the \index{prefix!silent}\findex[silent!prefix]{silent prefix} and means that the process performs an internal step.

Processes do not simply consist of prefixes, they also can for example be composed in parallel or appear in an alternative. The full syntax of \picalc{} processes is given in \refDef{def_syntax}. To define a recursion with parameters, there is the need for \findex[process!identifier]{process identifiers} typically denoted by upper case letters $A,B,C,\ldots$ and an abbreviation for a finite list of names. Thus, we can write $\vec{a}$ for a parameter list $a_1,a_2,\ldots,a_n$ for some $n\in\N\setminus\set{0}$ and $a_1,\ldots,a_n\in\names$. Additionally, we treat $\vec{a}$ as a set $\set{a_1,a_2,\ldots,a_n}$, whenever it is needed and no confusion arises. Moreover, in the \findex{recursive definition} $\procdef{A}{\vec{w}}\define{}P$ with the process identifier $A$ and a process $P$, the elements of $\vec{w}$ must be pairwise distinct.

\begin{definition}[Syntax]
\label{def_syntax}
The syntax of \picalc{} \findex[process]{processes} -- typically denoted by $P,Q,R,\ldots$ -- is defined inductively: 
\begin{align*}
 M & \syntdef \proczero \ebnf \pi.P \ebnf \procchoice{M_1}{M_2} \ebnf \procres{z}{M} \\
 P & \syntdef M \ebnf \procpar{P_1}{P_2} \ebnf \procres{z}{P} \ebnf \proccall{A}{\vec{v}}.
\end{align*}
The set of all \picalc{} processes is denoted by $\procs$. We call $M$ a \findex{summation} or \findex{sum} and the set of all summations is denoted by $\sums$. We further assume that there is a recursive definition $\procdef{A}{\vec{w}}\define{}Q$ for every $\proccall{A}{\vec{v}}$ and define that every process itself relies on a finite number of process identifiers.
\end{definition}

To gain an intuition of the intended interpretation of the syntax, at first we give a brief informal description of the behavior of the possible types of processes, before we define an operational semantics in \refSec{sec_pi_op_sem}.

The process $\proczero$ is called the \index{process!stop}\findex{stop process} or \findex{inaction} and stands for the process without behavior. Often we will -- as a convention -- omit a pending $\proczero$ and accordingly write, for example, $\procpar{\out{x}{y}}{\left(\procchoice{\inp{x}{z}}{\tau}\right)}$ instead of $\procpar{\out{x}{y}.\proczero}{\left(\procchoice{\inp{x}{z}.\proczero}{\tau.\proczero}\right)}$.

Let $\pi$ be one of the three defined prefixes and $P\in\procs$, then we call $\pi.P$ a \index{process!prefix}\findex[prefix!process]{prefix process}. The prefixes represent the communication possibilities of a process. As described we can send or receive values over a given channel or perform an internal step. From the semantical view we call it an \findex[action]{action}, if such a prefix is performed. Every action resulting from handling a prefix apart from the silent prefix will be called an \index{action!observable}\findex[observable action]{observable action}, since this behavior is visible to the environment. Thus, a process $P'\define\out{x}{y}.P$ has to perform an \findex[action!output]{output action} $\out{x}{y}$ before it behaves like $P$. Furthermore, we call $x$ the \findex{subject} and $y$ the \findex{object} or \findex{parameter} of a prefix $\out{x}{y}$ or $\inp{x}{y}$, as well as of the corresponding actions. The third prefix performs a \index{action!silent}\findex[silent!action]{silent action} or -- in contrast to the other prefixes -- an \index{action!unobservable}\findex{unobservable action} and is used to describe an \index{action!internal}\findex{internal action}. Thus, $\pi.P$ performs the $\pi$ prefix and after that it behaves like $P$ or in case that $\pi$ is an input prefix it behaves like a process $P'$, where every occurrence of the object of the prefix is replaced by the received name. For example the process $\inp{a}{x}.\out{x}{y}.\proczero$ can evolve with the visible \findex[action!input]{input action} $\inpa{a}{b}$ to the process $\out{b}{y}.\proczero$. Depending on the prefix $\pi$ we also call a process $\pi.P$ an \index{process!input}\findex[input!process]{input process}, \index{process!output}\findex[output!process]{output process} or \index{process!$\tau$}\findex{$\tau$ process}.

For a \index{process!parallel}\findex{parallel composition} $\procpar{P_1}{P_2}$ with $P_1,P_2\in\procs$ the behavior is an interleaving of the behavior of the process $P_1$ and $P_2$, with the added possibility of a communication between the two parts over a common channel. Consider, for example, $P_1\define\inp{x}{y}.\out{y}{z}$ and $P_2\define\out{x}{a}.\out{b}{b}$, then $P\define\procpar{P_1}{P_2}$ can evolve invisibly to $\procpar{\out{a}{z}}{\out{b}{b}}$, since $P_1$ and $P_2$ communicate over the common channel $x$ by sending the name $a$. Furthermore, $P_1$ and $P_2$ can act independently, so if, for instance, $P_2$ performs the visible action $\out{x}{a}$, $P$ can also perform the action $\out{x}{a}$ and thus evolve to $\procpar{\inp{x}{y}.\out{y}{z}}{\out{b}{b}}$.

The \index{process!restriction}\findex{restriction} $\procres{z}{P}$ for a process $P\in\procs$ and a name $z\in\names$, binds the name $z$ to the process $P$. Thus, the scope of $z$ is restricted to $P$. This concept can be seen similarly to the notion of private variables in the context of programming languages. We call $z$ \findex[restricted name]{restricted}\index{name!restricted}, since the name $z$ may only be used within $P$ and not for a communication with the environment. For instance $\procres[a]{z}{\procpar{\inp{z}{x}}{\out{z}{y}}}$ can only invisibly communicate over $z$, whereby $\procres[a]{z}{\out{x}{y}.\out{z}{a}}$ can perform the visible output action $\out{x}{y}$, since $z$ is not a part of the action. But there is one exception from this rule. It is also possible to widen the scope of a restricted name $z$ -- called \findex{scope extrusion} --, if the restricted name is passed via a nonrestricted channel. For instance consider $P\define\procres[a]{z}{\out{x}{z}.\inp{z}{y}}$ and $Q\define\inp{x}{a}.\out{a}{b}$, then $\procpar{P}{Q}$ can communicate over channel $x$ so that afterwards the former just to $P$ known channel $z$ is also known, but restricted, in the whole remaining process $\procres[a]{z}{\procpar{\inp{z}{y}}{\out{z}{b}}}$.

A \findex[process!choice]{choice process}, or simply \findex{choice}, is written by $\procchoice{M_1}{M_2}$ with $M_1,M_2\in\sums$. From the definition we know that every participant of the choice must either be an inaction, a prefix, a restriction of a sum or simply a sum itself. Since a sum with a restriction is also possible within a choice, this variant is a slightly adapted version of a \findex[guarded choice]{guarded choice}. A choice process $\procchoice{M_1}{M_2}$ with $M_1,M_2\in\sums$ has the possibility to behave like the process $M_1$ or like $M_2$. But if one process is chosen, the behavior of the other is no longer taken into account. Thus, for instance, the process $\procchoice{\out{a}{b}.\out{b}{c}.\proczero}{\out{x}{y}.\out{y}{z}.\proczero}$ can evolve to the process $\out{b}{c}.\proczero$ or to $\out{y}{z}.\proczero$.

Finally, $\proccall{A}{\vec{v}}$ with a process $P\in\procs$ and a recursive definition $\procdef{A}{\vec{w}}\define{}P$ is a \findex[process!call]{process call}, or simply \findex{call}. The behavior of the call $\proccall{A}{\vec{v}}$ is the same as the behavior of the process $P'$, where $P'$ results of $P$ by the simultaneous replacement of every occurrence of the elements of $\vec{w}$ by the elements of $\vec{v}$. Hence, $\vec{v}$ and $\vec{w}$ have to be of the same length.

As a convention, we agree that the unary operators bind stronger than the binary ones, and also the sum binds more tightly than the parallel composition. Additionally, the prefix operator binds stronger than restriction. Thus, we can, for example, write $\procpar{\procres{y}{\out{x}{y}.P}}{\procchoice{\inp{a}{b}.Q}{M_1}}$ with $P,Q\in\procs$ and $M_1\in\sums$ instead of $\procpar{\procres[a]{y}{\out{x}{y}.P}}{\left(\procchoice{\left(\inp{a}{b}.Q\right)}{M_1}\right)}$. Furthermore, we agree that we can abbreviate a sequence of restrictions, for example $\procres[()]{a_1}{\ldots \procres[()]{a_n}{P}\ldots}$, by $\procres[()]{a_1,\ldots,a_n}{P}$.

%The limiting of the operators, which are used to build the processes with, leads to some interesting syntactical subclasses of the \picalc{}.
We present two interesting syntactical subclasses of the \picalc{}: the subclass of \findex{restriction-free} processes and the subclass of \findex{recursion-free} ones, which are already defined, for example, in \cite{meyer}.

\begin{definition}[Restriction-free]
\label{def_res_free}
A process $P'\in\procs$, which is constructed from the syntax of \refDef{def_syntax} without using the $\procres{z}{M}$ with $M\in\sums$ and $\procres{z}{P}$ with $P\in\procs$ parts, is called \findex{restriction-free}. The set of all restriction-free processes is denoted by $\procsresf$.
\end{definition}

Thus, there is no restriction operator within a restriction-free process. Analogously, we define the recursion-free process by not using the call operator from \refDef{def_syntax}, while building the process.

\begin{definition}[Recursion-free]
\label{def_res_free}
A process $P\in\procs$, which is constructed from the syntax of \refDef{def_syntax} without using the $\proccall{A}{\vec{v}}$ part, is called \findex{recursion-free}. The set of all recursion-free processes is denoted by $\procsrecf$.
\end{definition}

These subclasses helps us in \refSec{sec_de_sem_trace_prop_comp} to investigate the compositionality of our in \refSec{sec_de_sem_trace_def} defined semantics.


\subsection{Names and substitution}
\label{sec_pi_names_substitution}
% mainfile: ../../Refinement.tex
As already seen in the informal description of the syntax of the \picalc{}, the restriction operator as well as the input prefix binds a name to a process. Intuitively, the \index{name!bound}\findex[bound name]{bound names} of a process are those, which occur as an object in an input prefix and those who are restricted within a process. The other names, which additionally appear in a process, are called \index{name!free}\findex[free name]{free names}. Formally, we introduce with \refDef{def_bound_free_names} two functions to calculate the bound respectively free names of a process. 

\begin{definition}[Bound and free names of a process]
\label{def_bound_free_names}%|see{free name}
Given $P,Q\in\procs$ and $M,N\in\sums$. Furthermore, let $a,b\in\names$. The function $\bnF:\procs\rightarrow\pom{\names}$ collects all \index{name!bound}\findex[bound name!process]{bound names} of a process and similarly the function $\fnF:\procs\rightarrow\pom{\names}$ computes the \index{name!free}\findex[free name!process]{free names} of a process. Both are inductively defined as follows:
\begin{equation*}	
	\begin{aligned}			
		\begin{array}{lcl}
			\bn{\proczero} &\define& \emptyset \\
			\bn{\tau.P} &\define& \bn{P}\\
			\bn{\out{a}{b}.P} &\define& \bn{P} \\
			\bn{\inp{a}{b}.P} &\define&\set{b}\cup\bn{P}\\
			\bn{\procchoice{M}{N}} &\define& \bn{M}\cup\bn{N}\\
			\bn{\procpar{P}{Q}} &\define& \bn{P}\cup\bn{Q} \\
			\bn{\procres{a}{P}} &\define& \set{a}\cup\bn{P} \\
			\bn{\proccall{A}{\vec{v}}} &\define& \emptyset
		\end{array}
	\end{aligned}
	\begin{aligned}
		\begin{array}{lcl}
			\fn{\proczero} &\define& \emptyset \\
			\fn{\tau.P} &\define &\fn{P} \\
			\fn{\out{a}{b}.P} &\define& \set{a,b}\cup\fn{P} \\
			\fn{\inp{a}{b}.P} &\define& \set{a}\cup\left(\fn{P}\setminus\set{b}\right) \\
			\fn{\procchoice{M}{N}} &\define &\fn{M}\cup\fn{N} \\
			\fn{\procpar{P}{Q}} &\define& \fn{P}\cup\fn{Q} \\
			\fn{\procres{a}{P}}& \define& \fn{P}\setminus\set{a} \\
			\fn{\proccall{A}{\vec{v}}}& \define& \vec{v}.
		\end{array}
	\end{aligned}
\end{equation*}
Furthermore, we call $\nF:\procs\rightarrow\pom{\names}$ with $\n{P}\define \fn{P}\cup\bn{P}$ for all $P\in\procs$ the \findex[name!process]{names} of a process $P$.
\end{definition}

According to that, we stipulate that for a recursive definition $\procdef{A}{\vec{w}}\define P$, the free names of $P$ are included in the parameter list; thus, $\fn{P}\subseteq\vec{w}$. This simplifies the handling of a recursive call and so the definition of the free names of a recursive call fits to this convention.

%According to that, we stipulate that for a recursive definition $\procdef{A}{\vec{w}}\define P$, the free names of $P$ are included in the parameter list; thus, $\fn{P}\subseteq\vec{w}$. This simplifies the handling of a recursive call and its names by not complicating its definition. Thus, the definition of the free names of a recursive call fits to this convention.
%Furthermore, we define all parameters of a call as free. From the definition of a call we know that the free names of the process must occur as parameters of the call, but the possibility for more parameters is not excluded. Since those cannot occur bound in the process -- bound names are unique (\refConv{conv_uni_bn})--, they cannot occur at all. Hence there is no problem by calling them also free.

For the informal description of the semantics of the input process and the process call, we already used an idea of the replacement of names in a process. We now give a formal definition of a function, which replaces the free names of a process by some other names.

\begin{definition}[Substitution]
\label{def_substitution}
We define a \findex{substitution} $\substF:\names\rightarrow\names$, as the simultaneous replacement of names, that is, $\substF$ maps a finite number of names to other names and behaves as the identity otherwise. We write $\subs{a_1,\ldots,a_n}{b_1,\ldots,b_n}$ for some $n\in\N$ and $a_1,\ldots,a_n,b_1,\ldots,b_n\in\names$ for the substitution
	\[\substF(x) =\left\{\begin{array}{ll}
									a_i & \falls\; x=b_i \;\text{and} \; i\in\set{1,\ldots,n}\\
									x & \text{else}
									\end{array}
								\right.\]
	as an abbreviation. Furthermore, we define the \findex[name!substitution]{names} of such a substitution with $\n{\subs{a_1,\ldots,a_n}{b_1,\ldots,b_n}}\define\set{a_1,\ldots,a_n,b_1,\ldots,b_n}$, the \index{substitution!support}\findex{support} with $\supp{\sigma}\define\set[\substF(x)\neq{}x]{x\in\names}$ and the \index{substitution!co-support}\findex{co-support} of a such a substitution by $\cosupp{\sigma}\define\set[x\in\supp{\substF}]{\substF(x)\in\names}$. Hence, $\n{\substF}=\supp{\substF}\cup\cosupp{\substF}$. 

If the substitution $\sigma$ just interchanges two names, for example, $\sigma=\subs{a,b}{b,a}$ we call $\sigma$ a \findex{transposition} and abbreviate it by $\transp{a}{b}$. Furthermore, we define the application of a substitution $\sigma$ on a set of names $X\subseteq\names$ by $X\sigma\define\set[x\in{}X]{\sigma(x)}$.

The \index{substitution!process}\findex[application substitution!process]{application} of a substitution $\substF$ to a process $P\in\procs$, with $\n{\sigma}\cap\bn{P}=\emptyset$, is written as $P\substF\in\procs$ and is inductively defined
	\begin{equation*}	
		\begin{aligned}			
			\begin{array}{lcl}
				\proczero\substF & \define & \proczero \\
				\subst{\tau.P_1} & \define & \tau.\left(P_1\substF\right) \\
				\subst{\inp{x}{y}.P_1} & \define & \inp{\substF(x)}{y}.\left(P_1\substF\right) \\
				\subst{\out{x}{y}.P_1} & \define & \out{\substF(x)}{\substF(y)}.\left(P_1\substF\right)
			\end{array}
		\end{aligned}
		\begin{aligned}
			\begin{array}{lcl}
				\subst{\procchoice{M_1}{M_2}} & \define & \procchoice{M_1\substF}{M_2\substF} \\
				\subst{\procpar{P_1}{P_2}} & \define & \procpar{P_1\substF}{P_2\substF} \\
				\subst{\procres{a}{P_1}} & \define & \procres[a]{a}{P_1\substF}\\
				\proccall{A}{\vec{v}}\substF & \define & \proccall{A}{\vec{v}\substF}
			\end{array}
		\end{aligned}
	\end{equation*}
for $x,y,a\in\names$, $\vec{v}\subseteq\names$, $P_1,P_2\in\procs$ and $M_1,M_2\in\sums$.
\end{definition}

%%%%%%%%%%%%%%%%%%%%%%%%%%%%%%%%%%%%%%%%%%%%%%%%%%%%%%%%%%%%%%%%%%%%%%%%%%%%%% START SAFE SUBSTITUTION %%%%%%%%%%%%%%%%%%%%%%%%%%%%%%%%%%%%%%%%%%%%%%%%%%%%%%%%%%%%%%
\begin{old}{Not needed definition of save substitution}
For the definition of the operational semantics as well, there is the need for a formal definition of the informally described replacement of names for example for the input prefix and the process call. To avoid the often used substitution where -- as a convention -- the names of the substitution differs from the names already occurring bound in the process (for example used in \cite{sangiorgi}), we accord to the substitution Lu\'{i}s Caires and Luca Cardelli presented in \cite{caires}. So we explicitly change the bound names of a process while applying a substitution so that no unintended capture of names by binders can occur.

Hence the secure substitution replaces names, no matter if they are bound, free or even existing in the process.
\begin{definition}[Safe substitution]
\label{def_substitution}
	We define a \findex{safe substitution} or simply \findex{substitution} $\substF:\names\rightarrow\names$, as the simultaneous replacement of names, such as $\substF$ maps a finite number of names to other names and behaves as identity otherwise. As abbreviation we write $\subs{a_1,\ldots,a_n}{b_1,\ldots,b_m}$ for some $n,m\in\N$ and $a_1,\ldots,a_n,b_1,\ldots,b_m\in\names$ for the substitution
	\[\substF\left(x\right) =\left\{\begin{array}{ll}
									a_i & \falls\; x=b_i \;\text{and} \; i\leq{}n\\
									x & \text{else}
									\end{array}
								\right..\]
	Furthermore we define the names of such a substitution as: $\n{\subs{a_1,\ldots,a_n}{b_1,\ldots,b_m}}\define\set{a_1,\ldots,a_n,b_1,\ldots,b_m}$. 

The \index{substitution!process}\findex[application substitution!process]{application} of a substitution $\substF$ to a process $P\in\procs$ is written $P\substF\in\procs$ and is inductively defined
	\begin{equation*}	
		\begin{aligned}			
			\begin{array}{lcl}
				\proczero\substF & \define & \proczero \\
				\subst{\tau.P} & \define & \tau.\left(P\substF\right) \\
				\subst{\inp{x}{y}.P} & \define & \inp{\substF\left(x\right)}{p}.\left(\subst{P\subs{p}{y}}\right) \;\text{for}\;p\nin\left(\fn{P}\cup\set{x}\cup\n{\substF}\right) \\
				\subst{\out{x}{y}.P} & \define & \out{\substF\left(x\right)}{\substF\left(y\right)}.\left(P\substF\right) \\
%			\end{array}
%		\end{aligned}\quad
%		\begin{aligned}
%			\begin{array}{lcl}
				\subst{\procchoice{P}{Q}} & \define & \procchoice{P\substF}{Q\substF} \\
				\subst{\procpar{P}{Q}} & \define & \procpar{P\substF}{Q\substF} \\
				\subst{\procres{a}{P}} & \define & \procres[a]{p}{\subst{P\subs{p}{a}}} \;\text{for}\;p\nin\left(\fn{P}\cup\n{\substF}\right)\\
				\proccall{A}{v_1,\ldots,v_n}\substF & \define & \proccall{A}{\substF\left(v_1\right),\ldots,\substF\left(v_n\right)} \\
			\end{array}
		\end{aligned}
	\end{equation*}
for $x,y,a,v_1,\ldots,v_n\in\names$ and $P,Q\in\procs$.
\end{definition}
\end{old}
%%%%%%%%%%%%%%%%%%%%%%%%%%%%%%%%%%%%%%%%%%%%%%%%%%%%%%%%%%% END SAFE SUBSTITUTION %%%%%%%%%%%%%%%%%%%%%%%%%%%%%%%%%%%%%%%%%%%%%%%%%%%%%%%%%%%%%%%%
%%%%%%%%%%%%%%%%%%%%%%%%%%%%%%%%%%%%%%%%%%%%%%%%%%%%%%%%%%%%%%%%%%%%%%%%%%%%%% START TRANSPOSITION %%%%%%%%%%%%%%%%%%%%%%%%%%%%%%%%%%%%%%%%%%%%%%%%%%%%%%%%%%%%%%
\begin{old}{Not needed definition for transposition}
\begin{definition}[Transposition]
\label{def_transposition}
	We define a \findex{transposition} $\theta_X:\names\rightarrow\names$ noted $\transp{m}{n}_X$ for some $n,m\in\names$ and $X\subset\names$ as the interchanging of the names such that
	\[\transp{m}{n}_X\left(x\right) =\left\{\begin{array}{ll}
					n & \falls\; x=m \;\text{and} \; m,n\nin{}X\\
					m & \falls\; x=n \;\text{and} \; m,n\nin{}X\\
					x & \text{else}
					\end{array}
				\right.\]
	holds with $\n{\transp{m}{n}_X}\define\set{m,n}$ the names of the transposition. Furthermore we define the transposition $\theta_X:\procs\rightarrow\procs$ also noted $\transp{m}{n}_X$ for some $n,m\in\names$ and $X\subset\names$ inductively for processes:
	\begin{equation*}	
		\begin{aligned}			
			\begin{array}{lcl}
				\transpT{X}{\proczero} & \define & \proczero \\
				\transpT{X}{\tau} & \define & \tau \\
				\transpT{X}{\inp{x}{y}} & \define & \inp{\transpT{X}{x}}{\transpT{X}{y}} \\
				\transpT{X}{\out{x}{y}} & \define & \out{\transpT{X}{x}}{\transpT{X}{y}} \\
				\transpT{X}{\pi.P} & \define & \transpT{X}{\pi}.\transpT{X}{P} \\
			\end{array}
		\end{aligned}\quad
		\begin{aligned}
			\begin{array}{lcl}
				\transpT{X}{\procchoice{P}{Q}} & \define & \procchoice{\transpT{X}{P}}{\transpT{X}{Q}} \\
				\transpT{X}{\procpar{P}{Q}} & \define & \procpar{\transpT{X}{P}}{\transpT{X}{Q}} \\
				\transpT{X}{\procres{a}{P}} & \define & \procres{\transpT{X}{a}}{\transpT{X}{P}} \\
				\transpT{X}{\proccall{A}{\vec{v}}} & \define & \proccall{B}{\transpT{X}{\vec{v}}} \\
			\end{array}
		\end{aligned}
	\end{equation*}
for $x,y,a\in\names$, $P,Q\in\procs$ and $\procdef{A}{\vec{w}}\define{}P$ and $\procdef{B}{\vec{w}}\define{}\transpT{X\cup\vec{w}}{P}$. The function for the names of a transposition is  equally defined.
\end{definition}
\end{old}
%%%%%%%%%%%%%%%%%%%%%%%%%%%%%%%%%%%%%%%%%%%%%%%%%%%%%%%%%%% END TRANSPOSITION %%%%%%%%%%%%%%%%%%%%%%%%%%%%%%%%%%%%%%%%%%%%%%%%%%%%%%%%%%%%%%%%

For the definition of the \findex{$\alpha$-convertibility}, a relation collecting processes which can be obtained from each other by only a finite number of replacements of bound names, we refer to \cite{caires}. Therefore, we first define the meaning of a \findex{congruence relation} on processes.

\begin{definition}[Congruence relation]%\todo{why is the process congruence a congruence? Does not fit for all in all contexts.}
\label{def_cong_rel}
A relation $\equiv{}\subseteq\procs\times\procs$ is an \findex{equivalence relation}, if
\begin{align}
	\forall P\in\procs&\colon P\equiv{}P \tag{Reflexivity} \\
	\forall P,Q\in\procs&\colon P\equiv{}Q\Rightarrow Q\equiv{}P \tag{Symmetry}\\
	\forall P,Q,R\in\procs&\colon P\equiv{}Q \wedge Q\equiv{}R \Rightarrow P\equiv{}R\tag{Transitivity}
\end{align}
holds. If such a relation is preserved under each defined operator, it is called a \findex[process!congruence]{process congruence}. In the notation according to \cite{meyer}, an equivalence relation is a process congruence if
\begin{align}
	\forall P,Q,R\in\procs&\colon P\equiv{}Q\Rightarrow \procpar{P}{R}\equiv{}\procpar{Q}{R} \tag{Cong Par R}\label{congparr}\\
	\forall P,Q,R\in\procs&\colon P\equiv{}Q\Rightarrow \procpar{R}{P}\equiv{}\procpar{R}{Q} \tag{Cong Par L}\label{congparl}\\
	\forall P,Q\in\procs,M\in\sums,\pi\;\text{prefix}&\colon P\equiv{}Q \Rightarrow \procchoice{\pi.P}{M}\equiv{}\procchoice{\pi.Q}{M}\tag{Cong Sum1 R}\label{congsum1r}\\
	\forall P,Q\in\procs,M\in\sums,\pi\;\text{prefix}&\colon P\equiv{}Q \Rightarrow \procchoice{M}{\pi.P}\equiv{}\procchoice{M}{\pi.Q}\tag{Cong Sum1 L}\label{congsum1l} \\\notag\\
	\forall M,M_1,M_2\in\sums,a\in\names&\colon \notag\\
				 M_1\equiv{}M_2 \Rightarrow\; & \procchoice{\procres{a}{M_1}}{M}\equiv{}\procchoice{\procres{a}{M_2}}{M}\tag{Cong Sum2 R}\label{congsum2r}\\
				 M_1\equiv{}M_2 \Rightarrow\; & \procchoice{M}{\procres{a}{M_1}}\equiv{}\procchoice{M}{\procres{a}{M_2}}\tag{Cong Sum2 L}\label{congsum2l}\\
	\forall P,Q\in\procs,a\in\names&\colon  P\equiv{}Q \Rightarrow \procres{a}{P}\equiv{}\procres{a}{Q}\tag{Cong Res}\label{congres}
\end{align}
holds.
\end{definition}

Since the choice operator is only defined on prefix processes or restrictions of sums, in general $\procchoice{P}{M}$ and $\procchoice{\procres{a}{P}}{M}$ is not a process. Thus, every possible choice process is covered by the (\ref{congsum1r}) respectively (\ref{congsum1l}) and (\ref{congsum2r}) respectively (\ref{congsum2l}) properties. Furthermore, note that with $M=\proczero{}$ the prefix operator is also covered by the first two sum implications and the restriction operator by the second two sum implications together with the last implication of the congruence.

By adding two additional rules, we get a relation which is preserved under the renaming of bound names within a process.

\begin{definition}[$\alpha$-convertibility]
\label{def_alpha_conv}
The least congruence relation on processes, denoted by $\alphaeq\;\subseteq\procs\times\procs$, with
	\begin{align}
		\forall P\in\procs,a\in\names{},p\nin\n{P}&: \procres{a}{P}\alphaeq{}\procres[a]{p}{P\subs{p}{a}} \tag{Alpha Res} \\
		\forall P\in\procs,x,y\in\names{},p\nin\left(\n{P}\cup\set{x}\right)&: \inp{x}{y}.P\alphaeq{}\inp{x}{p}.\left(P\subs{p}{y}\right) \tag{Alpha Inp} 
	\end{align}
is called the \findex[$\alpha$-convertibility]{$\alpha$-convertibility relation}. For all $(P,Q)\in\;\alphaeq$ we say $P$ and $Q$ are \findex{$\alpha$-convertible} and the equivalence classes are denoted by $\ec{P}\in\procs_\alpha$ for all $P\in\procs$, where $\procs_\alpha$ denotes the set of all equivalence classes modulo $\alpha$-convertibility.
\end{definition}

Intuitively, we can rename the bound names within a process without changing anything of its behavior or nature. This concept can similarly be seen as the idea of local variables in programming languages. %To avoid the necessity to take a closer look at the structure of a process when determining the scope of a bound name, we introduce the following convention that the bound names of an investigated set of processes and substitutions are unique.

Since there is the possibility to change the bound names anytime we want without changing the process' behavior, we introduce the convention that the bound names of an investigated set of processes and substitutions are unique.% for a more simplified handling.

\begin{conv}[Uniqueness of bound names]
\label{conv_uni_bn}\index{uniqueness of bound names!process}
Given $P_1,\ldots,P_n\in\procs$ a collection of \picalc{} processes and a collection $\sigma_1,\ldots,\sigma_m$ of substitutions for $n,m\in\N$, we stipulate that
\begin{enumerate}
	\item $\forall{}i,j\in\set{1,\ldots,n}\;\text{with}\;i\neq{}j: \bn{P_i}\cap\bn{P_j} = \emptyset$,
	\item $\bigcup_{i\in\set{1,\ldots,n}}\bn{P_i} \cap \bigcup_{i\in\set{1,\ldots,n}}\fn{P_i} = \emptyset$,
	\item For all $i\in\set{1,\dots,n}$ and $\circ\in\set{\procpar{}{},\procchoice{}{}}$ the following implications hold
		\begin{align*}
			\begin{array}{lcll}
				P_i&=&P\circ{}Q &\Rightarrow \bn{P}\cap\bn{Q}=\emptyset \\
				P_i&=&\inp{a}{x}.P &\Rightarrow x\nin\bn{P} \\
				P_i&=&\procres{z}{P} &\Rightarrow  z\nin\bn{P}
			\end{array}
		\end{align*}
		with $P,Q\in\procs$ and $a,x,z\in\names$,
	\item $\bn{P_i}\cap\n{\sigma_j}=\emptyset$ for all $i\in{}\set{1,\ldots,n}$ and $j\in\set{1,\dots,m}$
\end{enumerate}
holds.
\end{conv}

The first condition ensures that the bound names among the processes are different, the second states that the bound names of a process and its free names and all the free names of the other processes are different, the third one ensures that the bound names differ within a process, and the last one states that the bound names of a process are different from the names of the substitutions under consideration.

Hence, any time we receive a process which violates the convention, we silently rename the bound names so that we receive a proper process.


\subsection{Structural congruence and extended standard form}
\label{sec_pi_struct_cong}
% mainfile: ../../Refinement.tex
The $\alpha$-convertibility relation identifies all processes $P$ and $Q$ where $P$ can be obtained from $Q$ by a renaming of bound names. We now extend this relation, to gain the so-called \findex{structural congruence}.

% mainfile: ../Refinement.tex
\begin{figure}[h!]
\begin{align}
		P&\struc{}Q,\text{ if }P\alphaeq{}Q\tag{ALPHA}\label{sc_alpha}\\\notag\\
		\procchoice{M}{\proczero}&\struc{}M \tag{SUM-NEU}\label{sc_sum-neu} \\
		\procchoice{M_1}{M_2}&\struc{}\procchoice{M_2}{M_1}\tag{SUM-COM}\label{sc_sum-com}\\
		\procchoice{M_1}{\left(\procchoice{M_2}{M_3}\right)}&\struct{}\procchoice{\left(\procchoice{M_1}{M_2}\right)}{M_3}\tag{SUM-ASS\label{sc_sum-ass}}\\\notag\\
		\procpar{P}{\proczero}&\struc{}P \tag{PAR-NEU} \label{sc_par-neu}\\
		\procpar{P_1}{P_2}&\struc{}\procpar{P_2}{P_1}\tag{PAR-COM}\label{sc_par-com}\\
		\procpar{P_1}{\left(\procpar{P_2}{P_3}\right)}&\struct{}\procpar{\left(\procpar{P_1}{P_2}\right)}{P_3}\tag{PAR-ASS}\label{sc_par-ass}\\\notag\\
		\procres{a}{\proczero}&\struc{}\proczero \tag{RES-ABS} \label{sc_res-abs}\\
		\procres[()]{a}{\procres{b}{P}}&\struc{}\procres[()]{b}{\procres{a}{P}}\tag{RES-TRA}\label{sc_res-tra}\\
		\procres[()]{a}{\pi.P}&\struct{}\pi.\left(\procres{a}{P}\right),\text{ if }a\nin\fn{\pi}\tag{RES-PRE}\label{sc_res-pre}\\
		\procres[()]{a}{\procchoice{M_1}{M_2}}&\struct{}\procchoice{M_1}{\left(\procres{a}{M_2}\right)},\text{ if }a\nin\fn{M_1}\tag{RES-SUM}\label{sc_res-sum}\\
		\procres[()]{a}{\procpar{P_1}{P_2}}&\struct{}\procpar{P_1}{\left(\procres{a}{P_2}\right)},\text{ if }a\nin\fn{P_1}\tag{RES-PAR}\label{sc_res-par}%\\\notag\\
		%\proccall{A}{\vec{v}}\struct{}P&\subs{\vec{v}}{\vec{w}}, \text{ if }\procdef{A}{\vec{w}}\define{}P\label{sc_call}\tag{CALL}	
\end{align}
\caption{Axioms of the structural congruence on processes.}
\label{fig_def_struct_cong}
\end{figure}


\begin{definition}[Structural congruence]
\label{def_struct_cong}
The smallest congruence, denoted by $\struct{}\subseteq\procs\times\procs$, which satisfies the axioms of \refFig{fig_def_struct_cong}, is called \findex{structural congruence}.
\end{definition}

The (\ref{sc_res-par}) axiom is normally known as \findex{scope extrusion}. With the added (\ref{sc_res-pre}) and (\ref{sc_res-sum}) axioms, we now have the ability to extrude the scope in two more ways. Besides, note that within these three rules the side condition that the name $a$ is not allowed to appear free in the other part, is by \refConv{conv_uni_bn} the same as the name $a$ is not allowed to appear in the other part at all.

Milner's well-known \findex{standard form} \cite{milner} expands the scope of a restriction as much as possible. We added in \refFig{fig_def_struct_cong} the (\ref{sc_res-pre}) and (\ref{sc_res-sum}) axioms to Milner's structural congruence so that it is now possible to move \emph{all} restrictions right at the front of a process.

\begin{definition}[Extended standard form]
\label{def_extended_standard_form}
A process $P\in\procs$ with
\[P\define\procres[()]{\vec{z}}{\procpar{P_1}{\procpar{\ldots}{P_n}}}\]
and $\vec{z}\subseteq\names$ and $n\in\N$ is said to be in \index{standard form!extended}\findex{extended standard form}, if $P_1,\ldots,P_n\in\procsresf$. If $\vec{z}$ is an empty sequence, we omit the restriction operator and if $n=0$ then the form of the process is $P=\procres{\vec{z}}{\proczero}$. The set of all processes in extended standard form is denoted by $\procsesf$.
\end{definition}

Thus, if we consider a process $P\define\procchoice{\inp{a}{x}.\out{x}{a}}{\inp{z}{x}.\procres{y}{\out{x}{y}}}$, we know that referring to Milner, $P$ is in standard form. But with the (\ref{sc_res-pre}) and (\ref{sc_res-sum}) axioms we can get rid of the restriction operator inside the process and move it to the front. Hence, $P\struc\procchoice{\inp{a}{x}.\out{x}{a}}{\procres[()]{y}{\inp{z}{x}.\out{x}{y}}}\struc\procres[()]{y}{\procchoice{\inp{a}{x}.\out{x}{a}}{\inp{z}{x}.\out{x}{y}}}$. Thereby, we see the necessity of the slightly adapted version of the guarded choice within the syntax of the \picalc{}. For the intermediate step it is necessary that the parts of a sum can also start with a restriction and not only with a prefix. In general, the intermediate steps during a transformation of a process in extended standard form is the only reason for extending the syntax.

Moreover, it is possible to transform every process with the axioms of \refFig{fig_def_struct_cong} to a structurally congruent process in extended standard form.

\begin{lemma}[Extended standard form]
\label{lem_extended_standard_form}
Every process $P\in\procs$ can be transformed in a process $P'\in\procsesf$, with $\struct{P}{P'}$.
\end{lemma}
\begin{prf}
Let $P\in\procs$. We proceed by induction over the structure of processes.
\begin{description}
\item[Base case $P=\proczero$:] By the (\ref{sc_res-abs}) axiom of \refFig{fig_def_struct_cong}, we know $\struct{P}{\procres[()]{a}{\proczero}}$ holds for an arbitrary $a\in\names$. And with the definition of the extended standard form $\procres[()]{a}{\proczero}\in\procsesf$ holds.%We know that $P\in\procsesf$ holds, because with $n=0$ and $\vec{z}$ the empty sequence, $P=\procres[()]{\vec{z}}{\procpar{P_1}{\procpar{\ldots}{P_n}}}$ holds. Since the structural congruence is reflexive, we also know that $P\struct{}P$ holds

\item[Base case $P=\proccall{A}{\vec{v}}$:] With $\vec{z}$ the empty sequence and since then the restriction operator is omitted according to \refDef{def_extended_standard_form}, we know $P=\procres{\vec{z}}{\proccall{A}{\vec{v}}}$. Hence, since the structural congruence is reflexive $\struct{P}{P}$ and since $\proccall{A}{\vec{v}}\in\procsresf$, we know $P\in\procsesf$.

\item[Induction hypothesis:] For all structurally simpler $Q,R\in\procs$ there exist processes $Q',R'\in\procsesf$ with $Q\struct{}Q'$ and $R\struct{}R'$.

\item[Induction step:] Let $Q,R\in\procs$.
	\begin{description}		
\item[Case $P=\procpar{Q}{R}$:] From the induction hypothesis we know that there are processes $Q',R'\in\procsesf$ with $Q\struct{}Q'$ and $R\struct{}R'$. So there are sequences $\vec{z_1},\vec{z_2}\subseteq\names$ and processes $Q_1,\ldots,Q_n,R_1,\ldots,R_m\in\procsresf$ with $n,m\in\N$, such that $Q'\define\procres[()]{\vec{z_1}}{\procpar{Q_1}{\procpar{\ldots}{Q_n}}}$ and $R'\define\procres[()]{\vec{z_2}}{\procpar{R_1}{\procpar{\ldots}{R_m}}}$ holds. Since there is just a finite number of free names in a collection of processes, we can choose names $\vec{z_1'},\vec{z_2'}\subseteq{}\names$, with $\card{\vec{z_1'}}=\card{\vec{z_1}}$ and $\card{\vec{z_2'}}=\card{\vec{z_2}}$ such that for all names $x\in\vec{z_1'}$ it holds that
\[x\nin\left(\fn{\procpar{Q_1}{\procpar{\ldots}{Q_n}}}\cup\fn{\procpar{R_1}{\procpar{\ldots}{R_m}}}\cup{}z_2'\right)\]
and similarly $x'\nin\left(\fn{\procpar{Q_1}{\procpar{\ldots}{Q_n}}}\cup\fn{\procpar{R_1}{\procpar{\ldots}{R_m}}}\cup{}z_1'\right)$ holds for all names $x'\in\vec{z_2'}$. Since the structural congruence is preserved under $\alpha$-conversion, we know that $Q'\struct{}\procres[()]{\vec{z_1'}}{\left(\procpar{Q_1}{\procpar{\ldots}{Q_n}}\right)\subs{\vec{z_1'}}{\vec{z_1}}}$ and $R'\struct{}\procres[()]{\vec{z_2'}}{\left(\procpar{R_1}{\procpar{\ldots}{R_m}}\right)\subs{\vec{z_2'}}{\vec{z_2}}}$ holds. From the definition of the substitution on processes, we know that there are processes $Q_1',\ldots,Q_n'\in\procsresf$ and $R_1',\ldots,R_m'\in\procsresf$ with $Q_i'\define{}Q_i\subs{\vec{z_1'}}{\vec{z_1}}$ for all $i\in\set{1,\ldots,n}$ and $R_j'\define{}R_j\subs{\vec{z_2'}}{\vec{z_2}}$ for all $j\in\set{1,\ldots,m}$. Hence, there are processes $Q'',R''\in\procsesf$ with $Q''\define{}\procres[()]{\vec{z_1'}}{\procpar{Q_1'}{\procpar{\ldots}{Q_n'}}}$ and $R''\define{}\procres[()]{\vec{z_2'}}{\procpar{R_1'}{\procpar{\ldots}{R_m'}}}$. Since the structural congruence is transitive, we know that $Q\struct{}Q''$ and $R\struct{}R''$ holds. Furthermore, we know with the (\ref{congparr}) implication of \refDef{def_cong_rel} that $\procpar{Q}{R}\struct{}\procpar{Q''}{R}$ and with (\ref{congparl}) that $\procpar{Q''}{R}\struct{}\procpar{Q''}{R''}$ holds. Hence, with the transitivity of the structural congruence $\procpar{Q}{R}\struct{}\procpar{Q''}{R''}$ holds. Thus, consider $\procpar{Q''}{R''}=\procpar{\procres[()]{\vec{z_1'}}{\procpar{Q_1'}{\procpar{\ldots}{Q_n'}}}}{\procres[()]{\vec{z_2'}}{\procpar{R_1'}{\procpar{\ldots}{R_m'}}}}$. Since there is no $x\in\vec{z_2'}$ with $x\in\fn{Q''}$, we know from the (\ref{sc_res-par}) axiom of \refFig{fig_def_struct_cong} that $\procpar{Q''}{R''}\struct{}\procres[()]{\vec{z_2'}}{\procpar{Q''}{\left(\procpar{R_1'}{\procpar{\ldots}{R_m'}}\right)}}$. Moreover, with (\ref{sc_par-com}) and (\ref{sc_res-par}), we also know $\procpar{Q''}{R''}\struct{}\procres[()]{\vec{z_1'}}{\procres[()]{\vec{z_2'}}{\procpar{\left(\procpar{Q_1'}{\procpar{\ldots}{Q_n'}}\right)}{\left(\procpar{R_1'}{\procpar{\ldots}{R_m'}}\right)}}}$, since the definition of the congruence relation yields that the axioms are also applicable in the context of a restriction. Hence, we found a process $P'\in\procsesf$ with $P'\define{}\procres[()]{\vec{z_3}}{\procpar{\procpar{Q_1'}{\procpar{\ldots}{Q_n'}}}{\procpar{R_1'}{\procpar{\ldots}{R_m'}}}}$ with $\vec{z_3}=\vec{z_1'}\cup\vec{z_2'}$ and $P\struct{}P'$.
	
\item[Case $P=\pi.Q$:] Again the induction hypothesis yields that there are names $z\subseteq\names$ and processes $Q_1,\ldots,Q_n\in\procsresf$, with $n\in\N$, such that $Q\struct{}\procres[()]{\vec{z}}{\procpar{Q_1}{\procpar{\ldots}{Q_n}}}$. Additionally, we can find fresh names $\vec{z'}\subseteq\names$, with $\card{\vec{z'}}=\card{\vec{z}}$ such that for all $x\in{}\vec{z'}$, we know that $x\nin\fn{\pi}$ and $x\nin\fn{\procpar{Q_1}{\procpar{\ldots}{Q_n}}}$ holds. And with the (\ref{sc_alpha}) axiom of \refFig{fig_def_struct_cong} we know $Q\struct{}\procres[()]{\vec{z'}}{\left(\procpar{Q_1}{\procpar{\ldots}{Q_n}}\right)\subs{\vec{z'}}{\vec{z}}}$ holds. Thus, with $Q_i'\define{}Q_i\subs{\vec{z'}}{\vec{z}}$ for all $i\in\set{1,\ldots,n}$, we know $Q\struct{}\procres[()]{\vec{z'}}{\procpar{Q_1'}{\procpar{\ldots}{Q_n'}}}$ and so with the (\ref{congsum1l}) implication of \refDef{def_cong_rel}, we know $P=\struct{\pi.Q}{P'}$, with $P'\define\pi.\procres[()]{\vec{z'}}{\procpar{Q_1'}{\procpar{\ldots}{Q_n'}}}$. Furthermore, with the (\ref{sc_res-pre}) axiom of \refFig{fig_def_struct_cong} we know $
\struct{P'}{P''}$, with $P''\define\procres[()]{\vec{z'}}{\pi.\left(\procpar{Q_1'}{\procpar{\ldots}{Q_n'}}\right)}$. Hence, by transitivity of the structural congruence, there is a process $P''\in\procsesf$ with $P\struct{}P''$.

\item[Case $P=\procres{a}{Q}$:] Since there is a $Q'\in\procsesf$ with $\struct{Q}{Q'}$ due to the induction hypothesis, we know with \ref{congres} of \refDef{def_cong_rel} that $\struct{P}{\procres{a}{Q'}}$ holds. Furthermore, we know $\procres{a}{Q'}$ itself is in extended standard form, since $Q'\in\procsesf$ holds.

\item[Case $P=\procchoice{Q}{R}$:] The induction hypothesis yields that there are processes $Q',R'\in\procsesf$ with $\struct{Q}{Q'}$ and $\struct{R}{R'}$ and so there exists $\vec{z_1},\vec{z_2}\subseteq\names$ and $Q_1,\ldots,Q_n,R_1,\ldots,R_m\in\procsresf$ with $m,n\in\N$ such that $Q'=\procres[()]{\vec{z_1}}{\procpar{Q_1}{\procpar{\ldots}{Q_n}}}$ and $R'=\procres[()]{\vec{z_2}}{\procpar{R_1}{\procpar{\ldots}{R_m}}}$. From the definition of the syntax of the \picalc{} processes we know $Q,R\in\sums$, otherwise $P$ would not be a process. Hence, $Q$ and $R$ are each either an inaction, an input process, a choice process, or a restricted sum. Furthermore, we know that for all $M\in\sums$ and $P\in\procs$ that $\struct{M}{P}$ implies $P\in\sums$, since the only rules of the structural congruence, which influence the structure of a process significantly to possibly destroy the sum structure, are the (\ref{sc_res-pre}), (\ref{sc_res-sum}) and (\ref{sc_res-par}) axioms of \refFig{fig_def_struct_cong}. And since the (\ref{sc_res-pre}) and (\ref{sc_res-sum}) axioms preserve the sum structure and (\ref{sc_res-par}) is not applicable for sums, we know that $Q',R'\in\sums$. Thus, there are $M_1,M_2\in\sums\cap\procsresf$ such that $Q'=\procres{\vec{z_1}}{M_1}$ and $R'=\procres{\vec{z_2}}{M_2}$.

Analogously to the case of the parallel composition, we know that we can use $\alpha$-conversion to ensure that all bound names are different from each other and from all the other names and preserve the structural congruence. Thus, there are fresh names $\vec{z_1'},\vec{z_2'}\subseteq\names$ which do not appear free in any process under consideration with $Q'\struc{}\procres{\vec{z_1'}}{M_1'}$ and $R'\struc{}\procres{\vec{z_2'}}{M_2'}$ for some $M_1',M_2'\in\sums$ with $M_1'\define{}M_1\subs{\vec{z_1'}}{\vec{z_1}}$ and $M_2'\define{}M_2\subs{\vec{z_2'}}{\vec{z_2}}$. The transitivity of the structural congruence yields that $\struct{Q}{\procres{\vec{z_1'}}{M_1'}}$ and $\struct{R}{\procres{\vec{z_2'}}{M_2'}}$ holds. Since $\procres{\vec{z_1'}}{M_1'},\procres{\vec{z_2'}}{M_2'}\in\sums$ and since $\struc$ is a congruence and accordingly is preserved for every context, we know that $P\struct{}\procchoice{\procres[()]{\vec{z_1'}}{M_1'}}{\procres[()]{\vec{z_2'}}{M_2'}}$. Thus, with the (\ref{sc_res-sum}) axiom of \refFig{fig_def_struct_cong} we know $\struct{P}{\procres[()]{\vec{z_2'}}{\procchoice{\procres[()]{\vec{z_1'}}{M_1'}}{M_2}}}$ holds. The (\ref{sc_sum-com}) and the (\ref{sc_res-sum}) axiom of \refFig{fig_def_struct_cong} and the (\ref{congsum2l}) implication of \refDef{def_cong_rel} yields that $\struct{P}{P'}$ with $P'\define{}\procres[()]{\vec{z_2'}}{\procres[()]{\vec{z_1'}}{\procchoice{M_2'}{M_1'}}}$ holds and so $P'\in\procsesf$.
	\end{description}
\end{description}
\end{prf}

Thus, we know that we are able to transform every process in a structurally congruent process which has all the restrictions right at the front of the process.


\subsection{Operational semantics}
\label{sec_pi_op_sem}
% mainfile: ../../Refinement.tex
We already explained the intuition of the semantics of the \picalc{} in \refSec{sec_pi_syntax}. In this section we formalize this intuition by presenting a definition of an \index{semantics!operational}\findex{operational semantics} of the \picalc{}. This definition yields a labeled transition system called the \index{transition system!early}\findex{early transition system} as presented in \cite{sangiorgi} in Table 1.5 on page 38. For this thesis, we applied only minor changes to the original presentation. %one presented in \cite{sangiorgi} for the fitting in this thesis.
In particular, we replace the replication rules by the \ecall{} rule, which is also defined in \cite{sangiorgi}, and omit the matching rule due to the definition of our syntax. Furthermore, the labels of the transitions are adapted to the notion of this thesis.

\subsubsection{Definition}
For the labeling of the transitions, we define the set of all \index{action!output}\findex[output!action]{output actions} as $\outA\define\set[x,y\in\names]{\out{x}{y}}$, the set of all \index{action!input}\findex[input!action]{input actions} as $\inA\define\set[x,y\in\names]{\inpa{x}{y}}$ and the set of all \index{action!bound output}\findex[bound output action]{bound output actions} as $\boutA\define\set[\exists{}x,y\in\names:x\neq{}y]{\bout{x}{y}}$. The meaning of the bound output will be explained shortly. Thus, the set of all \index{action}\findex[action]{actions} is defined as $\actions\define\inA\cup\outA\cup\boutA\cup\set{\tau}$ with $\alpha,\beta,\ldots\in\actions$ as its standard representatives.

Furthermore, we expand the definition of the \index{name!free}\findex[free name!action]{free} and \index{name!bound}\findex[bound name!action]{bound names} of processes in \refFig{fig_names_act} such that they are also applicable to actions. Additionally, \refFig{fig_names_act} shows the application of a \findex[substitution!action]{substitution on actions} and the \findex{conjugation}.

%%%%%%%%%%%%%%%%%%%%%%%%%%%%%%%%%%%%%%%%%%%%%%%%%%%%%%%%%%% START SOME OLD DEFINITIONS %%%%%%%%%%%%%%%%%%%%%%%%%%%%%%%%%%%%%%%%%%%%%%%%%%%%%%%%%%%%%%%%
\begin{old}{some old definitions}
\todo{and is based on an early transition system Sangiorgi defined in his phd thesis in $1993$ \cite{sangiorgi_phd}}
\todo{to the best of our knowledge developed by Sangiorgi in its phd thesis, zitat paper das Milner das auch schon machen wollte und sein subvisor war}
\begin{description}
\item[Labels:] $\lambda, \mu, \ldots \in \labels = \set[x,y\in\names]{\inp{x}{y}} \cup \set[x,y\in\names]{\out{x}{y}}$
\item[Actions:] $\alpha, \beta, \ldots \in \actions = \labels \cup \set{\tau}$
\item[Syntax:] $P \syntdef \procsum \ebnf \procpar{P_1}{P_2} \ebnf \procres{x}{P} \ebnf \proccall{A}{\parl{w}}$ with $I$ finite indexset.
\item[Fragments:] $F\syntdef\procsum \ebnf \proccall{A}{\parl{w}} \ebnf \procres[e]{a}{\procpar{F_1}{\procpar{\ldots}{F_n}}}$ with $a\in fn(F_i)$ f.a. $i\in\set{1,\ldots,n}$, $I$ finite indexset, $\mid I\mid\neq0$.
\item[Restricted form:] $P_{\text{rf}}\syntdef \Pi_{i\in J} F_i$ with $J$ finite indexset and $F_i$ fragment f.a. $i\in J$.
\end{description}
\end{old}
%%%%%%%%%%%%%%%%%%%%%%%%%%%%%%%%%%%%%%%%%%%%%%%%%%%%%%%%%%% END SOME OLD DEFINITIONS %%%%%%%%%%%%%%%%%%%%%%%%%%%%%%%%%%%%%%%%%%%%%%%%%%%%%%%%%%%%%%%%

\begin{figure}[!h]
\centering
\begin{tabular}{c|l|c|c|c|c|c}%|c|c|c}
$\alpha$      & denotation   & $\n{\alpha}$ & $\bn{\alpha}$ & $\fn{\alpha}$ & $\substF(\alpha)$ & $\conj{\alpha}$\\\hline\hline%& $\alpha\subs{y}{a}$ & $\alpha\subs{y}{x}$ & $\alpha\subs{y}{z}$\\\hline\hline
$\tau$        & internal     & $\emptyset$  & $\emptyset$   & $\emptyset$   & $\tau$ & $\tau$\\%& $\tau$              & $\tau$        & $\tau$ \\
$\inpa{a}{x}$  & input        & $\set{a,x}$  & $\emptyset$   & $\set{a,x}$   & $\inpa{\substF(a)}{\substF(x)}$ & $\outa{a}{x}$\\%& $\inp{y}{x}$        & $\inp{a}{x}$  & $\inp{a}{x}$ \\
$\out{a}{x}$  & output       & $\set{a,x}$  & $\emptyset$   & $\set{a,x}$   & $\out{\substF(a)}{\substF(x)}$ & $\inpa{a}{x}$\\%& $\out{y}{x}$        & $\out{a}{y}$  & $\out{a}{x}$ \\
$\bout{a}{x}$ & bound output & $\set{a,x}$  & $\set{x}$     & $\set{a}$     & $\bout{\substF(a)}{x}$ & $\inpa{a}{x}$ %& $\bout{y}{x}$       & $\bout{a}{x}$ & $\bout{a}{x}$
\end{tabular}
\caption{Free and bound names of actions.}
\label{fig_names_act}
\end{figure}

It seems to be a bit unintuitive that the set of bound names of an input action $\inpa{a}{x}$ is empty, since in the view of processes the name $x$ is bound. But with the intuition that the actions present the behavior a process has performed, we see that the name $x$ has already been sent over the channel $a$ and is not a placeholder any longer. This fits to the definition of the operational semantics in \refFig{fig_ts_early}, where we see that a bound name in an input prefix is instantiated directly when the input transition is inferred. This principle is called an \index{instantiation!early}\findex{early instantiation}.

To abstract from the replacement of bound names, we define the transition relation on the equivalence classes modulo $\alpha$-converti"-bility of processes with the actions as its labels.

%%%%%%%%%%%%%%%%%%%%%%%%%%%%%%%%%%%%%%%%%%%%%%%%%%%%%%%%%%% START OLD DESCRIPTION TEXT %%%%%%%%%%%%%%%%%%%%%%%%%%%%%%%%%%%%%%%%%%%%%%%%%%%%%%%%%%%%%%%%
\begin{old}{some old description text}
In this section we introduce an operational semantics for \picalc{} processes. The desired manner of this semantics\todo{not good, is backward description.} is, that the behavior of a process -- meaning the internal as well as the external behavior -- is completely represented. With attention to the denotational semantics presented in \refChap{sec_de_semantics} it is also helpful that the semantics consists of a transition system which has all the processes behavior decoded in the labels of its transitions.

Below the \index{transition system!early}\findex{early transition system} for \picalc{} processes developed by Sangiorgi and Walker in \cite{sangiorgi} is defined and explained. Due to a better fitting in this thesis the notation of the early transition system is somewhat modified.

Primarily the rules for replication used in \cite{sangiorgi} are replaced by the rule for recursive calls, due to the fact that in this thesis no replication operator is defined and -- as mention in \cite{milner} -- replication and recursive calls are equivalent. Moreover the rule for matching is omitted by the same fact and obviously the labels of the transitions are adapted to the notion of this thesis. Furthermore in \cite{sangiorgi} processes are seen as equivalence classes modulo $\alpha$-conversion (compare \refDef{def_alpha_conv}), in this thesis we will -- in attention to \refChap{sec_reduc_semantics} -- explicitly differentiate between a process $P$ and its equivalence class $\ec{P}$.

All this leads to the semantically equivalent transition system presented in \refDef{def_early_trans_system}.
\end{old}
%%%%%%%%%%%%%%%%%%%%%%%%%%%%%%%%%%%%%%%%%%%%%%%%%%%%%%%%%%% END OLD DESCRIPTION TEXT %%%%%%%%%%%%%%%%%%%%%%%%%%%%%%%%%%%%%%%%%%%%%%%%%%%%%%%%%%%%%%%%

\begin{definition}
\label{def_early_trans_system}
The relation $\set[\alpha\in\actions]{\transs{\alpha}}\subseteq\procs_\alpha\times\procs_\alpha$ is called the \index{transition system!early}\findex{early transition system} and is defined by the rules in \refFig{fig_ts_early} apart from the omission of the rules \esumr{}, \eparr{}, \ecomr{} and \ecloser{} for a shorter presentation. They can be obtained from their related rules by interchanging the roles of $P$ and $Q$.
\end{definition}

% mainfile: ../../Refinement.tex
\begin{figure}[h!]
\begin{gather*}
\kalRule{E-TAU}{}{}{\ec{\tau.P} \transs{\tau} \ec{P}} \quad\quad \kalRule[\procdef{A}{\parl{w}}\define P]{E-CALL}{}{}{\ec{\proccall{A}{\parl{v}}} \tautrans \ec{P\subs{\parl{v}}{\parl{w}}}} \\\\
\kalRule{E-OUT}{}{}{\ec{\out{x}{y}.P} \transs{\out{x}{y}} \ec{P}} \quad\quad \kalRule{E-IN}{}{}{\ec{\inp{x}{z}.P} \intrans{x}{y} \ec{P\subs{y}{z}}} \\\\
\kalRule{E-SUM_L}{}{\ec{P} \transs{\alpha} \ec{P'}}{\ec{P+Q} \transs{\alpha} \ec{P'}} \quad\quad \kalRule[z\nin n(\alpha)]{E-RES}{}{\ec{P} \transs{\alpha} \ec{P'}}{\ec{\procres{z}{P}} \transs{\alpha} \ec{\procres{z}{P'}}} \\\\
\kalRule[\bn{\alpha}\cap\fn{Q}=\emptyset]{E-PAR_L}{}{\ec{P} \transs{\alpha} \ec{P'}}{\ec{\procpar{P}{Q}} \transs{\alpha} \ec{\procpar{P'}{Q}}} \\\\
\kalRule[z\neq x]{E-OPEN}{}{\ec{P} \outtrans{x}{z} \ec{P'}}{\ec{\procres{z}{P}} \bouttrans{x}{z} \ec{P'}} \quad\quad \kalRule{E-COM_L}{\ec{P} \outtrans{x}{y} \ec{P'}}{\ec{Q} \intrans{x}{y} \ec{Q'}}{\ec{\procpar{P}{Q}} \tautrans \ec{\procpar{P'}{Q'}}}\\\\
\kalRule[z\nin\fn{Q}]{E-CLOSE_L}{\ec{P} \bouttrans{x}{z} \ec{P'}}{\ec{Q} \intrans{x}{z} \ec{Q'}}{\ec{\procpar{P}{Q}} \tautrans \ec{\procres[a]{z}{\procpar{P'}{Q'}}}}
\end{gather*}
\caption{The \index{transition system!early}\findex{early transition system} \cite{sangiorgi}.}
\label{fig_ts_early}
\end{figure}



The labeled transition system fulfills two main tasks. On the one hand, it defines the activity within the process and on the other hand it exhibits the process' potential to interact with its environment.

Thereby, the internal behavior is represented by $\tau$ transitions, that is $\ec{P}\tautrans{}\ec{Q}$ expresses that $P$ and all processes $\alpha$-convertible to $P$ can invisibly evolve to $Q$ and to all processes which are $\alpha$-convertible to $Q$. For the external behavior there is a transition labeled with $\inpa{a}{x}$ for an input possibility, $\out{a}{x}$ for an output potential and the so-called bound output $\bout{a}{x}$, to handle the transfer of a name $x$ which is bound under the restriction. The intuition of a visible transition of a process is that we can compose another process with a corresponding visible transition in parallel and the composed processes can communicate with each other.

Note that we abbreviate multiple transitions like for example $\ec{\inp{a}{b}.\out{b}{c}}\intrans{a}{d}\ec{\out{d}{c}}$ and $\ec{\out{d}{c}}\outtrans{d}{c}\ec{\proczero}$ by $\ec{\inp{a}{b}.\out{b}{c}}\intrans{a}{d}\ec{\out{d}{c}}\outtrans{d}{c}\ec{\proczero}$.

%%%%%%%%%%%%%%%%%%%%%%%%%%%%%%%%%%%%%%%%%%%%%%%%%%%%%%%%%%% START OLD DESCRIPTION TEXT %%%%%%%%%%%%%%%%%%%%%%%%%%%%%%%%%%%%%%%%%%%%%%%%%%%%%%%%%%%%%%%%
\begin{old}{some old description text}
As mentioned, the hole transition system is build upon equivalence classes modulo $\alpha$-conversion. Thus, for a process $\inp{a}{z}.P$, with $P\in\procs$, the axiom $E-IN$ produces just because of the equivalence class, any number of transitions; one $\intrans{a}{x}$ for every channel $x\in\names\setminus\fn{P}$. In contrast to the late transition system -- also defined by Sangiorgi and Walker in \cite{sangiorgi} -- the missing transitions $\intrans{a}{y}$ for every channel $y\in\fn{P}$ are, as well as the former mentioned transitions, obtained by the rule and its substitution. Due to that, there is no need for a substitution in the conclusion and different names transmitted inside the premise of the \ecoml{} (respectively \ecomr{}) rule -- the standard rule for communication between two parallel processes --, because for every given output, there is, if possible, an associated deduction with the \ein{} rule.

Due to that, there is no need for a substitution in this rule. With this in mind, the substitution inside the $L-COM_L$ (respectively $L-COM_R$) rule, the standard rule for communication between two parallel processes, seems to be useless. It appears to be sufficient just allowing a communication between processes, where one can send a name over a channel and the other can receive the same name over the same channel and omitting the substitution. 

A main challenge while defining an operational semantics for \picalc{} processes is the treatment of bound names within a process. Four cases have to be considered. In the first case the bound name does not occur in the action of a transition. Then the restriction does not have any influence on the communication. Thus, the transition is just the same with or without restriction. This case is handled with the \eres{} rule.

If we consider $P\define\procres{a}{\inp{a}{x}.\proczero}$, we know the channel $a$ should locally be bound to the process $P$. Hence, there must not be a transition for the channel $a$, since a visible action stands for a possible communication of the process $P$ with its environment. Thus, for a communication of $P$ with its environment the channel $a$ must be known to the environment and so will not be local for $P$. Hence, no rule exists for this situation and with the side conditions $z\not\in\n{\alpha}$ in \eres{} and $z\neq x$ in \eopen{} this situation is excluded from the other rules.

\todo{It is not necessary to consider the case in which the restriced ... indepently, because ....}The case in which the restricted name is part of the input action -- for example $P\define\procres{a}{\inp{x}{a}.\proczero}$ -- is not necessary to be considered independently, because $P\in\ec{\procres{b}{\inp{x}{a}.\proczero}}$, hence every behavior is already regarded with the \eres{} rule. Otherwise it is either way not meaningful to define such a process in that way, because the ``$\procres{}{}$'' has no influence on the behavior of $P$, since $P\equiv\inp{x}{a}$. The focus on equivalence classes modulo $\alpha$-conversion instead of structural congruence makes it, however, necessary to consider it separately.

The most complicated case, which handles the transmission of bound names, is solved with the \eopen{} and \eclosel{} (respectively \ecloser{}) rule. For the external behavior as well as for the premise of the \eclosel{} rule, the \eopen{} rule is invented. There, the possibility for a hand over of a bound name is given. To distinguish between a normal and a bound output the parenthesis of the action are change angle bracket\todo{????}. It is necessary to mention that the conclusion omits the $\procres{}{}$-operator after the transition -- giving the rule the name ``open'' --, because without it, there could be no transition after a bound output to describe the full external behavior of a process. For example, consider the process $P\define \procres[klammer]{z}{\out{x}{z}.\inp{z}{w}.\proczero}$. Without the omission  of the ``$\procres{}{}$'' the external behavior of $P$ looks like

\[\ec{P}\bouttrans{x}{z}\ec{\procres[a]{z}{\inp{z}{w}.\proczero}}\nxrightarrow{}.\]

Thus, there is a deadlock after the bound output action $\bout{x}{z}$, but with a suitable parallel process, for example $Q\define\inp{x}{y}.\out{y}{a}.\proczero$, there should be more behavior possible: With the \eclosel{} and then \ecomr{} rule as last deductions, it holds

\[\ec{\procpar{P}{Q}}\tautrans\ec{\procres[a]{z}{\procpar{\inp{z}{w}.\proczero}{\out{z}{a}.\proczero}}}\tautrans\ec{\procres[a]{z}{\procpar{\proczero}{\proczero}}}.\]

On the way to reach this deductions, it is necessary that $P$ can perform $\ec{P}\bouttrans{x}{z}\ec{P'}\intrans{z}{a}\ec{\proczero}$, which is possible by omitting the ``$\procres{}{}$'' so that\todo{was will ich hier} $P'\define\procres[a]{z}{\inp{z}{w}.\proczero}$. Hence, $P$ can fully communicate with $\ec{Q}\intrans{x}{z}\ec{\out{z}{a}.\proczero}\outtrans{z}{a}\ec{\proczero}$.

The omission does not construct too much behavior, because the external behavior of a process $P$ means that an other process has to communicate with $P$. Hence, for a bound output like $\bout{x}{z}$, there has to be an input like $\inp{x}{z}$ and the only rule which handles this is the \eclosel{} (respectively \ecloser{}) rule, and this ``closes'' again in its conclusion the parallel process under the $\procres$-operator. Thus, another perception of this behavior is that those open and close rules handle a kind of scope expansion.

The side condition of the \eclosel{} (respectively \ecloser{}) rule is necessary because considering $P\define\procres[a]{z}{\out{x}{z}.\proczero}$ and $Q\define\inp{x}{a}.\out{z}{a}.\proczero$ then $\ec{P}\bouttrans{x}{z}\ec{\proczero}$ and $\ec{Q}\intrans{x}{z}\ec{\out{z}{z}.\proczero}$. Hence, without the $z\not\in\fn{Q}$ side condition $\ec{\procpar{P}{Q}}\tautrans\ec{\procres[a]{z}{\procpar{\proczero}{\out{z}{z}.\proczero}}}$ holds. This should not be allowed, because after this communication, the former unbound output channel $z$ is now bound; which is not the desired manner. Another perception is, that with $\procpar{P}{Q}=\procpar{\procres[a]{z}{\out{x}{z}.\proczero}}{\inp{x}{a}.\out{z}{a}.\proczero}$ there is no way to expand the scope of $z$ because $z\in\fn{Q}$, thus, a communication must not take place.

As opposed to the \esuml{} (respectively \esumr{}) rule, which just forwards the single transitions of a process in an alternative, the \eparl{} (respectively \eparr{}) rule needs a condition to prevent unwanted behavior. For example, consider two processes $P\define\procres[klammer]{a}{\out{w}{a}.\inp{a}{w}}$ and $Q\define\out{a}{y}$. With \eout{} holds $\ec{\out{w}{a}.\inp{a}{w}}\outtrans{w}{a}\ec{\inp{a}{w}}$ and using this with \eopen{}, then $\ec{P}\bouttrans{w}{a}\ec{\inp{a}{w}}$ holds. Hence, $\ec{R}\bouttrans{w}{a}\ec{\inp{a}{w}}$ holds with $R\define\procres[klammer]{b}{\out{w}{b}.\inp{b}{w}}$, because $R$ is formed out of $P$ by just a change of bound names. Thus, $R\in\ec{P}$. Hence would there be no condition in \eparl{} then $\ec{\procpar{R}{Q}}=\ec{\procpar{\procres[klammer]{b}{\out{w}{b}.\inp{b}{w}}}{\out{a}{y}}}\bouttrans{w}{a}\ec{\procpar{\inp{a}{w}}{\out{a}{y}}}$. So the resulting process could communicate over the former just for $P$ known channel $a$. This undesired behavior is prevented by the condition $\bn{\bout{w}{a}}\cap\fn{Q}=\set{a}\neq\emptyset$ of the \eparl{} rule.

The rule \ecall{} handles the function call in the \picalc{} by a $\tau$-transition, so every function call ``costs'' one $\tau$-transition. This may seem a bit constructed for the modeling of problems outside of the area of computer science, but in the topic of programming it is more meaningful that a computer takes an intern step to work off a function call.
\end{old}
%%%%%%%%%%%%%%%%%%%%%%%%%%%%%%%%%%%%%%%%%%%%%%%%%%%%%%%%%%% END OLD DESCRIPTION TEXT %%%%%%%%%%%%%%%%%%%%%%%%%%%%%%%%%%%%%%%%%%%%%%%%%%%%%%%%%%%%%%%%

A main challenge while defining an operational semantics for \picalc{} processes is the treatment of bound names within a process and especially the treatment of the transmission of bound names. Therefore, we have to take a closer look upon the \eres{}, \eparl{} (respectively \eparr{}), \eopen{} and \eclosel{} (respectively \ecloser{}) rules. The other rules straightforwardly fit to the intuition described in \refSec{sec_pi_syntax}.

Thus, with the \esuml{} (respectively \esumr{}) rule an alternative either evolves to the process arisen from the left or from the right part of the sum, whereby the other part vanishes. Besides, there are only the \etau{}, \eout{} and \ein{} axioms to handle a prefix in the described manner. Note that there is no condition within the \ein{} rule for the reception of a name. Thus, \textit{any} name can be send to an input process. Due to that, it is enough to consider the case where the object of the input is equal to the object of the output transition within the premise of the \ecoml{} (respectively \ecomr{}) rule for a communication between two processes. For instance, consider $P\define\out{a}{b}$ and $Q\define\inp{a}{x}.\out{x}{x}$. Since $\ec{P}\outtrans{a}{b}\ec{\proczero}$ and there is also a transition $\intrans{a}{b}$ such that $\ec{Q}\intrans{a}{b}\ec{\out{b}{b}}$, the \ecoml{} rule infers $\ec{\procpar{\out{a}{b}}{\inp{a}{x}.\out{x}{x}}}\tautrans{}\ec{\procpar{\proczero}{\out{b}{b}}}$. Furthermore, note that in \refFig{fig_ts_early} there is no inference rule for the inaction. Hence, an inaction has no transition and thus, no behavior.

By investigating the \ecall{} rule, we notice that a function call is resolved by a $\tau$ transition, so every function call ``costs'' one transition. This may seem a bit constructed for the modeling of problems outside the area of computer science, but in the topic of programming it is sensible that a computer takes an internal step to compute a function call.

From the \eres{} rule we know that a restriction does not delimit the behavior of a process as long as the bound channel is not used to compute the behavior. With the side condition $z\nin\n{\alpha}$ within the \eres{} rule, we prevent a private channel to be visible for the environment. Consider, for example, $P\define{}\procres[()]
{a}{\procpar{\inp{a}{x}.\out{x}{c}}{\out{a}{b}.\inp{b}{d}}}$. The bound channel $a$ can only be used to communicate within its scope -- for instance $\ec{P}\tautrans{}\ec{\procres[()]{a}{\procpar{\out{b}{c}}{\inp{b}{d}}}}$ -- but there is no visible transition starting in $\ec{P}$. Otherwise, each of such transitions would uncover the bound channel to the environment.

By \refConv{conv_uni_bn} we know that we do not have to consider the case that the bound name is the object of an input prefix. But note that nevertheless for $P\define\procres[()]{a}{\inp{x}{y}.\out{y}{a}}$ there is a transition labeled with $\inpa{x}{a}$ inferred by the \eres{} rule and starting in $\ec{P}$, since, for example, $\procres[()]{b}{\inp{x}{y}.\out{y}{b}}\in\ec{P}$ holds. But we see that the name used as object of the transition's label can in general not be the bound name of the resulting process: $\ec{P}\intrans{x}{a}\ec{\procres[]{c}{\out{a}{c}}}$, but $\procres[]{a}{\out{a}{a}}\nin\ec{\procres[]{c}{\out{a}{c}}}$.

Intuitively, the open and close rules resemble the structural law of scope extrusion for parallel processes. Consider, for example, $P\define\procres[]{a}{\out{x}{a}.\inp{a}{y}}$ and $Q\define\inp{x}{b}.\out{b}{x}$. We know from \refFig{fig_def_struct_cong} that $\procpar{\procres[()]{a}{\out{x}{a}.\inp{a}{y}}}{Q}\struc\procres[()]{a}{\procpar{\out{x}{a}.\inp{a}{y}}{Q}}$ holds, since $a\nin\fn{Q}$. With the \ein{}, \eout{}, \ecoml{} and \eres{} rules we know $\ec{\procres[()]{a}{\procpar{\out{x}{a}.\inp{a}{y}}{Q}}}\tautrans\ec{\procres[()]{a}{\procpar{\inp{a}{y}}{\out{a}{x}}}}$ and we reach the same for $\ec{\procpar{P}{Q}}$ by using the \eout{}, \eopen{}, \ein{} and \eclosel{} rules, since $a\nin\fn{Q}$. Thus, $\ec{\procpar{P}{Q}}\tautrans\ec{\procres[()]{a}{\procpar{\inp{a}{y}}{\out{a}{x}}}}$. Consequently, we see on the one hand that the side condition within the \eclosel{} (respectively \ecloser{}) rule is essential to prevent communication between processes without possible scope extrusion and on the other hand that there is a need for a visible transition for a process sending a bound name. Concluding, we see the necessity of the omission of the restriction within the conclusion of the \eopen{} rule and the reintroduction within the \eclosel{} (respectively \ecloser{}) rule. Otherwise, no scope extrusion would be possible. But if we consider, for example, a similar handling of the restriction operator as in the \eres{} and \ecoml{} (respectively \ecomr{}) rules, $\ec{\procpar{P}{Q}}$ would reach $\ec{\procpar{\procres[()]{b}{\inp{b}{y}}}{\out{a}{x}}}$ which has a significant different behavior.

With the intuition that the visible transitions describe the communication possibilities for a parallel composed process, we gain another hint for the omission and reintroduction of the restriction. We already saw for the processes $P$ and $Q$ from above that $\ec{\procpar{P}{Q}}\tautrans\ec{\procres[()]{a}{\procpar{\inp{a}{y}}{\out{a}{x}}}}$ and so $\ec{\procpar{P}{Q}}\tautrans\ec{\procres[()]{a}{\procpar{\inp{a}{y}}{\out{a}{x}}}}\tautrans\ec{\procres[()]{a}{\procpar{\proczero}{\proczero}}}$ holds. Since $\ec{Q}\intrans{x}{a}\ec{\out{a}{x}}\outtrans{a}{x}\ec{\proczero}$, it fits that $\ec{P}\bouttrans{x}{a}\ec{\inp{a}{y}}\intrans{a}{x}\ec{\proczero}$ holds. By not omitting the restriction, the process would stuck in a deadlock ($\ec{\procres[]{a}{\inp{a}{y}}}$) after conducting the first transition. Furthermore, we know that we do not create unwanted behavior with this omission, since the only rules which handle a bound output are the \esuml{} (respectively \esumr{}), \eclosel{} (respectively \ecloser{}) and the \eparl{} (respectively \eparr{}) rules. Whereby, for the summation rules this policy does not have any relevance, since only one part will be chosen and the other is rendered void. Within the closing rules the resulting process is again restricted by the same bound name. Thus, this is just the scope extrusion and all unintended behavior is prevented by its side condition. Within the \eparl{} and the \eparr{} rule, which handle the interleaving behavior of both processes, the unwanted behavior is also excluded by the side condition. For example, consider $P\define\procres{a}{\out{b}{a}.\inp{a}{b}}$ and so $\ec{P}\bouttrans{b}{a}\ec{\inp{a}{b}}$. Without the side condition we could reach $\ec{\procpar{\inp{a}{b}}{\out{a}{b}}}$ with the bound output action $\bout{b}{a}$ from $\ec{\procpar{\procres[()]{d}{\out{b}{d}.\inp{d}{b}}}{\out{a}{b}}}$. Thus, we would identify a bound name with a free one.

\subsubsection{Names and substitution}
A closer look upon the \eopen{} rule yields that if there is a bound output transition, we can apply the \eopen{} rule for every process within the equivalence class. Hence, for every process constructed through a replacement of the restricted name of the process with a fresh one, there is also a bound transition with the new name bound in its label. This context is written down in \refLem{lem_bn_trans} which is proven by Sangiorgi and Walker in \cite{sangiorgi}.

\begin{lemma}[Bound transition \cite{sangiorgi}]
\label{lem_bn_trans}
Given names $a,b\in\names$ and processes $P,Q\in\procs$, then
\[\ec{P}\bouttrans{a}{b}\ec{Q} \text{ and } z\nin\fn{\procres{b}{P}} \text{ implies } \ec{P}\bouttrans{a}{z}\ec{Q\subs{z}{b}}\]
holds.
\end{lemma}

Since we now know that we can replace a bound name in an action by every name which is not free in the process the transition is starting from, we can extend \refConv{conv_uni_bn} for actions.

\begin{conv}[Uniqueness of bound names and transitions]
\label{conv_uni_bn_trans}
We extend \refConv{conv_uni_bn} to actions. Thus, we additionally stipulate that the bound names of some actions under consideration are different from the free names of the processes, the free names of the other actions and the names of the substitutions under consideration. Thereby, we need the limitation that for $\ec{P}\bouttrans{a}{z}\ec{Q}$ the name $z$, which is bound within the action and may be bound in $P$, is also allowed to be free in $Q$. Otherwise, scope extrusion would not be possible.
\end{conv}

We collect some more facts proved in \cite{sangiorgi}, which simplify the treatment of transitions and will be lifted to a chain of transitions in \refChap{sec_big-step_semantics}. We start by investing a connection between transitions and the free names of its processes.

\begin{lemma}[Transitions and free names \cite{sangiorgi}]
\label{lem_trans_fn}
Given names $a,b\in\names$, an action $\alpha\in\actions$ and processes $P,Q\in\procs$ with $\ec{P}\transs{\alpha}\ec{Q}$, then
\begin{align}
%\begin{array}{lcl}
\alpha=&\;\tau &\text{ implies } &&&\fn{Q}\subseteq\fn{P} \tag{TAU}\label{eq_trans_fn_tau}, \\
\alpha=&\;\inpa{a}{b}&\text{ implies } &&&a\in\fn{P} \text{ and } \fn{Q}\subseteq\fn{P}\cup\set{b} \tag{INP}\label{eq_trans_fn_inp},\\
\alpha=&\;\outa{a}{b} &\text{ implies } &&&a,b\in\fn{P} \text{ and } \fn{Q}\subseteq\fn{P} \tag{OUT}\label{eq_trans_fn_out},\\
\alpha=&\;\bouta{a}{b} &\text{ implies } &&&a\in\fn{P} \text{ and } \fn{Q}\subseteq\fn{P}\cup\set{b} \tag{BOUT}\label{eq_trans_fn_bout},
%\end{array}
\end{align}
holds.
\end{lemma}

Furthermore, we know that a transition between two processes implies that there is also a transition where an arbitrary substitution is applied to all of its components.

\begin{lemma}[Substitution on transitions (Part I) \cite{sangiorgi}]
\label{lem_subst_trans_partI}
Given processes $P,Q\in\procs$ and a substitution $\sigma$, then
	\[\text{if }\ec{P}\transs{\alpha}\ec{Q}\text{ then }\ec{P\sigma{}}\transs{\sigma(\alpha)}\ec{Q\sigma{}}\]
holds.
\end{lemma}
\begin{prf}
It is proved by an induction over the inference of $\ec{P}\transs{\alpha}\ec{Q}$. All cases apart from the \ecall{} case are proved in \cite{sangiorgi}. Thus, let $P,Q\in\procs$, $\sigma$ a substitution and $\ec{P}\transs{\alpha}\ec{Q}$ inferred by the \ecall{} rule. So we know that $\alpha=\tau$ and there is a recursive definition $\procdef{A}{\vec{w}}\define{}P'$ and some parameter $\vec{v}\subseteq\names$ such that $P=\proccall{A}{\vec{v}}$ and $Q=P'\subs{\vec{v}}{\vec{w}}$. The definition of the application of a substitution to a recursive call yields that $\proccall{A}{\vec{v}}\sigma=\proccall{A}{\vec{v}\sigma}$. Hence, with the \ecall{} rule we know $\ec{\proccall{A}{\vec{v}}\sigma}\transs{\tau}\ec{P'\subs{\vec{v}\sigma}{\vec{w}}}$. Since $\fn{P'}\subseteq{}\vec{w}$ as stipulated, we know that $\fn{P'\subs{\vec{v}}{\vec{w}}}\subseteq{}\vec{v}$ and so its the same if we apply the substitution to the parameters or to the resulting process. Hence, $\ec{\proccall{A}{\vec{v}}\sigma}\transs{\tau}\ec{\left(P'\subs{\vec{v}}{\vec{w}}\right)\sigma}$.
\end{prf}

Note that \refConv{conv_uni_bn_trans} is important for this lemma. Consider, for example, $P\define\procres{a}{\out{x}{a}.\out{x}{a}}$. Then $\ec{P}\bouttrans{x}{b}\ec{\out{x}{b}}$ and for a substitution $\sigma\define\subs{z}{b}$ \refLem{lem_subst_trans_partI} would yield that $\ec{P}=\ec{P\sigma}\bouttrans{x}{b}\ec{\out{x}{b}\sigma}=\ec{\out{x}{z}}$, since $\sigma(\bout{x}{b})=\bout{x}{b}$. But such a transition leading to $\ec{\out{x}{z}}$ does not exist. Without \refConv{conv_uni_bn_trans} we would need a side condition that in such cases $b$ is neither allowed to be within the free names of $P\sigma$ nor within the names of the substitution.

The converse of \refLem{lem_subst_trans_partI} does not hold. That is, for $\ec{P\sigma}\transs{\beta}\ec{Q'}$ we are not always able to find an action $\alpha\in\actions$ and a process $Q\in\procs$ such that $\ec{P}\transs{\alpha}\ec{Q}$ with $\sigma(\alpha)=\beta$ and $\ec{Q\sigma}=\ec{Q'}$.

Consider, for example, $P\define\procpar{\inp{a}{x}}{\out{b}{y}}$ and $\sigma\define\subs{a}{b}$. Then, on the one hand $P\sigma=\procpar{\inp{a}{x}.\proczero}{\out{a}{y}.\proczero}$ holds and so $\ec{P\sigma}\tautrans\ec{Q\sigma}$ with $Q\define\procpar{\proczero}{\proczero}$. But on the other hand there is no $\tau$ transition starting in $\ec{P}$. This is a problem as long as the application of a substitution inserts a name, which already occurs free in the process and this free name is not also replaced by the application. In this situation new possibilities for communication are established.

Furthermore, consider $P\define\inp{y}{a}.\out{x}{a}$ and $\sigma\define\subs{b}{x}$. Hence, we know that $\ec{P\sigma}=\ec{\inp{y}{a}.\out{b}{a}}\intrans{y}{x}\ec{\out{b}{x}}$ holds, but there is no way to find an action $\alpha$ and a process $Q$ such that $\ec{P}\transs{\alpha}\ec{Q}$ with $\subs{b}{x}(\alpha)=\inpa{y}{x}$ and $\ec{Q\sigma}=\ec{\out{b}{x}}$ holds, since $x$ has to be free in $\subs{b}{x}(\alpha)$ and $\ec{Q\sigma}$. This is a problem every time an input process receives a name from $\supp{\sigma}\setminus\cosupp{\sigma}$, since then this name will be replaced in the label of the transition by the application of the substitution and there is no possibility to reintegrate it, if it is not in the co-support of $\sigma$.

The first problem can be handled by adding the side condition that the substitution is \findex[substitution!injective on set]{injective} on the free names of the process under consideration. This means, for all $x,y\in\fn{P}$ if $x\neq{}y$ then $\sigma(x)\neq\sigma(y)$. Therewith, all the cases where new communications can be established due to the application of the substitution will be excluded.

The second problem is no problem in the view of the behavior of the process, since neither more communication is created nor destroyed from the substitution. The application of the substitution has just a problematic effect to the label of the transitions and the resulting process. We only can not find a suitable action for the transition, or a suitable process, since the application of the substitution replaces the needed name. Therefore, we can find a new substitution which preserves the behavior of the given substitution (behaves equal on the free names) but solves the problem with the input names, by adding those names to the substitution for reintegrating them. Thus, for example, a single substitution is getting a transposition.

This leads us to the restricted converse of \refLem{lem_subst_trans_partI}.

\begin{lemma}[Substitution on transitions (Part II) \cite{sangiorgi}]
\label{lem_subst_trans_partII}
Given a process $P\in\procs$ and a substitution $\sigma$, which is injective on $\fn{P}$. Then, there is a bijection $\rho: (\fn{P}\sigma\setminus\fn{P}) \rightarrow (\fn{P}\setminus\fn{P}\sigma)$ and with that a bijective substitution $\theta$, with
\[\theta(x)=\left\{\begin{array}{ll}
			\sigma(x) & \text{if } x\in\fn{P} \\
			\rho(x) & \text{if } x\in\fn{P}\sigma\setminus\fn{P} \\
			x & \text{if } x\nin\fn{P}\sigma\cup\fn{P}  
		\end{array}\right.\]
such that $\ec{P\sigma}=\ec{P\theta}$ and
\[\ec{P\theta}\transs{\beta}\ec{Q'} \text{ implies } \exists\alpha\in\actions,Q\in\procs:\ec{P}\transs{\alpha}\ec{Q}\]
with $\theta(\alpha)=\beta$ and $\ec{Q\theta}=\ec{Q'}$.
\end{lemma}
\begin{prf}
The second part is proved by induction over the inference of $\ec{P\theta}\transs{\beta}\ec{Q'}$. All cases apart from the \ecall{} case as well as the existence of $\rho$ are proved in \cite{sangiorgi}. Thus, let $\sigma$ be a substitution, $\theta$ as described in \refLem{lem_subst_trans_partII} and $\ec{P\theta}\transs{\beta}\ec{Q'}$ be inferred by the \ecall{} rule. Hence, $\beta=\tau$ and there is a recursive definition $\procdef{A}{\vec{w}}\define{}P'$ and a parameter list $\vec{v}\subseteq\names$ such that $P\theta=\proccall{A}{\vec{v}}$ and $Q'=P'\subs{\vec{v}}{\vec{w}}$. The application of a substitution to a process yields that the only way to reach $P\theta=\proccall{A}{\vec{v}}$ is if there is a parameter list $\vec{v'}\subseteq\names$ such that $P=\proccall{A}{\vec{v'}}$ and $\vec{v}=\vec{v'}\theta$. With the \ecall{} rule we know that $\ec{P}\transs{\tau}\ec{P'\subs{\vec{v'}}{\vec{w}}}$. Since $\fn{P'}\subseteq{\vec{w}}$ and so $\fn{P'\subs{\vec{v'}}{\vec{w}}}\subseteq{\vec{v'}}$ and \refConv{conv_uni_bn} states that no bound names of the process can occur in the substitution, we know $\ec{\left(P'\subs{\vec{v'}}{\vec{w}}\right)\theta}=\ec{P'\subs{\vec{v'}\theta}{\vec{w}}}=\ec{Q'}$. So we found an action $\alpha=\tau$ with $\theta(\alpha)=\beta$ and a process $Q=P'\subs{\vec{v'}}{\vec{w}}$ with $\ec{Q\theta}=\ec{Q'}$ and $\ec{P}\transs{\alpha}\ec{Q}$.
\end{prf}

%%%%%%%%%%%%%%%%%%%%%%%%%%%%%%%%%%%%%%%%%%%%%%%%% OLD: PROBLEM BY DEFINING INPUT ACTION BOUND %%%%%%%%%%%%%%%%%%%%%%%%%%%%%%%%%%%%%%%%%%%%%%%%%%%%%%%%
\begin{old}[If input bound in an action then problem with previous lemma]
Note that Lemma \ref{lem_subst_trans} would not hold, if we chose to define the bound names of an input action $\inp{a}{x}$ as $\set{x}$ and so would not replace the $x$ under a substitution. Consider, for example, $P\define\inp{a}{b}.\out{b}{z}$ and $\sigma\define\subs{x}{y}$. Then $\ec{P}\intrans{a}{x}\ec{\out{x}{z}}$ holds and if the lemma would hold, $\ec{P\sigma}=\ec{\inp{a}{b}.\out{b}{z}}\intrans{a}{x}\ec{\out{x}{z}\sigma}=\ec{\out{y}{z}}$ would also hold. But this is not true. Since we however replace the $x$ in the action we gain $\ec{\inp{a}{b}.\out{b}{z}}\intrans{a}{y}\ec{\out{y}{z}}$, which is correct.
\end{old}
%%%%%%%%%%%%%%%%%%%%%%%%%%%%%%%%%%%%%%%%%%%%%%%%% END OLD: PROBLEM BY DEFINING INPUT ACTION BOUND %%%%%%%%%%%%%%%%%%%%%%%%%%%%%%%%%%%%%%%%%%%%%%%%%%%%%%%%

We can think of the bijection $\rho$ as the function which gives the possibility to reintegrate the names which are replaced by $\sigma$. So, for the example above with $P\define\inp{y}{a}.\out{x}{a}$ and $\sigma\define\subs{b}{x}$, we know $\fn{P}=\set{y,x}$ and $\fn{P}\sigma=\set{y,b}$. Hence, we get a bijection $\rho:\set{b}\rightarrow\set{x}$ and so $\theta=\transp{b}{x}$. Thus, we know there is an action $\alpha\define{}\inpa{y}{b}$ and a process $Q\define{\out{x}{b}}$ such that $\ec{P}\transs{\alpha}\ec{Q}$ with $\transp{b}{x}(\alpha)=\inpa{y}{x}$ and $\ec{Q\transp{b}{x}}=\ec{\out{b}{x}}$. With this idea, we consider the special case of \refLem{lem_subst_trans_partII} where $\sigma$ is a transposition.

%\begin{old}[With substitution inside] %%%%%%%%%%%%%%%%%%%%%%%% OLD: BOTH subs and trans %%%%%%%%%%%%%%%%%%%%%%%%%%%%%%%%%%%%%%%%%%%%%%%%%%%%%%%%%%
%\begin{cor}[Substitution on transitions (Part III)]
%\label{cor_subst_trans_partIII}
%Given $a,b\in\names$, $P\in\procs$ and $\sigma=\subs{a}{b}$ substitution with $a\nin\fn{P}$ or $\sigma=\transp{a}{b}$ transposition, then $\ec{P\sigma}=\ec{P\transp{a}{b}}$ and
%\[\ec{P\sigma}\transs{\beta}\ec{Q'}\text{ implies } \exists\alpha\in\actions,Q\in\procs: \ec{P}\transs{\alpha}\ec{Q}\]
%with $\transp{a}{b}(\alpha)=\beta$ and $\ec{Q\transp{a}{b}}=\ec{Q'}$.
%\end{cor}
%\begin{prf}
%Let $a,b\in\names$ and $P\in\procs$. If $\sigma$ is not directly the transposition we know since then $a\nin\fn{P}$ that $\ec{P\sigma}=\ec{P\transp{a}{b}}$.
%\begin{description}
%\item[Case $\sigma=\subs{a}{b}$, $a,b\nin\fn{P}$:] Let $\beta\in\actions,Q'\in\procs$ with $\ec{P\subs{a}{b}}\transs{\beta}\ec{Q'}$. Since $a,b\nin\fn{P}$ we know $\ec{P\subs{a}{b}}=\ec{P\transp{a}{b}}=\ec{P}$ and thus, $\ec{P}\transs{\beta}\ec{Q'}$. Let $\alpha\define{}\transp{a}{b}(\beta)$ and $Q\define{}Q'\transp{a}{b}$, with \refLem{lem_subst_trans_partI} we know $\ec{P\transp{a}{b}}\transs{\transp{a}{b}(\beta)}\ec{Q'\transp{a}{b}}$ and so, $\ec{P}\transs{\alpha}\ec{Q}$ with $\transp{a}{b}(\alpha)=\beta$ and $\ec{Q\transp{a}{b}}=\ec{Q'}$.

%\item[Case $\sigma=\subs{a}{b}$, $a\nin\fn{P}$, $b\in\fn{P}$:] Let $\beta\in\actions,Q'\in\procs$ with $\ec{P\subs{a}{b}}\transs{\beta}\ec{Q'}$. Since $a\nin\fn{P}$ and $b\in\fn{P}$, we know $\fn{P}\subs{a}{b}\setminus\fn{P}=\set{a}$ and $\fn{P}\setminus\fn{P}\subs{a}{b}=\set{b}$. Furthermore, the substitution $\subs{a}{b}$ is injective on $\fn{P}$, since $a\nin\fn{P}$ and so, with \refLem{lem_subst_trans_partII}, we know that there is a bijective substitution $\theta$, with
%\[\theta(x)=\left\{\begin{array}{ll}
%			\subs{a}{b}(x) & \text{if } x\in\fn{P} \\
%			\rho(x) & \text{if } x\in\set{a} \\
%			x & \text{else}
%		\end{array}\right.\]
%and

%\[\ec{P\theta}\transs{\beta}\ec{Q'} \text{ implies } \exists\alpha\in\actions,Q\in\procs:\ec{P}\transs{\alpha}\ec{Q}\]

%with $\theta(\alpha)=\beta$, $\ec{Q\theta}=\ec{Q'}$ and $\ec{P\subs{a}{b}}=\ec{P\theta}$ for a bijection $\rho: \set{a} \rightarrow \set{b}$. Since there is just one bijection from $\set{a}$ to $\set{b}$ and due to $b\in\fn{P},a\nin\fn{P}$, and $\subs{a}{b}$ just replaces the name $b$ for the name $a$ and behaves as identity otherwise, we know
%\[\theta(x)=\left\{\begin{array}{ll}
%			a & \text{if } x=b \\
%			b & \text{if } x=a \\
%			x & \text{else}
%		\end{array}\right.\]
%holds. Hence, $\theta=\transp{a}{b}$.

%\item[Case $\sigma=\transp{a}{b}$:] Let $\beta\in\actions,Q'\in\procs$ with $\ec{P\transp{a}{b}}\transs{\beta}\ec{Q'}$. We know a transposition is injective and so in particular injective on $\fn{P}$ and thus, the precondition of \refLem{lem_subst_trans_partII} is fulfilled.

%If $a,b\nin\fn{P}$, we know with the same argumentation as in the first case that the corollary holds for this case.

%If $a,b\in\fn{P}$ and since due to that $\fn{P}\transp{a}{b}\setminus\fn{P}=\fn{P}\setminus\fn{P}\transp{a}{b}=\emptyset$, we know from \refLem{lem_subst_trans_partII} that there is a bijection $\theta$ with
%\[\theta(x)=\left\{\begin{array}{ll}
%			\transp{a}{b}(x) & \text{if } x\in\fn{P} \\			
%			x & \text{else}
%		\end{array}\right.\]
%$\ec{P\transp{a}{b}}=\ec{P\theta}$ and
%\[\ec{P\theta}\transs{\beta}\ec{Q'} \text{ implies } \exists\alpha\in\actions,Q\in\procs:\ec{P}\transs{\alpha}\ec{Q}\]
%with $\theta(\alpha)=\beta$ and $\ec{Q\theta}=\ec{Q'}$. Since $a,b\in\fn{P}$ we know $\theta=\transp{a}{b}$ and thus, we found everything we need.

%If $a\in\fn{P}$, $b\nin\fn{P}$ and so $\fn{P}\transp{a}{b}\setminus\fn{P}=\set{b}$ and $\fn{P}\setminus\fn{P}\transp{a}{b}=\set{a}$. Since there is just one bijection $\rho:\set{b}\rightarrow\set{a}$, we know from \refLem{lem_subst_trans_partII} that there is a bijection $\theta$ with
%\[\theta(x)=\left\{\begin{array}{ll}
%			\transp{a}{b}(x) & \text{if } x\in\fn{P} \\	
%			a & \text{if } x=b	\\	
%			x & \text{else}
%		\end{array}\right.\]
%$\ec{P\transp{a}{b}}=\ec{P\theta}$ and
%\[\ec{P\theta}\transs{\beta}\ec{Q'} \text{ implies } \exists\alpha\in\actions,Q\in\procs:\ec{P}\transs{\alpha}\ec{Q}\]
%with $\theta(\alpha)=\beta$ and $\ec{Q\theta}=\ec{Q'}$. We know $\theta=\transp{a}{b}$, since $a\in\fn{P}$ and $b\nin\fn{P}$ and thus, in this case, the corollary holds.

%If $a\nin\fn{P}$ and $b\in\fn{P}$, we know with the analogous argumentation as in the prior case just with interchanging roles of the names $a$ and $b$ that this corollary holds.

%%%%%%%%%%%%%%%%%%%%%%%%%%%%%%%%%%%%%%%% BEGIN OLD ALL IN ONE PART
%\begin{old}{Confusing all in one part}
%\begin{equation*}
%	\begin{aligned}
%		A=\left\{\begin{array}{ll}
%				\emptyset & \text{if } a,b\in\fn{P}\vee{} \\
%							&\quad{}a,b\nin\fn{P} \\
%				\set{b} & \text{if } a\in\fn{P},b\nin\fn{P} \\
%				\set{a} & \text{if } a\nin\fn{P},b\in\fn{P} 
%				\end{array}\right.
%	\end{aligned}
%	\begin{aligned}
%		B=\left\{\begin{array}{ll}
%				\emptyset & \text{if } a,b\in\fn{P}\vee{}\\
%							&\quad{}a,b\nin\fn{P} \\
%				\set{a} & \text{if } a\in\fn{P},b\nin\fn{P} \\
%				\set{b} & \text{if } a\nin\fn{P},b\in\fn{P} 
%				\end{array}\right.
%	\end{aligned}
%\end{equation*}
%holds and since a transposition is injective and so in particular injective on $\fn{P}$, we know with \refLem{lem_subst_trans_partII} that there is a bijective substitution $\theta$, with
%\[\theta(x)=\left\{\begin{array}{ll}
%			\transp{a}{b}(x) & \text{if } x\in\fn{P} \\
%			a & \text{if } x=b, a\in\fn{P}, b\nin\fn{P} \\
%			b & \text{if } x=a, a\nin\fn{P}, b\in\fn{P} \\
%			x & \text{else}
%		\end{array}\right.\]
%and $\exists\alpha\in\actions,Q\in\procs:\ec{P}\transs{\alpha}\ec{Q}$ with $\theta(\alpha)=\beta$, $\ec{Q\theta}=\ec{Q'}$ and $\ec{P\transp{a}{b}}=\ec{P\theta}$. We define $\theta'$ with
%\[\theta'(x)=\left\{\begin{array}{ll}
%			a & \text{if } x=b, a,b\nin\fn{P}\\
%			b & \text{if } x=a, a,b\nin\fn{P}\\
%			x & \text{else}
%			\end{array}\right.
%\]
%such that $\transp{a}{b}(\theta'(x))=\theta(x)$ for all $x\in\names$, since if $x\nin\fn{P}$ then $\transp{a}{b}(\theta'(x))=\transp{a}{b}(\transp{a}{b}(x))=x=\theta(x)$.
%\end{old}
%%%%%%%%%%%%%%%%%%%%%%%%%%%%%%%%%%%%%%%% END OLD ALL IN ONE PART
%\end{description}

%Thus, for all cases we found the suitable action $\alpha\in\actions$ and process $Q\in\procs$ with the claimed properties.
%\end{prf}
%\end{old} %%%%%%%%%%%%%%%%%%%%%%%%%%%%%%%%%%%%%%%%%%%%%%%%% END OLD: Subst and transp in one %%%%%%%%%%%%%%%%%%%%%%%%%%%%%

\begin{cor}[Substitution on transitions (Part III)]
\label{cor_subst_trans_partIII}
Given $a,b\in\names$, $P\in\procs$ and a transposition $\sigma=\transp{a}{b}$, then
\[\ec{P\sigma}\transs{\beta}\ec{Q'}\text{ implies } \exists\alpha\in\actions,Q\in\procs: \ec{P}\transs{\alpha}\ec{Q}\]
with $\transp{a}{b}(\alpha)=\beta$ and $\ec{Q\transp{a}{b}}=\ec{Q'}$.
\end{cor}
\begin{prf}
Let $a,b\in\names$, $P\in\procs$ and $\sigma=\transp{a}{b}$ a transposition. We know a transposition is injective and so in particular injective on $\fn{P}$ and thus, the precondition of \refLem{lem_subst_trans_partII} is fulfilled.
\begin{description}
\item[Case $a,b\nin\fn{P}$:] Let $\beta\in\actions,Q'\in\procs$ with $\ec{P\transp{a}{b}}\transs{\beta}\ec{Q'}$. Since $a,b\nin\fn{P}$ we know $\ec{P\transp{a}{b}}=\ec{P}$ and thus, $\ec{P}\transs{\beta}\ec{Q'}$. Let $\alpha\define{}\transp{a}{b}(\beta)$ and $Q\define{}Q'\transp{a}{b}$, with \refLem{lem_subst_trans_partI} we know $\ec{P\transp{a}{b}}\transs{\transp{a}{b}(\beta)}\ec{Q'\transp{a}{b}}$ and so, $\ec{P}\transs{\alpha}\ec{Q}$ with $\transp{a}{b}(\alpha)=\beta$ and $\ec{Q\transp{a}{b}}=\ec{Q'}$.

\item[Case $a\in\fn{P}$, $b\in\fn{P}$:] Since in this case $\fn{P}\transp{a}{b}\setminus\fn{P}=\emptyset$ and $\fn{P}\setminus\fn{P}\transp{a}{b}=\emptyset$ holds, we know from \refLem{lem_subst_trans_partII} that there is a bijection $\theta$ with
\[\theta(x)=\left\{\begin{array}{ll}
			\transp{a}{b}(x) & \text{if } x\in\fn{P} \\			
			x & \text{else}
		\end{array}\right.\]
$\ec{P\transp{a}{b}}=\ec{P\theta}$ and
\[\ec{P\theta}\transs{\beta}\ec{Q'} \text{ implies } \exists\alpha\in\actions,Q\in\procs:\ec{P}\transs{\alpha}\ec{Q}\]
with $\theta(\alpha)=\beta$ and $\ec{Q\theta}=\ec{Q'}$. Since $a,b\in\fn{P}$ we know $\theta=\transp{a}{b}$ and thus, we found everything we need.

\item[Case $a\in\fn{P}$, $b\nin\fn{P}$:] Thus, we know $\fn{P}\transp{a}{b}\setminus\fn{P}=\set{b}$ and $\fn{P}\setminus\fn{P}\transp{a}{b}=\set{a}$ holds. Since there is just one bijection $\rho:\set{b}\rightarrow\set{a}$, we know from \refLem{lem_subst_trans_partII} that there is a bijection $\theta$ with
\[\theta(x)=\left\{\begin{array}{ll}
			\transp{a}{b}(x) & \text{if } x\in\fn{P} \\	
			a & \text{if } x=b	\\	
			x & \text{else}
		\end{array}\right.\]
$\ec{P\transp{a}{b}}=\ec{P\theta}$ and
\[\ec{P\theta}\transs{\beta}\ec{Q'} \text{ implies } \exists\alpha\in\actions,Q\in\procs:\ec{P}\transs{\alpha}\ec{Q}\]
with $\theta(\alpha)=\beta$ and $\ec{Q\theta}=\ec{Q'}$. We know $\theta=\transp{a}{b}$, since $a\in\fn{P}$ and $b\nin\fn{P}$ and thus, in this case, the corollary holds.

\item[Case $a\nin\fn{P}$, $b\in\fn{P}$:] The argumentation is similar to the prior case by interchanging the roles of the names $a$ and $b$.
\end{description}

Thus, for all cases we found the suitable action $\alpha\in\actions$ and process $Q\in\procs$ with the claimed properties.
\end{prf}

Furthermore, we can similarly apply \refLem{lem_subst_trans_partII} to a single substitution, if the inserted name do not occur free in the process under consideration.

\begin{cor}[Substitution on transitions (Part IV)]
\label{cor_subst_trans_partIV}
Given $a,b\in\names$, $P\in\procs$ and $\sigma=\subs{a}{b}$ substitution with $a\nin\fn{P}$ then $\ec{P\sigma}=\ec{P\transp{a}{b}}$ and
\[\ec{P\sigma}\transs{\beta}\ec{Q'}\text{ implies } \exists\alpha\in\actions,Q\in\procs: \ec{P}\transs{\alpha}\ec{Q}\]
with $\transp{a}{b}(\alpha)=\beta$ and $\ec{Q\transp{a}{b}}=\ec{Q'}$.
\end{cor}
\begin{prf}
Since $a\nin\fn{P}$ and $\sigma=\subs{a}{b}$, we know that $\ec{P\sigma}=\ec{P\transp{a}{b}}$. We can now similarly prove this corollary to \refCor{cor_subst_trans_partIII}, since $\rho$ extends $\sigma$ such that for the $\theta$ of \refLem{lem_subst_trans_partII} we know $\theta=\transp{a}{b}$ holds.
\end{prf}

So, we gain the same results for on the one hand a transposition and on the other hand a substitution which only replaces one name and the co-support is no subset of the free names of the process under consideration.\index{\picalc{}|)}

%%%%%%%%%%%%%%%%%%%%%%%%%%%%%%%%%%%%%%% Example %%%%%%%%%%%%%%%%%%%%%%%%%%%%%%

\subsubsection{Example}
\label{sec_exp}
As examples for the \picalc{} and its semantics, we regard a system which can read and write values of and to a buffer.
% mainfile: ../../Refinement.tex
\begin{figure}[h!]
\centering
\begin{tikzpicture}%[dots/.style={white}]
  \node[initial,state] (A)                          {$\procdef{B_1}{in,out}$};
  \node[state] 	       (B) [below of=A]             {$\inp{in}{val}.\proccall{O_1}{val,in,out}$};
  \node[state] 	       (X) [below of=B, sdots]		    {$\cdots$};
  \node[state]         (C) [below of=B, left of=B]  {$\proccall{O_1}{a,in,out}$};
  \node[state]         (D) [below of=B, right of=B] {$\proccall{O_1}{z,in,out}$};
  \node[state]         (E) [below of=C]             {$\out{out}{a}.\proccall{B_1}{in,out}$};
  \node[state]         (F) [below of=D]             {$\out{out}{z}.\proccall{B_1}{in,out}$};

  \path (A) edge		node[right] 		    {$\tau$}		(B)
	(B) edge		node[ldiag]		    {$\inpa{in}{a}$}     (C)
            edge		node[rdiag]		    {$\inpa{in}{z}$}     (D)
	   % edge[dots]		node[xshift=-9.0, pos=1.1, black]   {$\cdots$} (X)
        (C) edge		node[anchor=east]		    {$\tau$}		(E)
        (D) edge		node[right]		    {$\tau$}		(F)
        (E.west) edge[bend left]	node[anchor=east]		    {$\out{out}{a}$}	(A.west)
        (F.east) edge[bend right]	node[anchor=west]		    {$\out{out}{z}$}	(A.east);
\end{tikzpicture}
\caption{The operational semantics for buffer $B_1$ from \refEx{ex_one_cell_buffer}.}
\label{fig_ts_buffer_one_cell}
\end{figure}


\begin{example}[One cell buffer \cite{milner}] %page 84
\label{ex_one_cell_buffer}
We can define a buffer with just one cell, for example through $B_1$.
\begin{align*}
	\procdef{B_1}{in,out}&\define\inp{in}{val}.\proccall{O_1}{val,in,out} \\
	\procdef{O_1}{val,in,out}&\define\out{out}{val}.\proccall{B_1}{in,out}
\end{align*}
The buffer has one value $val$ and two channels $in$ and $out$ to write respectively read the value. Since the buffer uses recursive calls, it is always ready for a new pass, after one pass of reading and writing. Its semantics is represented in \refFig{fig_ts_buffer_one_cell}.
\end{example}

Since for every input there are countably infinite names which can be received, we use dots to visualize such alternatives in \refFig{fig_ts_buffer_one_cell}.

To extend the buffer so that it is possible to read and write more than one value, we introduce two versions of a buffer with two cells.

\begin{example}[Two cell buffer (Version I)]
\label{ex_two_cell_buffer_mine}
A buffer with two cells and some kind of FIFO strategy can possibly be modeled as $B_2$.
\begin{align*}
	\procdef{B_2}{i,o}&\define{\procchoice{\proccall{C_1}{i,o}}{\proccall{C_2}{i,o}}}  \\
	\procdef{C_1}{i,o}&\define\inp{i}{val_1}.\inp{i}{val_2}.\out{o}{val_1}.\out{o}{val_2}.\proccall{B_2}{i,o}	\\
	\procdef{C_2}{i,o}&\define\inp{i}{val_1'}.\out{o}{val_1'}.\inp{i}{val_2'}.\out{o}{val_2'}.\proccall{B_2}{i,o}	
\end{align*}
This buffer has also an input $i$ and an output $o$ channel for writing respectively reading values. The semantics of $B_2$ is visualized in \refFig{fig_ts_buffer_fifo_1}.
\end{example}

Since $\inp{i}{y}.\out{o}{a}.\out{o}{y}.\proccall{B_2}{i,o}$ and $\inp{i}{y}.\out{o}{z}.\out{o}{y}.\proccall{B_2}{i,o}$ are the same, apart from the name that had been received and this name has no influence on the communication of the process, we abbreviate the second part by the dots in \refFig{fig_ts_buffer_fifo_1}. Furthermore, we know that in the second part just the first output transition is changed from $\out{o}{a}$ to $\out{o}{z}$.

It is also possible to define a FIFO buffer with two cells as a linked list.

\begin{example}[Two cell buffer (Version II)\cite{milner}] %page 85 and 95
A FIFO buffer with two cells can be defined with $FIFO$.%\todo{sangiorgi has also a two cell buffer on page 133}
\label{ex_two_cell_buffer_milner}
\begin{align*}
	\procdef{FIFO}{in,out}&\define\procres[a]{com}{\procpar{\proccall{B_3}{in,com}}{\proccall{B_3}{com,out}}}\\
	\procdef{B_3}{in,out}&\define\inp{in}{val}.\proccall{O_2}{val,in,out} \\
	\procdef{O_2}{val,in,out}&\define\out{out}{val}.\proccall{B_3}{in,out}
\end{align*}
This buffer can write the first value in one cell and, before receiving another one, it passes this value via the private channel $com$ to the other cell. A reading is just possible if the value is located in the second cell.
\end{example}

% mainfile: ../../Refinement.tex
\begin{figure}[h!]
  \centering
  \begin{tikzpicture}
	%\begin{scope}[node distance=15mm and 10mm]
	  \node[initial,state] (A)                          				{$\procdef{B_2}{i,o}$};
	  \node[state] 	       (B) [below of=A]             				{$\procchoice{\proccall{C_1}{i,o}}{\proccall{C_2}{i,o}}$};
	  \node[state]         (C) [below of=B, left of=B, xshift=-10, yshift=-5mm]  	{$\inp{i}{x}.\inp{i}{y}.\out{o}{x}.\out{o}{y}.\proccall{B_2}{i,o}$};
	  
 	  \node[state]	       (D')[below of=B, right of=B, xshift=10, yshift=4mm] 	{$\inp{i}{x}.\out{o}{x}.\inp{i}{y}.\out{o}{y}.\proccall{B_2}{i,o}$};
	  \node[state]         (L1)[below of=D', yshift=8mm, xshift=-6mm] 				{$\cdots$};	
	  \node[state]         (L')[below of=D', left of=D', xshift=7mm, yshift=2mm]     {$\out{o}{a}.\inp{i}{y}.\out{o}{y}.\proccall{B_2}{i,o}$};
%	  \node[state]         (X')[below of=L', yshift=14mm] 				{$\vdots$};
	  \node[state]         (R')[below of=D', yshift=-9mm]				         {$\out{o}{z}.\inp{i}{y}.\out{o}{y}.\proccall{B_2}{i,o}$};

	  \node[state]         (D) [below of=R'] 					{$\inp{i}{y}.\out{o}{y}.\proccall{B_2}{i,o}$};%, right of=B, xshift=10, yshift=5mm
	% right strand
	  \node[state] 	       (X1)[below of=D, sdots]		   			{$\cdots$};
	  \node[state]         (D1)[below of=D, xshift=-40]				{$\out{o}{a}.\proccall{B_2}{i,o}$};
	  \node[state]         (D2)[below of=D, xshift=40] 				{$\out{o}{z}.\proccall{B_2}{i,o}$};
	% left strand
	  \node[state] 	       (X2)[below of=C, sdots]					{$\cdots$};
	  \node[state]         (C1)[below of=C, xshift=-65] 				{$\inp{i}{y}.\out{o}{a}.\out{o}{y}.\proccall{B_2}{i,o}$};
	  \node[state]         (C2)[below of=C, xshift=65] 				{$\inp{i}{y}.\out{o}{z}.\out{o}{y}.\proccall{B_2}{i,o}$};
	  \node[state]         (L1)[below of=C2, yshift=1cm] 				{$\vdots$};
	  \node[state] 	       (X3)[below of=C1, sdots]		    			{$\cdots$};
	  \node[state]         (C3)[below of=C1, xshift=-50] 				{$\out{o}{a}.\out{o}{a}.\proccall{B_2}{i,o}$};
	  \node[state]         (C4)[below of=C1, xshift=50] 				{$\out{o}{a}.\out{o}{z}.\proccall{B_2}{i,o}$};
	  \node[state]         (C5)[below of=C3]					{$\out{o}{a}.\proccall{B_2}{i,o}$};
	  \node[state]         (C6)[below of=C4]					{$\out{o}{z}.\proccall{B_2}{i,o}$};
	%\end{scope}
	  \path (A)		edge						node[right]		{$\tau$}	(B)
		(B)		edge						node[ldiag]		{$\tau$}     	(C)
		    		edge						node[rdiag,pos=0.1]	{$\tau$}     	(D')	
		(C) 		edge						node[ldiag]		{$\inpa{i}{a}$}	(C1)
		    		edge						node[rdiag]		{$\inpa{i}{z}$}	(C2)
		(C1) 		edge						node[ldiag]		{$\inpa{i}{a}$}	(C3)
		    		edge						node[rdiag]		{$\inpa{i}{z}$}	(C4)
		(C3)		edge						node[left]		{$\outa{o}{a}$}	(C5)
		(C4)		edge						node[right]		{$\outa{o}{a}$}	(C6)
		(D) 		edge						node[ldiag]		{$\inpa{i}{a}$}	(D1)
		    		edge						node[rdiag]		{$\inpa{i}{z}$}	(D2)
		(R')     	edge						node[right]		{$\outa{o}{z}$} (D)
		(D')		edge						node[ldiag]		{$\inpa{i}{a}$} (L')
				edge						node[right]		{$\inpa{i}{z}$} (R')
		(L')		edge[bend right=30]				node[left,pos=0.7]	{$\outa{o}{a}$} (D)
		(D2.east) 	edge[bend right=30, dopac]			node[right]		{$\outa{o}{z}$}	(A.east)
		(D1) 		edge[bend left=60, dopac]			node[left,pos=0.8]	{$\outa{o}{a}$}	(A.south west)
		(C6.east)	edge[bend right=40, dopac]	node[right,pos=0.3]		{$\outa{o}{z}$}	(A.south east)
		(C5.west) 	edge[bend left=40, dopac, looseness=1]		node[left]		{$\outa{o}{a}$}	(A.west);
  \end{tikzpicture}
\caption{The operational semantics for buffer $B_2$ from \refEx{ex_two_cell_buffer_mine}.}
\label{fig_ts_buffer_fifo_1}
\end{figure}


Since the behavior of those processes -- especially of the $FIFO$ process -- is hard to determine, we introduce in \refSec{sec_de_sem_trace} a denotational semantics, which simplifies the calculation of the external behavior of a process. Furthermore, we introduce a relation which enables comparing processes on the base of the denotational semantics. Thereby, we will see, that the buffer $B_2$ and the buffer $FIFO$ have a different visible behavior.


\subsection{Simulation and bisimulation}
\label{sec_pi_simulation}
% mainfile: ../../Refinement.tex
For comparing processes based on their behavior there is already a notion called \index{bisimulation!strong}\findex[strong!bisimulation]{strong bisimulation} defined in \cite{milnerParrowWalker}. Since they use a slightly different semantics for the transitions, we stick to the notion of bisimulations in \cite{sangiorgi}.

\begin{definition}[Strong simulation and bisimulation]
\label{def_strong_sim_bisim}
A relation $\mathcal{S}\subseteq\procs_\alpha\times\procs_\alpha$ is called a \index{simulation!strong}\findex[strong!simulation]{strong simulation}, if $(\ec{P},\ec{Q})\in\mathcal{S}$ implies that
\[\text{if } \ec{P}\transs{\alpha}\ec{P'} \text{ then }Q'\in\procs \text{ exists such that } \ec{Q}\transs{\alpha}\ec{Q'} \text{ and } (\ec{P'},\ec{Q'})\in\mathcal{S}.\]
$\mathcal{S}$ is called a \index{bisimulation!strong}\findex[strong!bisimulation]{strong bisimulation} if $\mathcal{S}$ and $\mathcal{S}^{-1}$ are strong simulations.

Furthermore, for a strong bisimulation $\mathcal{S}$, we call $\ec{P}$ \findex[strong!bisimilar]{strong bisimilar} to $\ec{Q}$, written as $\ec{P}\sbisim\ec{Q}$, if $(\ec{P},\ec{Q})\in\mathcal{S}$.
\end{definition}

Thereby, the internal as well as the external behavior are taken into account. To abstract from the invisible actions, the \index{bisimulation!weak}\findex[weak!bisimulation]{weak bisimulation} relates processes only according to their external behavior.

\begin{definition}[Weak simulation and bisimulation]
\label{def_weak_sim_bisim}
A relation $\mathcal{S}\subseteq\procs_\alpha\times\procs_\alpha$ is called a \index{simulation!weak}\findex[weak!simulation]{weak simulation}, if $(\ec{P},\ec{Q})\in\mathcal{S}$ implies 
\begin{itemize}%\itemsep-5mm
\item[(1)] if $\ec{P}\transs{\tau}\ec{P'}$ then $Q'\in\procs$ exists such that $\ec{Q}(\tautrans)^*\ec{Q'}$ and $(\ec{P'},\ec{Q'})\in\mathcal{S}$,
\item[(2)] if $\ec{P}\transs{\alpha}\ec{P'}$ then $Q'\in\procs$ exists such that $\ec{Q}(\tautrans)^*\fatsemi\transs{\alpha}\fatsemi(\tautrans)^*\ec{Q'}$ and $(\ec{P'},\ec{Q'})\in\mathcal{S}$, for $\alpha\in\actions\setminus\set{\tau}$.
\end{itemize}
A relation $\mathcal{S}\subseteq\procs_\alpha\times\procs_\alpha$ is called a \index{bisimulation!weak}\findex[weak!bisimulation]{weak bisimulation}, if $\mathcal{S}$ and $\mathcal{S}^{-1}$ are weak simulations.

Furthermore, for a weak bisimulation $\mathcal{S}$, we call $\ec{P}$ \findex[weak!bisimilar]{weak bisimilar} to $\ec{Q}$, written as $\ec{P}\wbisim\ec{Q}$, if $(\ec{P},\ec{Q})\in\mathcal{S}$.
\end{definition}
Note that with the definition of the big-step semantics in \refChap{sec_big-step_semantics} there is a somehow simpler way to describe the weak bisimulation, since there is an abbreviation for an arbitrary number of $\tau$ transitions.

\newpage %%%%%%%%%%%%%%% NEWPAGE!



\section{Discussion}
\label{sec_discussion}
% mainfile: ../../Refinement.tex
Many operational semantics and variants of the syntax of the \picalc{} exist. %, which are principally developed for the content in which they should be used.
The reasons for choosing this syntactical variant of the \picalc{} are already explained in \refSec{sec_pi_calculus}. In this section we motivate the usage of the particular operational semantics defined in \refSec{sec_pi_op_sem} and compare it to others.% Furthermore, some other design decisions are discussed.

In \cite{milnerParrowWalker} Milner, Parrow, and Walker introduce an operational semantics with \index{instantiation!late}\findex{late instantiation}. That is, the variable of an input process is instantiated at the moment the communication is inferred. This contrasts to the \index{instantiation!early}\findex{early instantiation}, where the variable is already instantiated when the input transition is inferred, which we consider in this thesis. The late instantiation constitutes the reason for the inapplicability of this operational semantics for our desired denotational semantics. The notion of the trace semantics defined in \refChap{sec_de_sem_trace} is to collect all of a process' behavior just by observing the labels of the transitions. Consider, for example, the process $P'\define\inp{x}{z}.\out{z}{q}$. The only rule which handles the external behavior of an input in the operational semantics presented in \cite{milnerParrowWalker}, is the axiom

\[\kalRule{INPUT-ACT}{}{}{\ec{\inp{x}{z}.P}\transs{\inpa{x}{w}}\ec{P\subs{w}{z}}}{\quad{}w\nin\fn{\procres{z}{P}}}.\]

Thus, there is no transition $\ec{P'}\intrans{x}{q}\ec{\out{q}{q}}$, which however is also a valid external behavior of $P'$. This is no problem for the internal behavior of the process, since for a communication -- according to the late instantiation -- the missing input variables are instantiated by the communication rule. But the single transition itself is missing, which makes it impossible to describe the whole external and internal behavior by simply observing the labels of the transition system. Note that these missing transitions are harmful for describing the behavior of a process. Thus, in the definition of bisimilarity in \cite{milnerParrowWalker} they also took this missing behavior into account while relating processes.

In \cite{milner} Robin Milner defines two other operational semantics. On the one hand, the \findex{commitments} developed for another definition of the strong bisimilarity, and on the other hand, the \findex{input-/output experiments} to gain a better handling of the internal behavior of processes to present another definition of the weak bisimilarity. Both are not suitable for the desired denotational semantics, since they also do not represent the whole behavior of the process in the transition system itself. For instance, the commitment rules are deadlocked for one part of a sum after conducting a simple input or output transition. Hence, there is a commitment rule $\inp{a}{x}.\inp{x}{y}\transs{a}(x).\inp{x}{y}$, but there is no further rule for a so-called agent $(x).\inp{x}{y}$. Thus, the information that the process can interact with another process by synchronizing over channel $a$ and, after that, synchronizing with another output process over this name, is lost by just observing the transitions.

Another problem is that for an output process $P\define\out{a}{x}.\proczero$ the name $x$, which is sent over channel $a$, is not specified in the label of the transition $\out{a}{x}.\proczero \transs{\bar{a}}\langle x\rangle.\proczero$. Thus, this information gets also lost by just observing the label of the transitions. So, the transitions of $P$ and the process $\out{a}{y}.\proczero$ are the same, which is not suitable for our trace semantics.

The input-/output experiments presented in \cite{milner} base upon the commitment rules. In this approach, input transitions are extended such that the problems mentioned above will not occur. Even though, the output transitions are also extended, the problems concerning output processes still exists. Thus, not the whole behavior of a process is mapped to its transitions.

The motivation to take a closer look at the transition systems defined by Sangiorgi and Walker in \cite{sangiorgi} is given by their statement, in which they explain that they did not adapt the notion of Milner ``because [they] consider the presentation [they] have given to be simpler and easier to work with''\footnote{\cite{sangiorgi}, page $162$.}. Moreover, Milner himself rates the book in the foreword ``as a storehouse of ideas and techniques which is unlikely to be equalled in the next decade or two''\footnote{\cite{sangiorgi}, page x.}.

It is a matter of interest and the field of application, whether the \index{transition system!late}\findex[late transition system]{late} or \findex{early transition system} presented in \cite{sangiorgi} is chosen for formalizing the semantics. This is because every transition of the early can also be made, with a slight adaption, in the late transition system and vice versa \cite{sangiorgi}. However this adaption has to be taken into account while working with the semantics. It is for this reason that the late semantics is not suitable for a denotational semantics, which just observes the labels of the transitions in order to gain the whole behavior of a process. Even though the input rule in the late transition system differs from the rule in \cite{milnerParrowWalker}:
			\[\kalRule{L-INP}{}{}{\ec{\inp{x}{z}.P}\intrans{x}{z}\ec{P}}{},\]
the problem stays the same. Even if the side condition and the substitution within the INPUT-ACT can be omitted in \cite{sangiorgi} by the reason of a similar convention as \refConv{conv_uni_bn}, there is also no transition $\ec{\inp{x}{z}.\out{z}{q}}\intrans{x}{q}\ec{\out{q}{q}}$, since $\inp{x}{q}.\out{q}{q}\nin\ec{\inp{x}{z}.\out{z}{q}}$.

Since the inapplicability of the semantics defined in \cite{milnerParrowWalker, sangiorgi} results by the late instantiation of the variables of an input process, the more intensively investigation of other transition systems with late instantiation (for example in \cite{canal, paolaDiss, alexandru}) is not expedient.

% Since the late transition system corresponds in the main parts to the transition system presented in \cite{milnerParrowWalker} -- especially in the late instantiation of the input variables --  
%\todo[inline]{hier kommen noch Erklärungen warum die Variante von der \ecall{} und \eopen{} rule gewählt wurden.}


%\todo{uncomment the rest of it and find suitable citations}
\begin{old}{s.th. what really should be inserted}

There are also some alternatives for a few of the transition rules. For example in \cite{} there is an alternative for the \ecall{} rule, which do not use a $\tau$ step for the computation of a function call:
\[\kalRule{}{}{P\subs{\vec{v}}{\vec{w}}\transs{\alpha}P'}{\proccall{A}{\vec{v}}\transs{\alpha}P'}{\quad\procdef{A}{\vec{w}}\define{}P}.\]

 

% Owing to the fact that the transition system presented in \cite{canal} is mostly an adaption of the late semantics of \cite{sangiorgi}, it has the same problem and is consequently also not suitable for the semantics in this thesis.

\todo[inline]{
-\eopen{} replacement
-\ecall{} replacement
}

\todo[inline]{often \eopen{} replaced by rule for example \cite{paola} because of \refLem{lem_bn_trans}}
\todo[inline]{\ecall{} with $\tau$ because of nicht wirklich darstellbar? how would it look like? or other cause cannot determine or distinguish the behavior properly?}
\todo[inline]{Dieses Kapitel (operational semantics)muss evtl. nochmal geändert werden, da ich es nicht so schön finde, dass Traces entstehen können, wo zweimal derselbe gebundene Name auftreten kann. Das liegt nur an den rekursiven Aufrüfen. Zum Beispiel $P\define\proccall{A}{b}$ mit $\procdef{A}{b}\define\procres[()]{a}{\out{b}{a}.\proccall{A}{b}}$. Diese Definition wiederspricht nicht \refConv{conv_uni_bn}. Dann gilt aber $P\tautrans\procres[()]{a}{\out{b}{a}.\proccall{A}{b}}\bouttrans{b}{a}\proccall{A}{b}\tautrans\procres[()]{a}{\out{b}{a}.\proccall{A}{b}}\bouttrans{b}{a}\proccall{A}{b}$. Somit gibt es einen Trace, wo der Name $a$ zweimal gebunden vorkommt, was ich eigentlich mit der Konvention unterbinden wollte. Dies könnte man lösen indem man die operationelle Semantik dahingehend anpasst, dass man eine Menge der verwendeten gebundenen Namen mitschleppt und damit in diesen Fällen einen Übergang unterbindet. Eventuell kann man aber die Idee der Kompositionalität für den Restrictions-Operator auch noch anpassen, da man den Scope trotzdem wieder herausfinden kann.
}
\todo[inline]{Operational semantics: Kopierregel. zweites new a in new a strich umbennen. systematische umbenennungsnamesvergabe entlang einer trace. Hier kein Baum wie $T_\pi$ bei Proceduren.}

\todo[inline]{not safe substituion from Definition 6 \cite{milnerParrowWalker} or \cite{caires} because of substitution on traces?}

\todo[inline]{the rest of this section has a problem for calls, since the bound names of a call are an empty set. \cite{meyer} regards in his convention just the first call, but that doesn't help us for the uniqueness of the bound names in traces. \cite{sangiorgi} and \cite{milner} using replication or in the additional part of \cite{sangiorgi} they do not consider this cases. In \cite{milnerParrowWalker} the chose the bound names, as the bound names of the process, but then fixpoint calculation for calculation of bound names?}

\todo[inline]{Since we mainly investigate the behavior of processes in this thesis and the bound names can be changed to all names except of the free names of a process, without changing its behavior, we take a look at the bound and free names of a process from the process' structure. Hence we refer to the definition from \cite{sangiorgi_phd,meyer} with $\bn{\proccall{A}{\vec{v}}} \define \emptyset$ instead of $\bn{\proccall{A}{\vec{v}}} \define \bn{P}$ from \cite{milnerParrowWalker}, which would facilitates the calculation.}
\end{old}

%%%%%%%%%%%%%%%%%%%%%%%%%%%%%%%%%%%%%%%%%%%%%%%%%%%%%%%%%%% START IDEAS AND SO ON %%%%%%%%%%%%%%%%%%%%%%%%%%%%%%%%%%%%%%%%%%%%%%%%%%%%%%%%%%%%%%%%
\begin{old}{ideas and so on}
- Sichten die eher von einer Seite die Prozesse beleuchten sind zum Beispiel die von Bravetti und Vavataro, wo jedoch die Kanalübergabe (also die MObilitaet) ausgespart ist oder auch von Canal Pimentel Troya, welche aber es aus der Sicht von Vererbungen betrachten. Die vernachlässigen die Sicht auf divergenzen und auch wenn sie damit deterministischere Prozesse beschreiben könne ist die Einschränkung zu groß, da diese Relation zum Beispiel verlangt, dass jedes Verhalten des Vater Prozesses auch von dem Kindprozess nachgespielt werden können muss, was für unsere Zwecke nicht relevant ist.


\todo[inline]{Verträgt sich die Relation mit Schwacher oder starken bisimulationen.Gibt es da zusammenhänge, oder müssten die evtl. verändert werden siehe paper canal seite 121
- Dinge wie aus canal 122 Theorem 4.8 beweisen?
- relation eingeschränkt auf den leeren schnitt der freien Namen? Sonst deadlock gefahr usw.
- name boundeness in pi calculus
\url{http://concurrency.cs.uni-kl.de/documents/Huechting\_Majumdar\_Meyer\_CONCUR\_2013.pdf}
- Nur endlich viele Kanäle, da nicht signifikant unterschiedliche Transitionssysteme.
divergence respecting bisimulation
\url{http://202.120.38.217/~yuehg/LICS/1988\%20Bisimulations\%20and\%20Divergence.pdf}
}


It is necessary for two processes wishing to communicate, that the channel via the communication should take place is free in both of them, otherwise there is no possibility to establish the connection. So the free names of a process can be understood as the possibility of the process to communicate with its environment. There is obviously one exception which is important to mention, because the scope of a bounded channel can be widened by passing a bounded channel through an open channel to the other process, for example:
\begin{align*}
	P &= \procres[a]{a}{\out{x}{a}.\inp{a}{y}.\proczero} \\
	Q &= \inp{x}{b}.\out{b}{z}.\proczero \\
\end{align*}
Even though $\fn{P}=\set{x}$ and $\fn{Q}=\set{x,z}$ the processes $P$ and $Q$ got the ability to communicate over channel $a$ after it is passed through channel $x$; compare figure \todo{ref}. Shurely it is another process which communicates over $a$ after the communication over $x$ has been established, but it must be taken in mind, for the ability of communication of the former process.\todo[inline]{damit das "uberhaupt klappt darf nicht die strukturellen "aquivalenzklassen von roland meyer genommern werden, da sonst bei den so definierten Regeln der Prozess gar nicht kommunizierne k"onnte}

This is important for understanding the limitations of some rules of the later (\todo{auch nur evtl. wahr :) ref chapter...}) defined transition system.
\todo[inline]{Brille strukturelle semantik, da dann endlich in der Breite bei Input (countably-infinite name space) and to save rules?}
\todo[inline]{geht aber wenn nur ohne das umbenennen von gebundenen Variablen, da dadurch siehe Beispiel oben auch Kommunikation stattfinden oder eben nicht stattfinden kann. Deswegen doch nicht endlich in der Breite, da m"usste man f"ur die visualisierung es auf die Menge der in den Prozessen vorkommenden Namen einschr"anken, hm ist da auch gar nicht dabei, aber was sagen sie bei der Seite 15 \cite{sangiorgi} mit der alpha... die sollen sogar gleich sein. Zu m"ude, sp"ater gucken...}
\todo[inline]{contexts and structurally congruenz}
\todo[inline]{example BAG? Linker Diplomarbeit page 71}
\todo[inline]{explain by buffer its not possible to save and save again without getting the value first. And that the buffer is continually ready to start a new pass}
\end{old}
%%%%%%%%%%%%%%%%%%%%%%%%%%%%%%%%%%%%%%%%%%%%%% END IDEAS AND SO ON %%%%%%%%%%%%%%%%%%%%%%%%%%%%%%%%%%%%%%%%%%%%%%%%%%%%%%%%%%%%%%%%%%%%%%%%%%%%%%


