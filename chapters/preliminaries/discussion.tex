% mainfile: ../../Refinement.tex
Many operational semantics and variants of the syntax of the \picalc{} exist. %, which are principally developed for the content in which they should be used.
The reasons for choosing this syntactical variant of the \picalc{} are already explained in \refSec{sec_pi_calculus}. In this section we motivate the usage of the particular operational semantics defined in \refSec{sec_pi_op_sem} and compare it to others.% Furthermore, some other design decisions are discussed.

In \cite{milnerParrowWalker} Milner, Parrow, and Walker introduce an operational semantics with \index{instantiation!late}\findex{late instantiation}. That is, the variable of an input process is instantiated at the moment the communication is inferred. This contrasts to the \index{instantiation!early}\findex{early instantiation}, where the variable is already instantiated when the input transition is inferred, which we consider in this thesis. The late instantiation constitutes the reason for the inapplicability of this operational semantics for our desired denotational semantics. The notion of the trace semantics defined in \refChap{sec_de_sem_trace} is to collect all of a process' behavior just by observing the labels of the transitions. Consider, for example, the process $P'\define\inp{x}{z}.\out{z}{q}$. The only rule which handles the external behavior of an input in the operational semantics presented in \cite{milnerParrowWalker}, is the axiom

\[\kalRule{INPUT-ACT}{}{}{\ec{\inp{x}{z}.P}\transs{\inpa{x}{w}}\ec{P\subs{w}{z}}}{\quad{}w\nin\fn{\procres{z}{P}}}.\]

Thus, there is no transition $\ec{P'}\intrans{x}{q}\ec{\out{q}{q}}$, which however is also a valid external behavior of $P'$. This is no problem for the internal behavior of the process, since for a communication -- according to the late instantiation -- the missing input variables are instantiated by the communication rule. But the single transition itself is missing, which makes it impossible to describe the whole external and internal behavior by simply observing the labels of the transition system. Note that these missing transitions are harmful for describing the behavior of a process. Thus, in the definition of bisimilarity in \cite{milnerParrowWalker} they also took this missing behavior into account while relating processes.

In \cite{milner} Robin Milner defines two other operational semantics. On the one hand, the \findex{commitments} developed for another definition of the strong bisimilarity, and on the other hand, the \findex{input-/output experiments} to gain a better handling of the internal behavior of processes to present another definition of the weak bisimilarity. Both are not suitable for the desired denotational semantics, since they also do not represent the whole behavior of the process in the transition system itself. For instance, the commitment rules are deadlocked for one part of a sum after conducting a simple input or output transition. Hence, there is a commitment rule $\inp{a}{x}.\inp{x}{y}\transs{a}(x).\inp{x}{y}$, but there is no further rule for a so-called agent $(x).\inp{x}{y}$. Thus, the information that the process can interact with another process by synchronizing over channel $a$ and, after that, synchronizing with another output process over this name, is lost by just observing the transitions.

Another problem is that for an output process $P\define\out{a}{x}.\proczero$ the name $x$, which is sent over channel $a$, is not specified in the label of the transition $\out{a}{x}.\proczero \transs{\bar{a}}\langle x\rangle.\proczero$. Thus, this information gets also lost by just observing the label of the transitions. So, the transitions of $P$ and the process $\out{a}{y}.\proczero$ are the same, which is not suitable for our trace semantics.

The input-/output experiments presented in \cite{milner} base upon the commitment rules. In this approach, input transitions are extended such that the problems mentioned above will not occur. Even though, the output transitions are also extended, the problems concerning output processes still exists. Thus, not the whole behavior of a process is mapped to its transitions.

The motivation to take a closer look at the transition systems defined by Sangiorgi and Walker in \cite{sangiorgi} is given by their statement, in which they explain that they did not adapt the notion of Milner ``because [they] consider the presentation [they] have given to be simpler and easier to work with''\footnote{\cite{sangiorgi}, page $162$.}. Moreover, Milner himself rates the book in the foreword ``as a storehouse of ideas and techniques which is unlikely to be equalled in the next decade or two''\footnote{\cite{sangiorgi}, page x.}.

It is a matter of interest and the field of application, whether the \index{transition system!late}\findex[late transition system]{late} or \findex{early transition system} presented in \cite{sangiorgi} is chosen for formalizing the semantics. This is because every transition of the early can also be made, with a slight adaption, in the late transition system and vice versa \cite{sangiorgi}. However this adaption has to be taken into account while working with the semantics. It is for this reason that the late semantics is not suitable for a denotational semantics, which just observes the labels of the transitions in order to gain the whole behavior of a process. Even though the input rule in the late transition system differs from the rule in \cite{milnerParrowWalker}:
			\[\kalRule{L-INP}{}{}{\ec{\inp{x}{z}.P}\intrans{x}{z}\ec{P}}{},\]
the problem stays the same. Even if the side condition and the substitution within the INPUT-ACT can be omitted in \cite{sangiorgi} by the reason of a similar convention as \refConv{conv_uni_bn}, there is also no transition $\ec{\inp{x}{z}.\out{z}{q}}\intrans{x}{q}\ec{\out{q}{q}}$, since $\inp{x}{q}.\out{q}{q}\nin\ec{\inp{x}{z}.\out{z}{q}}$.

Since the inapplicability of the semantics defined in \cite{milnerParrowWalker, sangiorgi} results by the late instantiation of the variables of an input process, the more intensively investigation of other transition systems with late instantiation (for example in \cite{canal, paolaDiss, alexandru}) is not expedient.

% Since the late transition system corresponds in the main parts to the transition system presented in \cite{milnerParrowWalker} -- especially in the late instantiation of the input variables --  
%\todo[inline]{hier kommen noch Erklärungen warum die Variante von der \ecall{} und \eopen{} rule gewählt wurden.}


%\todo{uncomment the rest of it and find suitable citations}
\begin{old}{s.th. what really should be inserted}

There are also some alternatives for a few of the transition rules. For example in \cite{} there is an alternative for the \ecall{} rule, which do not use a $\tau$ step for the computation of a function call:
\[\kalRule{}{}{P\subs{\vec{v}}{\vec{w}}\transs{\alpha}P'}{\proccall{A}{\vec{v}}\transs{\alpha}P'}{\quad\procdef{A}{\vec{w}}\define{}P}.\]

 

% Owing to the fact that the transition system presented in \cite{canal} is mostly an adaption of the late semantics of \cite{sangiorgi}, it has the same problem and is consequently also not suitable for the semantics in this thesis.

\todo[inline]{
-\eopen{} replacement
-\ecall{} replacement
}

\todo[inline]{often \eopen{} replaced by rule for example \cite{paola} because of \refLem{lem_bn_trans}}
\todo[inline]{\ecall{} with $\tau$ because of nicht wirklich darstellbar? how would it look like? or other cause cannot determine or distinguish the behavior properly?}
\todo[inline]{Dieses Kapitel (operational semantics)muss evtl. nochmal geändert werden, da ich es nicht so schön finde, dass Traces entstehen können, wo zweimal derselbe gebundene Name auftreten kann. Das liegt nur an den rekursiven Aufrüfen. Zum Beispiel $P\define\proccall{A}{b}$ mit $\procdef{A}{b}\define\procres[()]{a}{\out{b}{a}.\proccall{A}{b}}$. Diese Definition wiederspricht nicht \refConv{conv_uni_bn}. Dann gilt aber $P\tautrans\procres[()]{a}{\out{b}{a}.\proccall{A}{b}}\bouttrans{b}{a}\proccall{A}{b}\tautrans\procres[()]{a}{\out{b}{a}.\proccall{A}{b}}\bouttrans{b}{a}\proccall{A}{b}$. Somit gibt es einen Trace, wo der Name $a$ zweimal gebunden vorkommt, was ich eigentlich mit der Konvention unterbinden wollte. Dies könnte man lösen indem man die operationelle Semantik dahingehend anpasst, dass man eine Menge der verwendeten gebundenen Namen mitschleppt und damit in diesen Fällen einen Übergang unterbindet. Eventuell kann man aber die Idee der Kompositionalität für den Restrictions-Operator auch noch anpassen, da man den Scope trotzdem wieder herausfinden kann.
}
\todo[inline]{Operational semantics: Kopierregel. zweites new a in new a strich umbennen. systematische umbenennungsnamesvergabe entlang einer trace. Hier kein Baum wie $T_\pi$ bei Proceduren.}

\todo[inline]{not safe substituion from Definition 6 \cite{milnerParrowWalker} or \cite{caires} because of substitution on traces?}

\todo[inline]{the rest of this section has a problem for calls, since the bound names of a call are an empty set. \cite{meyer} regards in his convention just the first call, but that doesn't help us for the uniqueness of the bound names in traces. \cite{sangiorgi} and \cite{milner} using replication or in the additional part of \cite{sangiorgi} they do not consider this cases. In \cite{milnerParrowWalker} the chose the bound names, as the bound names of the process, but then fixpoint calculation for calculation of bound names?}

\todo[inline]{Since we mainly investigate the behavior of processes in this thesis and the bound names can be changed to all names except of the free names of a process, without changing its behavior, we take a look at the bound and free names of a process from the process' structure. Hence we refer to the definition from \cite{sangiorgi_phd,meyer} with $\bn{\proccall{A}{\vec{v}}} \define \emptyset$ instead of $\bn{\proccall{A}{\vec{v}}} \define \bn{P}$ from \cite{milnerParrowWalker}, which would facilitates the calculation.}
\end{old}

%%%%%%%%%%%%%%%%%%%%%%%%%%%%%%%%%%%%%%%%%%%%%%%%%%%%%%%%%%% START IDEAS AND SO ON %%%%%%%%%%%%%%%%%%%%%%%%%%%%%%%%%%%%%%%%%%%%%%%%%%%%%%%%%%%%%%%%
\begin{old}{ideas and so on}
- Sichten die eher von einer Seite die Prozesse beleuchten sind zum Beispiel die von Bravetti und Vavataro, wo jedoch die Kanalübergabe (also die MObilitaet) ausgespart ist oder auch von Canal Pimentel Troya, welche aber es aus der Sicht von Vererbungen betrachten. Die vernachlässigen die Sicht auf divergenzen und auch wenn sie damit deterministischere Prozesse beschreiben könne ist die Einschränkung zu groß, da diese Relation zum Beispiel verlangt, dass jedes Verhalten des Vater Prozesses auch von dem Kindprozess nachgespielt werden können muss, was für unsere Zwecke nicht relevant ist.


\todo[inline]{Verträgt sich die Relation mit Schwacher oder starken bisimulationen.Gibt es da zusammenhänge, oder müssten die evtl. verändert werden siehe paper canal seite 121
- Dinge wie aus canal 122 Theorem 4.8 beweisen?
- relation eingeschränkt auf den leeren schnitt der freien Namen? Sonst deadlock gefahr usw.
- name boundeness in pi calculus
\url{http://concurrency.cs.uni-kl.de/documents/Huechting\_Majumdar\_Meyer\_CONCUR\_2013.pdf}
- Nur endlich viele Kanäle, da nicht signifikant unterschiedliche Transitionssysteme.
divergence respecting bisimulation
\url{http://202.120.38.217/~yuehg/LICS/1988\%20Bisimulations\%20and\%20Divergence.pdf}
}


It is necessary for two processes wishing to communicate, that the channel via the communication should take place is free in both of them, otherwise there is no possibility to establish the connection. So the free names of a process can be understood as the possibility of the process to communicate with its environment. There is obviously one exception which is important to mention, because the scope of a bounded channel can be widened by passing a bounded channel through an open channel to the other process, for example:
\begin{align*}
	P &= \procres[a]{a}{\out{x}{a}.\inp{a}{y}.\proczero} \\
	Q &= \inp{x}{b}.\out{b}{z}.\proczero \\
\end{align*}
Even though $\fn{P}=\set{x}$ and $\fn{Q}=\set{x,z}$ the processes $P$ and $Q$ got the ability to communicate over channel $a$ after it is passed through channel $x$; compare figure \todo{ref}. Shurely it is another process which communicates over $a$ after the communication over $x$ has been established, but it must be taken in mind, for the ability of communication of the former process.\todo[inline]{damit das "uberhaupt klappt darf nicht die strukturellen "aquivalenzklassen von roland meyer genommern werden, da sonst bei den so definierten Regeln der Prozess gar nicht kommunizierne k"onnte}

This is important for understanding the limitations of some rules of the later (\todo{auch nur evtl. wahr :) ref chapter...}) defined transition system.
\todo[inline]{Brille strukturelle semantik, da dann endlich in der Breite bei Input (countably-infinite name space) and to save rules?}
\todo[inline]{geht aber wenn nur ohne das umbenennen von gebundenen Variablen, da dadurch siehe Beispiel oben auch Kommunikation stattfinden oder eben nicht stattfinden kann. Deswegen doch nicht endlich in der Breite, da m"usste man f"ur die visualisierung es auf die Menge der in den Prozessen vorkommenden Namen einschr"anken, hm ist da auch gar nicht dabei, aber was sagen sie bei der Seite 15 \cite{sangiorgi} mit der alpha... die sollen sogar gleich sein. Zu m"ude, sp"ater gucken...}
\todo[inline]{contexts and structurally congruenz}
\todo[inline]{example BAG? Linker Diplomarbeit page 71}
\todo[inline]{explain by buffer its not possible to save and save again without getting the value first. And that the buffer is continually ready to start a new pass}
\end{old}
%%%%%%%%%%%%%%%%%%%%%%%%%%%%%%%%%%%%%%%%%%%%%% END IDEAS AND SO ON %%%%%%%%%%%%%%%%%%%%%%%%%%%%%%%%%%%%%%%%%%%%%%%%%%%%%%%%%%%%%%%%%%%%%%%%%%%%%%
