% mainfile: ../../Refinement.tex
The definition of traces leads directly to the \findex[trace!refinement]{trace refinement}. Hence, the situation that a process has the same external behavior as another one and, in addition to that, can possibly have some more behavior, is intuitively achieved by set inclusion. This definition directly corresponds to the definition of trace refinement in \gls{CSP}, for example in \cite{roscoe}.

\begin{definition}[\index{refinement}{Trace refinement}]
\label{def_trace_ref}
	Let $P,Q\in\procs$, then $Q$ is a \findex[trace!refinement]{trace refinement} of $P$ ($P\refi Q$) if the inverse set inclusion holds:
		\[P\refi Q \text{ is equivalent to } \traces[Q]\subseteq\traces[P].\]	
	We also say for $P\refi{}Q$ that $Q$ \findex[refinement]{refines} $P$.

	Two processes are called \findex[trace!equivalent]{trace equivalent}, if and only if they have got the same external behavior: $\traces[P]=\traces[Q]$.
\end{definition}

Thus, $Q$ refines $P$ if and only if $Q$ has less behavior than $P$. In contrast to the simulations, where the processes are collected to compare their behavior, with the trace sets we do not observe the processes, but only save the labels of the possible transitions.

Since the refinement is just a set inclusion, we know that all the results of \refSec{sec_de_sem_trace_prop} for set inclusion also hold for the refinement.

\begin{lemma}[Refinement properties]
\label{lemma_ref_properties}
Given two processes $P,Q\in\procs$ and sums $M_1,M_2,M_3,M_4\in\sums$. Then
\begin{itemize}
\item[(1)] For all substitutions $\sigma=\subs{a}{b}$ with $a\nin\fn{P}\cup\fn{Q}$ or $\sigma=\transp{a}{b}$ transposition,
\[Q \refi P \text{ implies } Q\sigma \refi P\sigma\]
holds,
\item[(2)] $Q \refi P$ is equivalent to $\out{a}{x}.Q \refi \out{a}{x}.P$, for all $a,x\in\names$,
\item[(3)] $M_2  \refi M_1$ and $M_4 \refi M_3$ implies $\procchoice{M_2}{M_4} \refi \procchoice{M_1}{M_3}$,
\item[(4)] For $\simu\subseteq\procs_\alpha\times\procs_\alpha$ weak or strong simulation
\[(\ec{P},\ec{Q})\in{}\simu \text{ implies } Q \refi P\]
holds
\end{itemize}
holds.
\end{lemma}
\begin{prf}
\refCor{cor_subst_trace_inclusion}, \refLem{lem_pres_out_choice}, \refLem{lem_weak_sim_trace_inclusion}, \refCor{cor_strong_sim_trace_inclusion} directly yield the corresponding properties.
\end{prf}

In the next section we present some examples of applications of the trace refinement.
