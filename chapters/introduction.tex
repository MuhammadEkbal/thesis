% mainfile: ../Refinement.tex
\chapter{Introduction}
\pagestyle{scrheadings}	
\setcounter{page}{0}
\pagenumbering{arabic} 
\label{sec_introduction}
In many cases of modern computing it is of interest to describe and model concurrency. Computers no longer just solve a problem by subsequently working off the single tasks of their own, but they decompose and concurrently calculate the problem even together in a network. The increase in the number of CPU cores and more heavily of GPU cores within one single computer convincingly demonstrates how fundamental concurrency is for modern computing. Moreover, the rapidly increasing spread of the Internet is one of the most common examples which shows the importance of networks.

This thesis is divided into five chapters. In \refChap{sec_preliminaries} we briefly introduce sequences and properly investigate the \picalc{} and its operational semantics (the \emph{early transition system} \cite{sangiorgi}). Thereby, we investigate its properties and define the refinement based on the trace semantics. Finally, the conclusion in \refChap{sec_conclusion} gives a brief summary of our results and presents ideas for future work.

