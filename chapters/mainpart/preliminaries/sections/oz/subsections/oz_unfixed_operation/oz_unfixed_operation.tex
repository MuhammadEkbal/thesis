Let us have another look at \refFig{fig_oz_mobile_vending_machine_and_shops}. We can notice that before switching $Shop1$ was able to invoke the operation $talk$, but after switching this is not possible, since the link $talk$ moved to $Shop2$. In \oz{} words:
\begin{itemize}
\item Before switching:
	\begin{itemize}
	\item $Shop1$: has the operation schema $talk$
	\item $Shop2$: doesn't have the operation schema $talk$
	\end{itemize}
\item By switching: $Shop1$ sends his operation schema $talk$ to $Shop2$.
\item After switching:
	\begin{itemize}
	\item $Shop1$: doesn't have the operation schema $talk$
	\item $Shop2$: has the operation schema $talk$
	\end{itemize}
\end{itemize}
. That means, when we model the $talk$ operation in \oz{} we need to consider the following problems: 

\begin{itemize}
\item The operation schema $talk$ some times exists and some times does not.
\item In \oz{}, it is not possible to send an operation schema from $Shop1$ to $Shop2$.
\end{itemize}
Solution:
\begin{itemize}
\item We call $talk$ an unfixed operation. And we use a state variable $t$ to store the name of an unfixed operation. Thus, we need an unfixed operation schema. We need to rework the $talk$ operation schema in \refFig{fig_oz_overloaded_operation_shop} to make it an unfixed operation schema as shown in \refFig{fig_oz_unfixed_operation_schema_shop}. The new schema is an conditional schema, it says: if $t = talk$ then this schema is $talk$ operation schema, otherwise this schema doesn't exists.
\item After switching 
\begin{itemize}
\item $Shop1$ is the sender, so it sets its state variable $t$ to $nil$, which means the operation schema $talk$ is no more available on $Shop1$.
\item $Shop2$ is the receiver, so it sets its state variable $t$ to the value received from $Shop1$, .i.e., $talk$, which means the operation schema $talk$ is now available on $Shop2$.
\end{itemize}

\end{itemize}

\begin{figure}[H]
\centering
\begin{class}{Shop(ref: \integer)}
\\
\begin{state}
m, vmId, self: \integer
\\t: chan[\integer \times \integer]
\end{state} 
\\
\begin{init}
\\self = ref
\end{init} 
\\
\begin{op}{switch}
x!:  chan[\integer \times \integer]
\ST
x! = t
\\t' = nil
\end{op}
\\
\begin{op}{switch}
\Delta (t)
\\x?:  chan[\integer \times \integer]
\ST
x? = t'
\end{op}
\\
\begin{op}{\text{if } t = talk \text{ then } talk \text{ else } undefined}
\Delta (m,vmId)
\\y?: \integer
\\z?: \integer
\ST
y? = m'
\\z? = vmId'
\end{op}
\end{class}
\caption{$VM$ class: \textit{unfixed operation schema.}}
\label{fig_oz_unfixed_operation_schema_shop}
\end{figure}
