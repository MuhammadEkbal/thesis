% mainfile: ../../Refinement.tex
The Object-Z ,shortly OZ, is a specifications language used to describe the data part of an entity. 
The main concepts in OZ are:
\begin{itemize}
\item \findex{Schema}: It can been seen as a set.
\item \findex{Class}: It can been seen as a grouping of a \findex{state schema} , \findex{initial state schema} and \findex{operation schemas}.
\end{itemize}

To illustrate the idea, consider the following example of modeling a vending machine $VM$:
\begin{itemize}
\item \textit{Class}: To model the vending machine we need to define a class $VM$, since OZ is an object oriented language. Syntactically, in OZ
a class definition is a named box as shown in \refFig{oz_VM_class}.
\begin{figure}[H]
\centering
\begin{class}{VM}
\end{class}
\caption{VM class.}
\label{oz_VM_class}
\end{figure}
\item \textit{State space}: Our vending machine sells coffee and tea with maximum amount 3 and minimum amount 0 for each. This means it has a state space, which can be seen as a set of valid states. The set of all valid states is:
\begin{itemize}
\item In Mathematics: $State\_Space=\{coffee,tea: \integer\mid (0 \leq  coffee \leq 3) \wedge
(0 \leq  tea \leq 3)\bullet(coffee,tea)\}  =\{\pair{0}{0},\dots,\pair{3}{3}\}$.
\item In OZ: The set $State\_Space$ can be described using a \findex{state schema}, which is a box without name added to the class box as shown in \refFig{oz_VM_state}.
\end{itemize}
\begin{figure}[H]
\centering
\begin{class}{VM}
\begin{state}
coffee, tea: \integer
\ST
0 \leq  coffee \leq 3
\\
0 \leq  tea \leq 3
\end{state} 
\end{class}
\caption{$VM$ class: adding the \textit{state schema}.}
\label{oz_VM_state}
\end{figure}
\item \textit{Initial state}: Our vending machine has an initial state with $coffee = 3$ and $tea = 3$. The set of all possible initial states, that respects those conditions is:  
\begin{itemize}
\item In Mathematics: $Initial\_States =\{coffee,tea: \integer\mid (0 \leq  coffee \leq 3) \wedge
(0 \leq  tea \leq 3)\wedge (coffee = 3) \wedge (tea = 3)\bullet(coffee,tea)\}  =\{\pair{3}{3}\}$.
\item In OZ: the set $Initial\_States$ can be described using a \findex{initial state schema}, which is a box named $INIT$ added to the class box  as shown in \refFig{oz_VM_init}.
\end{itemize}
\begin{figure}[H]
\centering
\begin{class}{VM}
\begin{state}
coffee, tea: \integer
\ST
0 \leq  coffee \leq 3
\\
0 \leq  tea \leq 3
\end{state} 
\\
\begin{init}
coffee = 3
\\tea = 3
\end{init} 
\end{class}
\caption{$VM$ class: adding the \textit{initial state schema}.}
\label{oz_VM_init}
\end{figure}
\item \textit{State transition}: When the vending machine sells a coffee, the amount of coffee should be decreased by one. This is a state transition.
the set of all possible state transitions when the selling coffee operation occurs is:
\begin{itemize}
\item In Mathematics: $decrease\_coffee =\{coffee,tea,coffee',tea': \integer\mid (0 \leq  coffee \leq 3) \wedge
(0 \leq  tea \leq 3)\wedge (0 \leq  coffee' \leq 3) \wedge (0 \leq  tea' \leq 3)\wedge (tea' = tea) \wedge (coffee' = coffee - 1) \bullet\pair{\pair{coffee}{tea}}{\pair{coffee'}{tea'}}\}  =\{\pair{\pair{3}{3}}{\pair{2}{3}},\dots,\pair{\pair{1}{0}}{\pair{0}{0}}\}$, where $(coffee,tea)$ represents the \findex{pre state} and $(coffee',tea')$ represents the \findex{post state} of a state transition.
\item In OZ: the set $decrease\_coffee$ can be described using an \findex{operation schema}, which is a box named with the operation name added to the class box as shown in \refFig{oz_VM_op}.
\end{itemize}
\begin{figure}[H]
\centering
\begin{class}{VM}
\\
\begin{state}
coffee, tea: \integer
\ST
0 \leq  coffee \leq 3
\\
0 \leq  tea \leq 3
\end{state} 
\\
\begin{init}
coffee = 3
\\tea = 3
\end{init} 
\\
\begin{op}{decrease\_coffee}
coffee, tea,coffee', tea': \integer
\ST
0 \leq  coffee \leq 3
\\
0 \leq  tea \leq 3
\\
0 \leq  coffee' \leq 3
\\
0 \leq  tea' \leq 3
\\
tea' = tea
\\
coffee' = coffee - 1
\end{op}
\end{class}
\caption{$VM$ class: adding the \textit{operation schema}.}
\label{oz_VM_op}
\end{figure}
OZ offers a nicer way to through using delta: 
\begin{itemize}
\item Operation schemas have a 	$\Delta$-list of those attributes
whose values may change. By convention, no 	$\Delta$-list means
no attribute changes value.
\item Every operation schema implicitly
includes the state schema in un-primed form s.
\end{itemize}
Thus, since the schema operation $decrease\_coffee$ specifies changes on the $coffee$ value only, we can write it as shown in \refFig{oz_VM_op_delta}. 
\begin{figure}[H]
\centering
\begin{class}{VM}
\\
\begin{state}
coffee, tea: \integer
\ST
0 \leq  coffee \leq 3
\\
0 \leq  tea \leq 3
\end{state} 
\\
\begin{init}
coffee = 3
\\tea = 3
\end{init} 
\\
\begin{op}{decrease\_coffee}
\Delta (coffee)
\ST
coffee' = coffee - 1
\end{op}
\end{class}
\caption{$VM$ class: \textit{operation schema using delta operator}.}
\label{oz_VM_op_delta}
\end{figure}
\end{itemize}
\begin{figure}[H]
\centering
\begin{class}{VA}
\\dec\_coffee, dec\_tea: channel
\\
\begin{state}
coffee, tea : \integer
\ST
coffee > 0
\\
tea > 0
\end{state} 
\\
\begin{init}
coffee = 3
\\tea = 3
\end{init} 
\\
\begin{op}{dec\_coffee}
\Delta (coffee)
\ST
coffee' = coffee - 1
\end{op}
\\
\begin{op}{dec\_tea}
\Delta (tea)
\ST
tea' = tea - 1
\end{op}
\end{class}
\caption{OZ VA.}
\label{oz_VA}
\end{figure}

\begin{figure}[H]
\centering
\begin{tikzpicture}[scale = .9]
\path [draw, help lines, opacity=.3]  (0,0) grid (3,3);
\draw [->] (-1,0) -- (4,0) node [anchor=south] {$cv$};
\draw [->] (0,-1) -- (0,4) node [anchor=west] {$tv$};
\path [ -,draw=gray, ultra thick, text=black, densely dashed] (0,3) node [anchor=south west] {} -| (3,0) node [anchor=south west] {} node [anchor=south west, midway] {$VM(3,3)$};
\fill (3,3)  circle[radius=2pt](A);
\fill (2,3)  circle[radius=2pt](B);
\draw [->,thick] (3,3) to[bend right=40] node {} (2,3);

\end{tikzpicture}
\caption{State Space.}
\label{state_space}
\end{figure}
