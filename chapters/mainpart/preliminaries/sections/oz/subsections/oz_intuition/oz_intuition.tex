% mainfile: ../../Refinement.tex
To explain the \oz{} intuitively, we will start by examining the vending machine example, then we will explain how to build a set mathematically, finally we will explain the main concepts in \oz{}. 

\subsubsection{Vending Machine:}
As a preperation, let us imagine that we have the task: specifying a vending machine.
\begin{itemize}
\item Let $cv$ be the amount of coffee, and $tv$ be the amount of tea.
\item Let $coffee$ be the selling coffee operation, and $tea$ be the selling tea operation.
\end{itemize}
the specifications are:
\begin{itemize}
\item It sells $coffee$ and $tea$, and the maximum amount for each of them is 3.
\item Its initial state is  $cv = 3$ and $tv = 3$.
\item When the operation $coffee$ or $tea$, then the amount should be decreased by one.
\end{itemize} 
The state space of the vending machine can be visualized as shown in \refFig{state_space}, where we see the initial state $VM(3,3)$. The arrow indicates a state transition decrementing the amount of coffee. Later in \textbf{Main concepts of \oz{}} we will learn how to write the specifications using \oz{} language notations.
\begin{figure}[H]
\centering
\begin{tikzpicture}[scale = .9]
\path [draw, help lines, opacity=.3]  (0,0) grid (3,3);
\draw [->] (-1,0) -- (4,0) node [anchor=south] {$cv$};
\draw [->] (0,-1) -- (0,4) node [anchor=west] {$tv$};
\path [ -,draw=gray, ultra thick, text=black, densely dashed] (0,3) node [anchor=south west] {} -| (3,0) node [anchor=south west] {} node [anchor=south west, midway] {$VM(3,3)$};
\fill (3,3)  circle[radius=2pt](A);
\fill (2,3)  circle[radius=2pt](B);
\draw [->,thick] (3,3) to[bend right=40] node {} (2,3);

\end{tikzpicture}
\caption{The state space of VM.}
\label{state_space}
\end{figure}

\subsubsection{Set building:} A set is a collection of things. For example: $\{5, 7, 11\}$ is a set.
But we can also build a set by describing what is in it using the following notation: \[\{Deklaration \mid predicate \bullet expression \}\]
 For example: $\{x: \integer \mid x \geq 0 \bullet x^{2}\}$ means \textit{the set of all squared x's, such that x is an integer and greater than 0}

	
\subsubsection{\textit{Main concepts of \oz{}}:} 
\label{main_concepts_oz} 
The main concepts of \oz{} are:
\begin{itemize}
\item \textit{Schema}: It can been seen as a set \cite{woodcock}.
\item \textit{Class}: It can been seen as a grouping of a \textit{state schema}, \textit{initial state schema} and \textit{operation schemas} \cite{kenji}. It represents the object oriented approach.
\end{itemize}
To illustrate those main concepts, consider the vending machine example denoted by $VM$:
\begin{itemize}
\item \textit{Class}: To model the vending machine we need to define a class $VM$. Syntactically, in \oz{}
a class definition is a named box as shown in \refFig{oz_vm_class}, where the dots \textquoteleft\dots\textquoteright\ refer to details explained next.
\begin{figure}[H]
\centering
\begin{class}{VM}
\end{class}
\caption{VM \textit{class.}}
\label{oz_vm_class}
\end{figure}

\item \textit{State space}: The state space of our vending machine can be seen as a set of all valid states. The set of all valid states is:
\begin{itemize}
\item In mathematics:
\begin{equation*}
\begin{aligned}
State\_Space ={} & \{cv,tv: \integer\mid (0 \leq  cv \leq 3) \wedge
(0 \leq  tv \leq 3)\bullet(cv,tv)\}\\
      & =\{\pair{0}{0},\dots,\pair{3}{3}\}
\end{aligned}
\end{equation*}
\item In \oz{}: The set $State\_Space$ can be described using a \findex[schema!state schema]{state schema}, which is a box without name added to the class box as shown in \refFig{oz_vm_state_schema}.
\end{itemize}
\begin{figure}[H]
\centering
\begin{class}{VM}
\begin{state}
cv, tv: \integer
\ST
0 \leq  cv \leq 3
\\
0 \leq  tv \leq 3
\end{state} 
\end{class}
\caption{$VM$ class: adding the \textit{state schema}.}
\label{oz_vm_state_schema}
\end{figure}

\item \textit{Initial state}: Our vending machine has an initial state with $cv = 3$ and $tv = 3$. The set of all possible initial states, that respects those conditions is:  
\begin{itemize}
\item In mathematics:
\begin{equation*}
\begin{aligned}
Initial\_States ={} & \{cv,tv: \integer\mid (0 \leq  cv \leq 3) \wedge (0 \leq  tv \leq 3)\\
      & \wedge (cv = 3) \wedge (tv = 3)\bullet(cv,tv)\} \\
      &  =\{\pair{3}{3}\}
\end{aligned}
\end{equation*}
\item In \oz{}: the set $Initial\_States$ can be described using a \findex[schema!initial state schema]{initial state schema}, which is a box named \textit{I\scriptsize{NIT}} added to the class box  as shown in \refFig{oz_vm_init_schema}.
\end{itemize}
\begin{figure}[H]
\centering
\begin{class}{VM}
\begin{state}
cv, tv: \integer
\ST
0 \leq  cv \leq 3
\\
0 \leq  tv \leq 3
\end{state} 
\\
\begin{init}
cv = 3
\\tv = 3
\end{init} 
\end{class}
\caption{$VM$ class: \textit{initial state schema}.}
\label{oz_vm_init_schema}
\end{figure}

\item \textit{State transition}: When the vending machine sells a coffee, the amount of coffee should be decreased by one. This is a state transition.
The set of all possible state transitions when the selling coffee operation occurs is:
\begin{itemize}
\item In mathematics:
\begin{equation*}
\begin{aligned}
coffee ={} & \{cv,tv,cv',tv': \integer\mid (0 \leq  cv \leq 3) \wedge (0 \leq  tv \leq 3)\\
      & \wedge (0 \leq  cv' \leq 3) \wedge (0 \leq  tv' \leq 3)\wedge (tv' = tv)  \\
      & \wedge (cv' = cv - 1) \bullet\pair{\pair{cv}{tv}}{\pair{cv'}{tv'}}\} \\
      & =\{\pair{\pair{3}{3}}{\pair{2}{3}},\dots,\pair{\pair{1}{0}}{\pair{0}{0}}\}
\end{aligned}
\end{equation*}
 where $(cv,tv)$ represents the \textit{pre state} and $(cv',tv')$ represents the \textit{post state} of a state transition.
\item In \oz{}: the set $coffee$ can be described using an \findex[schema!operation schema]{operation schema}, which is a box named with the operation name added to the class box as shown in \refFig{fig_oz_vm_operation_schema} left.
\end{itemize}

\begin{figure}[H]
\centering
\begin{sidebyside}[3]
\begin{class}{VM}
\\
\begin{state}
cv, tv: \integer
\ST
0 \leq  cv \leq 3
\\
0 \leq  tv \leq 3
\end{state} 
\\
\begin{init}
cv = 3
\\tv = 3
\end{init} 
\\
\begin{op}{coffee}
cv, tv,cv', tv': \integer
\ST
0 \leq  cv \leq 3
\\
0 \leq  tv \leq 3
\\
0 \leq  cv' \leq 3
\\
0 \leq  tv' \leq 3
\\
tv' = tv
\\
cv' = cv - 1
\end{op}
\end{class}
\nextside
\begin{class}{VM}
\\
\begin{state}
cv, tv: \integer
\ST
0 \leq  cv \leq 3
\\
0 \leq  tv \leq 3
\end{state} 
\\
\begin{init}
cv = 3
\\tv = 3
\end{init} 
\\
\begin{op}{coffee}
\Delta (cv)
\ST
cv' = cv - 1
\end{op}
\end{class}
\nextside
\begin{class}{VM}
\\
\begin{state}
cv, tv: \integer
\ST
0 \leq  cv \leq 3
\\
0 \leq  tv \leq 3
\end{state} 
\\
\begin{init}
cv = 3
\\tv = 3
\end{init} 
\\
\begin{op}{coffee}
\Delta (cv)
\ST
cv' = cv - 1
\end{op}
\\
\begin{op}{tea}
\Delta (tv)
\ST
tv' = tv - 1
\end{op}
\end{class}
\end{sidebyside}
\caption{VM class, operation schema.}
\label{fig_oz_vm_operation_schema}
\end{figure}


\oz{} offers a nicer way to write the operation schema using $\Delta$-list. In \oz{}:
\begin{itemize}
\item Operation schema has a $\Delta$-list of state variables
whose values may change. By convention, no $\Delta$-list means
no attribute changes value.
\item Operation schema implicitly
includes the state schema and a primed version of it.
\end{itemize}
Thus, since the schema operation $coffee$ specifies changes on the $coffee$ value only, we can write it as shown in \refFig{fig_oz_vm_operation_schema} middle. 
Similarly, the operation schema $tea$ is shown in \refFig{fig_oz_vm_operation_schema} right.
\end{itemize}

\subsubsection{\textit{Operation's input and output parameters}:} 
\label{operation_input_output_parameters} 
Some operations can have input and output parameters, just like a method in a programming language, where the method's parameters represent the \findex[schema!operation schema!input parameter]{input}, and the returned values represent the \findex[schema!operation schema!output parameter]{output}. To illustrate the idea let us extend our vending machine. The new $VM$ can talk to a shop sending a message to it. So it has a new operation $talk$ and a state variable $m$ representing the message to be sent.

The set of all possible state transitions when the $talk$ operation occurs is:
\begin{itemize}
\item In mathematics: 
\begin{equation*}
\begin{aligned}
talk ={} & \{cv,tv,message,cv',tv',message',y: \integer\mid (0 \leq  cv \leq 3) \\
      &  \wedge (0 \leq  tv \leq 3)\wedge (0 \leq  cv' \leq 3) \wedge (0 \leq  tv' \leq 3) \wedge (tv' = tv)  \\
      & \wedge (cv' = cv) \wedge (message' = message) \wedge (y = message)\bullet \\
      &  \pair{\tripple{cv}{tv}{message}}{\tripple{cv'}{tv'}{message'}}\} \\
      & =\{\pair{\tripple{3}{3}{1}}{\tripple{3}{3}{1}},\dots,\pair{\tripple{0}{0}{1}}{\tripple{0}{0}{1}}\}
\end{aligned}
\end{equation*}
\item In \oz{}: the set $talk$ can be described using an operation schema, as shown in \refFig{oz_vm_with_operation_input_output_parameters}. We can notice that this operation does not change any state variable's value, it just says that the value of the output parameter $y$, written as $y!$, must be equal to the value of the state variable $message$. For input parameter we use the $?$ symbol.
\begin{figure}[H]
\centering
\begin{class}{VM}
\\
\begin{state}
cv, tv, m: \integer
\ST
0 \leq  cv \leq 3
\\
0 \leq  tv \leq 3
\end{state} 
\\
\begin{init}
cv = 3
\\tv = 3
\\ m= 1
\end{init} 
\\
\begin{op}{coffee}
\Delta (cv)
\ST
cv' = cv - 1
\end{op}
\\
\begin{op}{tea}
\Delta (tv)
\ST
tv' = tv - 1
\end{op}
\\
\begin{op}{talk}
y!: \integer
\ST
y! = m
\end{op}
\end{class}
\caption{$VM$ class: \textit{$talk$ operation with output parameter.}}
\label{oz_vm_with_operation_input_output_parameters}
\end{figure}
\end{itemize}

\subsubsection{\findex[schema!instance reference]{Instance reference}:} 
\label{instance_reference} 
\oz{} is an object oriented approach, thus every instance of a class needs a unique identifier, i.e., a reference name to refer to it. In \oz{} this can be seen simply as state constant $self$ initialized with some $id$ when the instance is created. Furthermore, operations can share the instance identity through output or input the reference name $self$ as shown in \refFig{oz_vm_reference_name} in the operation $talk$.
\begin{figure}[H]
\centering
\begin{class}{VM(id: \integer)}
\\
\begin{state}
self, cv, tv, message: \integer
\ST
0 \leq  cv \leq 3
\\
0 \leq  tv \leq 3
\end{state} 
\\
\begin{init}
self = id
\\cv = 3
\\tv = 3
\\ message= 1
\end{init} 
\\
\begin{op}{coffee}
\Delta (cv)
\ST
cv' = cv - 1
\end{op}
\\
\begin{op}{tea}
\Delta (tv)
\ST
tv' = tv - 1
\end{op}
\\
\begin{op}{talk}
y!: \integer
\\z!: \integer
\ST
y! = message
\\z! = self
\end{op}
\end{class}
\caption{VM class, instance reference.}
\label{oz_vm_reference_name}
\end{figure}