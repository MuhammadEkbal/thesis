% mainfile: ../../Refinement.tex
To explain the \oz{} intuitively, we will start by examining the vending machine example, then we will explain how to build a set mathematically, finally we will explain the main concepts in \oz{}. 

\subsubsection{\findex{Vending Machine}:}
As a preperation, let us imagine that we have the task: specifying a vending machine.
\begin{itemize}
\item Let $cv$ be the ammount of coffee, and $tv$ be the amount of tea.
\item Let $coffee$ be the selling coffee operation, and $tea$ be the selling tea operation.
\end{itemize}
the specifications are:
\begin{itemize}
\item It sell $coffee$ and $tea$, and the maximum amount for each if them is 3.
\item It's initial state is  $cv = 3$ and $tv = 3$.
\item When the operation $coffee$ or $tea$,then the amount should be decreased by one.
\end{itemize} 
Those specifications describe the data of the vending machine and the changes on them. Thus, the state space of the vending machine can be visualized as shown in \refFig{state_space}, where we see the initial state $VM(3,3)$ and a state transition to $VM(2,3)$. Later in \textbf{Main concepts of \oz{}} we will learn how to write the specifications using \oz{} language notations.
\begin{figure}[H]
\centering
\begin{tikzpicture}[scale = .9]
\path [draw, help lines, opacity=.3]  (0,0) grid (3,3);
\draw [->] (-1,0) -- (4,0) node [anchor=south] {$cv$};
\draw [->] (0,-1) -- (0,4) node [anchor=west] {$tv$};
\path [ -,draw=gray, ultra thick, text=black, densely dashed] (0,3) node [anchor=south west] {} -| (3,0) node [anchor=south west] {} node [anchor=south west, midway] {$VM(3,3)$};
\fill (3,3)  circle[radius=2pt](A);
\fill (2,3)  circle[radius=2pt](B);
\draw [->,thick] (3,3) to[bend right=40] node {} (2,3);

\end{tikzpicture}
\caption{The state space of VM.}
\label{state_space}
\end{figure}

\subsubsection{\findex{Set building}:} A Set is a collection of things. For example: $\{5, 7, 11\}$ is a set.
But we can also build a set by describing what is in it.
Here is a simple example of building a set: $\{x \mid (x \in \integer) \wedge (x \geq 0) \}$, it says: \textit{the set of all x's, such that x is integer and greater than 0}. The last example can be written in a nicer way as follow: $\{x: \integer \mid x \geq 0 \}$. T have a more control about the elements in a set, we can write: $\{x: \integer \mid x \geq 0 \bullet x\}$,it says the same as the last example. The $expretion$ after $\bullet$ decides how should be the elements of the set be transformed. another example is: $\{x: \integer \mid x \geq 0 \bullet x^{2}\}$,it says: \textit{the set of all  squared x's, such that x is integer and greater than 0}. Thus, to build a set: we can use the following notation: $\{Deklaration \mid predicate \bullet expression \}$.
	
\subsubsection{\findex{Main concepts of \oz{}}:} 
\label{main_concepts_oz} 
The main concepts of \oz{} are:
\begin{itemize}
\item \findex{Schema}: It can been seen as a set \cite{woodcock}.
\item \findex{Class}: It can been seen as a grouping of a \findex{state schema}, \findex{initial state schema} and \findex{operation schemas} \cite{kenji}. It represents the object oriented approach 
\end{itemize}

To illustrate those main concepts, consider the vending machine example,denoted by $VM$:
\begin{itemize}
\item \textit{Class}: To model the vending machine we need to define a class $VM$. Syntactically, in \oz{}
a class definition is a named box as shown in \refFig{oz_vm_class}.
\begin{figure}[H]
\centering
\begin{class}{VM}
\end{class}
\caption{VM \textit{class.}}
\label{oz_vm_class}
\end{figure}

\item \textit{State space}: The state space of our vending machine can be seen as a set of all valid states. The set of all valid states is:
\begin{itemize}
\item In Mathematics: $State\_Space=\{cv,tv: \integer\mid (0 \leq  cv \leq 3) \wedge
(0 \leq  tv \leq 3)\bullet(cv,tv)\}  =\{\pair{0}{0},\dots,\pair{3}{3}\}$.
\item In \oz{}: The set $State\_Space$ can be described using a \findex{state schema}, which is a box without name added to the class box as shown in \refFig{oz_vm_state_schema}.
\end{itemize}
\begin{figure}[H]
\centering
\begin{class}{VM}
\begin{state}
cv, tv: \integer
\ST
0 \leq  cv \leq 3
\\
0 \leq  tv \leq 3
\end{state} 
\end{class}
\caption{$VM$ class: adding the \textit{state schema}.}
\label{oz_vm_state_schema}
\end{figure}

\item \textit{Initial state}: Our vending machine has an initial state with $cv = 3$ and $tv = 3$. The set of all possible initial states, that respects those conditions is:  
\begin{itemize}
\item In Mathematics: $Initial\_States =\{cv,tv: \integer\mid (0 \leq  cv \leq 3) \wedge
(0 \leq  tv \leq 3)\wedge (cv = 3) \wedge (tv = 3)\bullet(cv,tv)\}  =\{\pair{3}{3}\}$.
\item In \oz{}: the set $Initial\_States$ can be described using a \findex{initial state schema}, which is a box named $INIT$ added to the class box  as shown in \refFig{oz_vm_init_schema}.
\end{itemize}
\begin{figure}[H]
\centering
\begin{class}{VM}
\begin{state}
cv, tv: \integer
\ST
0 \leq  cv \leq 3
\\
0 \leq  tv \leq 3
\end{state} 
\\
\begin{init}
cv = 3
\\tv = 3
\end{init} 
\end{class}
\caption{$VM$ class: \textit{initial state schema}.}
\label{oz_vm_init_schema}
\end{figure}

\item \textit{State transition}: When the vending machine sells a coffee, the amount of coffee should be decreased by one. This is a state transition.
the set of all possible state transitions when the selling coffee operation occurs is:
\begin{itemize}
\item In Mathematics: $coffee =\{cv,tv,cv',tv': \integer\mid (0 \leq  cv \leq 3) \wedge
(0 \leq  tv \leq 3)\wedge (0 \leq  cv' \leq 3) \wedge (0 \leq  tv' \leq 3)\wedge (tv' = tv) \wedge (cv' = cv - 1) \bullet\pair{\pair{cv}{tv}}{\pair{cv'}{tv'}}\}  =\{\pair{\pair{3}{3}}{\pair{2}{3}},\dots,\pair{\pair{1}{0}}{\pair{0}{0}}\}$, where $(cv,tv)$ represents the \findex{pre state} and $(cv',tv')$ represents the \findex{post state} of a state transition.
\item In \oz{}: the set $coffee$ can be described using an \findex{operation schema}, which is a box named with the operation name added to the class box as shown in \refFig{oz_vm_op_schema}.
\end{itemize}
\begin{figure}[H]
\centering
\begin{class}{VM}
\\
\begin{state}
cv, tv: \integer
\ST
0 \leq  cv \leq 3
\\
0 \leq  tv \leq 3
\end{state} 
\\
\begin{init}
cv = 3
\\tv = 3
\end{init} 
\\
\begin{op}{coffee}
cv, tv,cv', tv': \integer
\ST
0 \leq  cv \leq 3
\\
0 \leq  tv \leq 3
\\
0 \leq  cv' \leq 3
\\
0 \leq  tv' \leq 3
\\
tv' = tv
\\
cv' = cv - 1
\end{op}
\end{class}
\caption{$VM$ class: adding the \textit{operation schema}.}
\label{oz_vm_op_schema}
\end{figure}

\oz{} offers a more nice way to write the operation schema using $\Delta$-list. In \oz{}:
\begin{itemize}
\item Operation schema has a $\Delta$-list of state variables
whose values may change. By convention, no $\Delta$-list means
no attribute changes value.
\item Operation schema implicitly
includes the state schema and a primed version of it.
\end{itemize}
Thus, since the schema operation $coffee$ specifies changes on the $coffee$ value only, we can write it as shown in \refFig{oz_vm_op_schema_delta}. 
\begin{figure}[H]
\centering
\begin{class}{VM}
\\
\begin{state}
coffee, tea: \integer
\ST
0 \leq  coffee \leq 3
\\
0 \leq  tea \leq 3
\end{state} 
\\
\begin{init}
coffee = 3
\\tea = 3
\end{init} 
\\
\begin{op}{decrease\_coffee}
\Delta (coffee)
\ST
coffee' = coffee - 1
\end{op}
\end{class}
\caption{$VM$ class: \textit{operation schema using delta operator}.}
\label{oz_vm_op_schema_delta}
\end{figure}

Finally, we can model the $tea$ operation schema as we did with $coffee$ to get the class $VM$ as shown in \refFig{oz_vm_with_all_specifications}, which represents all the specification we had of the vending machine written in \oz{} language.
\begin{figure}[H]
\centering
\begin{class}{VM}
\\
\begin{state}
coffee, tea: \integer
\ST
0 \leq  coffee \leq 3
\\
0 \leq  tea \leq 3
\end{state} 
\\
\begin{init}
coffee = 3
\\tea = 3
\end{init} 
\\
\begin{op}{decrease\_coffee}
\Delta (coffee)
\ST
coffee' = coffee - 1
\end{op}
\\
\begin{op}{decrease\_tea}
\Delta (tea)
\ST
tea' = tea - 1
\end{op}
\end{class}
\caption{$VM$ class: with all specifications.}
\label{oz_vm_with_all_specifications}
\end{figure}
\end{itemize}

\subsubsection{\findex{Operation's input and output parameters}:} 
\label{operation_input_output_parameters} 
Some operations can have input and output parameters, just like method in programming language, where the method's parameters represent the input, and the returned values represent the output. To illustrate the idea let us extend our vending machine .The new $VM$ can talk to a shop sending a message to it. So it has a new operation $talk$ and a state variable $m$ representing the message to be be sent.

The set of all possible state transitions when the $talk$ operation occurs is:
\begin{itemize}
\item In Mathematics: $talk =\{cv,tv,m,cv',tv',m',x: \integer\mid (0 \leq  cv \leq 3) \wedge
(0 \leq  tv \leq 3)\wedge (0 \leq  cv' \leq 3) \wedge (0 \leq  tv' \leq 3)\wedge (tv' = tv) \wedge (cv' = cv)  \wedge (m' = m)  \wedge (x = m)\bullet\pair{\tripple{cv}{tv}{m}}{\tripple{cv'}{tv'}{m'}}\}  =\{\pair{\tripple{3}{3}{1}}{\tripple{3}{3}{1}},\dots,\pair{\tripple{0}{0}{1}}{\tripple{0}{0}{1}}\}$.
\item In \oz{}: the set $talk$ can be described using an \findex{operation schema}, as shown in \refFig{oz_vm_with_operation_input_output_parameters}. We can notice that the operation doesn't change any state variable's value, it just says that the value of the output parameter $x$, written as $x!$, must be equal to the value of the state variable $m$. For input parameter use $?$ symbol.
\begin{figure}[H]
\centering
\begin{class}{VM}
\\
\begin{state}
cv, tv, m: \integer
\ST
0 \leq  cv \leq 3
\\
0 \leq  tv \leq 3
\end{state} 
\\
\begin{init}
cv = 3
\\tv = 3
\\ m= 1
\end{init} 
\\
\begin{op}{coffee}
\Delta (cv)
\ST
cv' = cv - 1
\end{op}
\\
\begin{op}{tea}
\Delta (tv)
\ST
tv' = tv - 1
\end{op}
\\
\begin{op}{talk}
y!: \integer
\ST
y! = m
\end{op}
\end{class}
\caption{$VM$ class: \textit{$talk$ operation with output parameter.}}
\label{oz_vm_with_operation_input_output_parameters}
\end{figure}
\end{itemize}

\subsubsection{\findex{Instance reference}:} 
\label{instance_reference} 
Since \oz{} is an object oriented approach, every instance of a class needs a reference to refer to it. In \oz{} the reference can be seen simply as sate constant. Furthermore, operations can output or input the reference name $self$ as shown in \refFig{oz_vm_reference_name} in the operation $talk$.
\begin{figure}[H]
\centering
\begin{class}{VM(id: \integer)}
\\
\begin{state}
self, cv, tv, message: \integer
\ST
0 \leq  cv \leq 3
\\
0 \leq  tv \leq 3
\end{state} 
\\
\begin{init}
self = id
\\cv = 3
\\tv = 3
\\ message= 1
\end{init} 
\\
\begin{op}{coffee}
\Delta (cv)
\ST
cv' = cv - 1
\end{op}
\\
\begin{op}{tea}
\Delta (tv)
\ST
tv' = tv - 1
\end{op}
\\
\begin{op}{talk}
y!: \integer
\\z!: \integer
\ST
y! = message
\\z! = self
\end{op}
\end{class}
\caption{VM class, instance reference.}
\label{oz_vm_reference_name}
\end{figure}