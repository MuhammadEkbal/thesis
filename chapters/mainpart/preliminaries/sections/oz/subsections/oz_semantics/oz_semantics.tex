To understand the operational semantics of \oz{} we will use  a labelled transition system LTS. Using this LTS we can investigate the state evolution of a \oz{} object . The definition of this LTS is adapted from \refDef{def_pi_trans_system} with some changes.

\begin{definition}[LTS of \oz{}]
\label{def_oz_trans_system}
The labelled transition system $(\stats,\traces)$ of \oz{} class states over the set of operations, has the valid states $\stats$ as its states and its transitions $\traces$ are those which can be inferred from the following rule: \\$\underline{\scriptstyle{OPER}}: PRE\_STATE \transs{operation} POST\_STATE$.
\end{definition}
An example of using the transition rule of this LTS is: drawing the transition graph of vending machine shown in  \refFig{oz_vm_with_all_specifications}. The transition graph is shown in \refFig{fig_VM_transition_graph}, where we show only a small part of it. The transitions $coffee$ and $tea$ refer to the operation schema $coffee$ and $tea$. Thus, using the LTS we can enumerate all possible states transitions of an \oz{} state.
\begin{figure}[H]%
    \centering
    \scalebox{0.85}{
\begin{tikzpicture}[->,>=stealth',shorten >=1pt,auto,node distance=5cm,
                    semithick]
  \tikzstyle{every state}=[]

  \node[state] (VM33)                    {$VM(3,3)$};
  \node[state] (VM23) 	[right of=VM33]  {$VM(2,3)$};
  \node[state] (VM13) 	[right of=VM23]  {$VM(1,3)$};
  \node[state] (VM03) 	[right of=VM13]  {$VM(0,3)$};
  \node[state] (VM32) 	[below of=VM33]  {$VM(3,2)$};

  \path (VM33) edge              node {$decrease\_coffee$} (VM23)
  		(VM23) edge              node {$decrease\_coffee$} (VM13)
  		(VM13) edge              node {$decrease\_coffee$} (VM03)
  		(VM33) edge              node {$decrease\_tea$} (VM32);
  		
  \draw [] (VM32) -- +(0:2cm);
  \draw [] (VM32) -- +(270:2cm); 
  \draw [] (VM23) -- +(270:2cm); 
  \draw [] (VM13) -- +(270:2cm); 
  \draw [] (VM03) -- +(270:2cm);   
\end{tikzpicture}
}
    \caption{$VM$ \textit{Transition graph}}%
    \label{fig_VM_transition_graph}%
\end{figure}

