In this work when we use \oz{} to model an operation, we restrict ourself to use only one type of parameters in the operation schema. Either input or output. This can be noticed in the operation schema $talk$ in:
\begin{itemize}
\item In \refFig{oz_vm_reference_name} all the parameters of the operation schema $talk$ are output parameters.
\item In \refFig{fig_oz_active_idle_shop}  all the parameters of the operation schema $talk$ are input parameters.
\end{itemize}
Why this restriction? Because a channel in the \picalc{} is \findex[channel!unidirectional]{unidirectional} pro reaction. In the next chapter we will map the \oz{} class constructs to  \picalc{} constructs, so we will map an \oz{} operation to an \picalc{} name, i.e., channel. In \picalc{} a processes can send or receive over a channel pro reaction, but not the both together.