% mainfile: ../../Refinement.tex
The \findex{strong simulation} is comparison of processes based on their behavior. To understand this let us start with a simple example:
Let $P\define\tau.\tau.\proczero$ and $Q\define\tau.\proczero$. We can notice that $P$ can do two $\tau$ transitions, but $Q$ can do only one. Thus $Q$ doesn't strongly simulates $P$. The word \findex{strongly} refers to the point that: the strong simulation comparison takes the internal transition $\tau$ into account. There is another kind of comparison called the  \findex{weak simulation}, which doesn't consider the internal transition $\tau$, but this kind of comparison is not considered in this thesis. The formal definition of the \findex{strong simulation} is given in \refDef{def_strong_sim}, which is adapted from \cite{gieseking} page 32 with some changes.


\begin{definition}[Strong simulation]
\label{def_strong_sim}
A relation $\mathcal{S}\subseteq\procs\times\procs$ is called a \index{simulation!strong}\findex[strong!simulation]{strong simulation}, if $(P,Q)\in\mathcal{S}$ implies that
\[\text{if } P \transs{\alpha} P' \text{ then }Q'\in\procs \text{ exists such that } Q \transs{\alpha} Q' \text{ and } (P',Q')\in\mathcal{S}.\]
\end{definition}

An example of checking the strong simulation is:
\\Let
\begin{itemize}
\item $P\define\procres{x}{(\procpar{\proccall{A_1}{x}}{\proccall{B_1}{x}})}$ 
\item $Q\define\procres{x}{(\procchoice{(\procpar{\proccall{A_1}{x}}{\proccall{B_1}{x}})}{\tau.Q})}$
\end{itemize}
where:
\begin{itemize}
\item $\procdef{A_1}{y}\define{}\out{y}{{}}.\proczero$
\item $\procdef{B_1}{z}\define{}\inp{z}{{}}.\proczero$
\end{itemize}

Intuitively, The behavior of $P$ and $Q$ can be illustrated using transition graphs as shown in \refFig{transition_graphs}. $Q$'s transition graph is the same as $P$'s, except one thing: $Q$ has a loop with label $\tau$. This loop is due to the $\tau$ transition in $Q$'s definition. Hence, we can notice that $Q$ can do all the transitions that $P$ can, plus an extra transition $\tau$. In other words $Q$ simulates $P$, but $P$ doesn't simulate $Q$.
\begin{figure}[H]%
    \centering
    \subfloat[$P$]{{\fcolorbox{black}{white}{
    \begin{tikzpicture}[->,>=stealth',shorten >=1pt,auto,node distance=3cm,
                    semithick]
  \tikzstyle{every state}=[]

  \node[state] (A)                    {$\procres{x}{(\procpar{\out{x}{}.\proczero}{\inp{x}{}.\proczero})}$};
  \node[state]         (B) [right of=A] {$\proczero$};
  \path (A) edge              node {$\tau$} (B);
\end{tikzpicture}
 }}}%
    \qquad
    \subfloat[$Q$]{{\fcolorbox{black}{white}{
\begin{tikzpicture}[->,>=stealth',shorten >=1pt,auto,node distance=4cm,
                    semithick]
  \tikzstyle{every state}=[]

  \node[state] (A)                    {$\procres{x}{(\procchoice{(\procpar{\out{x}{}.\proczero}{\inp{x}{}.\proczero})}{\tau.Q})}$};
  \node[state]         (B) [right of=A] {$\proczero$};
  \path (A) edge              node {$\tau$} (B)
        (A) edge [loop above] node {$\tau$} (A);
\end{tikzpicture}    
    }}}%
    \caption{Transition graphs}%
    \label{transition_graphs}%
\end{figure}

To check the strong simulation we can use \findex{ABC (Another Bisimilarity Checker)} \cite{abc}. ABC is a tool that checks simulation between  \picalc{} processes. \refFig{pi_simulation_ABC_code} shows the code listing of the process $P$ and $Q$ in ABC syntax.
\begin{figure}[ht!]
\lstinputlisting[backgroundcolor=\color{white}]{listings/pi_simulation_ABC_code.abc}
\caption{ABC code for $P$ and $Q$.}
\label{pi_simulation_ABC_code}
\end{figure}

\refFig{pi_simulation_ABC_outputPsQ} and \refFig{pi_simulation_ABC_outputQsP} shows the result of running \refFig{pi_simulation_ABC_code}, where x0 stands for x, since ABC renames the channels and messages names internally.

In \refFig{pi_simulation_ABC_outputPsQ} we can see the result of the command $lt\ P\ Q$, which checks if $Q$ strongly simulates $P$. The result is $yes$ and the simulation relation is $\mathcal{S} = \set{(\proczero,\proczero),(\procres{x}{(\procpar{\proccall{A_1}{x}}{\proccall{B_1}{x}})},\procres{x}{(\procchoice{(\procpar{\out{x}{}.\proczero}{\inp{x}{}.\proczero})}{\tau.Q})})}$, where x0 stands for x. The simulation relation $S$ says that: 
\begin{itemize}
\item The state $\proczero$ of $Q$ is as powerful as  $\proczero$ of $P$.
\item The state $\procres{x}{(\procchoice{(\procpar{\out{x}{}.\proczero}{\inp{x}{}.\proczero})}{\tau.Q})}$ of $Q$ is as powerful as  $\procres{x}{(\procpar{\out{x}{}.\proczero}{\inp{x}{}.\proczero})}$ of $P$.
\end{itemize}
Thus, $Q$ strongly simulates the behavior of $P$.
\begin{figure}[H]
\lstinputlisting[backgroundcolor=\color{white}]{listings/pi_simulation_ABC_outputPsQ.pi}
\caption{ABC output: check if $Q$ strongly simulates $P$.}
\label{pi_simulation_ABC_outputPsQ}
\end{figure}

In \refFig{pi_simulation_ABC_outputQsP} we can see the result of the command $lt\ Q\ P$, which checks if $P$ strongly simulates $Q$. The result is $no$, since, when:
\begin{itemize}
\item $Q$ is in the state $\procres{x}{(\procchoice{(\procpar{\out{x}{}.\proczero}{\inp{x}{}.\proczero})}{\tau.Q})}$.
\item $P$ is in the state $\procres{x}{(\procpar{\out{x}{}.\proczero}{\inp{x}{}.\proczero})}$.
\end{itemize}
, then:
\begin{itemize}
\item $Q$ can do a $\tau$ transition, which is the loop, to the state $\procres{x}{(\procchoice{(\procpar{\out{x}{}.\proczero}{\inp{x}{}.\proczero})}{\tau.Q})}$.
\item $P$ can do a $\tau$ transition, which is a reaction, to the state $\proczero$.
\end{itemize}
, then:
\begin{itemize}
\item $Q$ can do a $\tau$ transition, which is a reaction, to the state $\proczero$.
\item $P$ cannot go ahead, denoted by `` $*$ '', since it is in the state $\proczero$.
\end{itemize}

Thus, $P$ doesn't strongly simulates the behavior of $Q$.
\begin{figure}[ht!]
\lstinputlisting[backgroundcolor=\color{white}]{listings/pi_simulation_ABC_outputQsP.pi}
\caption{ABC output: check if $P$ strongly simulates $Q$.}
\label{pi_simulation_ABC_outputQsP}
\end{figure}
