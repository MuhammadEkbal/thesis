% mainfile: ../../Refinement.tex
For comparing processes based on their behavior there is already a notion called \index{bisimulation!strong}\findex[strong!bisimulation]{strong bisimulation} defined in \cite{milnerParrowWalker}..

\begin{definition}[Strong simulation]
\label{def_strong_sim}
A relation $\mathcal{S}\subseteq\procs\times\procs$ is called a \index{simulation!strong}\findex[strong!simulation]{strong simulation}, if $(P,Q)\in\mathcal{S}$ implies that
\[\text{if } P \transs{\alpha} P' \text{ then }Q'\in\procs \text{ exists such that } Q \transs{\alpha} Q' \text{ and } (P',Q')\in\mathcal{S}.\]
\end{definition}

Thereby, the internal as well as the external behavior are taken into account. To abstract from the invisible actions, the \index{bisimulation!weak}\findex[weak!bisimulation]{weak bisimulation} relates processes only according to their external behavior.

An example of checking the strong simulation is:
\\Let
\begin{itemize}
\item $P\define\procres{x}{(\procpar{\proccall{A_1}{x}}{\proccall{B_1}{x}})}$ 
\item $Q\define\procres{x}{(\procchoice{(\procpar{\out{x}{}.\proczero}{\inp{x}{}.\proczero})}{\tau.Q})}$
\end{itemize}
where:
\begin{itemize}
\item $\procdef{A_1}{y}\define{}\out{y}{{}}.\proczero$
\item $\procdef{B_1}{z}\define{}\inp{z}{{}}.\proczero$
\end{itemize}

Intuitively, The behavior of $P$ and $Q$ can be illustrated using transition graphs as shown in \refFig{transition_graphs}. $Q$'s transition graph is the same as $P$'s, except one thing: $Q$ has a loop with label $\tau$. This loop is due to the $\tau$ transition in $Q$'s definition. Hence, we can notice that $Q$ can do all the transitions that $P$ can, plus an extra transition $tau$. In other words $Q$ simulates $P$, but $P$ doesn't simulate $Q$.
\begin{figure}[H]%
\centering
\subcaptionbox{$P$}{\fbox{{
    \begin{tikzpicture}[->,>=stealth',shorten >=1pt,auto,node distance=3cm,
                    semithick]
  \tikzstyle{every state}=[]

  \node[state] (A)                    {$\procres{x}{(\procpar{\out{x}{}.\proczero}{\inp{x}{}.\proczero})}$};
  \node[state]         (B) [right of=A] {$\proczero$};
  \path (A) edge              node {$\tau$} (B);
\end{tikzpicture}
 }}}%
\hfill
\subcaptionbox{$Q$}{\fbox{{
\begin{tikzpicture}[->,>=stealth',shorten >=1pt,auto,node distance=4cm,
                    semithick]
  \tikzstyle{every state}=[]

  \node[state] (A)                    {$\procres{x}{(\procchoice{(\procpar{\out{x}{}.\proczero}{\inp{x}{}.\proczero})}{\tau.Q})}$};
  \node[state]         (B) [right of=A] {$\proczero$};
  \path (A) edge              node {$\tau$} (B)
        (A) edge [loop above] node {$\tau$} (A);
\end{tikzpicture}    
    }}}%
\caption{Transition graphs.}
\label{transition_graphs}
\end{figure}
To check the strong simulation we will use \findex{ABC (Another Bisimilarity Checker)}\cite{abc}. ABC is a tool that checks for simulation between  \picalc{} processes. \refFig{pi_simulation_ABC_code} shows the code listing of the process $P$ and $Q$ in ABC syntax.
\begin{figure}[ht!]
\lstinputlisting[backgroundcolor=\color{white}]{listings/pi_simulation_ABC_code.abc}
\caption{ABC code for $P$ and $Q$.}
\label{pi_simulation_ABC_code}
\end{figure}

\begin{figure}[ht!]
\lstinputlisting[backgroundcolor=\color{white}]{listings/pi_simulation_ABC_output.pi}
\caption{ABC code for $P$ and $Q$.}
\label{pi_simulation_ABC_code}
\end{figure}