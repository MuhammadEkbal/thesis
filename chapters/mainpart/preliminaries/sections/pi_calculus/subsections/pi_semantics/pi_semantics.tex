% mainfile: ../../Refinement.tex
We already explained the intuition of the semantics of the \picalc{} in \refSec{sec_pi_syntax}. In this section we formalize this intuition by presenting a definition of an \index{semantics!operational}\findex{operational semantics} of the \picalc{}. This definition yields a labeled transition system called the \index{transition system!early}\findex{early transition system} as presented in \cite{sangiorgi} in Table 1.5 on page 38. For this thesis, we applied only minor changes to the original presentation. %one presented in \cite{sangiorgi} for the fitting in this thesis.
In particular, we replace the replication rules by the \ecall{} rule, which is also defined in \cite{sangiorgi}, and omit the matching rule due to the definition of our syntax. Furthermore, the labels of the transitions are adapted to the notion of this thesis.

\subsubsection{Definition}



\begin{definition}
\label{def_early_trans_system}
The relation $\set[\alpha\in\actions]{\transs{\alpha}}\subseteq\procs_\alpha\times\procs_\alpha$ is called the \index{transition system!early}\findex{early transition system} and is defined by the rules in \refFig{fig_ts_early} apart from the omission of the rules \esumr{}, \eparr{}, \ecomr{} and \ecloser{} for a shorter presentation. They can be obtained from their related rules by interchanging the roles of $P$ and $Q$.
\end{definition}

% mainfile: ../../Refinement.tex
\begin{figure}[h!]
\begin{gather*}
\kalRule{E-TAU}{}{}{\ec{\tau.P} \transs{\tau} \ec{P}} \quad\quad \kalRule[\procdef{A}{\parl{w}}\define P]{E-CALL}{}{}{\ec{\proccall{A}{\parl{v}}} \tautrans \ec{P\subs{\parl{v}}{\parl{w}}}} \\\\
\kalRule{E-OUT}{}{}{\ec{\out{x}{y}.P} \transs{\out{x}{y}} \ec{P}} \quad\quad \kalRule{E-IN}{}{}{\ec{\inp{x}{z}.P} \intrans{x}{y} \ec{P\subs{y}{z}}} \\\\
\kalRule{E-SUM_L}{}{\ec{P} \transs{\alpha} \ec{P'}}{\ec{P+Q} \transs{\alpha} \ec{P'}} \quad\quad \kalRule[z\nin n(\alpha)]{E-RES}{}{\ec{P} \transs{\alpha} \ec{P'}}{\ec{\procres{z}{P}} \transs{\alpha} \ec{\procres{z}{P'}}} \\\\
\kalRule[\bn{\alpha}\cap\fn{Q}=\emptyset]{E-PAR_L}{}{\ec{P} \transs{\alpha} \ec{P'}}{\ec{\procpar{P}{Q}} \transs{\alpha} \ec{\procpar{P'}{Q}}} \\\\
\kalRule[z\neq x]{E-OPEN}{}{\ec{P} \outtrans{x}{z} \ec{P'}}{\ec{\procres{z}{P}} \bouttrans{x}{z} \ec{P'}} \quad\quad \kalRule{E-COM_L}{\ec{P} \outtrans{x}{y} \ec{P'}}{\ec{Q} \intrans{x}{y} \ec{Q'}}{\ec{\procpar{P}{Q}} \tautrans \ec{\procpar{P'}{Q'}}}\\\\
\kalRule[z\nin\fn{Q}]{E-CLOSE_L}{\ec{P} \bouttrans{x}{z} \ec{P'}}{\ec{Q} \intrans{x}{z} \ec{Q'}}{\ec{\procpar{P}{Q}} \tautrans \ec{\procres[a]{z}{\procpar{P'}{Q'}}}}
\end{gather*}
\caption{The \index{transition system!early}\findex{early transition system} \cite{sangiorgi}.}
\label{fig_ts_early}
\end{figure}



