% mainfile: ../../Refinement.tex
To understand the operational semantics of \picalc{} we will use  a labelled transition system LTS. Using this LTS we can investigate \picalc{} process evolution. The definition of LTS is adapted from \cite{milner} pages 39\footnote{Transition Rules: LTS for concurrent processes not for \picalc{} processes.}, 91\footnote{Reaction Rules: no labels and no LTS.}, 132\footnote{Commitment Rules: abstractions and concretions are out of this thesis's scope.} with some changes.

\begin{definition}[LTS of \picalc{}]
\label{def_trans_system}
The labelled transition system $(\procs,\traces)$ of \picalc{} processes over the action set $\actions$ has the process expressions $\procs$ as its states, and its transitions $\traces$ are those which can be inferred from the rules in \refFig{fig_transition_rules}.
The rule REACT is the most important one. It shows the process evolution when a reaction occurs. The reaction requires two complementary transitions $P \transs{\out{x}{\vec{y}}} P'$ and $Q \transs{\inp{x}{\vec{z}}} Q'$, we call them commitments. so the process $P$ takes a commitment to take part in the reaction, and so does $Q$.
% mainfile: ../../Refinement.tex
\begin{figure}[H]
\begin{gather*}
\underline{\scriptstyle{OUT}}: \out{x}{\vec{y}}.P \transs{\out{x}{\vec{y}}} P
\quad\quad
\underline{\scriptstyle{IN}}: \inp{x}{\vec{y}}.P \transs{\inp{x}{\vec{y}}} P
\\\\
\underline{\scriptstyle{TAU}}: \tau.P \transs{\tau} P
\quad\quad
\underline{\scriptstyle{SUM}}: \procchoice{\alpha.P}{\procsum} \transs{\alpha} P
\\\\
\kalRule[]{L\_PAR}{}{P \transs{\alpha} P'}{\procpar{P}{Q} \transs{\alpha} \procpar{P'}{Q}}
\quad\quad
\kalRule[]{R\_PAR}{}{Q \transs{\alpha} Q'}{\procpar{P}{Q} \transs{\alpha} \procpar{P}{Q'}}
\\\\
\kalRule[if\ \alpha \nin \{\overline{x},x\}]{RESTRICTION}{}{P \transs{\alpha} P'}{\procres{x}{P} \transs{\alpha} \procres{x}{P'}}
\\\\
\kalRule[if\ \procdef{A}{\vec{z}}\define{}P]{PROCESS\_CALL}{}{\substitue{\vec{y}}{\vec{z}}P \transs{\alpha} P'}{\proccall{A}{\vec{y}} \transs{\alpha} P'}
\\\\
\kalRule[]{REACT}{P \transs{\out{x}{\vec{y}}} P'}{Q \transs{\inp{x}{\vec{z}}} Q'}{\procpar{P}{Q} \transs{\tau} \procpar{P'}{\substitue{\vec{y}}{\vec{z}}Q'}}
\end{gather*}
\caption{The transition rules \cite{milner}.}
\label{fig_transition_rules}
\end{figure}


\end{definition}


An example of using the transition rules of this LTS to infer a transition is: Let $P\define\procres{x}{(\procpar{\proccall{A_1}{x}}{\proccall{B_1}{x}})}$, where: $\procdef{A_1}{y}\define{}\out{y}{{}}.\proccall{A_2}{y}$ and $\procdef{B_1}{z}\define{}\inp{z}{{}}.\proccall{B_2}{z}$. $P$ can do the transition $\procres{x}{(\procpar{\proccall{A_1}{x}}{\proccall{B_1}{x}})} \transs{\tau} \procres{x}{(\procpar{\proccall{A_2}{x}}{\proccall{B_2}{x}})}$, which is a reaction. The inference tree of this transition is shown in \refFig{fig_inference_tree}. Thus, using the LTS we can enumerate sll possible transitions of a \picalc{} process.
% mainfile: ../../piOZ.tex
\begin{figure}[htbp]
\scalebox{0.93}{
\begin{prooftree} 
\infer0[by OUT]{\out{x}{{}}.\proccall{A_2}{x} \transs{\out{x}{{}}} \proccall{A_2}{x}}
\infer1[by PROCESS CALL]{\proccall{A_1}{x} \transs{\out{x}{{}}} \proccall{A_2}{x}}
\infer0[by IN]{\inp{x}{{}}.\proccall{B_2}{x} \transs{\inp{x}{{}}} \proccall{B_2}{x}}
\infer1[by PROCESS CALL]{\proccall{B_1}{x} \transs{\inp{x}{{}}} \proccall{B_2}{x}}
\infer2[by REACT]{ \procpar{\proccall{A_1}{x}}{\proccall{B_1}{x}} \transs{\tau} \procpar{\proccall{A_2}{x}}{\proccall{B_2}{x}} }
\infer1[by RESTRICTION]{ \procres{x}{(\procpar{\proccall{A_1}{x}}{\proccall{B_1}{x}})} \transs{\tau} \procres{x}{(\procpar{\proccall{A_2}{x}}{\proccall{B_2}{x}})} }                       
\end{prooftree}
}
\caption{The inference tree of the reaction of the process $P$.}
\label{fig_inference_tree}
\end{figure}


