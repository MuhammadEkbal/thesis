% mainfile: ../../Refinement.tex
\begin{definition}[Process syntax]
\label{def_process_syntax}
The syntax of a \picalc{} process \textit{P} is defined by: 
\begin{align*}
 P & \syntdef \procsum \ebnf \procpar{P_1}{P_2} \ebnf \procres{\vec{y}}{P} \ebnf \proccall{A}{\vec{v}}
\end{align*}
where:
\begin{itemize}
\item $\procsum$ is the guarded sum.
\item $\procpar{P_1}{P_2}$ is the parallel composition of processes.
\item $\procres{\vec{y}}{P}$ is the restriction of the scope of the names $\vec{y}$ to the process $P$
\item $\proccall{A}{\vec{v}}$ is a process call. 
\end{itemize}
\end{definition}

\subsubsection{\findex[guarded sum]{Guarded sum}:} The guarded sum is the choice between multiple guarded processes. If the guard of one process took place, other guarded processes will be discarded. For example, the processes: $\procchoice{\inp{x}{}.P_1}{\inp{y}{}.P_2}$ will evolve to the process $P_1$ if the guard $\inp{x}{}$ occurred.

Furthermore, The process $\proczero$ is called the \findex[process!stop]{stop process} stands for the process that can do nothing. It can be omitted.
\subsubsection{\findex[guard]{Guard}:} The guard is also called \findex[process!action prefix]{action prefix} and denoted by $\pi$. It's syntax is defined by:
\begin{definition}[Action prefix syntax]
\label{def_prefix_syntax}
\begin{align*}
 \pi \syntdef \out{x}{\vec{y}} \ebnf \inp{x}{\vec{y}} \ebnf \tau
\end{align*}
where:
\begin{itemize}
\item $\out{x}{\vec{y}}$\footnote{$\out{x}{{}}$ means: send a signal via $x$. $\out{x}{{y}}$ means: send the name $y$ via $x$.  $\out{x}{{\vec{y}}}$ means: send the sequence $\vec{y}$ via $x$.} represents the action: send $\vec{y}$ via the channel $x$.
\item $\inp{x}{\vec{y}}$\footnote{$\inp{x}{{}}$ means: receive a signal via $x$. $\inp{x}{{y}}$ means: receive any name $y$ via $x$.  $\inp{x}{{\vec{y}}}$ means: receive any sequence $\vec{y}$ via $x$. ``$y$ here plays the role of parameter''} represents the action: receive $\vec{y}$ via the channel $x$.
\item $\tau$ represents an internal non observable action.
\end{itemize}
\end{definition}

We use the symbols $\alpha$ or $\beta$ to denote an action. Furthermore, we call $x$ the subject and $\vec{y}$ the object of a an action $\out{x}{\vec{y}}$ or $\inp{x}{\vec{y}}$. The functions  $\sub{\alpha}$ and $\sub{\alpha}$ returns the subject and object of an action $\alpha$. More formally:
\[\sub{\alpha} \text{: returns the channel name through which the exchange occurs.}\]
\[\obj{\alpha} \text{: return the exchanged names across the channe.l}\]

The set of all actions is defined as $\actions\define\outA\cup\inA\cup\set{\tau}$, where:
\begin{itemize}
\item $\outA$ is the set of all output actions, defined as $\outA\define\set[x\in\names]{\out{x}{\vec{y}}}$.
\item $\inA$ is the set of all input actions, defined as $\inA\define\set[x\in\names]{\inp{x}{\vec{y}}}$.
\end{itemize}
\subsubsection{\findex[parallel composition]{Parallel composition}:}
The parallel composition operator $\procpar{}{}$ represents the concept of concurrency in the \picalc{}, where two processes can evolve in concurrent.It represents an interleaving behavior of the concurrency.
For example let:  $P\define\procpar{P_1}{(\procpar{P_2}{P_3})}$ where: $P_1\define\inp{x}{y}.Q_1$ , $P_2\define\out{x}{y}.Q_2$ and $P_3\define\inp{x}{y}.Q_3$. So $P\define\procpar{\inp{x}{y}.Q_1}{(\procpar{\out{x}{y}.Q_2}{\inp{x}{y}.Q_3})}$.
Possible evolution cases of $P$ are:
\begin{itemize}
\item $\procpar{P_1}{(\procpar{Q_2}{Q_3})}$. $P_2$ sends $y$ via $x$ to $P_3$.
\item $\procpar{Q_1}{(\procpar{Q_2}{P_3})}$. $P_2$ sends $y$ via $x$ to $P_1$.
\end{itemize}

The example above illustrated the privacy nature of the parallel operator in the \picalc{}. A process can via a channel communicate with only one process pro time, i.e., the channel represents a binary synchronization. $P_2$ cannot communicate with both $P_1$, $P_3$ in the same time, while in \gls{CSP} a process can communicate with multiple processes in the same time via the same channel by sending multiple copies of the same message, i.e., in CSP the channel represents a multiple synchronization.


\subsubsection{\findex[restriction]{Restriction}:}

The expression $\procres{\vec{y}}{P}$ binds the names $\vec{y}$ to the process $P$. In other words: the visibility scope of the  names $\vec{y}$ is restricted to the process $P$. It is similar to declaring a private variable in programming languages. Thus the names $\vec{y}$ are not visible outside $P$ and $P$ cannot use them to communicate with outside. For example, let $P\define\procpar{P_1}{P_2}$ where: $P_1\define\procres{y}{\out{y}{z}.Q_1}$ and $P_2\define\inp{y}{z}.Q_2$. The process $P$ cannot evolute to $\procpar{Q_1}{Q_2}$, since the name $y$ in $P_1$ is only visible inside it, i.e., from the $P_2$'s point of view $P_1$ doesn't have a channel called $y$. This takes us to the definition of the 
\index{name}\findex[name]{name}
\index{name!free}\findex[name!free]{free names}
and 
\index{name!bound}\findex[name!bound]{name bound}.

\begin{definition}[Bound names]
\label{def_bound_names}
 are all the restricted names in a process.
\end{definition}
\begin{definition}[Free names]
\label{def_free_names}
 are all the name that occur in a process except the bound names.
\end{definition}

For example, let $P_1\define\procres{x}{\out{x}{y}.P_2}$ where $P_2\define\procres{z}{\out{x}{z}.P_3}$. The name $x$ is bound in $P_1$ but free in $P_2$.


\subsubsection{\findex[process!call]{Process call}:}
\label{subsubsection_process_call}

Let $P$ be a process and let $A$ be a process identifier.To be able to use the process $P$ recursively we use the process identifier $A$ as follow: $\procdef{A}{\vec{w}}\define{}P$. Thus , when we write $\proccall{A}{\vec{v}}$ we are using the identifier $A$ to call the process $P$ with replacing the names $\vec{w}$ in $P$ with the names $\vec{v}$. This replacement is called the $\alpha$-conversion

For example, let $P\define\out{w}{y}.\proczero$ and let $\procdef{A}{w}\define{}P$ be the recursive definition of the process $P$, then the behavior of $\proccall{A}{v}$is equlivant to $\out{v}{y}.\proczero$ 

