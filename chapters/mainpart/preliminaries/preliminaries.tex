% mainfile: ../../Refinement.tex
At the heart of the refinement of \findex{\picalc{}} processes is the theory of \findex[sequence]{sequences}. Thus, in this chapter, we recall the model of sequences to gain a formal construct to handle ordered elements.% in an intuitive way.

Furthermore, we introduce the \picalc{} and investigate its behavior properly. In particular, we carefully explain the operational semantics of \picalc{} processes, since its peculiarities induce the characteristics of the refinement and its properties. Moreover, we discuss why we choose this particular operational semantics for the following work in this thesis and compare it to other semantics.

The majority of those definitions and notions can, for example, be found in \cite{milner,sangiorgi}.

As mathematical notations, we consider the natural numbers starting with zero ($\N=\set{0,1,2,\ldots}$) and use $\fatsemi$ as the composition of relations. Furthermore, we denote $R^*$ as the reflexive and transitive closure of a relation $R$.


\section{The \texorpdfstring{$\pi$}{pi}-calculus}
\label{sec_pi_calculus}
% mainfile: ../../Refinement.tex
The \findex[\picalc{}|(]{\picalc{}} is a process algebra that can be used to describe the behavior. This section introduces the polyadic pure version of the \picalc{} as decriped in \cite{milner}. 


\subsection{Syntax}
\label{sec_pi_syntax}
% mainfile: ../../Refinement.tex
\begin{definition}[Process syntax]
\label{def_process_syntax}
The syntax of a \picalc{} process \findex[process]{P} is defined by: 
\begin{align*}
 P & \syntdef \procsum \ebnf \procpar{P_1}{P_2} \ebnf \procres{\vec{y}}{P} \ebnf \proccall{A}{\vec{v}}
\end{align*}
where:
\begin{itemize}
\item $\procsum$ is the guarded sum.
\item $\procpar{P_1}{P_2}$ is the parallel composition of processes.
\item $\procres{\vec{y}}{P}$ is the restriction of the scope of the names $\vec{y}$ to the process $P$
\item $\proccall{A}{\vec{v}}$ is a process call. 
\end{itemize}
\end{definition}

\subsubsection{\findex{Guarded sum}:} The guarded sum is the \findex{choice} between multiple guarded processes. If the guard of one process took place, other guarded processes will be discarded. For example, the processes: $\procchoice{\inp{x}{}.P_1}{\inp{y}{}.P_2}$ will evolve to the process $P_1$ if the guard $\inp{x}{}$ occurred.

Furthermore, The process $\proczero$ is called the \index{process!stop}\findex{stop process} or \findex{inaction} and stands for the process that can do nothing. It can be omitted.
\subsubsection{\findex{Guard}:} The guard is also called \findex[process]{action prefix} and denoted by $\pi$. It's syntax is defined by:
\begin{definition}[Action prefix syntax]
\label{def_prefix_syntax}
\begin{align*}
 \pi \syntdef \out{x}{\vec{y}} \ebnf \inp{x}{\vec{y}} \ebnf \tau
\end{align*}
where:
\begin{itemize}
\item $\out{x}{\vec{y}}$ \footnote{$\out{x}{{}}$ means: send a signal via $x$. $\out{x}{{y}}$ means: send the name $y$ via $x$.  $\out{x}{{\vec{y}}}$ means: send the sequence $\vec{y}$ via $x$.} represents the action: send $\vec{y}$ via the channel $x$.
\item $\inp{x}{\vec{y}}$ \footnote{$\inp{x}{{}}$ means: receive a signal via $x$. $\inp{x}{{y}}$ means: receive any name $y$ via $x$.  $\inp{x}{{\vec{y}}}$ means: receive any sequence $\vec{y}$ via $x$. ``$y$ here plays the role of parameter''} represents the action: receive $\vec{y}$ via the channel $x$.
\item $\tau$ represents an internal non observable action.
\end{itemize}
\end{definition}
The set of all \index{action}\findex[action]{actions} is defined as $\actions\define\outA\cup\inA\cup\set{\tau}$, where:
\begin{itemize}
\item $\outA$ is the set of all \index{action!output}\findex[output!action]{output actions}, defined as $\outA\define\set[x\in\names]{\out{x}{\vec{y}}}$.
\item $\inA$ the set of all \index{action!input}\findex[input!action]{input actions}, defined as $\inA\define\set[x\in\names]{\inp{x}{\vec{y}}}$.
\end{itemize}
\subsubsection{\findex{Parallel composition}:}
The parallel composition operator $\procpar{}{}$ represents the concept of concurrency in the \picalc{}, where two processes can evolve in concurrent.It represents an interleaving behavior of the concurrency.
For example let:  $P\define\procpar{P_1}{(\procpar{P_2}{P_3})}$ where: $P_1\define\inp{x}{y}.Q_1$ , $P_2\define\out{x}{y}.Q_2$ and $P_3\define\inp{x}{y}.Q_3$. So $P\define\procpar{\inp{x}{y}.Q_1}{(\procpar{\out{x}{y}.Q_2}{\inp{x}{y}.Q_3})}$.
Possible evolution cases of $P$ are:
\begin{itemize}
\item $\procpar{P_1}{(\procpar{Q_2}{Q_3})}$. $P_2$ sends $y$ via $x$ to $P_3$.
\item $\procpar{Q_1}{(\procpar{Q_2}{P_3})}$. $P_2$ sends $y$ via $x$ to $P_1$.
\end{itemize}

The example above illustrated the privacy nature of the parallel operator in the \picalc{}. A process can via a channel communicate with only one process pro time, i.e., the channel represents a binary synchronization. $P_2$ cannot communicate with both $P_1$, $P_3$ in the same time, while in \gls{CSP} a process can communicate with multiple processes in the same time via the same channel by sending multiple copies of the same message, i.e., in CSP the channel represents a multiple synchronization.


\subsubsection{\findex{Restriction}:}

The expression $\procres{\vec{y}}{P}$ binds the names $\vec{y}$ to the process $P$. In other words: the visibility scope of the  names $\vec{y}$ is restricted to the process $P$. It is similar to declaring a private variable in programming languages. Thus the names $\vec{y}$ are not visible outside $P$ and $P$ cannot use them to communicate with outside. For example, let $P\define\procpar{P_1}{P_2}$ where: $P_1\define\procres{y}{\out{y}{z}.Q_1}$ and $P_2\define\inp{y}{z}.Q_2$. The process $P$ cannot evolute to $\procpar{Q_1}{Q_2}$, since the name $y$ in $P_1$ is only visible inside it, i.e., from the $P_2$'s point of view $P_1$ doesn't have a channel called $y$. This takes us to the definition of the Bound and free names.

\begin{definition}[Free names]
\label{def_bound_names}
 are all the restricted names in a process.
\end{definition}
\begin{definition}[Bound names]
\label{def_free_names}
 are all the name that occur in a process except the bound names.
\end{definition}

For example, let $P_1\define\procres{x}{\out{x}{y}.P_2}$ where $P_2\define\procres{z}{\out{x}{z}.P_3}$. The name $x$ is bound in $P_1$ but free in $P_2$.


\subsubsection{\findex{Process call}:}
\label{subsubsection_process_call}

Let $P$ be a process and let $A$ be a process identifier.To be able to use the process $P$ recursively we use the process identifier $A$ as follow: $\procdef{A}{\vec{w}}\define{}P$. Thus , when we write $\proccall{A}{\vec{v}}$ we are using the identifier $A$ to call the process $P$ with replacing the names $\vec{w}$ in $P$ with the names $\vec{v}$. This replacement is called the $\alpha$-conversion

For example, let $P\define\out{w}{y}.\proczero$ and let $\procdef{A}{w}\define{}P$ be the recursive definition of the process $P$, then the behavior of $\proccall{A}{v}$is equlivant to $\out{v}{y}.\proczero$ 



\newpage %%%%%%%%%%%%%%% NEWPAGE!


\section{The OZ}
\label{sec_oz}
% mainfile: ../../Refinement.tex
The Object-Z ,shortly \oz{}, is a specifications language used to describe the data part of an entity.
\subsection{Intuition}
\label{sec_oz_intuition}
% mainfile: ../../Refinement.tex
To explain the \oz{} intuitively, we will start by examining the vending machine example, then we will explain how to build a set mathematically, finally we will explain the main concepts in \oz{}. 

\subsubsection{Vending Machine:}
As a preperation, let us imagine that we have the task: specifying a vending machine.
\begin{itemize}
\item Let $cv$ be the ammount of coffee, and $tv$ be the amount of tea.
\item Let $coffee$ be the selling coffee operation, and $tea$ be the selling tea operation.
\end{itemize}
the specifications are:
\begin{itemize}
\item It sell $coffee$ and $tea$, and the maximum amount for each if them is 3.
\item It's initial state is  $cv = 3$ and $tv = 3$.
\item When the operation $coffee$ or $tea$,then the amount should be decreased by one.
\end{itemize} 
The state space of the vending machine can be visualized as shown in \refFig{state_space}, where we see the initial state $VM(3,3)$. The arrow indicates a state transition decrementing the amount of coffee. Later in \textbf{Main concepts of \oz{}} we will learn how to write the specifications using \oz{} language notations.
\begin{figure}[H]
\centering
\begin{tikzpicture}[scale = .9]
\path [draw, help lines, opacity=.3]  (0,0) grid (3,3);
\draw [->] (-1,0) -- (4,0) node [anchor=south] {$cv$};
\draw [->] (0,-1) -- (0,4) node [anchor=west] {$tv$};
\path [ -,draw=gray, ultra thick, text=black, densely dashed] (0,3) node [anchor=south west] {} -| (3,0) node [anchor=south west] {} node [anchor=south west, midway] {$VM(3,3)$};
\fill (3,3)  circle[radius=2pt](A);
\fill (2,3)  circle[radius=2pt](B);
\draw [->,thick] (3,3) to[bend right=40] node {} (2,3);

\end{tikzpicture}
\caption{The state space of VM.}
\label{state_space}
\end{figure}

\subsubsection{Set building:} A set is a collection of things. For example: $\{5, 7, 11\}$ is a set.
But we can also build a set by describing what is in it using the following notation: \[\{Deklaration \mid predicate \bullet expression \}\].  For example: $\{x: \integer \mid x \geq 0 \bullet x^{2}\}$ means \textit{the set of all  squared x's, such that x is integer and greater than 0}

	
\subsubsection{\textit{Main concepts of \oz{}}:} 
\label{main_concepts_oz} 
The main concepts of \oz{} are:
\begin{itemize}
\item \textit{Schema}: It can been seen as a set \cite{woodcock}.
\item \textit{Class}: It can been seen as a grouping of a \textit{state schema}, \textit{initial state schema} and \textit{operation schemas} \cite{kenji}. It represents the object oriented approach 
\end{itemize}
To illustrate those main concepts, consider the vending machine example denoted by $VM$:
\begin{itemize}
\item \textit{Class}: To model the vending machine we need to define a class $VM$. Syntactically, in \oz{}
a class definition is a named box as shown in \refFig{oz_vm_class}, where the dots \dots refer to details explained next.
\begin{figure}[H]
\centering
\begin{class}{VM}
\end{class}
\caption{VM \textit{class.}}
\label{oz_vm_class}
\end{figure}

\item \textit{State space}: The state space of our vending machine can be seen as a set of all valid states. The set of all valid states is:
\begin{itemize}
\item In mathematics:
\begin{equation*}
\begin{aligned}
State\_Space ={} & \{cv,tv: \integer\mid (0 \leq  cv \leq 3) \wedge
(0 \leq  tv \leq 3)\bullet(cv,tv)\}\\
      & =\{\pair{0}{0},\dots,\pair{3}{3}\}
\end{aligned}
\end{equation*}
\item In \oz{}: The set $State\_Space$ can be described using a \findex[schema!state schema]{state schema}, which is a box without name added to the class box as shown in \refFig{oz_vm_state_schema}.
\end{itemize}
\begin{figure}[H]
\centering
\begin{class}{VM}
\begin{state}
cv, tv: \integer
\ST
0 \leq  cv \leq 3
\\
0 \leq  tv \leq 3
\end{state} 
\end{class}
\caption{$VM$ class: adding the \textit{state schema}.}
\label{oz_vm_state_schema}
\end{figure}

\item \textit{Initial state}: Our vending machine has an initial state with $cv = 3$ and $tv = 3$. The set of all possible initial states, that respects those conditions is:  
\begin{itemize}
\item In mathematics:
\begin{equation*}
\begin{aligned}
Initial\_States ={} & \{cv,tv: \integer\mid (0 \leq  cv \leq 3) \wedge (0 \leq  tv \leq 3)\\
      & \wedge (cv = 3) \wedge (tv = 3)\bullet(cv,tv)\} \\
      &  =\{\pair{3}{3}\}
\end{aligned}
\end{equation*}
\item In \oz{}: the set $Initial\_States$ can be described using a \findex[schema!initial state schema]{initial state schema}, which is a box named $INIT$ added to the class box  as shown in \refFig{oz_vm_init_schema}.
\end{itemize}
\begin{figure}[H]
\centering
\begin{class}{VM}
\begin{state}
cv, tv: \integer
\ST
0 \leq  cv \leq 3
\\
0 \leq  tv \leq 3
\end{state} 
\\
\begin{init}
cv = 3
\\tv = 3
\end{init} 
\end{class}
\caption{$VM$ class: \textit{initial state schema}.}
\label{oz_vm_init_schema}
\end{figure}

\item \textit{State transition}: When the vending machine sells a coffee, the amount of coffee should be decreased by one. This is a state transition.
The set of all possible state transitions when the selling coffee operation occurs is:
\begin{itemize}
\item In mathematics:
\begin{equation*}
\begin{aligned}
coffee ={} & \{cv,tv,cv',tv': \integer\mid (0 \leq  cv \leq 3) \wedge (0 \leq  tv \leq 3)\\
      & \wedge (0 \leq  cv' \leq 3) \wedge (0 \leq  tv' \leq 3)\wedge (tv' = tv)  \\
      & \wedge (cv' = cv - 1) \bullet\pair{\pair{cv}{tv}}{\pair{cv'}{tv'}}\} \\
      & =\{\pair{\pair{3}{3}}{\pair{2}{3}},\dots,\pair{\pair{1}{0}}{\pair{0}{0}}\}
\end{aligned}
\end{equation*}
 where $(cv,tv)$ represents the \textit{pre state} and $(cv',tv')$ represents the \textit{post state} of a state transition.
\item In \oz{}: the set $coffee$ can be described using an \findex[schema!operation schema]{operation schema}, which is a box named with the operation name added to the class box as shown in \refFig{fig_oz_vm_operation_schema} left.
\end{itemize}

\begin{figure}[H]
\centering
\begin{sidebyside}[3]
\begin{class}{VM}
\\
\begin{state}
cv, tv: \integer
\ST
0 \leq  cv \leq 3
\\
0 \leq  tv \leq 3
\end{state} 
\\
\begin{init}
cv = 3
\\tv = 3
\end{init} 
\\
\begin{op}{coffee}
cv, tv,cv', tv': \integer
\ST
0 \leq  cv \leq 3
\\
0 \leq  tv \leq 3
\\
0 \leq  cv' \leq 3
\\
0 \leq  tv' \leq 3
\\
tv' = tv
\\
cv' = cv - 1
\end{op}
\end{class}
\nextside
\begin{class}{VM}
\\
\begin{state}
cv, tv: \integer
\ST
0 \leq  cv \leq 3
\\
0 \leq  tv \leq 3
\end{state} 
\\
\begin{init}
cv = 3
\\tv = 3
\end{init} 
\\
\begin{op}{coffee}
\Delta (cv)
\ST
cv' = cv - 1
\end{op}
\end{class}
\nextside
\begin{class}{VM}
\\
\begin{state}
cv, tv: \integer
\ST
0 \leq  cv \leq 3
\\
0 \leq  tv \leq 3
\end{state} 
\\
\begin{init}
cv = 3
\\tv = 3
\end{init} 
\\
\begin{op}{coffee}
\Delta (cv)
\ST
cv' = cv - 1
\end{op}
\\
\begin{op}{tea}
\Delta (tv)
\ST
tv' = tv - 1
\end{op}
\end{class}
\end{sidebyside}
\caption{VM class, operation schema.}
\label{fig_oz_vm_operation_schema}
\end{figure}


\oz{} offers a more nice way to write the operation schema using $\Delta$-list. In \oz{}:
\begin{itemize}
\item Operation schema has a $\Delta$-list of state variables
whose values may change. By convention, no $\Delta$-list means
no attribute changes value.
\item Operation schema implicitly
includes the state schema and a primed version of it.
\end{itemize}
Thus, since the schema operation $coffee$ specifies changes on the $coffee$ value only, we can write it as shown in \refFig{fig_oz_vm_operation_schema} middle. 
Similarly, the operation schema $tea$ is shown in \refFig{fig_oz_vm_operation_schema} right.
\end{itemize}

\subsubsection{\textit{Operation's input and output parameters}:} 
\label{operation_input_output_parameters} 
Some operations can have input and output parameters, just like method in programming language, where the method's parameters represent the \findex[schema!operation schema!input parameter]{input}, and the returned values represent the \findex[schema!operation schema!output parameter]{output}. To illustrate the idea let us extend our vending machine .The new $VM$ can talk to a shop sending a message to it. So it has a new operation $talk$ and a state variable $m$ representing the message to be be sent.

The set of all possible state transitions when the $talk$ operation occurs is:
\begin{itemize}
\item In mathematics: 
\begin{equation*}
\begin{aligned}
talk ={} & \{cv,tv,message,cv',tv',message',y: \integer\mid (0 \leq  cv \leq 3) \\
      &  \wedge (0 \leq  tv \leq 3)\wedge (0 \leq  cv' \leq 3) \wedge (0 \leq  tv' \leq 3) \wedge (tv' = tv)  \\
      & \wedge (cv' = cv) \wedge (message' = message) \wedge (y = message)\bullet \\
      &  \pair{\tripple{cv}{tv}{message}}{\tripple{cv'}{tv'}{message'}}\} \\
      & =\{\pair{\tripple{3}{3}{1}}{\tripple{3}{3}{1}},\dots,\pair{\tripple{0}{0}{1}}{\tripple{0}{0}{1}}\}
\end{aligned}
\end{equation*}
\item In \oz{}: the set $talk$ can be described using an operation schema, as shown in \refFig{oz_vm_with_operation_input_output_parameters}. We can notice that this operation doesn't change any state variable's value, it just says that the value of the output parameter $y$, written as $y!$, must be equal to the value of the state variable $message$. For input parameter use $?$ symbol.
\begin{figure}[H]
\centering
\begin{class}{VM}
\\
\begin{state}
cv, tv, m: \integer
\ST
0 \leq  cv \leq 3
\\
0 \leq  tv \leq 3
\end{state} 
\\
\begin{init}
cv = 3
\\tv = 3
\\ m= 1
\end{init} 
\\
\begin{op}{coffee}
\Delta (cv)
\ST
cv' = cv - 1
\end{op}
\\
\begin{op}{tea}
\Delta (tv)
\ST
tv' = tv - 1
\end{op}
\\
\begin{op}{talk}
y!: \integer
\ST
y! = m
\end{op}
\end{class}
\caption{$VM$ class: \textit{$talk$ operation with output parameter.}}
\label{oz_vm_with_operation_input_output_parameters}
\end{figure}
\end{itemize}

\subsubsection{\findex[schema!instance reference]{Instance reference}:} 
\label{instance_reference} 
\oz{} is an object oriented approach, Thus every instance of a class needs a unique identifier, i.e., a reference name to refer to it. In \oz{} this can be seen simply as state constant $self$ initialized with some $id$ when the instance is created. Furthermore, operations can can share the instance identity through output or input the reference name $self$ as shown in \refFig{oz_vm_reference_name} in the operation $talk$.
\begin{figure}[H]
\centering
\begin{class}{VM(id: \integer)}
\\
\begin{state}
self, cv, tv, message: \integer
\ST
0 \leq  cv \leq 3
\\
0 \leq  tv \leq 3
\end{state} 
\\
\begin{init}
self = id
\\cv = 3
\\tv = 3
\\ message= 1
\end{init} 
\\
\begin{op}{coffee}
\Delta (cv)
\ST
cv' = cv - 1
\end{op}
\\
\begin{op}{tea}
\Delta (tv)
\ST
tv' = tv - 1
\end{op}
\\
\begin{op}{talk}
y!: \integer
\\z!: \integer
\ST
y! = message
\\z! = self
\end{op}
\end{class}
\caption{VM class, instance reference.}
\label{oz_vm_reference_name}
\end{figure}

\subsection{Syntax}
\label{sec_oz_syntax}
This part is dedicated to give the reader a deeper understanding of \oz{}. The reader can skip this part and go directly to \refSubSec{sec_oz_sem}. since we already explained the most important syntax constructs in \refSubSec{sec_oz_intuition}.

\subsection{Semantics}
\label{sec_oz_sem}
To understand the operational semantics of \oz{} we will use  a labelled transition system LTS. Using this LTS we can investigate the state evolution of a \oz{} object. The definition of this LTS is adapted from \refDef{def_pi_trans_system} with some changes.

\begin{definition}[LTS of \oz{}]
\label{def_oz_trans_system}
The \findex[transition system!for \oz{}]{labelled transition system} $(\stats,\traces)$ of \oz{} \\class states over the set of operations, has the valid states $\stats$ as its states and its transitions $\traces$ are those which can be inferred from the following rule: \\$\underline{\scriptstyle{OPER}}: PRE\_STATE \transs{operation} POST\_STATE$.
\end{definition}
An example of using the transition rule of this LTS is: drawing the transition graph of the vending machine shown in  \refFig{fig_oz_vm_operation_schema}. The transition graph is shown in \refFig{fig_VM_transition_graph}, where we show only a small part of it. The transitions $coffee$ and $tea$ refer to the operation schema $coffee$ and $tea$. Thus, using the LTS we can enumerate all possible states transitions of an \oz{} state.
\begin{figure}[H]%
    \centering
    \scalebox{0.85}{
\begin{tikzpicture}[->,>=stealth',shorten >=1pt,auto,node distance=5cm,
                    semithick]
  \tikzstyle{every state}=[]

  \node[state] (VM33)                    {$VM(3,3)$};
  \node[state] (VM23) 	[right of=VM33]  {$VM(2,3)$};
  \node[state] (VM13) 	[right of=VM23]  {$VM(1,3)$};
  \node[state] (VM03) 	[right of=VM13]  {$VM(0,3)$};
  \node[state] (VM32) 	[below of=VM33]  {$VM(3,2)$};

  \path (VM33) edge              node {$decrease\_coffee$} (VM23)
  		(VM23) edge              node {$decrease\_coffee$} (VM13)
  		(VM13) edge              node {$decrease\_coffee$} (VM03)
  		(VM33) edge              node {$decrease\_tea$} (VM32);
  		
  \draw [] (VM32) -- +(0:2cm);
  \draw [] (VM32) -- +(270:2cm); 
  \draw [] (VM23) -- +(270:2cm); 
  \draw [] (VM13) -- +(270:2cm); 
  \draw [] (VM03) -- +(270:2cm);   
\end{tikzpicture}
}
    \caption{$VM$ \textit{Transition graph}}%
    \label{fig_VM_transition_graph}%
\end{figure}





