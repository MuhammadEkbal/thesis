Syntactically the $\pi$-OZ specification is divided into an interface, a $\pi$ part and an \oz{} part as shown in \refFig{PpiOZ}.
The idea of the combination is that communication
in the $\pi$ part has effects on the state space of the \oz{} part as specified in its operation schema. \refFig{comp_oz_pi_statefull_vm} shows the $\pi$-OZ specification of our vending machine $VM$.
In the interface, all channels are declared with the associated types.
The $\pi$ part is a system of recursive equations written according to the \picalc{} syntax represents the sequencing of operations
as shown in $VM\_PI$.
In the OZ part the state space, the initial schema and the operation schemes introduced, where the operation schemes defines the effect of communications on the in specified channels.

\begin{figure}[htbp]
\begin{schema}{S}
 Interface\\
 S_{\pi\_part}\\
 S_{OZ\_part}
\end{schema}
\caption{$\pi$-OZ specification of an entity $S$.}
\label{PpiOZ}
\end{figure}


\begin{figure}[H]
\centering
\begin{class}{VM(id: \integer)}
\ 
\\chan\ coffee,tea
\ 
\\chan\ talk:\integer \times \integer
\ \\ \
\\VM\_PI = coffee().VM\_PI + tea().VM\_PI 
\\ \ \qquad \qquad + \out{talk}{self,message}.VM\_PI
\\
\begin{state}
self, cv, tv, message: \integer
\ST
0 \leq  cv \leq 3
\\
0 \leq  tv \leq 3
\end{state} 
\\
\begin{init}
self = id
\\cv = 3
\\tv = 3
\\ message= 1
\end{init} 
\\
\begin{op}{coffee}
\Delta (cv)
\ST
cv' = cv - 1
\end{op}
\\
\begin{op}{tea}
\Delta (tv)
\ST
tv' = tv - 1
\end{op}
\\
\begin{op}{talk}
y!: \integer
\\z!: \integer
\ST
y! = message
\\z! = self
\end{op}
\end{class}
\caption{$\pi$-OZ specification of the $VM$.}
\label{comp_oz_pi_statefull_vm}
\end{figure}

\refFig{comp_oz_pi_statefull_shop} shows the $\pi$-OZ specification of the active and idle shop.
\begin{figure}[H]
\centering
\begin{sidebyside}
\begin{class}{ActiveShop(id: \integer)}
\ 
\\chan\ switch: nil | talk
\ 
\\chan\ talk:\integer \times \integer
\ \\ \
\\ActiveShop\_PI = 
\\ \ talk(vmId, message)
\ \\ \ \ \ .ActiveShop\_PI
\\ \ + \out{switch}{talk}
\ \\ \ \ \ .IdleShop\_PI
\\\begin{state}
self, vmId, message: \integer
\\transferableOperation: nil | talk
\end{state} 
\\
\begin{init}
\\self = id
\\transferableOperation = talk
\end{init} 
\\
\begin{op}{switch\_\_\_\_\ then\ IdleShop}
x!: nil | talk
\ST
x! = transferableOperation
\\transferableOperation' = nil
\end{op}
\\
\begin{op}{talk}
\Delta (vmId, message)
\\y?, z?: \integer
\ST
y? = message'
\\z? = vmId'
\end{op}
\end{class}
\nextside
\begin{class}{IdleShop(id: \integer)}
\ 
\\chan\ switch: nil | talk
\ \\ \
\\IdleShop\_PI = 
\\ \  switch(transferableOperation)
\ \\ \ \ \ .ActiveShop\_PI
\\
\begin{state}
self, vmId, message: \integer
\\transferableOperation: nil | talk
\end{state} 
\\
\begin{init}
\\self = id
\\transferableOperation = nil
\end{init} 
\\
\begin{op}{switch\_\_\_\_\ then\ ActiveShop}
\Delta (transferableOperation)
\\x?: nil | talk
\ST
x? = transferableOperation'
\end{op}
\end{class}
\end{sidebyside}
\caption{$\pi$-OZ specification of the active and idle shop.}
\label{comp_oz_pi_statefull_shop}
\end{figure}



