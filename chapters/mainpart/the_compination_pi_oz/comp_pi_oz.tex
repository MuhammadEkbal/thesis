This chapter study the syntactic transformation of \oz{} class into \picalc{} process. The resulting processes is intuitively defined as follow:\\

\begin{figure}[H]
\centering
\begin{class}{VM(id: \integer)}
\ 
\\VM\_PI = coffee().VM\_PI + tea().VM\_PI 
\\ \qquad \qquad \qquad + talk<self,message>.VM\_PI
\\
\begin{state}
self, cv, tv, message: \integer
\ST
0 \leq  cv \leq 3
\\
0 \leq  tv \leq 3
\end{state} 
\\
\begin{init}
self = id
\\cv = 3
\\tv = 3
\\ message= 1
\end{init} 
\\
\begin{op}{coffee}
\Delta (cv)
\ST
cv' = cv - 1
\end{op}
\\
\begin{op}{tea}
\Delta (tv)
\ST
tv' = tv - 1
\end{op}
\\
\begin{op}{talk}
y!: \integer
\\z!: \integer
\ST
y! = message
\\z! = self
\end{op}
\end{class}
\caption{the combination.}
\label{comp_oz_pi}
\end{figure}