We consider a finite set of natural numbers represented as binary numbers shown \refTab{two_bit_binary_numbers}. A value can be mapped to a \picalc{} processes. \refLis{values_ABC_code} shows the \picalc{} implementation of the values 0,1,2,3 in ABC syntax. The keyword \textit{agent} defines a new processes. The process \textit{Zero} is modeled using alternative choice: it either receives a signal via the channel $a$ and switch off, or it receives two channels \textit{tt,ff} via \textit{a}, then it sends two signals via the channel \textit{ff}.
\begin{table}[H]
\centering
\begin{tabular}{|c|c|}
\hline
Decimal & Binary \\ \hline
0       & 00     \\ \hline
1       & 01     \\ \hline
2       & 10     \\ \hline
3       & 11     \\ \hline
\end{tabular}%
\caption{Two bits binary numbers.}
\label{two_bit_binary_numbers}
\end{table}
\lstinputlisting[backgroundcolor=\color{white},caption={0,1,2,3 as \picalc{} processes.},captionpos=b, label={values_ABC_code}]{listings/values.abc}
