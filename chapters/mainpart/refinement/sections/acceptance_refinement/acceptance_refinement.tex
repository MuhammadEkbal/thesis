To compare \picalc{} processes we need to define the Acceptance-Refinement and relate it to the simulation. We start by defining the acceptance pair of a process.
The pair (t, X) is called a acceptance pair, where $t$ is a trace and $Y$ is a set of all possible next actions. Any process
$P$ is assigned a set of acceptance pairs $AC$. Formally, this means:
\begin{align}
    \accepatances[P] \define \{(t,Y) \mid \exists Q\in\procs: \ec{P} \bigstep{t} \ec{Q}\wedge Y\in AA\}
\label{failure}
\end{align}
where: \[AA \define \pom{\actions\setminus\set{\tau}}\] is the set of all acceptable actions

We can define the Acceptance-refinement of \picalc{} processes as follow:

\begin{definition}[\index{refinement}{Acceptance refinement}]
\label{def_failure_ref}
	Let $P,Q\in\procs$, then $\ec{P}$ is a \findex[acceptance!refinement]{acceptance refinement} of $\ec{Q}$ ($\ec{Q}\refiAC \ec{P}$) iff the inverse set inclusion of traces and acceptances holds:
\begin{align}
   \ec{Q} \refiAC \ec{P} \Leftrightarrow  \traces[P]\subseteq\traces[Q] \wedge \accepatances[P]\subseteq\accepatances[Q]
\end{align}
	We also say for $\ec{Q}\refiAC{}\ec{P}$ that $\ec{P}$ \findex[refinement]{refines} $\ec{Q}$.
\end{definition}

From \refPro{property1} and \refDef{def_failure_ref} we can to drive the following Corollary: 

\begin{cor}[Simulation and Acceptance refinement]
\label{cor_sim_acceptance_refinement}
Let $P,Q\in\procs$ processes. If $\ec{Q}$ strongly simulates $\ec{P}$, then $\ec{P}$ refines $\ec{Q}$ in Acceptance-Refinement model. Formally written:
\begin{align}
    (\ec{P},\ec{Q})\in{}\simu \Rightarrow \ec{Q} \refiAC \ec{P}
   \label{acceptance_model}
\end{align}
holds.
\end{cor}%%
which reads: $Q$ simulate $P$ implies $P$ refines $Q$, in acceptance-refinement model.

\begin{prf}
Let $(\ec{P},\ec{Q})\in{}\simu$, then $\traces[P]\subseteq\traces[Q] \wedge \accepatances[P]\subseteq\accepatances[Q]$ holds, Since:
\begin{itemize}
\item For $\traces[P]\subseteq\traces[Q]$: it holds using \refPro{property1}.
\item For $\accepatances[P]\subseteq\accepatances[Q]$: we need to show that,\\$\forall (t_P, Y_P) \in \accepatances[P] \textit{ then } \exists (t_Q, Y_Q) \in \accepatances[P]: t_p = t_Q \wedge Y_P \subseteq Y_Q$
\begin{itemize}
\item $t_p = t_Q$ holds using \refPro{property1}.
\item $Y_P \subseteq Y_Q$ holds, since $\ec{Q}$ strongly simulates $\ec{P}$ means that $\ec{Q}$ can do all the actions that $\ec{P}$ can do after any trace $t$.
\end{itemize}
\end{itemize}
\end{prf}
\subsubsection{Acceptance-Refinement use case:}
 We will show that if $\ec{VM\_PI}$ shown in \refFig{vm_and_vmHalf} strongly simulates $\ec{VM\_Half\_PI}$, then $\ec{VM\_Half\_PI}$ refines $\ec{VM\_PI}$ in acceptance-refinement model.
 
 \refLis{vm_and_vmHalf_ABC_check1} showed that $\ec{VM\_PI}$ strongly simulates $\ec{VM\_Half\_PI}$. This result implies, according to \refCor{cor_sim_acceptance_refinement}, that $\ec{VM\_Half\_PI}$ refines $\ec{VM\_PI}$ in the acceptance-refinement model. Thus we need to show that $\traces[VM\_Half\_PI]\subseteq\traces[VM\_PI] \wedge \accepatances[VM\_Half\_PI]\subseteq\accepatances[VM\_PI]$.
 
 \begin{itemize}
\item For $\traces[VM\_Half\_PI]\subseteq\traces[VM\_PI]$: previously we showed that it holds. 

\item For $\accepatances[VM\_Half\_PI]\subseteq\accepatances[VM\_PI]$: let $\epsilon$ be the empty trace, then
    \[\accepatances[VM\_PI] \define \{(\epsilon,\{tea(),coffee(),talk<>\}),\dots\}^\ast\]
    \[\accepatances[VM\_Half\_PI] \define \{(\epsilon,\{coffee(),talk<>\}),\dots\}^\ast\]
It is clear that $\accepatances[VM\_Half\_PI]\subseteq\accepatances[VM\_PI]$ holds, thus $\ec{VM\_PI} \refiAC \ec{VM\_Half\_PI}$.
\end{itemize}