To compare \picalc{} processes we need to define the \findex[refinement!success]success refinement and relate it to the strong simulation. We start by defining the \textit{success pair} of a process.

The pair $(t, Y)$ is called an \textit{success pair}, where $t$ is a trace and $Y$ is a set of acceptable next actions after $t$. Formally, $P \bigstep{t} Q$ and $Q$ acp $X$. Any process
$P$ is assigned a set of success pairs $\Successes[]$. Formally, this means:
\begin{align}
    \Successes[P] \define \{(t,Y) \mid \exists Q\in\procs: P \bigstep{t} Q \wedge Q \text{ acp } Y \wedge Y\in Acceptances\}
\end{align}

where: $Acceptances \define \pom{\actions}$
\\The the special inclusion of success sets is defined as follow:
\begin{definition}[special inclusion of success set]
\label{def_success_inclutuion_ref}
Let $P,Q\in\procs$, then:
\begin{equation*}
\begin{aligned}
\Successes[P] \ddot{\subseteq} \Successes[Q] \Leftrightarrow {} & \forall (t_P, Y_P) \in \Successes[P] \textit{ then }\\
      & \ \ \ \exists (t_Q, Y_Q) \in \Successes[Q]: t_p = t_Q \wedge Y_P \subseteq Y_Q
\end{aligned}
\end{equation*}
\ 
\end{definition}
To determine the acceptance set $Y$ we need to collect all possible the actions after the substitution of bound names, which requires collecting the available actions before the substitution of bound names.

To collect the available actions before the substitution of bound names we need to collect the available output actions $Av_\outA$, collect the available input actions $Av_\inA$, and the available $\tau$ actions $Av_\tau$ as follow:
\[Av_\outA\define\{ \alpha \in \outA \mid \alpha\text{ is a syntatically possible next action in P}\}\]
\[Av_\inA\define\{ \alpha \in \inA \mid \alpha\text{ is a syntatically possible next action in P}\}\]
\[Av_\tau\define\{ \tau \mid \tau\text{ is a syntatically possible next action in P}\}\]

Now, we can collect the all the possible output actions after the substitution of bound names, where $Acc_\outA$ is for the output actions, $Acc_\inA$ is for the input actions, and $Acc_\tau$ is for the $\tau$ actions as follow:
\[Acc_\outA \define Av_\outA\]
\begin{equation*}
\begin{aligned}
Acc_\inA \define{} & \{\alpha \in \inA \mid \exists \beta \in Av_\inA \wedge \sub{\alpha}=\sub{\beta} \wedge \alpha=\substitue{\vec{y}}{\vec{z}}\beta \\
      &  \text{ where } \vec{z}=\obj{\beta} \wedge \vec{y}=\obj{\alpha} \}
\end{aligned}
\end{equation*}
\begin{equation*}
\begin{aligned}
Acc_\tau \define{} & \{\tau \mid \forall \alpha \in Acc_\outA \text{ if } \exists \beta \in Acc_\inA  :\sub{\alpha}=\sub{\beta} \wedge \obj{\alpha}=\obj{\beta}  \} \\
      &  \cup Av_\tau
\end{aligned}
\end{equation*}
\[\sub{\alpha} \text{: returns the channel name through which the exchange occurs}\]
\[\obj{\alpha} \text{: return the exchanged names across the channel}\]

Finally, we can determine $Y$ as follow:
\[Y \define Acc_\tau \cup Acc_\outA \cup Acc_\inA\]

Let us take an example to illustrate our approach of calculating the acceptance set of a trace. Let $P\define(\inp{a}{x}.\out{x}{a} \mid \inp{b}{y}).P$.  \refFig{trace_set_acceptance} shows $\Traces[P]$ and how to calculate the acceptance set $Y$ of the empty trace $\out{}{}$. $P$ has only two syntactically available actions $\inp{a}{x}$ and $\inp{b}{d}$ as shown in the set $Av_\inA$. Once one of them occurs, its bound name must be substituted with the received name. This substitution is shown in $Acc_\inA$ for all possible received names. Finally, the acceptance set $Y$ of $\out{}{}$ is the union of $Acc_\outA$, $Acc_\inA$ and $Acc_\tau$. 
The sets $Av_\outA$ and $Av_\tau$ are empty, because $P$ can neither make an output action nor $\tau$ action. Thus, $Acc_\outA$ and $Acc_\tau$ are empty too.
    
\begin{figure}[H]
\centering
\resizebox{0.8\textwidth}{!}{%
\begin{tikzpicture}
    % draw the sets
   \draw[] (-7.5,0) ellipse [x radius = 1.2cm, y radius = 5cm];
    \draw[] (-4.5,0) ellipse [x radius = 0.9cm, y radius = 3cm];
    % the texts
    \node at (-7.5,5.3) {$\Traces[P]$};
    \node (x1) at (-4.5,3.3) {$\out{}{}$};

    \node (y1) at (-7.5,4.5) {$\out{}{}$};
    \node at (-7.5,3.9) {$\out{}{\inp{a}{a}}$};
    \node at (-7.5,3.3) {$\out{}{\inp{a}{b}}$};
    \node at (-7.5,2.7) {$\out{}{\inp{a}{c}}$};
    \node at (-7.5,2.1) {$\dot$};
    \node at (-7.5,1.5) {$\dot$};
    \node at (-7.5,0.9) {$\out{}{\inp{b}{a}}$};
    \node at (-7.5,0.3) {$\out{}{\inp{b}{d}}$};
    \node at (-7.5,-0.3) {$\dot$};
    \node at (-7.5,-0.9) {$\dot$};
    \node at (-7.5,-1.5) {$\out{}{\inp{a}{a},\inp{a}{a}}$};
    \node at (-7.5,-2.1) {$\out{}{\inp{b}{a},\inp{a}{a}}$};
    \node at (-7.5,-2.7) {$\dot$};
    \node at (-7.5,-3.3) {$\dot$};

    \node at (-4.5,2.4) {$\inp{a}{a}$};
    \node at (-4.5,1.8) {$\inp{a}{b}$};
    \node at (-4.5,1.2) {$\inp{a}{c}$};
    \node at (-4.5,0.6) {$\dot$};
    \node at (-4.5,0) {$\dot$};
    \node at (-4.5,-0.6) {$\inp{b}{a}$};
    \node at (-4.5,-1.2) {$\inp{b}{d}$};    
    \node at (-4.5,-1.8) {$\dot$};
    \node at (-4.5,-2.2) {$\dot$};
    
    \path (y1) edge (x1);

    % draw the sets
   \draw[] (-7.5,-10) ellipse [x radius = 0.9cm, y radius = 3cm];
    \draw[] (-5.5,-10) ellipse [x radius = 0.9cm, y radius = 3cm];
    \draw[] (-3.5,-10) ellipse [x radius = 0.9cm, y radius = 3cm];
    \draw[] (0.5,-10) ellipse [x radius = 0.9cm, y radius = 3cm];
    \draw[] (2.5,-10) ellipse [x radius = 0.9cm, y radius = 3cm];
    \draw[] (4.5,-10) ellipse [x radius = 0.9cm, y radius = 3cm];

    % the texts
    \node (w1) at (-7.5,-6.7) {$Av_\outA$};
    \node (w2) at (-5.5,-6.7) {$Av_\inA$};
    \node (w3) at (-3.5,-6.7) {$Av_\tau$};
    \node (w4) at (0.5,-6.7) {$Acc_\outA$};
    \node (w5) at (2.5,-6.7) {$Acc_\inA$};
    \node (w6) at (4.5,-6.7) {$Acc_\tau$};

    \node at (-7.5,-9.7) {$\varnothing$};
    
    \node at (-5.5,-9.1) {$\inp{a}{x}$};
    \node at (-5.5,-9.7) {$\inp{b}{y}$};

    \node at (-3.5,-9.7) {$\varnothing$};
   
    \node at (0.5,-9.7) {$\varnothing$};
        
    \node at (2.5,-7.5) {$\inp{a}{a}$};
    \node at (2.5,-8.1) {$\inp{a}{b}$};
    \node at (2.5,-8.7) {$\inp{a}{c}$};
    \node at (2.5,-9.4) {$\dot$};
    \node at (2.5,-10) {$\dot$};
    \node at (2.5,-10.6) {$\inp{b}{a}$};
    \node at (2.5,-11.2) {$\inp{b}{d}$};    
    \node at (2.5,-11.8) {$\dot$};
    \node at (2.5,-12.4) {$\dot$};
    
    \node at (4.5,-9.7) {$\varnothing$};

    \node (m1)at (-4.5,-3.3) {$Y$};
    \path (w4) edge (m1);
    \path (w5) edge (m1);
    \path (w6) edge (m1);

    \node (k1)at (-7.5,-13) {$ $};
    \node (l1)at (0.5,-13) {$ $};
	\draw [dashed] (k1.south) to [out=-10,in=-170] (l1.south);
	
	\node (k2)at (-5.5,-13) {$ $};
    \node (l2)at (2.5,-13) {$ $};
	\draw [->] (k2.south) to [out=-10,in=-170] (l2.south);
	
	\node (k3)at (-3.5,-13) {$ $};
    \node (l3)at (4.5,-13) {$ $};
	\draw [dashed] (k3.south) to [out=-10,in=-170] (l3.south);
	
\end{tikzpicture}
}
\caption{Calculating the acceptance set $Y$ of the trace $\out{}{}$ of the processes $P$}
\label{trace_set_acceptance}
\end{figure}
We can now define the success-refinement of \picalc{} processes as follow:

\begin{definition}[Successes refinement]
\label{def_Successes_ref}
	Let $P,Q\in\procs$, then $P$ is a {successes refinement} of $Q$ iff the inverse set inclusion of traces and successes holds:
   \[Q \refiAC P \Leftrightarrow  \Traces[P]\subseteq\Traces[Q] \wedge \Successes[P]\ddot{\subseteq}\Successes[Q]\]
\end{definition}

From \refPro{property2} and \refDef{def_Successes_ref} we can drive the following Corollary: 

\begin{cor}[Simulation and Success refinement]
\label{cor_sim_acceptance_refinement}
Let $P,Q\in\procs$. If $Q$ strongly simulates $P$, then $P$ refines $Q$ in Success-Refinement model. Formally written:
    \[(P,Q)\in{}\simu \Rightarrow Q \refiSuc P\]
holds.
\end{cor}%%

\begin{prf}\footnote{Our proof is not complete and can be enhanced in later works.}
Let $(P,Q)\in{}\simuv$, then $\Traces[P]\subseteq\Traces[Q] \wedge \Successes[P]\subseteq\Successes[Q]$ holds, Since:
\begin{itemize}
\item For $\Traces[P]\subseteq\Traces[Q]$: it holds using \refPro{property2}.
\item For $\Successes[P]\ddot{\subseteq}\Successes[Q]$: we need to show that,\\$\forall (t_P, Y_P) \in \Successes[P] \textit{ then } \exists (t_Q, Y_Q) \in \Successes[Q]: t_p = t_Q \wedge Y_P \subseteq Y_Q$
\begin{itemize}
\item $t_p = t_Q$ holds using \refPro{property2}.
\item $Y_P \subseteq Y_Q$ holds, since $Q$ strongly simulates $P$ means that $Q$ can do any the actions that $P$ can do respecting the use of the same free and bound names, what implies that $Q$ can do any trace $t$ that $P$ does. This implies that after any trace $t$, $Q$ accepts all the actions that $P$ accepts. This means that the acceptance set of $P$ is included in the acceptance set of $Q$ after any trace $t$. Formally, $Y_P \subseteq Y_Q$ for any trace $t$.
\end{itemize}
\end{itemize}
\end{prf}
\subsubsection{Success-Refinement use case:}
Let us take an example to inspect if \refCor{cor_sim_acceptance_refinement} and our approach of calculating the acceptance set works. let:\\
$P\define(\inp{a}{x}.\out{x}{a} \mid \inp{b}{y}).P$ and 
$Q\define(\inp{a}{x}.\out{x}{a} \mid \inp{b}{y}).Q + \tau.Q$
It is clear that $Q$ strongly simulates $P$. This result implies, according to \refCor{cor_sim_acceptance_refinement}, that $P$ refines $Q$ in the success-refinement model. In other words $\forall t$ then $Y_P \subseteq Y_Q$. We will show that this holds for the empty trace $\out{}{}$ using our approach of determining the acceptance set $Y$.
This can be seen in \refFig{trace_set_acceptance} and \refFig{trace_set_acceptance2}. We can notice that there is no differences between $\Traces[P]$ and $\Traces[Q]$, since the $\tau$ action that $Q$ can do, cannot appear in the trace. But $\tau$ appears in the acceptance set $Y$ of the empty trace $\out{}{}$ of $Q$ shown in \refFig{trace_set_acceptance2}. In the same manner, we can determine $Y_P$ and $Y_Q$ for any trace $t$ and see that $Y_P \subseteq Y_Q$ holds. Thus, $\Successes[P]\ddot{\subseteq}\Successes[Q]$ holds and so $Q \refiSuc P$.

\begin{figure}[H]
\centering
\resizebox{0.8\textwidth}{!}{%
\begin{tikzpicture}
    % draw the sets
   \draw[] (-7.5,0) ellipse [x radius = 1.2cm, y radius = 5cm];
    \draw[] (-4.5,0) ellipse [x radius = 0.9cm, y radius = 3cm];
    % the texts
    \node at (-7.5,5.3) {$\Traces[Q]$};
    \node (x1) at (-4.5,3.3) {$\out{}{}$};

    \node (y1) at (-7.5,4.5) {$\out{}{}$};
    \node at (-7.5,3.9) {$\out{}{\inp{a}{a}}$};
    \node at (-7.5,3.3) {$\out{}{\inp{a}{b}}$};
    \node at (-7.5,2.7) {$\out{}{\inp{a}{c}}$};
    \node at (-7.5,2.1) {$\dot$};
    \node at (-7.5,1.5) {$\dot$};
    \node at (-7.5,0.9) {$\out{}{\inp{b}{a}}$};
    \node at (-7.5,0.3) {$\out{}{\inp{b}{d}}$};
    \node at (-7.5,-0.3) {$\dot$};
    \node at (-7.5,-0.9) {$\dot$};
    \node at (-7.5,-1.5) {$\out{}{\inp{a}{a},\inp{a}{a}}$};
    \node at (-7.5,-2.1) {$\out{}{\inp{b}{a},\inp{a}{a}}$};
    \node at (-7.5,-2.7) {$\dot$};
    \node at (-7.5,-3.3) {$\dot$};
    \node at (-7.5,-3.9) {$\tau$};

    \node at (-4.5,2.4) {$\inp{a}{a}$};
    \node at (-4.5,1.8) {$\inp{a}{b}$};
    \node at (-4.5,1.2) {$\inp{a}{c}$};
    \node at (-4.5,0.6) {$\dot$};
    \node at (-4.5,0) {$\dot$};
    \node at (-4.5,-0.6) {$\inp{b}{a}$};
    \node at (-4.5,-1.2) {$\inp{b}{d}$};    
    \node at (-4.5,-1.8) {$\dot$};
    \node at (-4.5,-2.2) {$\tau$};
    
    \path (y1) edge (x1);

    % draw the sets
   \draw[] (-7.5,-10) ellipse [x radius = 0.9cm, y radius = 3cm];
    \draw[] (-5.5,-10) ellipse [x radius = 0.9cm, y radius = 3cm];
    \draw[] (-3.5,-10) ellipse [x radius = 0.9cm, y radius = 3cm];
    \draw[] (0.5,-10) ellipse [x radius = 0.9cm, y radius = 3cm];
    \draw[] (2.5,-10) ellipse [x radius = 0.9cm, y radius = 3cm];
    \draw[] (4.5,-10) ellipse [x radius = 0.9cm, y radius = 3cm];

    % the texts
    \node (w1) at (-7.5,-6.7) {$Av_\outA$};
    \node (w2) at (-5.5,-6.7) {$Av_\inA$};
    \node (w3) at (-3.5,-6.7) {$Av_\tau$};
    \node (w4) at (0.5,-6.7) {$Acc_\outA$};
    \node (w5) at (2.5,-6.7) {$Acc_\inA$};
    \node (w6) at (4.5,-6.7) {$Acc_\tau$};

    \node at (-7.5,-9.7) {$\varnothing$};
    
    \node at (-5.5,-9.1) {$\inp{a}{x}$};
    \node at (-5.5,-9.7) {$\inp{b}{y}$};

    \node at (-3.5,-9.7) {$\tau$};
   
    \node at (0.5,-9.7) {$\varnothing$};
        
    \node at (2.5,-7.5) {$\inp{a}{a}$};
    \node at (2.5,-8.1) {$\inp{a}{b}$};
    \node at (2.5,-8.7) {$\inp{a}{c}$};
    \node at (2.5,-9.4) {$\dot$};
    \node at (2.5,-10) {$\dot$};
    \node at (2.5,-10.6) {$\inp{b}{a}$};
    \node at (2.5,-11.2) {$\inp{b}{d}$};    
    \node at (2.5,-11.8) {$\dot$};
    \node at (2.5,-12.4) {$\dot$};
    
    \node at (4.5,-9.7) {$\tau$};

    \node (m1)at (-4.5,-3.3) {$Y$};
    \path (w4) edge (m1);
    \path (w5) edge (m1);
    \path (w6) edge (m1);

    \node (k1)at (-7.5,-13) {$ $};
    \node (l1)at (0.5,-13) {$ $};
	\draw [dashed] (k1.south) to [out=-10,in=-170] (l1.south);
	
	\node (k2)at (-5.5,-13) {$ $};
    \node (l2)at (2.5,-13) {$ $};
	\draw [->] (k2.south) to [out=-10,in=-170] (l2.south);
	
	\node (k3)at (-3.5,-13) {$ $};
    \node (l3)at (4.5,-13) {$ $};
	\draw [dashed] (k3.south) to [out=-10,in=-170] (l3.south);
\end{tikzpicture}
}
\caption{Calculating the acceptance set $Y$ of the trace $\out{}{}$ of the processes $Q$}
\label{trace_set_acceptance2}
\end{figure}
