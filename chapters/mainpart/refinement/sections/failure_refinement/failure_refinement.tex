We check whether the very strong simulation implies failure-refinement. We found that this is not the case. We start by defining the failure of a process.
The pair $(t, X)$ is called a $failure$, where $t$ is a trace and $X$ is a set of impossible next actions. Any process
$P$ is assigned a set of failures $F$. Formally, this means:
\begin{align}
    \Failures[P] \define \{(t,X) \mid \exists Q\in\procs: P \bigstep{t} Q \wedge Q ref X \wedge X\in Refusals\}
\label{failure}
\end{align}
where: $Refusals \define \pom{\actions\setminus\set{\tau}}$.

We can define the failure-refinement of \picalc{} processes as follow:

\begin{definition}[\index{refinement}{Failure refinement}]
\label{def_failure_ref}
	Let $P,Q\in\procs$, then $P$ is a \findex[trace!refinement]{failure refinement} of $Q$ iff the inverse set inclusion of traces and failure holds:
\begin{align}
   Q \refiF P \Leftrightarrow  \Traces[P]\subseteq\Traces[Q] \wedge \Failures[P]\subseteq\Failures[Q]
\end{align}
\end{definition}

From \refPro{property2} and \refDef{def_failure_ref} we can drive the following Corollary: 

\begin{cor}[Simulation and Failure refinement]
\label{cor_sim_failure_refinement}
Let $P,Q\in\procs$ processes. If $Q$ very strongly simulates $P$, this does not imply that $P$ refines $Q$ in Failure-Refinement model. Formally written:
\begin{align}
    (P,Q)\in{}\simuv  \not\Rightarrow Q \refiF P
   \label{failure_model}
\end{align}
\end{cor}%%
\begin{prf}
by counter example. Assume that $(P,Q)\in{}\simuv \Rightarrow Q \refiF P$ and let $P\define\out{a}{}.P$ and $Q\define\out{a}{}.Q + \out{b}{}.Q$, shown in \refFig{vm_and_vmHalf}.
\begin{figure}[H]%
\centering
\subcaptionbox{}{\fbox{{
\begin{tikzpicture}[->,>=stealth',shorten >=1pt,auto,node distance=4cm,
                    semithick]
  \tikzstyle{every state}=[]

  \node[state] (A)                    {$P$};
  \path (A) edge [loop above] node {$\out{a}{}$} (A);
\end{tikzpicture}    
    }}}%
\qquad
\subcaptionbox{}{\fbox{{
    \begin{tikzpicture}[->,>=stealth',shorten >=1pt,auto,node distance=3cm,
                    semithick]
  \tikzstyle{every state}=[]

  \node[state] (A)                    {$Q$};
  \path (A) edge [loop above] node {$\out{a}{}, \out{b}{}$} (A);
\end{tikzpicture}
 }}}%
\caption{$P$ and $Q$}
\label{vm_and_vmHalf}
\end{figure}

It is clear that $Q$ very strongly simulates $P$. This should imply, according to our assumption, that $P$ refines $Q$ in the failure model, thus we need to show that $\Traces[P]\subseteq\Traces[Q] \wedge \Failures[P]\subseteq\Failures[Q]$.
\begin{itemize}
\item For $\Traces[P]\subseteq\Traces[Q]$: we need to determine the traces of $Q$ and $P$  shown in \refFig{vm_and_vmHalf}. According to \ref{determine_trace}:

    \[\Traces[Q] \define \{a(),b()\}^\ast\]
    \[\Traces[P] \define \{a()\}^\ast\]
It is clear that $\Traces[P]\subseteq\Traces[Q]$ holds. 

\item For $\Failures[P]\subseteq\Failures[Q]$: let $\epsilon$ be the empty trace, then
    \[\Failures[Q] \define \{(\epsilon,\{\}),\dots\}^\ast\]
    \[\Failures[P] \define \{(\epsilon,\{b()\}),\dots\}^\ast\]
It is clear that $\Failures[P]\not\subseteq\Failures[Q]$, thus $Q \not\refiF P$.

So, very strong simulation does not imply failure-refinement. Thus, in the next section we will introduce the Acceptance-Refinement model and use it instead of the Failure-Refinement model.
\end{itemize}
\end{prf}
