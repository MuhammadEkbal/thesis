We checked whether the strong simulation implies failure-refinement. We found that this is not the case. We start by defining the failure of a process.
The pair $(t, X)$ is called a \textit{failure pair}, where $t$ is a trace and $X$ is a set of refused next actions after $t$. Formally, $P \bigstep{t} Q$ and $Q$ ref $X$. Any process
$P$ is assigned a set of failures $\Failures[]$. Formally, this means:
\begin{align}
    \Failures[P] \define \{(t,X) \mid \exists Q\in\procs: P \bigstep{t} Q \wedge Q \text{ ref } X \wedge X\in Refusals\}
\end{align}
where: $Refusals \define \pom{\actions}$.
\\The special inclusion of failures sets is defined as follows:
\begin{definition}[special inclusion of failures set]
\label{def_failure_inclutuion_ref}
Let $P,Q\in\procs$, then:
\begin{equation*}
\begin{aligned}
\Failures[P] \ddot{\subseteq} \Failures[Q] \Leftrightarrow {} & \forall (t_P, Y_P) \in \Failures[P] \textit{ then }\\
      & \ \ \ \exists (t_Q, Y_Q) \in \Failures[Q]: t_p = t_Q \wedge Y_P \subseteq Y_Q
\end{aligned}
\end{equation*}
\ 
\end{definition}
Notice, we are neither focusing on how to calculate the $\Traces[P]$, since it has been studied in \cite{gieseking} nor on how to calculate the refusal set $X$ of a trace $t$, since the failure-refinement is not useful as we will show later in \refRem{cor_sim_failure_refinement}. Later in \refSec{sec_acceptance-refinement} we will focus on calculating the acceptance set of a trace.
For now, let us go ahead and define the failure-refinement of \picalc{} processes as follows: 
\begin{definition}[Failure refinement]
\label{def_failure_ref}
	Let $P,Q\in\procs$, then $P$ is a \findex[refinement!failure]{failure refinement} of $Q$ iff the inverse set inclusion of traces and failures holds:
   \[Q \refiF P \Leftrightarrow  \Traces[P]\subseteq\Traces[Q] \wedge \Failures[P]\ddot{\subseteq}\Failures[Q]\]
\end{definition}

From \refPro{property2} and \refDef{def_failure_ref} we can drive the following remark: 

\begin{rem}[Simulation and Failure refinement]
\label{cor_sim_failure_refinement}
Let $P,Q\in\procs$. If $Q$ strongly simulates $P$, this does not imply that $P$ refines $Q$ in Failure-Refinement model. Formally written:
    \[(P,Q)\in{}\simu  \not\Rightarrow Q \refiF P\]
\end{rem}%%
\begin{prf}
by counter-example. Assume that $(P,Q)\in{}\simu \Rightarrow Q \refiF P$ and let $P\define\out{a}{}.P$ and $Q\define\out{a}{}.Q + \out{b}{}.Q$, shown in \refFig{vm_and_vmHalf}.
\begin{figure}[H]%
\centering
\subcaptionbox{}{\fbox{{
\begin{tikzpicture}[->,>=stealth',shorten >=1pt,auto,node distance=4cm,
                    semithick]
  \tikzstyle{every state}=[]

  \node[state] (A)                    {$P$};
  \path (A) edge [loop above] node {$\out{a}{}$} (A);
\end{tikzpicture}    
    }}}%
\qquad
\subcaptionbox{}{\fbox{{
    \begin{tikzpicture}[->,>=stealth',shorten >=1pt,auto,node distance=3cm,
                    semithick]
  \tikzstyle{every state}=[]

  \node[state] (A)                    {$Q$};
  \path (A) edge [loop above] node {$\out{a}{}, \out{b}{}$} (A);
\end{tikzpicture}
 }}}%
\caption{$P$ and $Q$.}
\label{vm_and_vmHalf}
\end{figure}

It is clear that $Q$ strongly simulates $P$. This should imply, according to our assumption, that $P$ refines $Q$ in the failure model, thus we need to show that $\Traces[P]\subseteq\Traces[Q] \wedge \Failures[P]\ddot{\subseteq}\Failures[Q]$.
\begin{itemize}
\item For $\Traces[P]\subseteq\Traces[Q]$:
    \[\Traces[P] \define \{a()\}^\ast\]
    \[\Traces[Q] \define \{a(),b()\}^\ast\]
It is clear that $\Traces[P]\subseteq\Traces[Q]$ holds. 
\item For $\Failures[P]\ddot{\subseteq}\Failures[Q]$:
    \[\Failures[P] \define \{(\out{}{},\{b()\}),\dots\}\]
    \[\Failures[Q] \define \{(\out{}{},\{\}),\dots\}\]
It is clear that $\Failures[P]\ddot{\not\subseteq}\Failures[Q]$, since $\{b()\} \not\subseteq \{\}$ for $\out{}{}$ . Thus $Q \not\refiF P$.
\end{itemize}
\end{prf}
So, strong simulation does not imply failure-refinement. Thus, in the next section we will introduce the Success-Refinement model and use it instead of the Failure-Refinement model.
