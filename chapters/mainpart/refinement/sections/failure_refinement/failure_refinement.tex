To compare \picalc{} processes we need to define the Failure-Refinement and relate it to the simulation. We start by defining the failure of a process.
The pair (t, X) is called a failure, where $t$ is a trace and $X$ is a set of impossible next actions. Any process
$P$ is assigned a set of failures $F$. Formally, this means:
\begin{align}
    \failures[P] \define \{(t,X) \mid \exists Q\in\procs: \ec{P} \bigstep{t} \ec{Q}\wedge X\in Refusals\}
\label{failure}
\end{align}
where: \[Refusals \define \pom{\actions\setminus\set{\tau}}\]

We can define the failure-refinement of \picalc{} processes as follow:

\begin{definition}[\index{refinement}{Failure refinement}]
\label{def_failure_ref}
	Let $P,Q\in\procs$, then $P$ is a \findex[trace!refinement]{failure refinement} of $Q$ ($Q\refiF P$) iff the inverse set inclusion of traces and failure holds:
\begin{align}
   \ec{Q} \refiF \ec{P} \Leftrightarrow  \traces[P]\subseteq\traces[Q] \wedge \failures[P]\subseteq\failures[Q]
\end{align}
	We also say for $Q\refi{}P$ that $P$ \findex[refinement]{refines} $Q$.
\end{definition}

From \refPro{property1} and \refDef{def_failure_ref} we can to drive the following Corollary: 

\begin{cor}[Simulation and Failure refinement]
\label{cor_sim_failure_refinement}
Let $P,Q\in\procs$ processes, then there exists a weak or strong bisimulation $\simu\subseteq\procs_\alpha\times\procs_\alpha$ such that
\begin{align}
    (\ec{P},\ec{Q})\in{}\simu  \Rightarrow \ec{Q} \refiF \ec{P}
   \label{failure_model}
\end{align}
holds.
\end{cor}%%
which reads:  $Q$ simulate $P$ implies $P$ refines $Q$, in Failure-Refinement.

\begin{prf}
it yields directly from \refPro{property1} and \refDef{def_failure_ref}.
\end{prf}