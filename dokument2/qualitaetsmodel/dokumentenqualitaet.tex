\section{Dokumentenqualität}
In diesem Abschnitt werden die Kriterien und Maßnahmen zur Dokumentenqualität für unsere Gesamtdokumentation und Protokolle kurz dargestellt:
\\
\subsection{Projektdokumentation}
\textbf{Kriterium: Vollständigkeit}
\begin{itemize}
\item	Alle relevanten Aspekte zum Vorgehen und der Ergebnisse der Projektgruppe sind erfasst
\item	Literaturverzeichnis ist vollständig
\item	Anhänge sind vollständig
\end{itemize}

\textbf{Maßnahmen: }
\begin{itemize}
\item	Gesamt-Inhaltsverzeichnis erarbeiten und regelmäßig im Team ergänzen und überprüfen
\item	Unterpunkte des Inhaltsverzeichnisses so bald wie möglich ausarbeiten 
\item	Regelmäßige Abgabe von Einzelkapiteln und Einarbeitung von Feedback
\item	Phase zum Ende der Projektgruppe zur Überarbeitung der Dokumentation
\end{itemize}

\textbf{Kriterium: Rechtschreibung}
\begin{itemize}
\item	Minimale Anzahl an Rechtschreibfehlern
\end{itemize}

\textbf{Maßnahmen: }
\begin{itemize}
\item	Einsatz von Rechtschreibtools
\item	Gegenlesen von anderen Teammitgliedern
\end{itemize}

\textbf{Kriterium: Format}
\begin{itemize}
\item	Randbreite: 2,5 cm überall
\item	Schrift: Computern Modern (ähnlich zu Times New Roman)
\item	Schriftgröße 12 pt.
\item	Zeilenabstand 1,5
\item	Blocksatz
\item	einseitiger Druck
\item	durchlaufende Seitenzählung (beginnend mit der Einleitung)
\end{itemize}

\textbf{Maßnahmen: }\\
Verwendung Latex:
\begin{itemize}
\item	Passende Vorlage
\item	keine Warnungen
\item	Überprüfung mit Jenkins
\end{itemize}

\textbf{Kriterium: Zugriff}\\
Alle Projektgruppenmitglieder haben Zugriff auf
\begin{itemize}
\item	Quelldateien
\item	aktuellen Stand als PDF
\end{itemize}

\textbf{Maßnahmen: }
\begin{itemize}
\item	Verwendung von Source-Code-Verwaltung
\item	Veröffentlichung des PDFs im Confluence durch Jenkins nach Commit
\end{itemize}

\textbf{Kriterium: Zitierweise}\\
In der gesamten Dokumentation muss eine einheitliche Zitierweise verfolgt werden, um Plagiatsvorwürfen vorzubeugen und die Standards einer wissenschaftlichen Ausarbeitung zu erfüllen.\\
\\
\textbf{Maßnahmen: }
\begin{itemize}
\item	Verwendung der LNI-Autorenstandards
\end{itemize}

\subsection{Protokollqualität}
\textbf{Kriterium: Vollständigkeit}
\begin{itemize}
\item	Datum/Uhrzeit
\item	Teilnehmer (alle) und Abwesende (nur PG-Mitglieder)
\item	Art der Sitzung und Agenda
\item	wesentliche Wortbeiträge nach Punkten der Agenda
\item	Zusammenfassung mit getroffenen und vertagten Entscheidungen sowie offenen Aufgaben
\end{itemize}

\textbf{Maßnahmen: }
\begin{itemize}
\item	Confluence-Vorlagen je nach Art der Sitzung
\item	Einteilung genau eines Protokollanten, der wochenweise in jeder der stattfindenden Sitzungen protokolliert und ansonsten keine weitere Aufgabe in der Sitzungsgestaltung hat
\item	Protokollant stellt Rückfragen und fasst Entscheidungen und Aufgaben zusammen 
\item	Referenzen zu besprochenen/vorgestellten Dokumenten/Artefakten sind angegeben
\end{itemize}

\textbf{Kriterium: Formulierung}
\begin{itemize}
\item	Verständlichkeit für Abwesende
\end{itemize}

\textbf{Maßnahmen: }
\begin{itemize}
\item	Protokollant stellt Rückfragen während der Sitzung
\item	Gegenlesen durch nächsten Protokollanten
\end{itemize}

\textbf{Kriterium: Rechtschreibung}
\begin{itemize}
\item	Geringe Anzahl an Rechtschreibfehlern
\end{itemize}

\textbf{Maßnahmen: }
\begin{itemize}
\item	Gegenlesen durch nächsten Protokollanten
\end{itemize}

\textbf{Kriterium: Format}
\begin{itemize}
\item	Informelle Beurteilung durch Ersteller
\end{itemize}

\textbf{Maßnahmen: }
\begin{itemize}
\item	Export Confluence
\end{itemize}

\textbf{Kriterium: Verständlichkeit}
\begin{itemize}
\item	Das Protokoll muss so geschrieben sein, dass klar wird, warum über welches Thema gesprochen wird. So wird gewährleistet, dass der Leser die Zusammenhänge versteht und auch Beziehungen zu anderen Protokollen herstellen kann
\end{itemize}

\textbf{Maßnahmen: }
Einleitungssatz zu jedem Thema: 
\begin{itemize}
\item	Kurz erklären, warum über das Thema geredet wird und was das Ziel ist.
\end{itemize}

\textbf{Kriterium: Zugriff}
\begin{itemize}
\item	alle Projektgruppenmitglieder haben Zugriff auf bearbeitbare Version der Protokolle
\item	Betreuern wird lesbares Protokoll zeitnah zur Verfügung gestellt
\end{itemize}

\textbf{Maßnahmen: }
\begin{itemize}
\item	Protokollierung im Confluence
\item	Versand der letzten Protokolle mit Einladung zur nächsten Sitzung (nach Gegenlesen)
\end{itemize}


