\section{Zusammenfassung und Evaluation}
Zusammenfassend zeigt sich, dass das Qualitätsmodell viele Qualitätsaspekte abdeckt. 
So wurden Kriterien und Maßnahmen zu Softwarequalität, Datenqualität und Dokumentenqualität festgelegt, die in jeder Phase des Projektes von hoher Relevanz sind. 
Dinge wie eine Definition of Done, Definition of Ready und ein Branchmodell haben geholfen, die Software qualitätsgesichert zu entwickeln. 
Auf der anderen Seite muss auch gesagt werden, dass viele Aspekte des Qualitätsmodells vor allem aus Zeitgründen nicht eingehalten werden konnten. Dabei handelt es sich insbesondere um Inhalte, die nicht konstant durchgeführt werden, sondern in bestimmten Abständen immer wieder überprüft werden müssen. Ein gutes Beispiel dafür ist das gezielte Testen mit Hilfe von Testdrehbüchern. Der hierfür benötigte Zeitaufwand konnte von der Projektgruppe nicht geleistet werden, zumal die eindeutige Zuständigkeit gefehlt hat. Ein weiterer Punkt, der nicht allen Gruppen eingehalten wurde oder eingehalten werden konnte ist die Nutzung des TDD-Ansatzes in der Softwareentwicklung. 
Abschließend kann jedoch festgehalten werden, dass es sinnvoll ist, frühzeitig ein Qualitätsmodell zu erarbeiten, alleine um den Aspekt des Qualitätssicherns einen hohen Stellenwert zukommen zu lassen. 