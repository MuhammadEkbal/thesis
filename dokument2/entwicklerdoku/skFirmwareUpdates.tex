Dieser muss dazu im Webroot eine JSON"=Datei mit dem Namen \filename{availableUpdates.json} und folgendem Aufbau bereitstellen:
\begin{lstlisting}[language=json]
{
  "updates": [
    {
      "_v": "1",
      "path": "/realtivePathTo.bin",
      "version": "version",
      "crc32": "crc32inLowerLetters",
      "description": "Description"
    },
    {
      "_v": "1",
      "path": "/files/empty.bin",
      "version": "0.0.190911000000",
      "crc32": "deadbeef",
      "description": "Example-File"
    }
  ]
}
\end{lstlisting}

Die Firmware"=Datei wird dann relativ zur \filename{availableUpdates.json} abgelegt, hier im Beispiel im Verzeichnis \dirname{files}.
Da der Puffer für die \filename{availableUpdates.json}"=Datei auf dem Sensorknoten begrenzt ist, sollten nicht mehr als fünf Versionen in der Datei zur Verfügung stehen.
Die Auswertung der \filename{availableUpdates.json} geschieht dann "von oben nach unten", wobei die erste Firmware mit neuerer Version ausgewählt und heruntergeladen wird.
Dieses Vorgehen ermöglicht eine gezielte Sequenzialisierung der Updates, sofern einmal eine Interims"=Firmware benötigt werden sollte. 
