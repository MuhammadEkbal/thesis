\section{Kommunikation}
\subsection{Interne Kommunikation}
Die interne Kommunikation der Projektgruppe findet über 2 Kommunikationskanäle statt:
\begin{itemize}
	\item Slack
	\item Confluence
\end{itemize}

\paragraph{Slack} Über Slack können Privat- und Gruppen-nachrichten ausgetauscht werden. Diese Kommunikations Ebene ist von unformeller Natur und dient ausschließlich zum schnellen Austausch von Informationen und Fragen.

\paragraph{Confluence} Confluence dient zur Archivierung der internen Kommunikation und in diesen stattfindenden Entscheidungen. Dies betrifft zum Beispiel einen Kalender, in welchen Urlaubstermine und Arbeitstermine eingetragen werden, aber auch Sitzungs- und Entscheidungsprotokolle. 

\subsection{Kommunikation mit den Betreuern/Projektpartnern}
Bei der Kommunikation mit den Betreuern und anderen Projektpartnern wird ausschließlich die E-Mail verwendet. In diesen werden unter anderen Dokumente (immer als PDF), sowie Einladungen (mit Agenda und Ziel) verschickt. Die Kommunikation läuft ausschließlich über den Projektleiter. \\
\textbf{Mail-Verteiler:} \mailaddr{pg-rio@informatik.uni-oldenburg.de} (Enthält alle Projektgruppenmitglieder und -betreuer)

\subsection{Kommunikation mit externen Personen}
Die Kommunikation zu außen stehenden Personen erfolgt über den Projektleiter, um möglichen Verwirrungen vorzubeugen. In Einzelfällen kann die Kommunikation von einer im entsprechenden Thema involvierten Person erfolgen. Dann ist der Projektleiter beim Versenden externer Mails in cc: zu nehmen.\\
\textbf{Mail-Verteiler:} \mailaddr{pg-rio-alle@informatik.uni-oldenburg.de} (Enthält alle Projektgruppenmitglieder, -betreuer und externen Partner)