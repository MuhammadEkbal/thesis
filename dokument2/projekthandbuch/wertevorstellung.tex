\section{Wertevorstellung}
Unter Wertevorstellung verstehen wir u.A. Regeln, die jedes Mitglied der Gruppe zum Beispiel beim Gruppengespräch beachten muss. Dazu gehört:
\begin{itemize}
	\item Transparenz gegenüber allen Gruppenmitgliedern, sodass Probleme nicht "`verschluckt"' werden und später wieder aufkommen
	\begin{itemize}
		\item Evtl. zusammen ein Ticket besprechen in den Teams, sodass nicht ein Mitglied alleine aufkommende Probleme behandeln muss
	\end{itemize}
	\item Dem Kommunikationspartner beim Gespräch nicht ins Wort fallen - Ausreden lassen
	\item Keine unnötige Rechtfertigungen
	\begin{itemize}
		\item Kritik annehmen und in der Gruppe die Auswirkungen diskutieren
		\item Keine persönlichen Rechtfertigungen oder Ausreden wie Zeitmangel o.Ä.
		\item Entscheidungen und Abläufe dürfen und sollen aber erklärt werden
	\end{itemize}
	\item Auch über den Tellerrand schauen und nicht nur die eigenen Aufgaben betrachten, sodass z.B. montags nicht alles komplett neu ist
	\item Probleme können offen in den kleinen Gruppen oder um gesamten Team angesprochen werden
	\item Wenn jemand eine Aufgabe erledigt und diese dann im gesamten Team besprochen und kritisiert wird, ist die Kritik nicht persönlich gemeint, sondern sachlich. 
	\begin{itemize}
		\item Kritik muss auch sachlich vorgetragen werden
	\end{itemize}
\end{itemize}