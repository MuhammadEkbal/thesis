\section{Change Management}
Da ein Hardware/Software Projekt stetigen Änderungen von Anforderungen ausgesetzt ist, folgt in diesen Abschnitt ein Leitfaden zur Bearbeitungen dieser Anforderungsänderungen (Change Requests).
Ein Change Request ist eine Anfrage zu Änderung des Systems und damit in den meisten Fällen eine Anfrage zur Änderung von Anforderungen.
Dieser Änderungswunsch kann mehrere Ursprünge haben, jedoch sind die häufigsten:
\begin{itemize}
	\item Fehler des Systems (Bug)
	\item Erweiterungswunsch eines Nutzers
	\item Event bei der Entwicklung eines anderen Systems
	\item Änderungen in der Struktur oder bei den Standards
\end{itemize}
Tritt einer dieser Punkte auf, so wird im ersten Schritt ein Change Request als Issue in Jira angelegt, welcher durch einen Status gekennzeichnet wird.
In jeder Sitzung wird überprüft, ob Change Requests vorhanden sind.
Sind Change Requests vorhanden, so werden diese auf ihre Möglichkeit in der Umsetzung (Kosten und Nutzen Analyse) geprüft.
Sind Kosten und Nutzen in einen akzeptablen Verhältnis und ist der Change Request in der Projektdauer umsetzbar, so wird dieser Change Request angenommen und in eine normale Anforderung übersetzt, bzw. es werden bestehende Anforderungen angepasst.
Wird der Change Request abgelehnt, so wird dieser derartig gekennzeichnet und mit dem Grund der Ablehnung (als Beschreibung) in Jira archiviert.