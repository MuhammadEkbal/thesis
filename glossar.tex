%%Ein eigenes Symbolverzeichnis erstellen
%\newglossary[slg]{symbolslist}{syi}{syg}{Symbolverzeichnis}
%\newglossary[tlg]{translationlist}{tyi}{tyg}{Übersetzungsliste Deutsch-Englisch}

%%Den Punkt am Ende jeder Beschreibung deaktivieren
%\renewcommand*{\glspostdescription}{}

%Befehle für Symbole
% Zahlmengen
%\newglossaryentry{symb:N}{
%name=$\N$,
%description={nat\"urliche Zahlen, einschlie\ss{}lich $0$},
%sort=00n, type=symbolslist
%}
%\newglossaryentry{symb:Z}{
%name=$\Z$,
%description={ganze Zahlen},
%sort=01z, type=symbolslist
%}
%\newglossaryentry{symb:Q}{
%name=$\Q$,
%description={rationale Zahlen},
%sort=02q, type=symbolslist
%}
%\newglossaryentry{symb:R}{
%name=$\R$,
%description={reelle Zahlen},
%sort=03r, type=symbolslist
%}
%\newglossaryentry{symb:C}{
%name=$\K$,
%description={komplexe Zahlen},
%sort=04k, type=symbolslist
%}

%\newglossaryentry{symb:pot}{
%name=$\mathscr{P}$ ,
%description={Potenzmenge},
%sort=Potenzmenge, type=symbolslist
%}

%\newglossaryentry{symb:dis_ver}{
%name=$\bigcupdot$ ,
%description={die Vereinigung paarweise disjunkter Mengen},
%sort=disjunkte Vereinigung, type=symbolslist
%}

%\newglossaryentry{symb:emptyset}{
%name=$\emptyset$ ,
%description={leere Menge},
%sort=leere Menge, type=symbolslist
%}

%Befehle für Abkürzungen
\newacronym{CCS}{CCS}{Calculus of Communicating Systems}
\newacronym{CSP}{CSP}{Communicating Sequential Processes}
%Eine Abkürzung mit Glossareintrag
%\newacronym{AD}{AD}{Active Directory\protect\glsadd{glos:AD}}

%Übersetzungen Deutsch-Englisch
%\loadglsentries [translationlist] {translationlist.tex} 

\makeglossaries

% Einf"ugen der Zeichen ohne anzeige
%\glsadd{symb:N}
%\glsadd{symb:Z}
%\glsadd{symb:Q}
%\glsadd{symb:R}
%\glsadd{symb:C}
%\glsadd{symb:pot}
%\glsadd{symb:dis_ver}
%\glsadd{symb:emptyset}
