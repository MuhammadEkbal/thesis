\chapter{Evaluation}
In der Projektphase Stand die Projektgruppe vor vielen Hindernisse: von der Projektorganisation über die Erhebung der Anforderungen bis zur tatsächlichen Entwicklung des Systems. Daher soll an dieser Stelle eine Evaluation der wichtigsten Aspekte stattfinden. 

Hierfür wird zunächst eine Bewertung der Projektorganisation vorgenommen. Anschließend daran werden die Anforderungen sowie die Handhabung der Meilensteine näher betrachtet. In diesem Zusammenhang soll evaluiert werden, inwiefern die Erhebung der Anforderungen sinnvoll gestaltet wurde und an welcher Stelle Optimierungspotenzial besteht. 

Hinsichtlich unseres Visionsdokumentes soll zudem beurteilt werden, inwiefern die darin formulierten Ziele umgesetzt werden konnten. Dazu findet eine Untersuchung der Machbarkeit zum Beispiel für das Routing anhand von Feinstaubdaten statt. Um dies angemessen bewerten zu können, muss zudem eine evaluierende Betrachtung der Datenanalyse vorgenommen werden.


\section{Projektorganisation}
Die generelle Projektorganisation der Projektgruppe ist im Projekthandbuch festgelegt. 
Um das Projekt strukturiert starten zu können, mussten zunächst mehrere wöchentliche Gruppentreffen festgelegt werden.
Zu diesen Treffen wurden die Betreuer sowohl von der Universität als auch externe Projektbeteiligte eingeladen.
Ziel dabei war es, aktuelle Themen rund um das Projekt zu diskutieren und projektrelevante Entscheidungen zu treffen.
Diese wöchentlichen Termine waren für alle Projektgruppen Mitglieder verpflichtend.
Die entsprechenden Termin sind üblicherweise von einem Scrum-Master beziehungsweise Projektleiter vorbereitet worden.
Insgesamt hat die Organisation bei diesem Termin innerhalb der Gruppe und zwischen der Gruppe mit den Betreuern sehr gut funktioniert.
Abschließend zu den wöchentlichen Treffen lässt sich festhalten, dass gerade die Findungsphase im Projekt durch die regelmäßigen Treffen erheblich erleichtert wurde.
Die Mitglieder konnten sich recht schnell kennenlernen und auch die Verhältnisse zu den Betreuern wurde verbessert.
Dem gegenüber steht ein eher unregelmäßiges Teilnehmen der externen Betreuer an den regelmäßigen Treffen der Projektgruppe.
Die erschwert insbesondere das Einholen von Feedback zum Projektfortschritt.
Ein vermehrtes Erscheinen wäre an dieser Stelle wünschenswert gewesen. \par\medskip
Nachdem die Projektgruppe die Vorbereitungsphase inklusive Visions- und Anforderungsdokument etc. abgeschlossen hat, fand eine Einteilung in verschiedene Teilgruppen statt:Sensorknoten, IoT-Plattform, Routing und Navigation statt. 
Trotz der Einteilung in unterschiedliche Teilgruppen bestand weiterhin ein wöchentliches Treffen der Gesamtgruppe, um sich zum einen hinsichtlich der diversen Schnittstellen abzustimmen sowie ein Review gegenüber den Betreuern zu halten. 
Dabei zeigte sich, dass ein wöchentliches Treffen nicht ausreicht, um beispielsweise die Integration der Teilsysteme zu besprechen. 
Dementsprechend wurden weitere Termine beschlossen, in denen speziell die Integration sowie der Projektfortschritt thematisiert wurden. \par\medskip
Zur Dokumentation der Gruppentreffen und es aktuellen Arbeitsfortschritts in den Teilgruppen wurde die von der Universität Oldenburg bereitgestellte Wiki-Software Confluence verwendet. 
Des Weiteren wurde mit dieser Software auch die Schnittstellendefinition und Urlaubsplanung für alle Gruppenmitglieder offengelegt. 
Die Verwendung hat sehr gut funktioniert und unterstütze die Organisation in allen Phasen des Projekts.
Jedoch wurde der Inhalt der Wiki-Seite mit fortschreitender Projektzeit immer unübersichtlicher, sodass ein regelmäßiges Aufräumen nötig war.
Dies hat zum Teil Ressourcen gekostet, die an anderer Stelle sinnvoller hätten eingesetzt werden können. 
Trotzdem lässt sich festhalten, dass die Projektorganisation und Kommunikation innerhalb der Projektgruppe von der Wiki-Seite sehr profitiert hat.\par\medskip
Für die Planung, Zuweisung und Fortschrittskontrolle von Aufgaben wurde die von der Universität Oldenburg bereitgestellt Projektmanagement-Software Jira verwendet. 
Nach anfänglichen Schwierigkeiten bei der Bedienung der Software, konnte diese im weiteren Verlauf für ein effektiven Projektmanagement eingesetzt werden.
Insbesondere für die agile Arbeitsweise der Projektgruppe erwies sich Jira als geeignetes Tool.
Durch dieses konnten Sprints und Releases geplant sowie der Projektfortschritt einfach verfolgt werden. 
Auch in der Definition der User-Stories und beim Verfassen von Fehlerticktes wurde intensiv auf Jira zurückgegriffen.\par\medskip
Für alle organisatorischen, aufgabenbezogenen und sonstigen, aufkommenden Fragen außerhalb der Treffen in der Gruppe und in den Teilgruppen wurde die Chat-Software Slack verwendet. 
Diese Software erwies sich als sehr geeignet, da auf Funktionen wie Gruppenchats oder Einzelchats zurückgegriffen werden konnte. 
Um Probleme hinsichtlich Kommunikationsverzögerungen entgegen zu wirken, ist zudem eine Regel im Projekthandbuch festgelegt, welche festlegt, dass ein Projektgruppenmitglied innerhalb von 24 Stunden mindestens einmal die Nachrichten in Slack überprüfen muss.


Zusammenfassend kann festgehalten werden, dass die Projektgruppe von Anfang an einen hohen Wert auf die Projektorganisation gelegt hat. 
Dies zeigt sich auch im Entstehen des Projekthandbuchs, was chronologisch gesehen die erste Aktivität der Projektgruppe war. 
Zudem konnten die bereitgestellten Tools wie Jira und Confluence sinnvoll eingesetzt werden, sodass sie die Projektorganisation effizient verbessert haben.

\section{Anforderungen und Meilensteine}
Im Rahmen der Projektorganisation hat der Projektverlauf eine große Rolle gespielt. 
Aus diesem Grund soll in dem folgendem Kapitel die Erhebung der Anforderungen sowie das Planen der Meilensteine evaluiert werden.


Für die Erhebung der Anforderungen wurden aus der Vision heraus die Rollen: Navigationsnutzer, Umweltinformationssystemnutzer, Datenanalyst, Sensorknotenbetreiber und IoT-Administrator, abgeleitet. Aus Sicht dieser Rollen wurden für die Projektgruppe Anforderungen initial in der ersten Projektphase nach dem Wasserfallmodell erhoben. Schwierigkeiten traten bei der Erhebung der Anforderungen in dieser Projektphase bei Anforderungen auf, die sehr technisch geprägt waren. Diese konnten aus Sicht der abgeleiteten Rollen nur schwer erhoben werden.
Dies zeigte sich daraufhin auch in der Umsetzung der Anforderungen, da beispielsweise die IoT-Plattform in vielen nutzergetriebenen Anforderungen eine Rolle gespielt hat.
Aus diesen Grund sind viele Anforderungen aus Sicht des IoT-Administrators nicht umgesetzt worden.
Der Fokus lag dabei oft eher auf den funktionalen Anforderungen aus Sicht des Navigationsnutzers oder des UIS-Nutzers. 
An diesen Stellen waren die Services der IoT-Plattform sehr wichtig, da nur so Sensordaten angezeigt beziehungsweise verarbeitet werden konnten.\par\medskip
In der Entwicklungsphase fand zudem eine Aufwandsschätzung in Form eines Refinements der erhobenen Anforderungen statt. 
Zu Beginn der Entwicklungsphase waren diese Schätzungen noch sehr ungenau, sodass der Aufwand oft zu gering geschätzt wurde. Zum Ende der Projektgruppe wurden die Schätzungen jedoch immer genauer. Hier konnte eine deutliche Steigerung der Abstimmung und der Selbsteinschätzung in der Projektgruppe festgestellt werden.\par\medskip
Auf Grund der Vielzahl an Schnittstellen zwischen den einzelnen Komponenten des Gesamtsystems, fanden viele Überschneidungen in der Anforderungserhebung aus Sicht der abgeleiteten Rollen statt. 
Aus diesem Grund wurde im Verlauf der Entwicklungsphase bei der Erhebung neuer Anforderungen die eingeschränkte Sichtweise einzelner Rollen verlassen. 
Zum Ende der Projektgruppe wurden Anforderungen stark aus der Sicht des Endbenutzers erhoben.
Anforderungen aus Sicht des Endnutzers konnten die Problematik der eingeschränkten Sichtweise einzelner Rollen, sowie auch die anfänglichen Schwierigkeiten bei sehr technisch geprägter Anforderungen, stark entgegenwirken. 
Bei der Erhebung der Anforderungen konnte, wie auch beim Schätzen der Anforderungen, eine deutliche Steigerung Leistungsfähigkeit in der Projektgruppe festgestellt werden.\par\medskip
Zur Umsetzung der erhobenen Anforderungen wurden in der Entwicklungsphase insgesamt fünf Meilensteine geplant. Ein Meilenstein galt als vollständig erreicht, wenn alle in einen Meilenstein eingeplanten Anforderungen vollständig umgesetzt wurden. Der erste Meilenstein enthielt lediglich Anforderungen zum Aufsetzen der einzelnen Projektsetups der Teilgruppen. Die Abweichung der eingeplanten Anforderungen und tatsächlich umgesetzten Anforderungen war im zweiten Meilenstein im Vergleich zu den folgenden Meilensteinen noch recht hoch. Grund dafür war insbesondere noch die sehr ungenaue Schätzung des Aufwands zu Beginn der Entwicklungsphase. Die Abweichung des folgenden dritten und vierten Meilensteins ist im Vergleich zum zweiten Meilenstein deutlich geringer ausgefallen. Abweichungen bei diesen beiden Meilensteinen sind insbesondere mit der Schnittstellenproblematik und der Integration der Sensoren aus dem SCHIT-Projekt begründet. Im fünften und letzten Meilenstein konnten alle eingeplanten Anforderungen umgesetzt werden.\par\medskip
Auch bei der Planung der Meilensteine konnte im Verlauf des Projekts eine deutliche Steigerung der Leistungsfähigkeit in der Projektgruppe festgestellt werden. Die Abweichung von eingeplanten und umgesetzten Anforderungen im zweiten Meilenstein war recht hoch, im letzten Meilenstein konnten alle eingeplanten Anforderungen ohne Abweichung umgesetzt werden. Diese Verbesserung bei der Planung der Meilensteine ist insbesondere damit Begründet, dass Anforderungen zum Ende der Projektgruppe deutlich besser geschätzt wurden und das Anforderungen aus Endnutzersicht betrachtet wurden. Eine nähere Betrachtung und Evaluation der einzelnen Meilensteine kann auch in dem Kapitel zur Projektumsetzung eingesehen werden.

\section{Routing anhand von Feinstaub}
In diesem Abschnitt wird reflektiert, wie sinnvoll es ist in Oldenburg nach Feinstaubdaten zu navigieren. Die Betrachtung muss jedoch immer in den Kontext der Projektgruppe und deren Ziele und Möglichkeiten gesetzt werden.

Im Routing-Dienst werden Routen berechnet, indem die Route gewählt wird, dessen Gesamtgewicht am niedrigsten ist. Das Gesamtgewicht ist dabei die gewichtete Summe der einzelnen Streckenabschnitte. Diese Gewichte werden zum einen aus der Streckenlänge und zum anderen einem zugehörigen Feinstaubwert berechnet.

Die Differenz der zu einem Zeitpunkt gemessenen minimal und maximal Werte ist jedoch im Allgemeinen sehr gering, da vermutlich in Oldenburg allgemein entweder eine geringe oder hohe Feinstaubkonzentration vorzufinden ist. Deshalb unterscheiden sich die Gewichte der einzelnen Strecken meistens nur durch ihre Distanz und nicht signifikant durch den vor Ort gemessenen Feinstaubwert. 
Die im Rahmen dieses Projekts aufgenommenen Feinstaubwerte liefern leider keine umfangreiche Stichprobe, um das System mit realistischen Werten validieren zu können.
Dabei muss jedoch auch die Position der Feinstaubsensoren berücksichtigt werden. Um diese Werte entsprechend zu berücksichtigen, muss jedoch eine umfassende Datenanalyse stattfinden, in der unter anderem die Ausbreitung von Feinstaubdaten auch unter Windeinfluss und anderen Einflussfaktoren berücksichtigt werden.

Abschließend lässt sich festhalten, dass der Routing-Algorithmus funktioniert und erfolgreich um Gebiete mit hohen Feinstaubbelastungen herum routet. 
Theoretisch ist es möglich, bei der Berechnung der Gewichtung einer Strecke für den Routing-Algorithmus den Einfluss der Feinstaubwerte zu erhöhen. Um jedoch hierfür eine sinnvolle Entscheidung treffen zu können, wird Wissen in der Datenanalyse, zum Beispiel über Einfluss von Feinstaub menschlichen Körper, benötigt. In diesem Kontext muss untersucht werden, wie sich die zeitliche Belastung auf den menschlichen Organismus auswirkt. Konkret stellt sich die Frage, ob es gesünder ist, über einen kurzen Zeitraum hohen Belastungen ausgesetzt zu sein oder ob dies bei einer längeren Belastung mit geringeren Werten der Fall ist. Da dieses Wissen in der Projektgruppe nicht gegeben ist und die Ressourcen nicht vorhanden waren, um dieses Wissen im Rahmen des Projekts anzueignen, wurde eine eins-zu-eins Gewichtung von Distanz zu Feinstaubwert angewendet. Durch die entsprechende Architektur des Routing-Algorithmus ist es jedoch mit geringem Aufwand möglich, den Algorithmus so anzupassen, dass Erkenntnisse aus der Datenanalyse in diesen einfließen können.


\section{Datenanalyse}
Um die von unseren Sensoren erzeugten Daten zu analysieren, haben wir einige Datenanalyseprozesse durchgeführt, um Wissen über die Datengeneration unserer Sensoren zu erhalten.
Mit Rapidminer wurden die Daten generell überprüft und grundlegende Korrelationen berechnet.
Außerdem wurden Lücken herausgefiltert, um Ausfälle in den Messungen näher analysieren zu können.
Dabei haben einige Sensoren Auffälligkeiten gezeigt.

Allerdings konnte eine kurz gehaltene Analyse keine validen Ergebnisse mit sich bringen. 
Für weitere Analysen im Rahmen der Projektgruppe fehlte zum Einen der naturwissenschaftliche Hintergrund innerhalb der Gruppe sowie die Ressourcen, um eine ausführliche Analyse durchzuführen.
Eine Verlagerung der Kapazitäten einzelner PG-Mitglieder in die Analyse hätte andere Aufgaben aufgehalten, sodass wir uns gegen die Verteilung unserer Ressourcen in diesen Bereich entschieden haben.

Um eine umfangreiche Datenanalyse dennoch durchführen zu können, bietet sich eine Zusammenarbeit mit anderen Modulen wie dem Datenbankpraktikum an.
Hierbei muss jedoch erwähnt werden, dass das Gesamtsystem so aufgesetzt ist, dass Datenanalysten das Projekt erweitern können.
Demnach ist eine Datenanalyse seitens der Projektgruppe nicht zwingend notwendig, sondern könnte im weiteren Betrieb des Systems durch externe Datenanalysten übernommen und umgesetzt werden.
Nähere Infos dazu befinden sich im Kapitel Ausblick.
