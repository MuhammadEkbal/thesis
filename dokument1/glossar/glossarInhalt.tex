\textbf{Aufbereitungsdienst:}  Ein Aufbereitungsdienst verändert Umweltdaten, die der IoT-Plattform bereitgestellt werden.\\
\\
\textbf{aufbereitete Daten:} Aufbereitete Daten sind Daten, die das Ergebnis eines Berechnungsverfahren eines Aufbereitungsdienstes sind. \\
\\
\textbf{Akzeptanzkriterium:} Ein Akzeptanzkriterium ist eine Anforderung an eine User-Story, die zum Zeitpunkt der Fertigstellung der User-Story erfüllt sein muss. \\
\\
\textbf{BME280:} Der BME280 ist der Sensor für Luftdruck, Temperatur und ralative Luftfeuchte. \\
\\
\textbf{Datenqualität:}  Unter Datenqualität werden Daten verstanden, die vom Sensor gemessen wurden und nicht durch Einflussfaktoren verfälscht sind.\\
\\
\textbf{Definition of Done:} Eine Definition of Done ist eine Art Checkliste. Erst wenn alle Kriterien der Definition of Done erfüllt sind, darf die Story (Anforderung) als "erledigt" beziehungsweise fertig implementiert angesehen werden. Die Definition of Done wird in jedem Story-Ticket gepflegt. \\
\\
\textbf{Definition of Ready:} Eine Definition of Ready ist auf User Stories bezogen und gibt an, ab wenn eine User Story so gepflegt ist, dass sie "ready" ist, um in einem Sprint bearbeitet zu werden.  \\
\\
\textbf{Dienst:}  Ein Dienst stellt der IoT-Plattform Daten bereit oder fragt Daten ab.\\
\\
\textbf{Dynamisches Routing:}
Dynamisches Routing ist ein Verfahren, in welchem aufgrund von veränderten Umweltdaten die Route während der Navigation neu berechnet wird.\\
\\
\textbf{Dynamische Navigation:}
Dynamische Navigation ist die Anpassung der Leitung des Nutzers entlang der Route. Diese Anpassung erfolgt aufgrund der Nichteinhaltung der Route oder aufgrund der Auswahl einer alternativen Route, die aufgrund des dynamischen Routings zur Verfügung steht.\\
\\
\textbf{Epic:} Ein Epic ist die Abstraktionsebene unter den Super Epics. Sie zerlegt das Super Epic in mehrere Epics, die das jeweilige Super Epic spezifizieren. \\
\\
\textbf{Externer Dienst:} Ein externer Dienst ist ein Dienst, der außerhalb der IoT-Plattform liegt und Daten von dieser abfragt. Das kann z.B. der Routing Dienst sein.\\
\\
\textbf{Event - Benachrichtigung:} Eine Event - Benachrichtigung ist eine Mitteiöung, die dem Nutzer angezeigt wird, wenn z.B. ein Verbindungsverlust auftritt oder der Standort eines Sensorknotens verändert wurde. \\
\\
\textbf{Feinstaub:} Feinstaub ist Staub, der aus kleinen Partikeln besteht, wie z.B. gefährliche Stoffe wie Schwermetall aber auch Staubpartike selbst. Dieser kann verschiedene Größen haben; PM 2.5 und PM 10. \\
\\
\textbf{Firmware:} Unter Firmware wird die Software verstanden, die auf dem Sensorknoten läuft.\\
\\
\textbf{GNSS:} Ein GNSS ist ein Global Navigation Satellite-System. Dazu gehören zum Beispiel "GPS" als amerikanisches Satellitensystem, oder "Galileo" als europäisches System.\\
\\
\textbf{Highcharts:}
Bei Highcharts handelt es sich um ...\\
\\
\textbf{historische Daten:} Historische Daten sind Daten aus früheren Messungen der Sensorknoten.  \\
\\
\textbf{IoT:}
IoT ist die Abkürzung für \dq Internet of Things\dq  und ist ein Sammelbegriff für Technologien einer globalen Infrastruktur der Informationsgesellschaften, die es ermöglicht, physische und virtuelle Gegenstände miteinander zu vernetzen und sie durch Informations-und Kommunikationstechniken zusammenarbeiten zu lassen.\\
\\
\textbf{IoT-Plattform:} Die IoT-Plattform nimmt Daten von Sensoren an, speichert diese ab und stellt sie externen Diensten bereit. Außerdem authentifizieren sich Sensoren externe Dienste und Aufbereitungsdienste mit der IoT-Plattform. \\
\\
\textbf{JIRA:} JIRA ist ein Ticketing-System im Projektmanagement, mithilfe dessen für unterschiedliche Anliegen Tickets erstellt und zugewiesen werden können.  \\
\\
\textbf{JIRA-Label:} Ein Label beschreibt in JIRA ein Stichwort, welches einheitlich für Tickets gepflegt werden kann.  \\
\\
\textbf{JWT:} Ein JSON-Web-Token (kurz JWT) ist ein auf JSON basiertes Zugriffstoken.  \\
\\
\textbf{korrekte Daten:} Korrekte Daten sind die tatsächlich vom Sensor gemessenen Daten.\\
\\
\textbf{Metadaten oder Metainformationen:} Metadaten oder Metainformationen beschreiben die eigentlichen Daten, indem sie Informationen über Merkmale der Daten enthalten, wie z.B. Messverfahren, ID oder den Standort. \\
\\
\textbf{MongoDB:} MongoDB ist eine dokumentenorientierte NoSQL-Datenbank, mit Hilfe derer man unter anderem JSON-ähnliche Dokumente verwalten kann.  \\
\\
\textbf{Navigation:}
Navigation ist die Leitung eines Nutzers entlang einer berechneten Route.\\
\\
\textbf{Navigationsanwendung:} Die Navigationsanwendung ist die Endanwendung, mithilfe derer sich der Nutzer eine Route generieren lassen kann und auf ebendieser Route navigiert wird.  \\
\\
\textbf{Navigationsnutzer:} Der Navigationsnutzer ist der Nutzer der Navigationsanwendung. Er kann sich anhand festgelegter Parameter eine Route generieren lassen und hat die Möglichkeit über die ausgewählte Route navigiert zu werden.  \\
\\
\textbf{nicht - funktionale Anforderungen:} Nicht - funktionale Anforderungen sind Anforderungen an die Qualität der Funktionalität, wie z.B.Performance. \\
\\
\textbf{NodeMCU:} Das NodeMCU-Board ist ein Kommunikations - und Verarbeitungssystem für den Sensorknoten. \\
\\
\textbf{PM 2,5:} Der Feinstaub besteht aus 50 Prozent der Teilchen mit einem Durchmesser von 2.5  $\mu$m.\\
\\
\textbf{PM10:} Der Feinstaub besteht aus 50 Prozent der Teilchen mit einem Durchmesser von 10 $\mu$m. \\
\\
\textbf{Prognose:} Eine Prognose ist eine Vorhersage über die Feinbelastung oder über andere Umweltdaten. \\
\\
\textbf{Rohdaten:} Rohdaten sind die Daten, die nicht durch einen Dienst aufbereitet oder verändert wurden. \\
\\
\textbf{Routing:}
Routing ist ein Verfahren, in welchem aufgrund von Umweltdaten eine Strecke von einem eingegebenen Start- und Endpunkt berechnet wird.\\
\\
\textbf{SDS011:} Der SDS011 ist der Feinstaubsensor.\\
\\
\textbf{Stakeholder:} Ein Stakeholder ist eine Person oder Gruppe, die ein berechtigtes Interesse am Verlauf oder Ergebnis eines Prozesses oder Projektes hat. \\
\\
\textbf{Sensor:} Ein Sensor ist ein technisches Bauteil, das bestimmte physikalische (oder chemische Eigenschaften) seiner Umgebung qualitativ oder als Messgröße quantitativ erfassen kann.\\
\\
\textbf{Sensorknoten:} Ein Sensorknoten ist eine Komposition bestehend aus einem Verarbeitungssystem und mehreren Messsystemen. Dabei besteht das Verarbeitungssystem aus einem Controller und einer zugehörigen Kommunikationseinheit. Das Messsystem sind Sensoren, die die Umweltdaten messen, wie zum Beispiel Luftdruck, Feinstaub oder Temperatur. \\
\\
\textbf{Super Epic:} Ein Super Epic ist die höchste Abstraktionsebene beim Formulieren der Anforderungen. Sie beschreibt grob einen Themenkomplex und ist in mehrere darunterliegende Epics aufgeteilt. \\
\\
\textbf{Unixformat:}
Das Unixformat ist....\\
\\
\textbf{Umweltdaten:} Umweltdaten sind Werte über den Zustand von Umweltbestandteile, die durch Beobachtungen, Messungen u.a. gewonnen wurden.
\\
\textbf{UserStory:} Eine User Story ist eine Nutzergeschichte, die eine funktionale Anforderung an das System aus Sicht eines bestimmten Stakeholders formuliert. Sie zerlegt das Epic in konkrete funktionale Anforderungen.  \\
\\
\textbf{Task:} Eine Task ist ein Tickettyp in Jira, der üblicherweise User Stories zugeordnet wird und die technische Umsetzung der Story beschreibt. Tasks können aber auch unabhängig von User Stories verwendet werden, um Aufgaben zu verteilen.  \\
\\
\textbf{UI:} Die UI (User Interface) stellt die visuelle Schnittstelle zum Nutzer dar. Hier ist insbesondere die Benutzeroberfläche des Systems gemeint, welche durch die Oberfläche der Navigationsanwendung und der Webanwendung abgebildet wird.\\
\\
\textbf{Umweltdaten:} Umweltdaten sind Werte über den Zustand von Umweltbestandteilen, die durch Beobachtungen, Messungen u.a. gewonnen wurden.
Zu den Umweltbestandteilen gehören z.B. Boden, Luft oder Gewässer.\\
\\
\textbf{virtuelle Sensorknoten:} Ein virtueller Sensorknoten ist ein Sensorknoten, der nicht physisch Daten mittels Sensoren misst sondern Daten bereitstellt, die mittels verschiedener Berechnungsverfahren, wie z.B. der Mittelwert, berechnet wurden. \\
