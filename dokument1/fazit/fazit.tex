\chapter{Fazit}
In diesem Kapitel wird ein Fazit über das gesamte Projekt gezogen. Um die Beurteilung nachvollziehbar zu gestalten, wird zuerst ein kurzer Überblick über das Projekt und dessen Verlauf gegeben.
Im Anschluss daran wird ein Fazit über die Umsetzung des Projekts und insbesondere der anfangs definierten Vision gezogen. 
Dabei werden wichtige Erkenntnisse hervorgehoben.
\section{Rückblick}

Wie in der Vision beschrieben (vgl. \Fref{sec:Motivation}) ist die Feinstaubbelastung ungesund für den Menschen und gerade in Ballungsgebieten häufig erhöht. 
Um Menschen eine Möglichkeit zu bieten, diese Feinstaubbelastung wahrzunehmen und stark belastete Gebiete zu umgehen, hat diese Projektgruppe das Ziel gehabt, ein Produkt zu entwickeln, welches die Navigation anhand von Umweltdaten in Oldenburg ermöglicht (vgl. \Fref{sec:Zieldefinition}). 
Dafür musste zum einen ein sensorbasiertes Umweltinformationssystem entwickelt werden, welches die Umweltdaten der ausgebrachten Sensorknoten entgegen nimmt und auswertet. 
Zum anderen musste eine Navigationsanwendung entwickelt werden, die basierend auf den Daten des Umweltinformationssystems Routen mit einer geringeren Feinstaubbelastung bereitstellen kann.
Dieses Projektziel wurde dabei in vier Teilprojektziele aufgeteilt: 
Die Entwicklung von \textbf{Sensorknoten}, einer \textbf{IoT-Plattform}, einer \textbf{Routing}- und \textbf{Navigationskomponente} und der Einhaltung einer hohen \textbf{Datenqualität} über alle zuvor genannten Komponenten.

Nach der  Definition der Vision und des darin beschriebenen Projektziels, musste die grundsätzliche Organisation und ein generelles Vorgehen bei der Umsetzung des Projektes festgelegt werden. 
Für diesen Zweck wurde ein Projekthandbuch angefertigt und ein Vorgehensmodell (vgl. Dokument 2 - Vorgehensmodell) erarbeitet. 
Das Vorgehensmodell beschreibt dabei die Phasen, die bei der Entwicklung des oben genannten Produkts durchlaufen werden sollen. 
Diese lassen sich in zwei Phasen aufteilen.
Die \textbf{Defintionsphase} beschreibt ein wasserfall-ähnliches Vorgehen von der Zieldefinition über die Anforderungsanalyse bis hin zur Beschreibung der System- und Softwarearchitektur. 
Die \textbf{Entwicklungsphase} beschreibt ein auf dem agilen Manifest beruhenden Ablauf, der auf die Arbeiten der Definitonphase aufbaut.

Wie im Vorgehensmodell beschrieben, wurde nach der Zieldefinition eine Anforderungsanalyse durchgeführt. Hierbei wurden die Stakeholder des Projekts von der Projektgruppe befragt, welche Funktionen das Produkt bereitstellen soll.
Die Interviewergebnisse wurden nach einer sorgfältigen Analyse innerhalb der Projektgruppe in einem Anforderungsdokument zusammengefasst. 
Da das Vorgehensmodell im späteren Projektverlauf einen agilen Ansatz nutzt, wurden bereits im Anforderungsdokument Begriffe aus dem agilen Entwicklungsprozess genutzt. 
Anforderungen wurden hier zu  Epics und zur weiteren Gruppierung den sogenannten SuperEpics zugewiesen.

Basierend auf den Anforderungen wurde nun das genauere Vorgehen besprochen. Um die nächste Phase, die Definition der System- und Softwarearchitektur, zu vereinfachen, wurden Kleingruppen festgelegt, die an unterschiedlichen Komponenten des Systems arbeiten sollten. 
Um im späteren Projektverlauf mögliche Mehrbelastungen einzelner Gruppen entgegenzuwirken, wurde jede Person zwei Kleingruppen zugeteilt. 
Innerhalb der Gruppen wurden die einzelnen Architekturen ausgearbeitet. 
Schnittstellen zwischen den einzelnen Kleingruppen wurden innerhalb der Projektgruppe besprochen.

Nach der Festlegung der Architektur wurde die Definitionsphase abgeschlossen und es konnte mit dem agilen Entwicklungsprozess begonnen werden, welcher in Sprints gegliedert wurde. 
Diese Sprints waren meist zwei Wochen lang waren und inhaltlich so aufgestellt, dass eine gewisse Anzahl an Sprints auf einen Meilenstein hinarbeiten, welcher zuvor innerhalb der Projektgruppe festgelegt wurden. 
Dieser iterative Prozess dauerte mehrere Monate.

In der Definionsphase wurde von der Projektgruppe festgelegt, dass es einen Monat vor Projektende einen sogenannten "`Feature Freeze"' gibt, sodass ab diesem Zeitpunkt die Entwicklung endet und nur noch an bekannten Fehlern und der Dokumentation und Präsentation gearbeitet werden darf. 
Bis zu diesem Zeitpunkt sollte das Minimalziel der Vision erreicht worden sein.

\section{Beurteilung des Verlaufs}
In den Anfangswochen der Projektgruppe, in denen viel Organisationsarbeit geleistet werden musste, konnte in den wöchentlichen Gruppentreffen viel konstruktiv besprochen werden. Aus diesen Treffen resultieren unter anderem die Vision und das Projekthandbuch, welches das Vorgehensmodell beinhaltet.
Die Interviews zur Erhebung der Anforderungen wurden von allen Teilnehmern in kleinen Gruppen gut vorbereitet und durchgeführt, sodass die Erstellung eines Anforderungsdokumentes möglich war.

Während der Phase der Anforderungsanalyse wurde ein Nebenprojekt gestartet. Der Schülerinformationstag, kurz SCHIT, sollte genutzt werden, um das Projekt nach außen sichtbarer zu machen und um Sensorknoten an Schulen auszubringen. Gerade Letzteres wäre für das Projekt selbst sinnvoll, da eine gewisse Abdeckung in der Vision vorgeschrieben ist. 
Zu diesem Zweck sollte ein SCHIT-Prototyp entwickelt werden, der Sensorknoten, eine prototypische IoT-Plattform und ein Frontend zur Anzeige der Sensorknoten-Daten umfasst. 
Obwohl die Anforderungsphase des eigentlichen Projektes nicht abgeschlossen war, konnte bereits zu diesem frühen Zeitpunkt erfolgreich ein Prototyp unseren späteren Systems entwickelt und das Projekt an dem Schülerinformationstag vorgestellt werden. 
Dies konnte durch einen gewissen Mehraufwand der Projektteilnehmer erreicht werden. 
Parallel wurde die Erstellung des Anforderungsdokuments erfolgreich beendet.

Die Erfahrung aus der Entwicklung des Prototypen wurde bei der Erstellung der Architekturen der einzelnen Komponenten berücksichtigt. 
Die Abstimmungen der Kleingruppen bezüglich der Schnittstellen lief gut, da durch die wöchentlichen Gesamttreffen gewisse Dinge angesprochen werden konnten. 

Die Sprints innerhalb der agilen Entwicklungsphase wurden gut vorbereitet, sodass jeder Projektteilnehmer beim Start in einen neuen Sprint klar definierte Aufgaben hatte. 
Die Aufgaben konnten häufig innerhalb des Sprints abgeschlossen werden. 
Bei Problemen innerhalb der Aufgaben war die Projektgruppe untereinander sehr hilfsbereit, sodass auch öfters inoffizelle Treffen zustande kamen.
Bei krankheitsbedingten Ausfällen konnte durch die Doppelbelegung der Kleingruppen oft ein Ausgleich bei der Belastung geschaffen werden.

\section{Beurteilung des Projekterfolgs}
Durch die oben dargestellte Arbeitseinstellung der Projektteilnehmer war es möglich das Minimalziel des Projekts (vgl. \Fref{sec:Zieldefinition}) zu erreichen und in gewissen Punkten zu übertreffen. 
Das Umweltinformationssystem, bestehend aus den Sensorknoten und der IoT-Plattform, funktioniert wie im Ziel definiert. 
Darauf aufbauend gibt es eine Navigationsanwendung, welche die Berechnung von Routen anhand der Sensordaten der ausgebrachten Sensorknoten zulässt.
Darüber hinaus gibt es noch ein Umweltinformations-Frontend, welches Informationen zu den Sensorknoten darstellt.

Die im Rückblick genannten Projektteilziele wurden dennoch nicht vollständig erreicht. Gewisse Probleme, wie die geringe Auswirkung der Feinstaubwerte auf die Navigation, wurden erst spät festgestellt.
Auch das Teilziel \textbf{Datenqualität} konnte nicht vollständig erfüllt werden. 
Es werden wohl Feinstaubwerte von der IoT-Plattform aufbereitet, allerdings nicht in dem Umfang, wie es eingangs festgelegt wurde. 
Dies ist auf das fehlende Fachwissen mit dem Umgang solcher Werte zurückzuführen. 
Eine frühzeitige fachübergreifende Zusammenarbeit mit Umweltwissenschaftlern wäre hier sinnvoll gewesen. 

Insgesamt wurde das Projekt sehr erfolgreich beendet.
