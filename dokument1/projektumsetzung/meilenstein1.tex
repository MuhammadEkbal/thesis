\section{Meilenstein 1: 10.04.2019 - 24.04.2019}
Der erste Meilenstein 1 in der Entwicklungsphase zeichnet sich dadurch aus, dass in diesem noch keine konkreten funktionalen Anforderungen umgesetzt werden.
Der Fokus des Meilensteins liegt auf der Einarbeitung und dem Aufbau der unterschiedlichen Softwareprojekte.

Diese Phase erweist sich als sehr wichtig für die Projektgruppe, da die einzelnen Komponenten zum Teil in unterschiedlichen Programmiersprachen erstellt werden.
Zudem werden verschiedenste Frameworks genutzt, um die Entwicklungsphase zu erleichtern.
So steht neben dem Projektsetup auch noch viel Recherchearbeit zu möglichen Plugins und Frameworks an, die zwar in der Architektur bereits betrachtet wurden, jedoch bis zu diesem Zeitpunkt noch nicht zum tatsächlichen Einsatz gekommen sind.
Das Ziel ist es, dass mit Abschluss des Meilensteins, die PG=Mitglieder die tatsächliche Umsetzung von funktionalen Anforderungen in Form von User Stories abarbeiten können.
Innerhalb des ersten Meilensteins werden folgende Komponenten als Projekt aufgesetzt: der Sensorknoten, die IoT"=Plattform, der Routing"=Service und die Navigationsapplikation.

Da im ersten Meilenstein fast keine Anforderungen aus Sicht eines Endnutzers erfüllt werden, gibt es hier keine Demo beziehungsweise eine konkrete Präsentation, mit welcher der Meilenstein als abgeschlossen gilt.
Als Ergebnis lässt sich jedoch festhalten, dass alle geplanten Ziele größtenteils erreicht wurden.
Jedoch ist der Aufwand des Aufsetzens der Projekte in den verschiedenen Teilbereichen unterschiedlich Komplex.
Aus diesem Grund konnte in einigen Gruppen auch schon mit der Umsetzung von funktionalen Anforderungen begonnen werden.
Durch die Fertigstellung der Projektsetups kann im folgenden zweiten Meilenstein mit der Umsetzung der Anforderungen begonnen werden.
Dabei muss jedoch berücksichtigt werden, dass neben den oben genannten Komponenten noch weitere Komponenten im Laufe der Entwicklungsphase hinzukommen werden, welche auch wieder eine kurze Einarbeitungszeit benötigen.
Dazu gehören beispielsweise ein Frontend für den Umweltinformationssystem"=Nutzer oder eine Oberfläche zum Verwalten der Sensorknoten.
Diese Einarbeitungszeit wird dann im Refinement der einzelnen User Stories berücksichtigt.