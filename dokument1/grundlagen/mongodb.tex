\section{MongoDB}
\label{sec:grundlagen:datenbank:mongodb}
In diesem Kapitel wird näher auf die Entscheidung für eine MongoDB als Datenbank eingegangen.
MongoDB ist eine dokumentenorientierte NoSQL-Datenbank.
Somit ist MongoDB nicht an einem festen Datenbankschema gebunden, was als Nachteil betrachet werden kann.
Allerdings ist es möglich Validierungsregeln für Collections einzurichten.
Dadurch können nur Daten der Datenbank hinzugefügt werden, die ein gefordertes Format besitzen.
Eine Collection ist eine Sammlung von MongoDB-Dokumenten, also eine Sammlung von Datensätzen\cite{mongodbManual}.
Außerdem können so einfach neue Daten bzw. Collections hinzugefügt werden.
Also ist die Datenbank erweiterbar für zum Beispiel andere Umweltdaten.
Die Daten innerhalb einer Collection werden in einem JSON-ähnlichen Dokumentenformat, intern im BSON-Format, gespeichert.
Somit ist die Speicherung aussagekräftiger und leistungsfähiger als das Zeilen-Spalten-Modell in SQL Datenbanken.
Denn sie Umformatierung ist nicht mehr notwendig, weil bereits das JSON verwendet werden kann\cite{mongodb}.
\newline
MongoDB bietet verschiedene Abfragen, um Dokumente in einer Collection zu finden.
Hierbei ist zu beachten, dass lediglich eine Collection mit einer Abfrage durchsucht werden kann.
Somit ist der Join-Operator, den es in SQL Datenbanken gibt, nicht verfügbar.
Daher sollten die Abfragen breits bei der Erstellung der Collections berücksichtigt werden.
Ab Version 3.2 ermöglicht MongoDB Joins als \textit{lookups}\cite{mongodbJoin}.
Abgesehen von dieser Einschränkung ist die Abfragepsrache, die MongoDB bietet leistungsfähig.
Denn die Filterung und Sortierung der Daten ist unabhängig von der Tiefe der Verschachtelung in dem Dokument.
Ein weiterer Vorteil der Abfragesprache sind die unterstützten Aggregationen.
Insbesondere sind hier geobasierte Suchen möglich, die in dem PG RiO Projekt von Vorteil, für die Abfragen des Routing Dienstes, sind.
Außerdem haben Abfragen das JSON-Format und können so einfach zusammengesetzt werden.
Also müssenkeine Zeichenfolgen mehr verkettet werden, um SQL-Abfragen zu generieren\cite{mongodb}.
\newline
MongoDB bietet außerdem das Erstellen von Indizes auf Collections.
So können die Abfragen effizient in MongoDB ausgeführt werden, denn mittels des Indizes wird die Anzahl der zu prüfenden Dokumente verringert.
Ohne Indizes müssen Collection-Scans durchgeführt werden.
Also muss jedes Dokument einer Collection überprüft werden, um festzustellen, ob dieses Dokument mit der Query-Anweisung übereinstimmt.
Da in der Datenbank eine große Datenmenge gespeichert wird, ist das Verwenden der Indizes sinnvoll\cite{mongodbIndices}.
\newline
Der ausschlaggebene Grund für die Entscheidung eine MongoDB zu verwenden, neben den zuvor genannten Vorteilen, ist, dass das verwendete Framework NestJS eine sehr gute Dokumentation zur Verwendung von MongoDB in Kombination mit NestJS bietet\cite{mongoNest}.