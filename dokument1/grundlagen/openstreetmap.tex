\section{OpenStreetMap}
OpenStreetMap (kurz OSM) ist ein 2004 gegründetes Projekt, in dem eine frei verfügbare Weltkarte fortlaufend geschaffen wird. Dafür werden weltweit Daten über Straßen, Wälder, Häuser und andere Dinge gesammelt und in die Karte eingefügt. Die Daten von OSM werden täglich erneuert, sodass auch Gegebenheiten wie Baustellen auf Autobahn oder Ähnliches berücksichtigt wird.

Für die Projektgruppe kann die OpenStreetMap in mehreren Bereichen eingesetzt werden. So dient das Kartenmaterial von OSM als Grundlage für den Routing-Algorithmus, die Navigationsapplikation und das UIS-Frontend. Der große Vorteil von OSM gegenüber zum Beispiel Google Maps besteht darin, dass OSM nicht nur kostenlos, sondern die Nutzung auch komplett lizenzfrei ist\cite{OpenStreetMapFAQ}.
