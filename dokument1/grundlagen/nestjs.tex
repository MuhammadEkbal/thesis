\section{NestJS}
NestJS wurde als Framework für die Entwicklung der IoT-Plattform ausgewählt, weil NestJS die Möglichkeit zum Erstellen von effizienten und skalierbaren Node.js serverseitigen Anwendungen bietet. Dieses Framwork nutzt progressives JavaScript und ist mit Typescript aufgebaut. Daher unterstützt NestJS vollständig Typescript. Da JavaScript sich durch Node.js als häufig genutzte Programmiersprache für das Front - und Backend entwickelt hat, sind dadurch viele Projekte enstanden, wie zum Beispiel Angular oder Vue. Diese haben die Produktivität, also die Erstellung von schnellen, testbaren und erweiterbaren Applikationen ermöglicht. NestJS hingehen bietet nicht nur diese Vorteile, sondern auch eine sofort einsatzbereite Anwendungsarchitektur. So können die Anwendungen testbar, skalierbar und lose gekoppelt erstellt werden. \cite{nestjs} 
