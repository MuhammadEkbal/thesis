\section{Mongoose}
Mongoose ist ein Framework, das auf den MongoDB-Driver aufsetzt, also eine Object Data Modeling Bibliothek für MongoDB. Da als Datenspeicherung eine MongoDB ausgewählt wurde und Nest.js genutzt wird, ist es sinnvoll Mongoose zu verwenden. Denn auch hier ist die Dokumentation von Mongoose in Kombination mit MongoDB undNestJS sehr gut erläutert. Mongoose verwendet zum einen Schematas und zum anderen Models, um Beziehungen zwischen Daten zu verwalten. Dabei definiert ein Mongoose Schema die Struktur des Dokuments, also der Collection in der MongoDB. Models hingegen bieten die Schnittstellen zur Datenbank. Mit Hilfe dieser Schnittstelle können letztendlich die Daten von der Datenbank abgefragt, verändert oder ähnliches werden. \cite{mongoose}