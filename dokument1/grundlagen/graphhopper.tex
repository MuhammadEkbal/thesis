\section{Graphhopper}
\label{sec:basics:routing:gh}
Die Grundlage für den Routing-Algorithmus bildet die Open-Source-Routing Bibliothek Graphhopper.
Graphhopper ist in der Programmiersprache Java geschrieben und wird von der Graphhopper GmbH bereitgestellt.
Es kann frei zu privaten oder kommerziellen Zwecken genutzt werden.
Mit Graphhopper ist es möglich, verschiedene Algorithmen zur Kantengewichtung zu nutzen, wie zum Beispiel Dijkstra oder A*.
Das grundsätzliche Kartenmaterial von Graphhopper beruht auf der Open Street Map\cite{OpenStreetMapGraphHopper} \cite{GraphHopperGithub}.

Somit ist Graphhopper sehr gut als Basis für die Navigationsapplikation der Projektgruppe, beziehungsweise für die Berechnung der Kantengewichte im Routingalgorithmus geeignet.
Für einen bestimmten Kartenausschnitt übernimmt Graphhopper die Gewichtung der einzelnen Kanten.
Das Ergebnis dessen ist eine vollständige Route, die zwischen mindestens zwei gegeben Punkten eine Strecke berechnet, die aus Kanten mit möglichst niedrigen Gewichten besteht \cite{GraphHopperGithub}.

Weiterhin unterstützt Graphhopper unterschiedliche Fortbewegungsmittel wie Fahrräder, Autos oder auch Busse.
Je nach Vorgaben werden beispielsweise bei der Eingabe eines Fahrrads als Fortbewegungsmittel alle Kanten aus dem Kartenausschnitt entfernt, die nicht für Fahrradfahrer geeignet sind.
Zudem können Navigationsanweisungen durch Graphhopper bereitgestellt werden, die letztendlich über die Navigationsapplikation an den Endnutzer weitergeleitet werden \cite{GraphHopperGithub}.


Es zeigt sich, dass Graphhopper sehr viele Funtionen in der Berechnung von Routen übernehmen kann.
Um es jedoch sinnvoll in der Projektgruppe nutzen zu können, muss die Gewichtung der einzelnen Kanten angepasst werden, sodass auch verschiedene Umweltfaktoren wie die Feinstaubbelastung oder die Temperatur bei der Gewichtung der Kanten berücksichtigt werden und nicht nur die Distanz zwischen zwei Knoten (siehe \Fref{sec:arch:routing}).