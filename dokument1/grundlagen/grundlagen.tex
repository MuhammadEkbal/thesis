\chapter{Grundlagen}
In diesem Kapitel werden die dem Projekt zu Grunde liegenden Technologien und ihr Nutzen in Bezug auf das oben genannte Vorhaben erläutert. 
Damit werden die grundlegenden Information über die genutzten Technologien erklärt und es dient als eine Einführung in die folgenden Kapitel. 

Für die verschiedenen Frontendbereiche wird Angular genauer betrachtet, da die auf Grundlage dieses Frameworks aufgebaut wurden.
Für die Navigationsapplikation wird darüber hinaus noch das Ionic-Framework verwendet. 
In dem UIS-Frontend und der Navigationsapplikation wird außerdem noch OpenStreetMap für das Kartenmaterial eingesetzt. 
Der Routingbereiche verwendet die Graphhopper-Bibliothek, deshalb wird im folgenden eine Einführung in diese Bibliothek gegeben. 
Außerdem wird erklärt, welche Technologien für die IoT-Plattform benötigt werden. 
Dafür ist ein Grundwissen über Docker, NestJS und die Datenbank MongoDB notwendig. 
Abschließend wird das Entwicklerframework ESP8266 Arduino erklärt. 

\section{Angular}
Angular ist ein auf TypeScript basierendes Framework zum Erstellen einer Webapplikation. 
Es ist ein Open-Source-Projekt, welches von einer großen Community weiterentwickelt wird. 
Es ist dabei von der Vorgägerversion AngularJS abzugrenzen, da es von einem JavaScript- zu einem TypeScript-basierten Framework umentwickelt wurde.
Angular kommt in den drei Frontendprojekten zum Einsatz. 
Dabei ist zu erwähnen, dass es bei der Navigationsapplikation in Verbindung mit Ionic eingesetzt wird. 
Die Wahl dieses Frameworks basierte auf Vorwissen von Personen innerhalb der PG und der großen Unterstützung von vielen nutzbaren Paketen\cite{AngularInfoPage} \cite{AngularArchitektur}.


\section{Ionic}

Das Ionic-Framework unterstützt die Navigationsapplikation beim Entwerfen einer Android-Applikation mit normalen Webtechnologien, wie Html, Scss und JavaScript. 
Ionic nutzt dabei standardmäßig Angular, um die Webapplikation aufzubauen.
Die Erstellung der Android-Applikation wird durch das Framework Cordova durchgeführt, welches auch standardmäßig in Ionic integriert ist. 
Dadurch, dass Angular in mehreren Frontendnavigationen genutzt wird, gibt es einen gemeinsammen Wissenstand, sodass mehrere Personen ohne großen Lernaufwand in die Entwicklung mit einsteigen können. 
Durch die Einbindung von Cordova sind grundlegende Funktionen des mobilen Gerätes, wie beispielsweise die Standortbestimmung, verfügbar. 
Technologien, wie Leaflet, sind leicht in das Projekt integrierbar und die Designmöglichkeiten für kleinere Bildschirmgrößen werden auch erleichtert.
Dieses Framework unterstützt alle bekannten Funktionen, die für die Entwicklung einer Navigationsapplikation benötigt werden\cite{IonicInfoSite}.


\section{OpenStreetMap}
OpenStreetMap (kurz OSM) ist ein 2004 gegründetes Projekt, in dem eine frei verfügbare Weltkarte fortlaufend geschaffen wird. Dafür werden weltweit Daten über Straßen, Wälder, Häuser und andere Dinge gesammelt und in die Karte eingefügt. Die Daten von OSM werden täglich erneuert, sodass auch Gegebenheiten wie Baustellen auf Autobahn oder Ähnliches berücksichtigt wird.

Für die Projektgruppe kann die OpenStreetMap in mehreren Bereichen eingesetzt werden. So dient das Kartenmaterial von OSM als Grundlage für den Routing-Algorithmus, die Navigationsapplikation und das UIS-Frontend. Der große Vorteil von OSM gegenüber zum Beispiel Google Maps besteht darin, dass OSM nicht nur kostenlos, sondern die Nutzung auch komplett lizenzfrei ist\cite{OpenStreetMapFAQ}.


\section{Graphhopper}
\label{sec:basics:routing:gh}
Die Grundlage für den Routing-Algorithmus bildet die Open-Source-Routing Bibliothek Graphhopper.
Graphhopper ist in der Programmiersprache Java geschrieben und wird von der Graphhopper GmbH bereitgestellt.
Es kann frei zu privaten oder kommerziellen Zwecken genutzt werden.
Mit Graphhopper ist es möglich, verschiedene Algorithmen zur Kantengewichtung zu nutzen, wie zum Beispiel Dijkstra oder A*.
Das grundsätzliche Kartenmaterial von Graphhopper beruht auf der Open Street Map\cite{OpenStreetMapGraphHopper} \cite{GraphHopperGithub}.

Somit ist Graphhopper sehr gut als Basis für die Navigationsapplikation der Projektgruppe, beziehungsweise für die Berechnung der Kantengewichte im Routingalgorithmus geeignet.
Für einen bestimmten Kartenausschnitt übernimmt Graphhopper die Gewichtung der einzelnen Kanten.
Das Ergebnis dessen ist eine vollständige Route, die zwischen mindestens zwei gegeben Punkten eine Strecke berechnet, die aus Kanten mit möglichst niedrigen Gewichten besteht \cite{GraphHopperGithub}.

Weiterhin unterstützt Graphhopper unterschiedliche Fortbewegungsmittel wie Fahrräder, Autos oder auch Busse.
Je nach Vorgaben werden beispielsweise bei der Eingabe eines Fahrrads als Fortbewegungsmittel alle Kanten aus dem Kartenausschnitt entfernt, die nicht für Fahrradfahrer geeignet sind.
Zudem können Navigationsanweisungen durch Graphhopper bereitgestellt werden, die letztendlich über die Navigationsapplikation an den Endnutzer weitergeleitet werden \cite{GraphHopperGithub}.


Es zeigt sich, dass Graphhopper sehr viele Funtionen in der Berechnung von Routen übernehmen kann.
Um es jedoch sinnvoll in der Projektgruppe nutzen zu können, muss die Gewichtung der einzelnen Kanten angepasst werden, sodass auch verschiedene Umweltfaktoren wie die Feinstaubbelastung oder die Temperatur bei der Gewichtung der Kanten berücksichtigt werden und nicht nur die Distanz zwischen zwei Knoten (siehe \Fref{sec:arch:routing}).

\section{Docker} 
Für das Deployment der IoT-Plattform wurde Docker benutzt. Docker ist eine Containerplattform, die es ermöglicht die einzelnen Komponenten der IoT-Plattform zu erstellen, auszuführen und zu starten. So wird jedes Projekt in einem Container gepackt, der alle Abhängigkeiten, den Code und die Konfiguration isoliert enthält. Durch die Benutzung von Docker Containern wird eine höhere Flexibilität erreicht. Die Flexibilität wird dadurch erreicht, dass sich die Anwendung innerhalb eines Docker Containers auf beliebig vielen Systemen portieren lässt. Anders wäre dies bei dem Deployment auf einer virtuellen Maschine. Hier ist die Flexibilität eingeschränkt, weil sich die Anwendungen nur sehr aufwendig aktualisieren sowie portieren lassen. Allerdings machen die Container eine virtuelle Maschine nicht obsolet. Denn auch in diesem Projekt laufen die Container auf einer virtuellen Maschine. Unter anderem weil die Container keine Sicherheitsprobleme lösen. Zwar ist die Anwendung innerhalb des Containers durch die Kapselung sicher, allerdings sagt dies nichts über die Sicherheit außerhalb des Containers aus. Allerdigs verhilft Docker dazu schnell neue Softwareversionen zu veröfentlichen. So können Anwendungen innerhalb eines Docker Containers mit verschiedenen Tags, also Versionen, markiert werden. Bei Problemen mit einer Version kann ebenfalls sehr schnell auf eine andere Version zurückgegriffen werden. \cite{docker}

\section{NestJS}
NestJS wurde als Framework für die Entwicklung der IoT-Plattform ausgewählt, weil NestJS die Möglichkeit zum Erstellen von effizienten und skalierbaren Node.js serverseitigen Anwendungen bietet. Dieses Framwork nutzt progressives JavaScript und ist mit Typescript aufgebaut. Daher unterstützt NestJS vollständig Typescript. Da JavaScript sich durch Node.js als häufig genutzte Programmiersprache für das Front - und Backend entwickelt hat, sind dadurch viele Projekte enstanden, wie zum Beispiel Angular oder Vue. Diese haben die Produktivität, also die Erstellung von schnellen, testbaren und erweiterbaren Applikationen ermöglicht. NestJS hingehen bietet nicht nur diese Vorteile, sondern auch eine sofort einsatzbereite Anwendungsarchitektur. So können die Anwendungen testbar, skalierbar und lose gekoppelt erstellt werden. \cite{nestjs} 


\section{MongoDB}
\label{sec:grundlagen:datenbank:mongodb}
In diesem Kapitel wird näher auf die Entscheidung für eine MongoDB als Datenbank eingegangen.
MongoDB ist eine dokumentenorientierte NoSQL-Datenbank.
Somit ist MongoDB nicht an einem festen Datenbankschema gebunden, was als Nachteil betrachet werden kann.
Allerdings ist es möglich Validierungsregeln für Collections einzurichten.
Dadurch können nur Daten der Datenbank hinzugefügt werden, die ein gefordertes Format besitzen.
Eine Collection ist eine Sammlung von MongoDB-Dokumenten, also eine Sammlung von Datensätzen\cite{mongodbManual}.
Außerdem können so einfach neue Daten bzw. Collections hinzugefügt werden.
Also ist die Datenbank erweiterbar für zum Beispiel andere Umweltdaten.
Die Daten innerhalb einer Collection werden in einem JSON-ähnlichen Dokumentenformat, intern im BSON-Format, gespeichert.
Somit ist die Speicherung aussagekräftiger und leistungsfähiger als das Zeilen-Spalten-Modell in SQL Datenbanken.
Denn sie Umformatierung ist nicht mehr notwendig, weil bereits das JSON verwendet werden kann\cite{mongodb}.
\newline
MongoDB bietet verschiedene Abfragen, um Dokumente in einer Collection zu finden.
Hierbei ist zu beachten, dass lediglich eine Collection mit einer Abfrage durchsucht werden kann.
Somit ist der Join-Operator, den es in SQL Datenbanken gibt, nicht verfügbar.
Daher sollten die Abfragen breits bei der Erstellung der Collections berücksichtigt werden.
Ab Version 3.2 ermöglicht MongoDB Joins als \textit{lookups}\cite{mongodbJoin}.
Abgesehen von dieser Einschränkung ist die Abfragepsrache, die MongoDB bietet leistungsfähig.
Denn die Filterung und Sortierung der Daten ist unabhängig von der Tiefe der Verschachtelung in dem Dokument.
Ein weiterer Vorteil der Abfragesprache sind die unterstützten Aggregationen.
Insbesondere sind hier geobasierte Suchen möglich, die in dem PG RiO Projekt von Vorteil, für die Abfragen des Routing Dienstes, sind.
Außerdem haben Abfragen das JSON-Format und können so einfach zusammengesetzt werden.
Also müssenkeine Zeichenfolgen mehr verkettet werden, um SQL-Abfragen zu generieren\cite{mongodb}.
\newline
MongoDB bietet außerdem das Erstellen von Indizes auf Collections.
So können die Abfragen effizient in MongoDB ausgeführt werden, denn mittels des Indizes wird die Anzahl der zu prüfenden Dokumente verringert.
Ohne Indizes müssen Collection-Scans durchgeführt werden.
Also muss jedes Dokument einer Collection überprüft werden, um festzustellen, ob dieses Dokument mit der Query-Anweisung übereinstimmt.
Da in der Datenbank eine große Datenmenge gespeichert wird, ist das Verwenden der Indizes sinnvoll\cite{mongodbIndices}.
\newline
Der ausschlaggebene Grund für die Entscheidung eine MongoDB zu verwenden, neben den zuvor genannten Vorteilen, ist, dass das verwendete Framework NestJS eine sehr gute Dokumentation zur Verwendung von MongoDB in Kombination mit NestJS bietet\cite{mongoNest}.

\section{Mongoose}
Mongoose ist ein Framework, das auf den MongoDB-Driver aufsetzt, also eine Object Data Modeling Bibliothek für MongoDB. Da als Datenspeicherung eine MongoDB ausgewählt wurde und Nest.js genutzt wird, ist es sinnvoll Mongoose zu verwenden. Denn auch hier ist die Dokumentation von Mongoose in Kombination mit MongoDB undNestJS sehr gut erläutert. Mongoose verwendet zum einen Schematas und zum anderen Models, um Beziehungen zwischen Daten zu verwalten. Dabei definiert ein Mongoose Schema die Struktur des Dokuments, also der Collection in der MongoDB. Models hingegen bieten die Schnittstellen zur Datenbank. Mit Hilfe dieser Schnittstelle können letztendlich die Daten von der Datenbank abgefragt, verändert oder ähnliches werden. \cite{mongoose}

\section{ESP8266 Arduino}
ESP8266 Arduino ist ein Entwicklungsframework, das die Arduino"=Umgebung für den Mikrocontroller ESP8266 verfügbar macht.
So ist es möglich Programme mit aus Arduino vertrauten Bibliotheken und Funktionalitäten zu entwickeln und direkt auf dem ESP8266 aufzuspielen und auszuführen.
ESP8266 Arduino stellt u.a. Bibliotheken für die Kommunikation über WLAN, TCP sowie UDP, und für die Nutzung des Flash"=Dateisystems zur Verfügung \cite{arduinoESP}.
Arduino selbst ist eine offene Elektronik-Plattform basierend auf einfach zu verwendender Hard- und Software \cite{arduino}.

Innerhalb der Projektgruppe wird ESP8266 Arduino als Grundlage zur Entwicklung der Sensorknoten"=Firmware verwendet.
Details zur Einrichtung des Frameworks finden sich im Entwicklerhandbuch (siehe Dokumentation Teil II, Abschnitt 3.1.3).




