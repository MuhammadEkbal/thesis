\section{Docker} 
Für das Deployment der IoT-Plattform wurde Docker benutzt. Docker ist eine Containerplattform, die es ermöglicht die einzelnen Komponenten der IoT-Plattform zu erstellen, auszuführen und zu starten. So wird jedes Projekt in einem Container gepackt, der alle Abhängigkeiten, den Code und die Konfiguration isoliert enthält. Durch die Benutzung von Docker Containern wird eine höhere Flexibilität erreicht. Die Flexibilität wird dadurch erreicht, dass sich die Anwendung innerhalb eines Docker Containers auf beliebig vielen Systemen portieren lässt. Anders wäre dies bei dem Deployment auf einer virtuellen Maschine. Hier ist die Flexibilität eingeschränkt, weil sich die Anwendungen nur sehr aufwendig aktualisieren sowie portieren lassen. Allerdings machen die Container eine virtuelle Maschine nicht obsolet. Denn auch in diesem Projekt laufen die Container auf einer virtuellen Maschine. Unter anderem weil die Container keine Sicherheitsprobleme lösen. Zwar ist die Anwendung innerhalb des Containers durch die Kapselung sicher, allerdings sagt dies nichts über die Sicherheit außerhalb des Containers aus. Allerdigs verhilft Docker dazu schnell neue Softwareversionen zu veröfentlichen. So können Anwendungen innerhalb eines Docker Containers mit verschiedenen Tags, also Versionen, markiert werden. Bei Problemen mit einer Version kann ebenfalls sehr schnell auf eine andere Version zurückgegriffen werden. \cite{docker}