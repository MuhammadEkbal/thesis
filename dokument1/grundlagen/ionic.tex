\section{Ionic}

Das Ionic-Framework unterstützt die Navigationsapplikation beim Entwerfen einer Android-Applikation mit normalen Webtechnologien, wie Html, Scss und JavaScript. 
Ionic nutzt dabei standardmäßig Angular, um die Webapplikation aufzubauen.
Die Erstellung der Android-Applikation wird durch das Framework Cordova durchgeführt, welches auch standardmäßig in Ionic integriert ist. 
Dadurch, dass Angular in mehreren Frontendnavigationen genutzt wird, gibt es einen gemeinsammen Wissenstand, sodass mehrere Personen ohne großen Lernaufwand in die Entwicklung mit einsteigen können. 
Durch die Einbindung von Cordova sind grundlegende Funktionen des mobilen Gerätes, wie beispielsweise die Standortbestimmung, verfügbar. 
Technologien, wie Leaflet, sind leicht in das Projekt integrierbar und die Designmöglichkeiten für kleinere Bildschirmgrößen werden auch erleichtert.
Dieses Framework unterstützt alle bekannten Funktionen, die für die Entwicklung einer Navigationsapplikation benötigt werden\cite{IonicInfoSite}.
