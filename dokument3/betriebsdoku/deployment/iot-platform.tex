\subsection{IoT-Plattform}
In diesem Abschnitt wird das Deployment der IoT-Plattform auf dem Produktiv-System beschrieben. 
Das Deployment auf dem Produktiv-System ähnelt dem Deployment auf dem Development-System sehr stark, denn auch hier wird dafür docker-compose verwendet. 
Im Folgenden werden die Schritte genannt, die erforderlich sind, um die IoT-Plattform produktiv zu deployen. \newline
\begin{enumerate}
	\item Zuerst muss sich der Entwickler auf die VM einloggen
	\item Danach müssen die \textit{env.} Datein angelegt oder gegebenenfalls angepasst werden. Die \textit{env.} Dateien wurden bereits im Dokument 2 in Abschnitt 3.3.4.1 erläutert. Insbesondere muss hier auf die Anpassung der Pfade, zum Beispiel die der Datenbank, geachtet werden. \newline
	Wenn beispielsweise ein neuer Microservice deployed werden soll muss eine neue \textit{env.} Datei angelegt werden. Unter \textit{home/pgrio/microservices/} muss nun ein neuer Ordner BeispielMicroservice angelegt werden, der die \textit{env.} Datei enthält \newline
	Die \textit{env.} Datei muss den folgenden Inhalt haben: \newline
	\begin{lstlisting}[language=json,firstnumber=1,basicstyle=\footnotesize]
	DB_URL = ProduktivDatenbankURL
	API_GATEWAY_URL = ProduktivAPIGatewayURL
	SERVICE_USERNAME = BeispielMicroservice
	SERVICE_PASSWORD = BeispielMicroservicePassword
	RENEW_INTERVALL = 5
	SERVICE_PORT = 3102
	SERVICE_HOST = beispielmicroservice
	\end{lstlisting}
	\begin{itemize}
		\item DB\_URL: URL der Datenbank, die auf dem Produktiv-System läuft
		\item API\_GATEWAY\_URL: URL des API Gateways, das auf dem Produktiv-System läuft
		\item SERVICE\_USERNAME: Der Benutzername des Microservices, der auch in der Datenbank hinterlegt ist
		\item SERVICE\_PASSWORD: Das Passwort des Microservices, das auch in der Datenbank hinterlegt ist
		\item SERVICE\_PORT: Der Port des Microservices, auf dem der Microservice erreichbar ist. Dieser ist nur dür das API Gateway offen und nicht außerhalb des Docker Netzwerkes
		\item RENEW\_INTERVAL Das Zeitintervall, in dem der Microservice ein neues Token anfragt
		\item SERVICE\_HOST: Entspricht dem Namen des Microservices in der yaml Datei und ist notwendig, damit das API Gateway die Anfragen, an der richtigen Microservice weiterleiten kann
		\end{itemize}
	\item Nachdem die \textit{env.} Dateien erstellt wurden muss die yaml Datei angepasst werden. Für das zuvor genannte Beispiel, einen neuen Microservice zu deployen, ist die Anpassung der yaml Datei ab Zeile 85 notwendig. Folgend ist die aktuelle yaml Datei der Produktionsumgebung gezeigt: \newline
	\begin{lstlisting}[language=json,basicstyle=\footnotesize]
	version: '3.5'
	services:
	mqttbroker:
	image: pgrio/iot-mqtt-broker-hive:stable
	volumes:
	- /home/pgrio/iot-platform/mqtt/config.properties:/opt/hivemq/config.properties
	ports:
	- "127.0.0.1:1883:1883"
	restart: always
	depends_on:
	- identityservice
	identityservice:
	image: pgrio/iot-identity-service:stable
	volumes:
	- /home/pgrio/iot-platform/identityservice/production.env:/usr/src/app/production.env
	environment:
	- NODE_ENV=production
	expose:
	- "9090"
	restart: always
	depends_on:
	- collection_creater
	api_gateway:
	image: pgrio/iot-api-gateway:stable
	volumes:
	- /home/pgrio/iot-platform/apigateway/production.env:/usr/src/app/production.env
	environment:
	- NODE_ENV=production
	ports:
	- "127.0.0.1:8080:8080"
	restart: always
	depends_on:
	- identityservice
	iot-data-collector:
	image: pgrio/iot-data-collector:stable
	volumes:
	- /home/pgrio/iot-platform/datacollector/config.env:/usr/src/app/config.env
	environment:
	- NODE_ENV=production
	restart: always
	depends_on:
	- collection_creater
	collection_creater:
	image: pgrio/iot-collection-creater:stable
	volumes:
	- /home/pgrio/iot-platform/collectioncreater/config.env:/usr/src/app/config.env
	environment:
	- NODE_ENV=production
	pm25microservice:
	image: pgrio/iot-microservice-pm25:stable
	expose:
	- "3000"
	volumes:
	- /home/pgrio/iot-platform/microservices/pm25/production.env:/usr/src/app/production.env
	environment:
	- NODE_ENV=production
	restart: always
	tempmicroservice:
	image: pgrio/iot-microservice-temp:stable
	expose:
	- "3004"
	volumes:
	- /home/pgrio/iot-platform/microservices/temp/production.env:/usr/src/app/production.env
	environment:
	- NODE_ENV=production
	restart: always
	svmicroservice:
	image: pgrio/iot-microservice-sv:stable
	expose:
	- "3002"
	volumes:
	- /home/pgrio/iot-platform/microservices/sv/production.env:/usr/src/app/production.env
	environment:
	- NODE_ENV=production
	restart: always
	uismicroservice:
	image: pgrio/iot-microservice-uis:stable
	expose:
	- "3001"
	volumes:
	- /home/pgrio/iot-platform/microservices/uis/production.env:/usr/src/app/production.env
	environment:
	- NODE_ENV=production
	restart: always
	beispielmicroservice:
	image: pgrio/iot-microservice-beispiel:stable
	expose:
	- "3102"
	volumes:
	- /home/pgrio/iot-platform/microservices/BeispielMicorservice/production.env:/usr/src/app/production.env
	environment:
	- NODE_ENV=production
	restart: always
	\end{lstlisting}
	\begin{itemize}
		\item image: Der Name des Image, wie er auf Docker-Hub veröffentlicht wurde. Außerdem wird die Versionierung angegeben
		\item expose: Der Port unter dem der Service intern erreichbar ist
		\item volumes: Der Pfad, unter dem die \textit{env.} Datei des Service zu finden ist
		\item environment: Setzt die Umgebungsvariable auf eine Umgebungsdatei. Sobald der Container gestartet wird, hat die definierte Umgebungsvariable Vorrang
		\item restart: In diesem Fall werden die Container immer neu gestartet
	\end{itemize}
	\item Der Unterschied zum Deployment auf der Development Umgebung ist die Versionierung der Images. Für die Produktiv Umgebung wird die Version \textit{stable} verwendet, siehe dazu unter \textit{image} in der yaml Datei
	\item Sobald der Master eines Repositoriums der IoT-Plattform aktualisiert wurde veröffentlicht der Jenkins ein neues Image mit der Versionierung \textit{stable}
	\item Für das Beispiel, das Deployment eines neuen Microservices, ist es notwendig das aktuelle Image von Docker Hub zu beziehen. Dieser Schritt ist ebenfalls notwendig, wenn der Jenkins ein anderes Image der IoT-Plattform aktualisiert hat. Mit dem folgenden Befehl wird das aktuellste Image von Docker Hub bezogen: \newline
	\console{docker pull pgrio/iot-microservice-beispielmicroservice:stable} (\textit{imagename:stable})
	Insbesondere muss hier die Versionierung des Image angegeben werden. Auf dem Deployment-System ist dies nicht notwendig, da die Standardversionierung \textit{latest} ist
	\item Mit dem Befehl \console{docker-compose up} kann nun die IoT-Plattform auf dem Produktiv-System gestartet werden. \newline
	Falls bereits Container aktiv sind, muss die IoT-Plattform nicht komplett heruntergefahren werden, wenn ein neuer Service deployed wurde oder lediglich ein neues Image eines Containers verfügbar ist. Mit dem zuvor genannten Befehl werden die Container neu gestartet, die ein neues Image besitzen
\end{enumerate}