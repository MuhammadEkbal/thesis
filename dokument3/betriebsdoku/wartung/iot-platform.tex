\subsection{IoT-Plattform}
In diesem Abschnitt wird erläutert, wie die IoT-Plattform gewartet werden kann. 
Insbesondere wird im folgenden auf die Fehlerbehebung eingegangen. 
Falls die IoT-Plattform oder einer der Services der IoT-Plattform nicht erreichbar ist, wird dies an zwei Stellen deutlich. Entweder können externe Dienste die IoT-Plattform nicht mehr erreichen oder die Sensorknoten können keine Daten mehr an den MQTT-Broker senden. Für diese beiden groben Fehlerquellen werden im Folgenden Problemerkennungs - und Lösungsmaßnahmen aufgezeigt. \newline
Falls die IoT-Plattform für die externen Dienste nicht mehr erreichbar ist kann dies entweder an dem API-Gateway, dem Identity-Service oder an einem der Microservices liegen. 
Für die Fehlererkennung ist es empfehlenswert die Logs des Containers auf der VM anzuschauen. Dafür sind folgende Schritte notwendig:
\begin{enumerate}
	\item Mit dem Befehl \console{docker ps} werden alle aktiven Container angezeigt. Um sich die Logs eines Containers anschauen zu können muss die ID des Containers bekannt sein
	\item Bei dem Aufruf wird nicht nur die ID aller aktiven Container angezeigt, sondern auch deren Laufzeit. Falls sich ein Container immer wieder neu startet wird das hier deutlich, da dieser dann erst seit ein paar Sekunden wieder aktiv ist. Falls dem so ist, sollten sich die Logs dieses Containers zuerst angeschaut werden. Dies tritt beispielsweise bei den Microservices auf, wenn diese sich nicht an dem API-Gateway authentifzieren können. Gründe hierfür können sein: fehlende Parameter in der env Datei oder falsche Zugangsdaten des Microservices in der Datenbank hinterlegt.
	\item Mit dem Befehl \console{docker logs \textit{Container ID}} werden die Logs des Containers angezeigt. In den meisten Fällen zeigen die Logs das Problem an
	\item Bug lösen und danach mit dem jeweiligen Branch, entweder Develop oder Master, zusammenführen
	\item Nachdem Jenkins das Image auf Docker Hub veröffentlicht hat kann mit \console{docker pull \textit{Image Name}} das aktuellste Image von Docker Hub bezogen werden
	\item Mit \console{docker-compose up} die IoT-Plattform starten
\end{enumerate}
Folgend werden Fehler augfelistet, die häufig auftreten:
\begin{enumerate}
	\item Imports innerhalb des Codes. Sobald ein fixer Pfad angegeben wird, findet das Projekt die Imports nicht und wirft deshalb Fehler
	\item Außerdem starten sich die Projekte nicht neu, wenn eine Datenbankverbindung auftritt. Das hat zur Folge, dass sich die Projekte danach nicht mehr mit der Datenbank verbinden und somit Fehler werfen. Dieses Problem ist ein bekannter Bug und wird ebenfalls in  \Fref{sec:bugsundeinschraenkungen} aufgeführt
	\item Microservice kann sich nicht an dem API-Gateway authentifizieren, wie zuvor bereits erwähnt
\end{enumerate}
Falls das Problem mit Hilfe der Logs nicht bekannt wird, ist es hilfreich den betreffenden Container neu zu starten. Um zu filtern, welcher Container gestoppt werden sollte, ist es hilfreich die nicht funktionierenden Abfragen zu definieren. 
\begin{itemize}
	\item Falls sich der externe Dienst nicht authentifizieren kann, also kein Token abfragen kann, sollte der API-Gateway-Container neu gestartet werden
	\item  Falls das Problem danach weiterhin besteht, sollte zusätzlich der Identity-Service-Container neu gestartet werden
	\item  Wenn sich ein externer Dienst jedoch authentifizieren kann und lediglich keine Antwort eines Microservices bekommt, sollte der jeweilige Microservice neu gestartet werden
\end{itemize}
Sobald herausgefunden wurde, welcher Container gestoppt und neu gestartet werden sollte, müssen folgende Schritte ausgeführt werden: 
\begin{enumerate}
	\item \console{docker stop \textit{Container ID}}
	\item \console{docker-compose up}
	\item \console{docker ps}, um zu kontrollieren, ob der Container erfolgreich neu gestartet wurde
\end{enumerate}

Eine weitere Schnittstelle zur IoT-Plattform bietet der MQTT-Broker. 
Falls an dieser Schnittstelle Fehler auftreten wird dies deutlich, wenn Sensorknoten keine Daten mehr an die IoT-Plattform senden. Gründe dafür können zum einen der MQTT-Broker sein oder zum anderen der Data-Collector. Für die Fehlerbehebung an dieser Schnittstelle sind folgende Schritte empfehlenswert:
\begin{enumerate}
	\item Zuerst sollte geprüft werden, ob Daten der Sensorknoten an dem MQTT-Broker ankommen
	\item Falls dies nicht der Fall ist, sollten sich die Logs des MQTT-Broker-Containers angeschaut werden
	\item Falls die Daten ankommen, aber nicht in der Datenbank gespeichert werden sollten sich die Logs des Data-Collector-Containers angeschaut werden
\end{enumerate}
Das Vorgehen zum Anzeigen der Logs wurde zuvor bereits erläutert. Wie auch bei der anderen Schnittstelle, sollte erst danach der Container gestoppt und neu gestartet werden, falls das Problem weiterhin besteht. 
